\tabulinesep = 5pt
\begin{longtabu} to \textwidth {
    |l|X[3l]|X[0.75l]|}
\caption{Attributes for Measurement} \label{table:attributes-for-measurement} \\

\hline
Attribute & Description & Occurrence \\
\hline
\endfirsthead

\hline
\multicolumn{3}{|c|}{Continuation of Table \ref{table:attributes-for-measurement}}\\
\hline
Attribute & Description & Occurrence \\
\hline
\endhead

\gls{code measurement}
&
\glsentrydesc{code measurement}
\newline \gls{code measurement} is a optional attribute.
&
0..1 \\
\hline
 
\gls{maximum attribute}
&
The maximum value for this measurement. Exceeding this
value would indicate the tool is not usable.
\newline \gls{maximum attribute} is a optional attribute.
&
0..1 \\
\hline

\gls{minimum attribute}
&
The minimum value for this measurement. Exceeding this
value would indicate the tool is not usable.
\newline \gls{minimum attribute} is a optional attribute.
&
0..1 \\
\hline

\gls{nominal attribute}
&
The as advertised value for this measurement.
\newline \gls{nominal attribute} is a optional attribute.
&
0..1 \\
\hline

\gls{significantdigits}
&
The number of significant digits in the reported value. This is used by applications to determine accuracy of values. This \MAY be specified for all numeric values.
\newline \gls{significantdigits} is a optional attribute.
&
0..1 \\
\hline

\gls{units}
&
The units for the measurements. MTConnect Standard defines all the units for each measurement, so this is mainly for documentation sake. See MTConnect \citetitle{MTCPart2} 7.2.2.5 for the full list of units.
\newline \gls{units} is a optional attribute.
&
0..1 \\
\hline

\gls{nativeunits}
&
The units the measurement was originally recorded in. This is only necessary if they differ from units. See \citetitle{MTCPart2} Section 7.2.2.6 for the full list of units.
\newline \gls{nativeunits} is a optional attribute.
&
0..1 \\
\hline


\end{longtabu}