% Generated 2021-02-02 18:15:32 +0530
\subsection{Common Measurement Types} \label{sec:Common Measurement Types}


This section lists the common \block{Measurement} types for \block{CuttingTool} and \block{CuttingItem}.


\subsubsection{BodyDiameterMax}
\label{sec:BodyDiameterMax}



The largest diameter of the body of a Tool Item.



\subsubsection{BodyLengthMax}
\label{sec:BodyLengthMax}



The distance measured along the X axis from that point of the item closest to the workpiece, including the Cutting Item for a Tool Item but excluding a protruding locking mechanism for an Adaptive Item, to either the front of the flange on a flanged body or the beginning of the connection interface feature on the machine side for cylindrical or prismatic shanks.



\subsubsection{ChamferFlatLength}
\label{sec:ChamferFlatLength}



The flat length of a chamfer.



\subsubsection{ChamferWidth}
\label{sec:ChamferWidth}



The width of the chamfer.



\subsubsection{CornerRadius}
\label{sec:CornerRadius}



The nominal radius of a rounded corner measured in the X Y-plane.



\subsubsection{CuttingDiameter}
\label{sec:CuttingDiameter}



The diameter of a circle on which the defined point Pk located on this Cutting Tool. The normal of the machined peripheral surface points towards the axis of the Cutting Tool.



\subsubsection{CuttingDiameterMax}
\label{sec:CuttingDiameterMax}



The maximum diameter of a circle on which the defined point Pk of each of the master inserts is located on a Tool Item. The normal of the machined peripheral surface points towards the axis of the Cutting Tool.



\subsubsection{CuttingEdgeLength}
\label{sec:CuttingEdgeLength}



The theoretical length of the cutting edge of a Cutting Item over sharp corners.



\subsubsection{CuttingHeight}
\label{sec:CuttingHeight}



The distance from the basal plane of the Tool Item to the cutting point.



\subsubsection{CuttingReferencePoint}
\label{sec:CuttingReferencePoint}



The theoretical sharp point of the Cutting Tool from which the major functional dimensions are taken.



\subsubsection{DepthOfCutMax}
\label{sec:DepthOfCutMax}



The maximum engagement of the cutting edge or edges with the workpiece measured perpendicular to the feed motion.



\subsubsection{DriveAngle}
\label{sec:DriveAngle}



Angle between the driving mechanism locator on a Tool Item and the main cutting edge.



\subsubsection{FlangeDiameter}
\label{sec:FlangeDiameter}



The dimension between two parallel tangents on the outside edge of a flange.



\subsubsection{FlangeDiameterMax}
\label{sec:FlangeDiameterMax}



The dimension between two parallel tangents on the outside edge of a flange.



\subsubsection{FunctionalLength}
\label{sec:FunctionalLength}



The distance from the gauge plane or from the end of the shank to the furthest point on the tool, if a gauge plane does not exist, to the cutting reference point determined by the main function of the tool. The \block{CuttingTool} functional length will be the length of the entire tool, not a single Cutting Item. Each \block{CuttingItem} can have an independent \block{FunctionalLength} represented in its measurements. 



\subsubsection{FunctionalWidth}
\label{sec:FunctionalWidth}



The distance between the cutting reference point and the rear backing surface of a turning tool or the axis of a boring bar.



\subsubsection{IncribedCircleDiameter}
\label{sec:IncribedCircleDiameter}



The diameter of a circle to which all edges of a equilateral and round regular insert are tangential.



\subsubsection{InsertWidth}
\label{sec:InsertWidth}



W1 is used for the insert width when an inscribed circle diameter is not practical.



\subsubsection{OverallToolLength}
\label{sec:OverallToolLength}



The largest length dimension of the Cutting Tool including the master insert where applicable.



\subsubsection{PointAngle}
\label{sec:PointAngle}



The angle between the major cutting edge and the same cutting edge rotated by 180 degrees about the tool axis.



\subsubsection{ProtrudingLength}
\label{sec:ProtrudingLength}



The dimension from the yz-plane to the furthest point of the Tool Item or Adaptive Item measured in the -X direction.



\subsubsection{ShankDiameter}
\label{sec:ShankDiameter}



The dimension of the diameter of a cylindrical portion of a Tool Item or an Adaptive Item that can participate in a connection.



\subsubsection{ShankHeight}
\label{sec:ShankHeight}



The dimension of the height of the shank.



\subsubsection{ShankLength}
\label{sec:ShankLength}



The dimension of the length of the shank.



\subsubsection{StepDiameterLength}
\label{sec:StepDiameterLength}



The length of a portion of a stepped tool that is related to a corresponding cutting diameter measured from the cutting reference point of that cutting diameter to the point on the next cutting edge at which the diameter starts to change.



\subsubsection{StepIncludedAngle}
\label{sec:StepIncludedAngle}



The angle between a major edge on a step of a stepped tool and the same cutting edge rotated 180 degrees about its tool axis.



\subsubsection{ToolCuttingEdgeAngle}
\label{sec:ToolCuttingEdgeAngle}



The angle between the tool cutting edge plane and the tool feed plane measured in a plane parallel the xy-plane.



\subsubsection{ToolLeadAngle}
\label{sec:ToolLeadAngle}



The angle between the tool cutting edge plane and a plane perpendicular to the tool feed plane measured in a plane parallel the xy-plane.



\subsubsection{ToolOrientation}
\label{sec:ToolOrientation}



The angle of the tool with respect to the workpiece for a given process. The value is application specific.



\subsubsection{UsableLengthMax}
\label{sec:UsableLengthMax}



Maximum length of a Cutting Tool that can be used in a particular cutting operation including the non-cutting portions of the tool.



\subsubsection{Weight}
\label{sec:Weight}



The total weight of the Cutting Tool in grams. The force exerted by the mass of the Cutting Tool.



\subsubsection{WiperEdgeLength}
\label{sec:WiperEdgeLength}



The measure of the length of a wiper edge of a Cutting Item.


