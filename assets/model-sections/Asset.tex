% Generated 2020-05-20 17:19:31 -0400
\subsection{Asset} \label{sec:Asset}

The MTConnect Standard supports a simple distributed storage mechanism that allows applications and equipment to share and exchange complex information models in a similar way to a distributed data store.  The \gls{asset information model} associates each electronic \gls{mtconnectassets} document with a unique identifier and allows for some predefined mechanisms to find, create, request, updated, and delete these electronic documents in a way that provides for consistency across multiple pieces of equipment.

The protocol provides a limited mechanism of accessing \glspl{mtconnect asset} using the following properties: \gls{assetid}, \gls{asset} type (element name of \gls{asset} root), and the piece of equipment associated with the \gls{asset}.  These access strategies will provide the following services and answer the following questions: What \glspl{asset} are from a particular piece of equipment?  What are the \glspl{asset} of a particular type? What \glspl{asset} is stored for a given \gls{assetid}?

Although these mechanisms are provided, an \gls{agent} should not be considered a data store or a system of reference.  The \gls{agent} is providing an ephemeral storage capability that will temporarily manage the data for applications wishing to communicate and manage data as need-ed by the various processes.  An application cannot rely on an \gls{agent} for long term persistence or durability since the \gls{agent} is only required to temporarily store the \gls{asset} data and may require an-other system to provide the source data upon initialization.  An \gls{agent} is always providing the best-known equipment centric view of the data given the limitations of that piece of equipment.

\begin{note}
Note: Currently only cutting tools have been addressed by the MTConnect Standard and other MTConnect Assets will be defined in later versions of the Standard.

\end{note}


\subsubsection{Asset}
  \label{sec:Asset}



  
		
			p {padding:0px; margin:0px;}
		
	
  
An Asset&#160;is something that is used in the manufacturing process, but is not permanently associated with a single piece of equipment, can be removed from the piece of equipment without compromising its function, and can be associated with other pieces of equipment during its lifecycle.



\paragraph{Attributes of Asset}\mbox{}
\label{sec:Attributes of Asset}

\tbl{attributes of Asset} lists the attributes of \texttt{Asset}.

\begin{table}[ht]
\centering 
  \caption{Attributes of Asset}
  \label{table:attributes of Asset}
\tabulinesep=3pt
\begin{tabu} to 6in {|l|l|l|} \everyrow{\hline}
\hline
\rowfont\bfseries {Attribute} & {Type} & {Multiplicity} \\
\tabucline[1.5pt]{}
\texttt{assetId} & \texttt{ID} & 1 \\
\texttt{deviceUuid} & \texttt{NMTOKEN} & 1 \\
\texttt{removed} & \texttt{boolean} & 0..1 \\
\texttt{timestamp} & \texttt{dateTime} & 1 \\
\end{tabu}
\end{table}
\FloatBarrier


Descriptions for attributes of \texttt{Asset}:

\begin{itemize}
\item \texttt{assetId} : The unique identifier for an Asset.
\item \texttt{deviceUuid} : The piece of equipment's uuid that supplied the Asset's data.
\item \texttt{removed} : An indicator that the Asset has been removed from the piece of equipment.
\item \texttt{timestamp} : The point in time time the Asset data was last modified.
\end{itemize}

\paragraph{Elements of Asset}\mbox{}
\label{sec:Elements of Asset}

\tbl{elements of Asset} lists the elements of \texttt{Asset}.

\begin{table}[ht]
\centering 
  \caption{Elements of Asset}
  \label{table:elements of Asset}
\tabulinesep=3pt
\begin{tabu} to 6in {|l|l|l|} \everyrow{\hline}
\hline
\rowfont\bfseries {Association Name} & {Element} & {Multiplicity} \\
\tabucline[1.5pt]{}
\texttt{Description} & \texttt{Description} & 0..1 \\
\end{tabu}
\end{table}
\FloatBarrier


Descriptions for elements of \texttt{Asset}:

\begin{itemize}
\item \texttt{Description} : An element that can contain any descriptive content.
\end{itemize}
\FloatBarrier
