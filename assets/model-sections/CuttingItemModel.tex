% Generated 2020-02-28 19:23:52 -0500
\subsection{CuttingItemModel} \label{model:CuttingItemModel}
\subsubsection{CuttingItem}
  \label{type:CuttingItem}

\FloatBarrier

A CuttingItem is the portion of the tool that physically removes the material from the workpiece by shear deformation.

\begin{table}[ht]
\centering 
  \caption{\texttt{Properties of CuttingItem}}
  \label{properties:CuttingItem}
\tabulinesep=3pt
\begin{tabu} to 6in {|l|l|l|} \everyrow{\hline}
\hline
\rowfont\bfseries {Property} & {Type} & {Multiplicity} \\
\tabucline[1.5pt]{}
\texttt{Description} & \texttt{string} & 0..1 \\
\texttt{grade} & \texttt{string} & 0..1 \\
\texttt{indices} & \texttt{string} & 1 \\
\texttt{itemId} & \texttt{ID} & 0..1 \\
\texttt{Locus} & \texttt{string} & 0..1 \\
\texttt{manufacturers} & \texttt{string} & 0..1 \\
\texttt{ProgramToolGroup} & \texttt{string} & 0..1 \\
\texttt{CutterStatus} & \texttt{Status} & 1..* \\
\texttt{ItemLife} & \texttt{ItemLife} & 0..1 \\
\texttt{ItemLife} & \texttt{ItemLife} & 0..1 \\
\texttt{ItemLife} & \texttt{ItemLife} & 0..1 \\
\texttt{Measurements} & \texttt{Measurement} & 0..* \\
\end{tabu}
\end{table}
\FloatBarrier


\paragraph{\texttt{Description}}\mbox{}
\newline\tab A free-form description of the Cutting Item.

\paragraph{\texttt{grade}}\mbox{}
\newline\tab The material composition for this Cutting Item.


\paragraph{\texttt{indices}}\mbox{}
\newline\tab The number or numbers representing the individual Cutting Item or items on the tool.


\paragraph{\texttt{itemId}}\mbox{}
\newline\tab The manufacturer identifier of this Cutting Item.

\paragraph{\texttt{Locus}}\mbox{}
\newline\tab A free form description of the location on the Cutting Tool.

\paragraph{\texttt{manufacturers}}\mbox{}
\newline\tab The manufacturers of the Cutting Item or Tool.

\paragraph{\texttt{ProgramToolGroup}}\mbox{}
\newline\tab The tool group this item is assigned in the part program.

\paragraph{\texttt{CutterStatus}}\mbox{}
\newline\tab Placeholder for documentation!

\paragraph{\texttt{ItemLife}}\mbox{}
\newline\tab Placeholder for documentation!

\paragraph{\texttt{ItemLife}}\mbox{}
\newline\tab Placeholder for documentation!

\paragraph{\texttt{ItemLife}}\mbox{}
\newline\tab Placeholder for documentation!

\paragraph{\texttt{Measurements}}\mbox{}
\newline\tab Placeholder for documentation!
\FloatBarrier
\subsubsection{ItemLife}
  \label{type:ItemLife}

\FloatBarrier

The life of this Cutting Item.

\begin{table}[ht]
\centering 
  \caption{\texttt{Properties of ItemLife}}
  \label{properties:ItemLife}
\tabulinesep=3pt
\begin{tabu} to 6in {|l|l|l|} \everyrow{\hline}
\hline
\rowfont\bfseries {Property} & {Type} & {Multiplicity} \\
\tabucline[1.5pt]{}
\texttt{countDirection} & \texttt{CountDirectionType} & 1 \\
\texttt{initial} & \texttt{float} & 0..1 \\
\texttt{limit} & \texttt{string} & 0..1 \\
\texttt{warning} & \texttt{float} & 0..1 \\
\end{tabu}
\end{table}
\FloatBarrier


\paragraph{\texttt{countDirection}}\mbox{}
\newline\tab Indicates if the item life counts from zero to maximum or maximum to zero.

Enumeration for countDirection types.

\begin{table}[ht]
\centering 
  \caption{\texttt{CountDirectionType} Enumeration}
\tabulinesep=3pt
\begin{tabu} to 6in {|l|X|} \everyrow{\hline}
\hline
\rowfont\bfseries {Name} & {Description} \\
\tabucline[1.5pt]{}
\texttt{UP} & The tool life counts up from zero to the maximum.
 \\
\texttt{DOWN} & The tool life counts down from the maximum to zero. \\
\end{tabu}
\end{table} 
\FloatBarrier

\paragraph{\texttt{initial}}\mbox{}
\newline\tab The initial life of the item when it is new

\paragraph{\texttt{limit}}\mbox{}
\newline\tab The end of life limit for this item.

\paragraph{\texttt{warning}}\mbox{}
\newline\tab The point at which a item life warning will be raised.

\FloatBarrier
