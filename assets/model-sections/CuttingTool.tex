% Generated 2021-01-06 15:42:30 +0530
\subsection{CuttingTool} \label{sec:CuttingTool}


The \block{CuttingTool} \gls{Information Model} illustrated in \ref{cuttingtool-schema} has the identical structure as the \block{CuttingToolArchetype} \gls{Information Model} except for the XML element \block{CuttingToolDefinition} that has been \textbf{DEPRECATED} in the \block{CuttingTool} schema.


\subsubsection{CuttingTool}
\label{sec:CuttingTool}



A CuttingTool physically removes the material from the workpiece by shear deformation.


\paragraph{Attributes of CuttingTool}\mbox{}
\label{sec:Attributes of CuttingTool}

\tbl{Attributes of CuttingTool} lists the attributes of \texttt{CuttingTool}.

\begin{table}[ht]
\centering 
  \caption{Attributes of CuttingTool}
  \label{table:Attributes of CuttingTool}
\tabulinesep=3pt
\begin{tabu} to 6in {|l|l|l|} \everyrow{\hline}
\hline
\rowfont\bfseries {Attribute} & {Type} & {Multiplicity} \\
\tabucline[1.5pt]{}

\property{manufacturers}[CuttingTool] & \texttt{string} & 0..1 \\
\property{serialNumber}[CuttingTool] & \texttt{string} & 1 \\
\property{toolId}[CuttingTool] & \texttt{string} & 1 \\
\end{tabu}
\end{table}
\FloatBarrier

Descriptions for attributes of \block{CuttingTool}:

\begin{itemize}

\item \property{manufacturers}[CuttingTool] \newline The manufacturers of the Cutting Item or Tool.

\item \property{serialNumber}[CuttingTool] \newline The unique identifier for this assembly.

\item \property{toolId}[CuttingTool] \newline The identifier for a class of cutting tools.
\end{itemize}

\paragraph{Elements of CuttingTool}\mbox{}
\label{sec:Elements of CuttingTool}

\tbl{Elements of CuttingTool} lists the elements of \texttt{CuttingTool}.

\begin{table}[ht]
\centering 
  \caption{Elements of CuttingTool}
  \label{table:Elements of CuttingTool}
\tabulinesep=3pt
\begin{tabu} to 6in {|l|l|} \everyrow{\hline}
\hline
\rowfont\bfseries {Element} & {Multiplicity} \\
\tabucline[1.5pt]{}
\texttt{CuttingToolLifeCycle} & 0..1 \\
\texttt{CuttingToolArchetypeReference} & 0..1 \\
\texttt{CuttingToolDefinition} & 0..1 \\
\end{tabu}
\end{table}
\FloatBarrier


Descriptions for elements of \block{CuttingTool}:

\begin{itemize}

\item \block{CuttingToolLifeCycle} \newline Data regarding the use of the cutting tool.

\item \block{CuttingToolArchetypeReference} \newline CuttingToolArchetypeReference has reference information about the assetId and/or the URL of the data source of CuttingToolArchetype.

\item \block{CuttingToolDefinition} \newline Reference to an ISO 13399.
\end{itemize}

\subsubsection{CuttingToolArchetype}
\label{sec:CuttingToolArchetype}



The CuttingToolArchetype represents the static cutting tool geometries and nominal values as one would expect from a tool catalog.


\paragraph{Attributes of CuttingToolArchetype}\mbox{}
\label{sec:Attributes of CuttingToolArchetype}

\tbl{Attributes of CuttingToolArchetype} lists the attributes of \texttt{CuttingToolArchetype}.

\begin{table}[ht]
\centering 
  \caption{Attributes of CuttingToolArchetype}
  \label{table:Attributes of CuttingToolArchetype}
\tabulinesep=3pt
\begin{tabu} to 6in {|l|l|l|} \everyrow{\hline}
\hline
\rowfont\bfseries {Attribute} & {Type} & {Multiplicity} \\
\tabucline[1.5pt]{}

\property{manufacturers}[CuttingToolArchetype] & \texttt{string} & 0..1 \\
\property{serialNumber}[CuttingToolArchetype] & \texttt{string} & 1 \\
\property{toolId}[CuttingToolArchetype] & \texttt{string} & 1 \\
\end{tabu}
\end{table}
\FloatBarrier

Descriptions for attributes of \block{CuttingToolArchetype}:

\begin{itemize}

\item \property{manufacturers}[CuttingToolArchetype] \newline 

\item \property{serialNumber}[CuttingToolArchetype] \newline The unique identifier for this assembly.

\item \property{toolId}[CuttingToolArchetype] \newline The identifier for a class of cutting tools.
\end{itemize}

\paragraph{Elements of CuttingToolArchetype}\mbox{}
\label{sec:Elements of CuttingToolArchetype}

\tbl{Elements of CuttingToolArchetype} lists the elements of \texttt{CuttingToolArchetype}.

\begin{table}[ht]
\centering 
  \caption{Elements of CuttingToolArchetype}
  \label{table:Elements of CuttingToolArchetype}
\tabulinesep=3pt
\begin{tabu} to 6in {|l|l|} \everyrow{\hline}
\hline
\rowfont\bfseries {Element} & {Multiplicity} \\
\tabucline[1.5pt]{}
\texttt{CuttingToolDefinition} & 0..1 \\
\texttt{CuttingToolLifeCycle} & 0..1 \\
\end{tabu}
\end{table}
\FloatBarrier


Descriptions for elements of \block{CuttingToolArchetype}:

\begin{itemize}

\item \block{CuttingToolDefinition} \newline Reference to an ISO 13399.

\item \block{CuttingToolLifeCycle} \newline Data regarding the use of the cutting tool.
\end{itemize}

\subsubsection{CuttingToolArchetypeReference}
\label{sec:CuttingToolArchetypeReference}



CuttingToolArchetypeReference has reference information about the assetId and/or the URL of the data source of CuttingToolArchetype.


The value of \texttt{CuttingToolArchetypeReference} \MUST be \texttt{CuttingToolArchetype}.


\paragraph{Attributes of CuttingToolArchetypeReference}\mbox{}
\label{sec:Attributes of CuttingToolArchetypeReference}

\tbl{Attributes of CuttingToolArchetypeReference} lists the attributes of \texttt{CuttingToolArchetypeReference}.

\begin{table}[ht]
\centering 
  \caption{Attributes of CuttingToolArchetypeReference}
  \label{table:Attributes of CuttingToolArchetypeReference}
\tabulinesep=3pt
\begin{tabu} to 6in {|l|l|l|} \everyrow{\hline}
\hline
\rowfont\bfseries {Attribute} & {Type} & {Multiplicity} \\
\tabucline[1.5pt]{}

\property{source}[CuttingToolArchetypeReference] & \texttt{string} & 0..1 \\
\end{tabu}
\end{table}
\FloatBarrier

Descriptions for attributes of \block{CuttingToolArchetypeReference}:

\begin{itemize}

\item \property{source}[CuttingToolArchetypeReference] \newline The URL of the CuttingToolArchetype Information Model.


\item \property{value}[CuttingToolArchetypeReference] \newline 
\end{itemize}

\subsubsection{CuttingToolDefinition}
\label{sec:CuttingToolDefinition}



Reference to an ISO 13399.


The value of \texttt{CuttingToolDefinition} \MUST be \texttt{string}.


\paragraph{Attributes of CuttingToolDefinition}\mbox{}
\label{sec:Attributes of CuttingToolDefinition}

\tbl{Attributes of CuttingToolDefinition} lists the attributes of \texttt{CuttingToolDefinition}.

\begin{table}[ht]
\centering 
  \caption{Attributes of CuttingToolDefinition}
  \label{table:Attributes of CuttingToolDefinition}
\tabulinesep=3pt
\begin{tabu} to 6in {|l|l|l|} \everyrow{\hline}
\hline
\rowfont\bfseries {Attribute} & {Type} & {Multiplicity} \\
\tabucline[1.5pt]{}

\property{format}[CuttingToolDefinition] & \texttt{FormatType} & 0..1 \\
\end{tabu}
\end{table}
\FloatBarrier

Descriptions for attributes of \block{CuttingToolDefinition}:

\begin{itemize}

\item \property{format}[CuttingToolDefinition] \newline Identifies the expected representation of the enclosed data.

\texttt{FormatType} Enumeration:

\begin{itemize}
\item \texttt{EXPRESS} \newline The document will confirm to the ISO 10303 Part 21 standard.
 
\item \texttt{TEXT} \newline The document will be a text representation of the tool data.
 
\item \texttt{UNDEFINED} \newline The document will be provided in an undefined format. 
\item \texttt{XML} \newline The default value for the definition. The content will be an XML document. 
\end{itemize}


\item \property{value}[CuttingToolDefinition] \newline 
\end{itemize}

\subsubsection{CuttingToolLifeCycle}
\label{sec:CuttingToolLifeCycle}



Data regarding the use of the cutting tool.


\paragraph{Elements of CuttingToolLifeCycle}\mbox{}
\label{sec:Elements of CuttingToolLifeCycle}

\tbl{Elements of CuttingToolLifeCycle} lists the elements of \texttt{CuttingToolLifeCycle}.

\begin{table}[ht]
\centering 
  \caption{Elements of CuttingToolLifeCycle}
  \label{table:Elements of CuttingToolLifeCycle}
\tabulinesep=3pt
\begin{tabu} to 6in {|l|l|} \everyrow{\hline}
\hline
\rowfont\bfseries {Element} & {Multiplicity} \\
\tabucline[1.5pt]{}
\texttt{ConnectionCodeMachineSide} & 0..1 \\
\texttt{ProgramToolGroup} & 0..1 \\
\texttt{ProgramToolNumber} & 0..1 \\
\texttt{ProcessFeedRate} & 0..1 \\
\texttt{ToolLife} & 0..1 \\
\texttt{ToolLife} & 0..1 \\
\texttt{ProcessSpindleSpeed} & 0..1 \\
\texttt{ToolLife} & 0..1 \\
\texttt{Status} (organized by \block{CutterStatus}) & 1..* \\
\texttt{CuttingItem} (organized by \block{CuttingItems}) & 0..* \\
\texttt{Measurement} (organized by \block{Measurements}) & 0..* \\
\texttt{ReconditionCount} & 0..1 \\
\texttt{Location} & 0..1 \\
\end{tabu}
\end{table}
\FloatBarrier


Descriptions for elements of \block{CuttingToolLifeCycle}:

\begin{itemize}

\item \block{ConnectionCodeMachineSide} \newline Identifier for the capability to connect any Component of the cutting tool together, except Assembly Items, on the machine side. Code: CCMS

The value of \block{ConnectionCodeMachineSide} \MUST be \texttt{string}.

\item \block{ProgramToolGroup} \newline The tool group this tool is assigned in the part program.

The value of \block{ProgramToolGroup} \MUST be \texttt{string}.

\item \block{ProgramToolNumber} \newline The number of the tool as referenced in the part program.

The value of \block{ProgramToolNumber} \MUST be \texttt{integer}.

\item \block{ProcessFeedRate} \newline The constrained process feed rate for this tool in mm/s.

\item \block{ToolLife} \newline The cutting tool life as related to this assembly.

\item \block{ToolLife} \newline The cutting tool life as related to this assembly.

\item \block{ProcessSpindleSpeed} \newline The constrained process spindle speed for this tool.


\item \block{ToolLife} \newline The cutting tool life as related to this assembly.

\item \block{CutterStatus} \newline 

\item \block{CuttingItems} \newline 

\item \block{Measurements} \newline 

\item \block{ReconditionCount} \newline The number of times this cutter has been reconditioned.


\item \block{Location} \newline The Pot or Spindle the cutting tool currently resides in.
\end{itemize}

\subsubsection{Location}
\label{sec:Location}



The Pot or Spindle the cutting tool currently resides in.


\paragraph{Attributes of Location}\mbox{}
\label{sec:Attributes of Location}

\tbl{Attributes of Location} lists the attributes of \texttt{Location}.

\begin{table}[ht]
\centering 
  \caption{Attributes of Location}
  \label{table:Attributes of Location}
\tabulinesep=3pt
\begin{tabu} to 6in {|l|l|l|} \everyrow{\hline}
\hline
\rowfont\bfseries {Attribute} & {Type} & {Multiplicity} \\
\tabucline[1.5pt]{}

\property{negativeOverlap}[Location] & \texttt{integer} & 0..1 \\
\property{positiveOverlap}[Location] & \texttt{integer} & 0..1 \\
\property{type}[Location] & \texttt{LocationType} & 1 \\
\property{turret}[Location] & \texttt{NMTOKEN} & 0..1 \\
\property{toolMagazine}[Location] & \texttt{NMTOKEN} & 0..1 \\
\property{toolBar}[Location] & \texttt{NMTOKEN} & 0..1 \\
\property{toolRack}[Location] & \texttt{NMTOKEN} & 0..1 \\
\property{automaticToolChanger}[Location] & \texttt{NMTOKEN} & 0..1 \\
\end{tabu}
\end{table}
\FloatBarrier

Descriptions for attributes of \block{Location}:

\begin{itemize}

\item \property{negativeOverlap}[Location] \newline The number of location at lower index values from this location.

\item \property{positiveOverlap}[Location] \newline The number of locations at higher index value from this location.


\item \property{type}[Location] \newline The type of location being identified. 

\texttt{LocationType} Enumeration:

\begin{itemize}
\item \texttt{POT} \newline The number of the pot in the tool handling system. 
\item \texttt{STATION} \newline The tool location in a horizontal turning machine. 
\item \texttt{CRIB} \newline The location with regard to a tool crib. 
\item \texttt{SPINDLE} \newline A location associated with a \gls{Spindle}. 
\item \texttt{TRANSFER\textunderscore POT} \newline A location for a tool awaiting transfer from a tool magazine to spindle or a turret. 
\item \texttt{RETURN\textunderscore POT} \newline A location for a tool removed from a \gls{Spindle} or turret and awaiting return to a tool magazine.
 
\item \texttt{STAGING\textunderscore POT} \newline A location for a tool awaiting transfer to a tool magazine or turret from outside of the piece of equipment. 
\item \texttt{REMOVAL\textunderscore POT} \newline A location for a tool removed from a tool magazine or turret awaiting transfer to a location outside of the piece of equipment.
 
\item \texttt{EXPIRED\textunderscore POT} \newline A location for a tool that is no longer usable and is awaiting removal from a tool magazine or turret. 
\item \texttt{END\textunderscore EFFECTOR} \newline A location associated with an end effector. 
\end{itemize}


\item \property{turret}[Location] \newline The turret associated with a tool.

\item \property{toolMagazine}[Location] \newline The tool magazine associated with a tool.


\item \property{toolBar}[Location] \newline The tool bar associated with a tool.

\item \property{toolRack}[Location] \newline The tool rack associated with a tool.

\item \property{automaticToolChanger}[Location] \newline The automatic tool changer associated with a tool.
\end{itemize}

\subsubsection{Measurement}
\label{sec:Measurement}



A constrained scalar value associated with this cutting tool.


\paragraph{Attributes of Measurement}\mbox{}
\label{sec:Attributes of Measurement}

\tbl{Attributes of Measurement} lists the attributes of \texttt{Measurement}.

\begin{table}[ht]
\centering 
  \caption{Attributes of Measurement}
  \label{table:Attributes of Measurement}
\tabulinesep=3pt
\begin{tabu} to 6in {|l|l|l|} \everyrow{\hline}
\hline
\rowfont\bfseries {Attribute} & {Type} & {Multiplicity} \\
\tabucline[1.5pt]{}

\property{code}[Measurement] & \texttt{CodeEnum} & 1 \\
\property{maximum}[Measurement] & \texttt{float} & 0..1 \\
\property{minimum}[Measurement] & \texttt{float} & 0..1 \\
\property{nativeUnits}[Measurement] & \texttt{NativeUnitEnum} & 0..1 \\
\property{nominal}[Measurement] & \texttt{float} & 0..1 \\
\property{significantDigits}[Measurement] & \texttt{integer} & 0..1 \\
\property{units}[Measurement] & \texttt{UnitEnum} & 0..1 \\
\end{tabu}
\end{table}
\FloatBarrier

Descriptions for attributes of \block{Measurement}:

\begin{itemize}

\item \property{code}[Measurement] \newline A shop specific code for this measurement. ISO 13399 codes MAY be used for these codes as well.

\texttt{CodeEnum} Enumeration:

\begin{itemize}
\item \texttt{BDX} \newline The largest diameter of the body of a Tool Item. 
\item \texttt{LBX} \newline The distance measured along the X axis from that point of the item closest to the workpiece, including the Cutting Item for a Tool Item but excluding a protruding locking mechanism for an Adaptive Item, to either the front of the flange on a flanged body or the beginning of the connection interface feature on the machine side for cylindrical or prismatic shanks. 
\item \texttt{APMX} \newline The maximum engagement of the cutting edge or edges with the workpiece measured perpendicular to the feed motion. 
\item \texttt{DC} \newline The maximum diameter of a circle on which the defined point Pk of each of the master inserts is located on a Tool Item. The normal of the machined peripheral surface points towards the axis of the Cutting Tool. 
\item \texttt{DF} \newline The dimension between two parallel tangents on the outside edge of a flange. 
\item \texttt{OAL} \newline The largest length dimension of the Cutting Tool including the master insert where applicable. 
\item \texttt{DMM} \newline The dimension of the diameter of a cylindrical portion of a Tool Item or an Adaptive Item that can participate in a connection. 
\item \texttt{H} \newline The dimension of the height of the shank. 
\item \texttt{LS} \newline The dimension of the length of the shank. 
\item \texttt{LUX} \newline Maximum length of a Cutting Tool that can be used in a particular cutting operation including the non-cutting portions of the tool. 
\item \texttt{LPR} \newline The dimension from the yz-plane to the furthest point of the Tool Item or Adaptive Item measured in the -X direction. 
\item \texttt{WT} \newline The total weight of the Cutting Tool in grams. The force exerted by the mass of the Cutting Tool. 
\item \texttt{LF} \newline The distance from the gauge plane or from the end of the shank to the furthest point on the tool, if a gauge plane does not exist, to the cutting reference point determined by the main function of the tool. The \block{CuttingTool} functional length will be the length of the entire tool, not a single Cutting Item. Each \block{CuttingItem} can have an independent \block{FunctionalLength} represented in its measurements.  
\item \texttt{CRP} \newline The theoretical sharp point of the Cutting Tool from which the major functional dimensions are taken. 
\item \texttt{L} \newline The theoretical length of the cutting edge of a Cutting Item over sharp corners. 
\item \texttt{DRVA} \newline Angle between the driving mechanism locator on a Tool Item and the main cutting edge. 
\item \texttt{WF} \newline The distance between the cutting reference point and the rear backing surface of a turning tool or the axis of a boring bar. 
\item \texttt{IC} \newline The diameter of a circle to which all edges of a equilateral and round regular insert are tangential. 
\item \texttt{SIG} \newline The angle between the major cutting edge and the same cutting edge rotated by 180 degrees about the tool axis. 
\item \texttt{KAPR} \newline The angle between the tool cutting edge plane and the tool feed plane measured in a plane parallel the xy-plane. 
\item \texttt{PSIR} \newline The angle between the tool cutting edge plane and a plane perpendicular to the tool feed plane measured in a plane parallel the xy-plane. 
\item \texttt{N/A} \newline The angle of the tool with respect to the workpiece for a given process. The value is application specific. 
\item \texttt{BS} \newline The measure of the length of a wiper edge of a Cutting Item. 
\item \texttt{SDLx} \newline The length of a portion of a stepped tool that is related to a corresponding cutting diameter measured from the cutting reference point of that cutting diameter to the point on the next cutting edge at which the diameter starts to change. 
\item \texttt{STAx} \newline The angle between a major edge on a step of a stepped tool and the same cutting edge rotated 180 degrees about its tool axis. 
\item \texttt{DCx} \newline The diameter of a circle on which the defined point Pk located on this Cutting Tool. The normal of the machined peripheral surface points towards the axis of the Cutting Tool. 
\item \texttt{HF} \newline The distance from the basal plane of the Tool Item to the cutting point. 
\item \texttt{RE} \newline The nominal radius of a rounded corner measured in the X Y-plane. 
\item \texttt{LFx} \newline The distance from the gauge plane or from the end of the shank of the Cutting Tool, if a gauge plane does not exist, to the cutting reference point determined by the main function of the tool. This measurement will be with reference to the Cutting Tool and \textbf{MUSTNOT} exist without a Cutting Tool. 
\item \texttt{BCH} \newline The flat length of a chamfer. 
\item \texttt{CHW} \newline The width of the chamfer. 
\item \texttt{W1} \newline W1 is used for the insert width when an inscribed circle diameter is not practical. 
\end{itemize}


\item \property{maximum}[Measurement] \newline The maximum value for this measurement. 

\item \property{minimum}[Measurement] \newline The minimum value for this measurement. 

\item \property{nativeUnits}[Measurement] \newline The units the measurement was originally recorded in.

\item \property{nominal}[Measurement] \newline The as advertised value for this measurement.


\item \property{significantDigits}[Measurement] \newline The number of significant digits in the reported value. 

\item \property{units}[Measurement] \newline The units for the measurements. 
\end{itemize}

\subsubsection{ProcessFeedRate}
\label{sec:ProcessFeedRate}



The constrained process feed rate for this tool in mm/s.


\paragraph{Attributes of ProcessFeedRate}\mbox{}
\label{sec:Attributes of ProcessFeedRate}

\tbl{Attributes of ProcessFeedRate} lists the attributes of \texttt{ProcessFeedRate}.

\begin{table}[ht]
\centering 
  \caption{Attributes of ProcessFeedRate}
  \label{table:Attributes of ProcessFeedRate}
\tabulinesep=3pt
\begin{tabu} to 6in {|l|l|l|} \everyrow{\hline}
\hline
\rowfont\bfseries {Attribute} & {Type} & {Multiplicity} \\
\tabucline[1.5pt]{}

\property{maximum}[ProcessFeedRate] & \texttt{float} & 0..1 \\
\property{minimum}[ProcessFeedRate] & \texttt{float} & 0..1 \\
\property{nominal}[ProcessFeedRate] & \texttt{float} & 0..1 \\
\end{tabu}
\end{table}
\FloatBarrier

Descriptions for attributes of \block{ProcessFeedRate}:

\begin{itemize}

\item \property{maximum}[ProcessFeedRate] \newline The upper bound for the tool’s process target feedrate.

\item \property{minimum}[ProcessFeedRate] \newline The lower bound for the tools feedrate.

\item \property{nominal}[ProcessFeedRate] \newline The nominal feedrate the tool is designed to operate at.

\end{itemize}

\subsubsection{ProcessSpindleSpeed}
\label{sec:ProcessSpindleSpeed}



The constrained process spindle speed for this tool.



\paragraph{Attributes of ProcessSpindleSpeed}\mbox{}
\label{sec:Attributes of ProcessSpindleSpeed}

\tbl{Attributes of ProcessSpindleSpeed} lists the attributes of \texttt{ProcessSpindleSpeed}.

\begin{table}[ht]
\centering 
  \caption{Attributes of ProcessSpindleSpeed}
  \label{table:Attributes of ProcessSpindleSpeed}
\tabulinesep=3pt
\begin{tabu} to 6in {|l|l|l|} \everyrow{\hline}
\hline
\rowfont\bfseries {Attribute} & {Type} & {Multiplicity} \\
\tabucline[1.5pt]{}

\property{maximum}[ProcessSpindleSpeed] & \texttt{float} & 0..1 \\
\property{minimum}[ProcessSpindleSpeed] & \texttt{float} & 0..1 \\
\property{nominal}[ProcessSpindleSpeed] & \texttt{float} & 0..1 \\
\end{tabu}
\end{table}
\FloatBarrier

Descriptions for attributes of \block{ProcessSpindleSpeed}:

\begin{itemize}

\item \property{maximum}[ProcessSpindleSpeed] \newline The upper bound for the tool’s target spindle speed.

\item \property{minimum}[ProcessSpindleSpeed] \newline The lower bound for the tools spindle speed.


\item \property{nominal}[ProcessSpindleSpeed] \newline The nominal speed the tool is designed to operate at.
\end{itemize}

\subsubsection{ReconditionCount}
\label{sec:ReconditionCount}



The number of times this cutter has been reconditioned.



\paragraph{Attributes of ReconditionCount}\mbox{}
\label{sec:Attributes of ReconditionCount}

\tbl{Attributes of ReconditionCount} lists the attributes of \texttt{ReconditionCount}.

\begin{table}[ht]
\centering 
  \caption{Attributes of ReconditionCount}
  \label{table:Attributes of ReconditionCount}
\tabulinesep=3pt
\begin{tabu} to 6in {|l|l|l|} \everyrow{\hline}
\hline
\rowfont\bfseries {Attribute} & {Type} & {Multiplicity} \\
\tabucline[1.5pt]{}

\property{maximumCount}[ReconditionCount] & \texttt{integer} & 0..1 \\
\end{tabu}
\end{table}
\FloatBarrier

Descriptions for attributes of \block{ReconditionCount}:

\begin{itemize}

\item \property{maximumCount}[ReconditionCount] \newline The maximum number of times this tool may be reconditioned.

\end{itemize}

\subsubsection{Status}
\label{sec:Status}



The status of the cutting tool.


The value of \texttt{Status} \MUST be \texttt{CutterStatusType}.


\subsubsection{ToolLife}
\label{sec:ToolLife}



The cutting tool life as related to this assembly.


\paragraph{Attributes of ToolLife}\mbox{}
\label{sec:Attributes of ToolLife}

\tbl{Attributes of ToolLife} lists the attributes of \texttt{ToolLife}.

\begin{table}[ht]
\centering 
  \caption{Attributes of ToolLife}
  \label{table:Attributes of ToolLife}
\tabulinesep=3pt
\begin{tabu} to 6in {|l|l|l|} \everyrow{\hline}
\hline
\rowfont\bfseries {Attribute} & {Type} & {Multiplicity} \\
\tabucline[1.5pt]{}

\property{countDirection}[ToolLife] & \texttt{CountDirectionType} & 1 \\
\property{initial}[ToolLife] & \texttt{float} & 0..1 \\
\property{limit}[ToolLife] & \texttt{float} & 0..1 \\
\property{warning}[ToolLife] & \texttt{float} & 0..1 \\
\end{tabu}
\end{table}
\FloatBarrier

Descriptions for attributes of \block{ToolLife}:

\begin{itemize}

\item \property{countDirection}[ToolLife] \newline Indicates if the tool life counts from zero to maximum or maximum to zero.

\texttt{CountDirectionType} Enumeration:

\begin{itemize}
\item \texttt{UP} \newline The tool life counts up from zero to the maximum.
 
\item \texttt{DOWN} \newline The tool life counts down from the maximum to zero. 
\end{itemize}


\item \property{initial}[ToolLife] \newline The initial life of the tool when it is new.

\item \property{limit}[ToolLife] \newline The end of life limit for this tool.

\item \property{warning}[ToolLife] \newline The point at which a tool life warning will be raised.
\end{itemize}
