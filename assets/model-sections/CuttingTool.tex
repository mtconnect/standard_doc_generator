% Generated 2020-05-20 17:19:31 -0400
\subsection{CuttingTool} \label{sec:CuttingTool}

The \gls{cuttingtool} \gls{information model} illustrated in \fig{cuttingtool-schema} has the identical structure as the \gls{cuttingtoolarchetype} \gls{information model} except for the XML element \gls{cuttingtooldefinition} that has been \DEPRECATED in the \gls{cuttingtool} schema.


\subsubsection{CuttingTool}
  \label{sec:CuttingTool}


A CuttingTool physically removes the material from the workpiece by shear deformation.


\paragraph{Attributes of CuttingTool}\mbox{}
\label{sec:Attributes of CuttingTool}

\tbl{attributes of CuttingTool} lists the attributes of \texttt{CuttingTool}.

\begin{table}[ht]
\centering 
  \caption{Attributes of CuttingTool}
  \label{table:attributes of CuttingTool}
\tabulinesep=3pt
\begin{tabu} to 6in {|l|l|l|} \everyrow{\hline}
\hline
\rowfont\bfseries {Attribute} & {Type} & {Multiplicity} \\
\tabucline[1.5pt]{}
\texttt{manufacturers} & \texttt{string} & 0..1 \\
\texttt{serialNumber} & \texttt{string} & 1 \\
\texttt{toolId} & \texttt{string} & 1 \\
\end{tabu}
\end{table}
\FloatBarrier


Descriptions for attributes of \texttt{CuttingTool}:

\begin{itemize}
\item \texttt{manufacturers} : The manufacturers of the Cutting Item or Tool.
\item \texttt{serialNumber} : The unique identifier for this assembly.
\item \texttt{toolId} : The identifier for a class of cutting tools.
\end{itemize}

\paragraph{Elements of CuttingTool}\mbox{}
\label{sec:Elements of CuttingTool}

\tbl{elements of CuttingTool} lists the elements of \texttt{CuttingTool}.

\begin{table}[ht]
\centering 
  \caption{Elements of CuttingTool}
  \label{table:elements of CuttingTool}
\tabulinesep=3pt
\begin{tabu} to 6in {|l|l|l|} \everyrow{\hline}
\hline
\rowfont\bfseries {Association Name} & {Element} & {Multiplicity} \\
\tabucline[1.5pt]{}
\texttt{CuttingToolLifeCycle} & \texttt{CuttingToolLifeCycle} & 0..1 \\
\texttt{CuttingToolArchetypeReference} & \texttt{CuttingToolArchetypeReference} & 0..1 \\
\texttt{CuttingToolDefinition} & \texttt{CuttingToolDefinition} & 0..1 \\
\end{tabu}
\end{table}
\FloatBarrier


Descriptions for elements of \texttt{CuttingTool}:

\begin{itemize}
\item \texttt{CuttingToolLifeCycle} : Data regarding the use of the cutting tool.
\item \texttt{CuttingToolArchetypeReference} : CuttingToolArchetypeReference has reference information about the assetId and/or the URL of the data source of CuttingToolArchetype.
\item \texttt{CuttingToolDefinition} : Reference to an ISO 13399.
\end{itemize}
\FloatBarrier

\subsubsection{CuttingToolArchetype}
  \label{sec:CuttingToolArchetype}


The CuttingToolArchetype represents the static cutting tool geometries and nominal values as one would expect from a tool catalog.


\paragraph{Attributes of CuttingToolArchetype}\mbox{}
\label{sec:Attributes of CuttingToolArchetype}

\tbl{attributes of CuttingToolArchetype} lists the attributes of \texttt{CuttingToolArchetype}.

\begin{table}[ht]
\centering 
  \caption{Attributes of CuttingToolArchetype}
  \label{table:attributes of CuttingToolArchetype}
\tabulinesep=3pt
\begin{tabu} to 6in {|l|l|l|} \everyrow{\hline}
\hline
\rowfont\bfseries {Attribute} & {Type} & {Multiplicity} \\
\tabucline[1.5pt]{}
\texttt{manufacturers} & \texttt{string} & 0..1 \\
\texttt{serialNumber} & \texttt{string} & 1 \\
\texttt{toolId} & \texttt{string} & 1 \\
\end{tabu}
\end{table}
\FloatBarrier


Descriptions for attributes of \texttt{CuttingToolArchetype}:

\begin{itemize}
\item \texttt{manufacturers} : 
\item \texttt{serialNumber} : The unique identifier for this assembly.
\item \texttt{toolId} : The identifier for a class of cutting tools.
\end{itemize}

\paragraph{Elements of CuttingToolArchetype}\mbox{}
\label{sec:Elements of CuttingToolArchetype}

\tbl{elements of CuttingToolArchetype} lists the elements of \texttt{CuttingToolArchetype}.

\begin{table}[ht]
\centering 
  \caption{Elements of CuttingToolArchetype}
  \label{table:elements of CuttingToolArchetype}
\tabulinesep=3pt
\begin{tabu} to 6in {|l|l|l|} \everyrow{\hline}
\hline
\rowfont\bfseries {Association Name} & {Element} & {Multiplicity} \\
\tabucline[1.5pt]{}
\texttt{CuttingToolDefinition} & \texttt{CuttingToolDefinition} & 0..1 \\
\texttt{CuttingToolLifeCycle} & \texttt{CuttingToolLifeCycle} & 0..1 \\
\end{tabu}
\end{table}
\FloatBarrier


Descriptions for elements of \texttt{CuttingToolArchetype}:

\begin{itemize}
\item \texttt{CuttingToolDefinition} : Reference to an ISO 13399.
\item \texttt{CuttingToolLifeCycle} : Data regarding the use of the cutting tool.
\end{itemize}
\FloatBarrier

\subsubsection{CuttingToolArchetypeReference}
  \label{sec:CuttingToolArchetypeReference}


CuttingToolArchetypeReference has reference information about the assetId and/or the URL of the data source of CuttingToolArchetype.


\paragraph{Attributes of CuttingToolArchetypeReference}\mbox{}
\label{sec:Attributes of CuttingToolArchetypeReference}

\tbl{attributes of CuttingToolArchetypeReference} lists the attributes of \texttt{CuttingToolArchetypeReference}.

\begin{table}[ht]
\centering 
  \caption{Attributes of CuttingToolArchetypeReference}
  \label{table:attributes of CuttingToolArchetypeReference}
\tabulinesep=3pt
\begin{tabu} to 6in {|l|l|l|} \everyrow{\hline}
\hline
\rowfont\bfseries {Attribute} & {Type} & {Multiplicity} \\
\tabucline[1.5pt]{}
\texttt{source} & \texttt{string} & 0..1 \\
\texttt{value} & \texttt{CuttingToolArchetype} & 0..1 \\
\end{tabu}
\end{table}
\FloatBarrier


Descriptions for attributes of \texttt{CuttingToolArchetypeReference}:

\begin{itemize}
\item \texttt{source} : The URL of the CuttingToolArchetype Information Model.

\item \texttt{value} : 
\end{itemize}
\FloatBarrier

\subsubsection{CuttingToolDefinition}
  \label{sec:CuttingToolDefinition}


Reference to an ISO 13399.


\paragraph{Attributes of CuttingToolDefinition}\mbox{}
\label{sec:Attributes of CuttingToolDefinition}

\tbl{attributes of CuttingToolDefinition} lists the attributes of \texttt{CuttingToolDefinition}.

\begin{table}[ht]
\centering 
  \caption{Attributes of CuttingToolDefinition}
  \label{table:attributes of CuttingToolDefinition}
\tabulinesep=3pt
\begin{tabu} to 6in {|l|l|l|} \everyrow{\hline}
\hline
\rowfont\bfseries {Attribute} & {Type} & {Multiplicity} \\
\tabucline[1.5pt]{}
\texttt{format} & \texttt{FormatType} & 0..1 \\
\texttt{value} & \texttt{string} & 0..* \\
\end{tabu}
\end{table}
\FloatBarrier


Descriptions for attributes of \texttt{CuttingToolDefinition}:

\begin{itemize}
\item \texttt{format} : Identifies the expected representation of the enclosed data.
\tabulinesep = 5pt
\begin{longtabu} to \textwidth {
    |l|X|}
  \caption{FormatType Enumeration}
  \label{enum:FormatType} \\
\hline
Name & Description \\
\hline
\endfirsthead
\hline
\multicolumn{2}{|c|}{Continuation of Table \texttt{FormatType} Enumeration} \\
\hline
Name & Description \\
\hline
\endhead
\texttt{EXPRESS} & The document will confirm to the ISO 10303 Part 21 standard.
 \\ \hline
\texttt{TEXT} & The document will be a text representation of the tool data.
 \\ \hline
\texttt{UNDEFINED} & The document will be provided in an undefined format. \\ \hline
\texttt{XML} & The default value for the definition. The content will be an XML document. \\ \hline
\end{longtabu}
\FloatBarrier
\item \texttt{value} : 
\end{itemize}
\FloatBarrier

\subsubsection{CuttingToolLifeCycle}
  \label{sec:CuttingToolLifeCycle}


Data regarding the use of the cutting tool.


\paragraph{Elements of CuttingToolLifeCycle}\mbox{}
\label{sec:Elements of CuttingToolLifeCycle}

\tbl{elements of CuttingToolLifeCycle} lists the elements of \texttt{CuttingToolLifeCycle}.

\begin{table}[ht]
\centering 
  \caption{Elements of CuttingToolLifeCycle}
  \label{table:elements of CuttingToolLifeCycle}
\tabulinesep=3pt
\begin{tabu} to 6in {|l|l|l|} \everyrow{\hline}
\hline
\rowfont\bfseries {Association Name} & {Element} & {Multiplicity} \\
\tabucline[1.5pt]{}
\texttt{ConnectionCodeMachineSide} & \texttt{string} & 0..1 \\
\texttt{ProgramToolGroup} & \texttt{string} & 0..1 \\
\texttt{ProgramToolNumber} & \texttt{integer} & 0..1 \\
\texttt{ProcessFeedRate} & \texttt{ProcessFeedRate} & 0..1 \\
\texttt{ToolLife} & \texttt{ToolLife} & 0..1 \\
\texttt{ToolLife} & \texttt{ToolLife} & 0..1 \\
\texttt{ProcessSpindleSpeed} & \texttt{ProcessSpindleSpeed} & 0..1 \\
\texttt{ToolLife} & \texttt{ToolLife} & 0..1 \\
\texttt{CutterStatus} & \texttt{Status} & 1..* \\
\texttt{CuttingItems} & \texttt{CuttingItem} & 0..* \\
\texttt{Measurements} & \texttt{Measurement} & 0..* \\
\texttt{ReconditionCount} & \texttt{ReconditionCount} & 0..1 \\
\texttt{Location} & \texttt{Location} & 0..1 \\
\end{tabu}
\end{table}
\FloatBarrier


Descriptions for elements of \texttt{CuttingToolLifeCycle}:

\begin{itemize}
\item \texttt{ConnectionCodeMachineSide} : Identifier for the capability to connect any Component of the cutting tool together, except Assembly Items, on the machine side. Code: CCMS
\item \texttt{ProgramToolGroup} : The tool group this tool is assigned in the part program.
\item \texttt{ProgramToolNumber} : The number of the tool as referenced in the part program.
\item \texttt{ProcessFeedRate} : The constrained process feed rate for this tool in mm/s.
\item \texttt{ToolLife} : The cutting tool life as related to this assembly.
\item \texttt{ToolLife} : The cutting tool life as related to this assembly.
\item \texttt{ProcessSpindleSpeed} : The constrained process spindle speed for this tool.

\item \texttt{ToolLife} : The cutting tool life as related to this assembly.
\item \texttt{CutterStatus} : 
\item \texttt{CuttingItems} : 
\item \texttt{Measurements} : 
\item \texttt{ReconditionCount} : The number of times this cutter has been reconditioned.

\item \texttt{Location} : The Pot or Spindle the cutting tool currently resides in.
\end{itemize}
\FloatBarrier

\subsubsection{Location}
  \label{sec:Location}


The Pot or Spindle the cutting tool currently resides in.


\paragraph{Attributes of Location}\mbox{}
\label{sec:Attributes of Location}

\tbl{attributes of Location} lists the attributes of \texttt{Location}.

\begin{table}[ht]
\centering 
  \caption{Attributes of Location}
  \label{table:attributes of Location}
\tabulinesep=3pt
\begin{tabu} to 6in {|l|l|l|} \everyrow{\hline}
\hline
\rowfont\bfseries {Attribute} & {Type} & {Multiplicity} \\
\tabucline[1.5pt]{}
\texttt{negativeOverlap} & \texttt{integer} & 0..1 \\
\texttt{positiveOverlap} & \texttt{integer} & 0..1 \\
\texttt{type} & \texttt{LocationType} & 1 \\
\end{tabu}
\end{table}
\FloatBarrier


Descriptions for attributes of \texttt{Location}:

\begin{itemize}
\item \texttt{negativeOverlap} : The number of location at lower index values from this location.
\item \texttt{positiveOverlap} : The number of locations at higher index value from this location.

\item \texttt{type} : The type of location being identified. 
\end{itemize}
\FloatBarrier

\subsubsection{Measurement}
  \label{sec:Measurement}


A constrained scalar value associated with this cutting tool.


\paragraph{Attributes of Measurement}\mbox{}
\label{sec:Attributes of Measurement}

\tbl{attributes of Measurement} lists the attributes of \texttt{Measurement}.

\begin{table}[ht]
\centering 
  \caption{Attributes of Measurement}
  \label{table:attributes of Measurement}
\tabulinesep=3pt
\begin{tabu} to 6in {|l|l|l|} \everyrow{\hline}
\hline
\rowfont\bfseries {Attribute} & {Type} & {Multiplicity} \\
\tabucline[1.5pt]{}
\texttt{code} & \texttt{CodeEnum} & 1 \\
\texttt{maximum} & \texttt{float} & 0..1 \\
\texttt{minimum} & \texttt{float} & 0..1 \\
\texttt{nativeUnits} & \texttt{NativeUnitEnum} & 0..1 \\
\texttt{nominal} & \texttt{float} & 0..1 \\
\texttt{significantDigits} & \texttt{integer} & 0..1 \\
\texttt{units} & \texttt{UnitEnum} & 0..1 \\
\end{tabu}
\end{table}
\FloatBarrier


Descriptions for attributes of \texttt{Measurement}:

\begin{itemize}
\item \texttt{code} : A shop specific code for this measurement. ISO 13399 codes MAY be used for these codes as well.
\tabulinesep = 5pt
\begin{longtabu} to \textwidth {
    |l|X|}
  \caption{CodeEnum Enumeration}
  \label{enum:CodeEnum} \\
\hline
Name & Description \\
\hline
\endfirsthead
\hline
\multicolumn{2}{|c|}{Continuation of Table \texttt{CodeEnum} Enumeration} \\
\hline
Name & Description \\
\hline
\endhead
\texttt{BDX} & The largest diameter of the body of a Tool Item. \\ \hline
\texttt{LBX} & The distance measured along the X axis from that point of the item closest to the workpiece, including the Cutting Item for a Tool Item but excluding a protruding locking mechanism for an Adaptive Item, to either the front of the flange on a flanged body or the beginning of the connection interface feature on the machine side for cylindrical or prismatic shanks. \\ \hline
\texttt{APMX} & The maximum engagement of the cutting edge or edges with the workpiece measured perpendicular to the feed motion. \\ \hline
\texttt{DC} & The maximum diameter of a circle on which the defined point Pk of each of the master inserts is located on a Tool Item. The normal of the machined peripheral surface points towards the axis of the Cutting Tool. \\ \hline
\texttt{DF} & The dimension between two parallel tangents on the outside edge of a flange. \\ \hline
\texttt{OAL} & The largest length dimension of the Cutting Tool including the master insert where applicable. \\ \hline
\texttt{DMM} & The dimension of the diameter of a cylindrical portion of a Tool Item or an Adaptive Item that can participate in a connection. \\ \hline
\texttt{H} & The dimension of the height of the shank. \\ \hline
\texttt{LS} & The dimension of the length of the shank. \\ \hline
\texttt{LUX} & Maximum length of a Cutting Tool that can be used in a particular cutting operation including the non-cutting portions of the tool. \\ \hline
\texttt{LPR} & The dimension from the yz-plane to the furthest point of the Tool Item or Adaptive Item measured in the -X direction. \\ \hline
\texttt{WT} & The total weight of the Cutting Tool in grams. The force exerted by the mass of the Cutting Tool. \\ \hline
\texttt{LF} & The distance from the gauge plane or from the end of the shank to the furthest point on the tool, if a gauge plane does not exist, to the cutting reference point determined by the main function of the tool. The {model:CuttingTool} functional length will be the length of the entire tool, not a single Cutting Item. Each {model:CuttingItem} can have an independent {model:FunctionalLength} represented in its measurements.  \\ \hline
\texttt{CRP} & The theoretical sharp point of the Cutting Tool from which the major functional dimensions are taken. \\ \hline
\texttt{L} & The theoretical length of the cutting edge of a Cutting Item over sharp corners. \\ \hline
\texttt{DRVA} & Angle between the driving mechanism locator on a Tool Item and the main cutting edge. \\ \hline
\texttt{WF} & The distance between the cutting reference point and the rear backing surface of a turning tool or the axis of a boring bar. \\ \hline
\texttt{IC} & The diameter of a circle to which all edges of a equilateral and round regular insert are tangential. \\ \hline
\texttt{SIG} & The angle between the major cutting edge and the same cutting edge rotated by 180 degrees about the tool axis. \\ \hline
\texttt{KAPR} & The angle between the tool cutting edge plane and the tool feed plane measured in a plane parallel the xy-plane. \\ \hline
\texttt{PSIR} & The angle between the tool cutting edge plane and a plane perpendicular to the tool feed plane measured in a plane parallel the xy-plane. \\ \hline
\texttt{N/A} & The angle of the tool with respect to the workpiece for a given process. The value is application specific. \\ \hline
\texttt{BS} & The measure of the length of a wiper edge of a Cutting Item. \\ \hline
\texttt{SDLx} & The length of a portion of a stepped tool that is related to a corresponding cutting diameter measured from the cutting reference point of that cutting diameter to the point on the next cutting edge at which the diameter starts to change. \\ \hline
\texttt{STAx} & The angle between a major edge on a step of a stepped tool and the same cutting edge rotated 180 degrees about its tool axis. \\ \hline
\texttt{DCx} & The diameter of a circle on which the defined point Pk located on this Cutting Tool. The normal of the machined peripheral surface points towards the axis of the Cutting Tool. \\ \hline
\texttt{HF} & The distance from the basal plane of the Tool Item to the cutting point. \\ \hline
\texttt{RE} & The nominal radius of a rounded corner measured in the X Y-plane. \\ \hline
\texttt{LFx} & The distance from the gauge plane or from the end of the shank of the Cutting Tool, if a gauge plane does not exist, to the cutting reference point determined by the main function of the tool. This measurement will be with reference to the Cutting Tool and *MUSTNOT* exist without a Cutting Tool. \\ \hline
\texttt{BCH} & The flat length of a chamfer. \\ \hline
\texttt{CHW} & The width of the chamfer. \\ \hline
\texttt{W1} & W1 is used for the insert width when an inscribed circle diameter is not practical. \\ \hline
\end{longtabu}
\FloatBarrier
\item \texttt{maximum} : The maximum value for this measurement. 
\item \texttt{minimum} : The minimum value for this measurement. 
\item \texttt{nativeUnits} : The units the measurement was originally recorded in.
\item \texttt{nominal} : The as advertised value for this measurement.

\item \texttt{significantDigits} : The number of significant digits in the reported value. 
\item \texttt{units} : The units for the measurements. 
\end{itemize}
\FloatBarrier

\subsubsection{ProcessFeedRate}
  \label{sec:ProcessFeedRate}


The constrained process feed rate for this tool in mm/s.


\paragraph{Attributes of ProcessFeedRate}\mbox{}
\label{sec:Attributes of ProcessFeedRate}

\tbl{attributes of ProcessFeedRate} lists the attributes of \texttt{ProcessFeedRate}.

\begin{table}[ht]
\centering 
  \caption{Attributes of ProcessFeedRate}
  \label{table:attributes of ProcessFeedRate}
\tabulinesep=3pt
\begin{tabu} to 6in {|l|l|l|} \everyrow{\hline}
\hline
\rowfont\bfseries {Attribute} & {Type} & {Multiplicity} \\
\tabucline[1.5pt]{}
\texttt{maximum} & \texttt{float} & 0..1 \\
\texttt{minimum} & \texttt{float} & 0..1 \\
\texttt{nominal} & \texttt{float} & 0..1 \\
\end{tabu}
\end{table}
\FloatBarrier


Descriptions for attributes of \texttt{ProcessFeedRate}:

\begin{itemize}
\item \texttt{maximum} : The upper bound for the tool’s process target feedrate.
\item \texttt{minimum} : The lower bound for the tools feedrate.
\item \texttt{nominal} : The nominal feedrate the tool is designed to operate at.

\end{itemize}
\FloatBarrier

\subsubsection{ProcessSpindleSpeed}
  \label{sec:ProcessSpindleSpeed}


The constrained process spindle speed for this tool.



\paragraph{Attributes of ProcessSpindleSpeed}\mbox{}
\label{sec:Attributes of ProcessSpindleSpeed}

\tbl{attributes of ProcessSpindleSpeed} lists the attributes of \texttt{ProcessSpindleSpeed}.

\begin{table}[ht]
\centering 
  \caption{Attributes of ProcessSpindleSpeed}
  \label{table:attributes of ProcessSpindleSpeed}
\tabulinesep=3pt
\begin{tabu} to 6in {|l|l|l|} \everyrow{\hline}
\hline
\rowfont\bfseries {Attribute} & {Type} & {Multiplicity} \\
\tabucline[1.5pt]{}
\texttt{maximum} & \texttt{float} & 0..1 \\
\texttt{minimum} & \texttt{float} & 0..1 \\
\texttt{nominal} & \texttt{float} & 0..1 \\
\end{tabu}
\end{table}
\FloatBarrier


Descriptions for attributes of \texttt{ProcessSpindleSpeed}:

\begin{itemize}
\item \texttt{maximum} : The upper bound for the tool’s target spindle speed.
\item \texttt{minimum} : The lower bound for the tools spindle speed.

\item \texttt{nominal} : The nominal speed the tool is designed to operate at.
\end{itemize}
\FloatBarrier

\subsubsection{ReconditionCount}
  \label{sec:ReconditionCount}


The number of times this cutter has been reconditioned.



\paragraph{Attributes of ReconditionCount}\mbox{}
\label{sec:Attributes of ReconditionCount}

\tbl{attributes of ReconditionCount} lists the attributes of \texttt{ReconditionCount}.

\begin{table}[ht]
\centering 
  \caption{Attributes of ReconditionCount}
  \label{table:attributes of ReconditionCount}
\tabulinesep=3pt
\begin{tabu} to 6in {|l|l|l|} \everyrow{\hline}
\hline
\rowfont\bfseries {Attribute} & {Type} & {Multiplicity} \\
\tabucline[1.5pt]{}
\texttt{maximumCount} & \texttt{integer} & 0..1 \\
\end{tabu}
\end{table}
\FloatBarrier


Descriptions for attributes of \texttt{ReconditionCount}:

\begin{itemize}
\item \texttt{maximumCount} : The maximum number of times this tool may be reconditioned.

\end{itemize}
\FloatBarrier

\subsubsection{Status}
  \label{sec:Status}


The status of the cutting tool.


\paragraph{Attributes of Status}\mbox{}
\label{sec:Attributes of Status}

\tbl{attributes of Status} lists the attributes of \texttt{Status}.

\begin{table}[ht]
\centering 
  \caption{Attributes of Status}
  \label{table:attributes of Status}
\tabulinesep=3pt
\begin{tabu} to 6in {|l|l|l|} \everyrow{\hline}
\hline
\rowfont\bfseries {Attribute} & {Type} & {Multiplicity} \\
\tabucline[1.5pt]{}
\texttt{value} & \texttt{CutterStatusType} & 1 \\
\end{tabu}
\end{table}
\FloatBarrier


Descriptions for attributes of \texttt{Status}:

\begin{itemize}
\item \texttt{value} : The status value of the cutting tool.
\tabulinesep = 5pt
\begin{longtabu} to \textwidth {
    |l|X|}
  \caption{CutterStatusType Enumeration}
  \label{enum:CutterStatusType} \\
\hline
Name & Description \\
\hline
\endfirsthead
\hline
\multicolumn{2}{|c|}{Continuation of Table \texttt{CutterStatusType} Enumeration} \\
\hline
Name & Description \\
\hline
\endhead
\texttt{NEW} & A new tool that has not been used or first use. Marks the start of the tool history. \\ \hline
\texttt{AVAILABLE} & Indicates the tool is available for use. If this is not present, the tool is currently not ready to be used. \\ \hline
\texttt{UNAVAILABLE} & Indicates the tool is unavailable for use in metal removal. If this is not present, the tool is currently not ready to be used.
 \\ \hline
\texttt{ALLOCATED} & Indicates if this tool is has been committed to a piece of equipment for use and is not available for use in any other piece of equipment. \\ \hline
\texttt{UNALLOCATED} & Indicates this cutting tool has not been committed to a process and can be allocated. \\ \hline
\texttt{MEASURED} & The tool has been measured.
 \\ \hline
\texttt{RECONDITIONED} & The Cutting Tool has been reconditioned. \\ \hline
\texttt{USED} & The cutting tool is in process and has remaining tool life. \\ \hline
\texttt{EXPIRED} & The cutting tool has reached the end of its useful life. \\ \hline
\texttt{BROKEN} & Premature tool failure. \\ \hline
\texttt{NOT_REGISTERED} & This cutting tool cannot be used until it is entered into the system. \\ \hline
\texttt{UNKNOWN} & The cutting tool is an indeterminate state. This is the default value. \\ \hline
\end{longtabu}
\FloatBarrier
\end{itemize}
\FloatBarrier

\subsubsection{ToolLife}
  \label{sec:ToolLife}


The cutting tool life as related to this assembly.


\paragraph{Attributes of ToolLife}\mbox{}
\label{sec:Attributes of ToolLife}

\tbl{attributes of ToolLife} lists the attributes of \texttt{ToolLife}.

\begin{table}[ht]
\centering 
  \caption{Attributes of ToolLife}
  \label{table:attributes of ToolLife}
\tabulinesep=3pt
\begin{tabu} to 6in {|l|l|l|} \everyrow{\hline}
\hline
\rowfont\bfseries {Attribute} & {Type} & {Multiplicity} \\
\tabucline[1.5pt]{}
\texttt{countDirection} & \texttt{CountDirectionType} & 1 \\
\texttt{initial} & \texttt{float} & 0..1 \\
\texttt{limit} & \texttt{float} & 0..1 \\
\texttt{warning} & \texttt{float} & 0..1 \\
\end{tabu}
\end{table}
\FloatBarrier


Descriptions for attributes of \texttt{ToolLife}:

\begin{itemize}
\item \texttt{countDirection} : Indicates if the tool life counts from zero to maximum or maximum to zero.
\tabulinesep = 5pt
\begin{longtabu} to \textwidth {
    |l|X|}
  \caption{CountDirectionType Enumeration}
  \label{enum:CountDirectionType} \\
\hline
Name & Description \\
\hline
\endfirsthead
\hline
\multicolumn{2}{|c|}{Continuation of Table \texttt{CountDirectionType} Enumeration} \\
\hline
Name & Description \\
\hline
\endhead
\texttt{UP} & The tool life counts up from zero to the maximum.
 \\ \hline
\texttt{DOWN} & The tool life counts down from the maximum to zero. \\ \hline
\end{longtabu}
\FloatBarrier
\item \texttt{initial} : The initial life of the tool when it is new.
\item \texttt{limit} : The end of life limit for this tool.
\item \texttt{warning} : The point at which a tool life warning will be raised.
\end{itemize}
\FloatBarrier
