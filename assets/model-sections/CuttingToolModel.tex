% Generated 2020-02-28 19:23:52 -0500
\subsection{CuttingToolModel} \label{model:CuttingToolModel}
\subsubsection[CuttingTool]{CuttingTool \\ {\small Subtype of Asset}}
  \label{type:CuttingTool}

\FloatBarrier

A CuttingTool physically removes the material from the workpiece by shear deformation.

\begin{table}[ht]
\centering 
  \caption{\texttt{Properties of CuttingTool}}
  \label{properties:CuttingTool}
\tabulinesep=3pt
\begin{tabu} to 6in {|l|l|l|} \everyrow{\hline}
\hline
\rowfont\bfseries {Property} & {Type} & {Multiplicity} \\
\tabucline[1.5pt]{}
\texttt{manufacturers} & \texttt{string} & 0..1 \\
\texttt{serialNumber} & \texttt{string} & 1 \\
\texttt{toolId} & \texttt{string} & 1 \\
\texttt{CuttingToolLifeCycle} & \texttt{CuttingToolLifeCycle} & 0..1 \\
\texttt{CuttingToolArchetypeReference} & \texttt{CuttingToolArchetypeReference} & 0..1 \\
\texttt{<<deprecated>> CuttingToolDefinition} & \texttt{CuttingToolDefinition} & 0..1 \\
\end{tabu}
\end{table}
\FloatBarrier


\paragraph{\texttt{manufacturers}}\mbox{}
\newline\tab The manufacturers of the Cutting Item or Tool.

\paragraph{\texttt{serialNumber}}\mbox{}
\newline\tab The unique identifier for this assembly.

\paragraph{\texttt{toolId}}\mbox{}
\newline\tab The identifier for a class of cutting tools.

\paragraph{\texttt{CuttingToolLifeCycle}}\mbox{}
\newline\tab Placeholder for documentation!

\paragraph{\texttt{CuttingToolArchetypeReference}}\mbox{}
\newline\tab Placeholder for documentation!

\paragraph{\texttt{CuttingToolDefinition}}\mbox{}
\newline\tab Placeholder for documentation!
\FloatBarrier
\subsubsection[CuttingToolArchetype]{CuttingToolArchetype \\ {\small Subtype of Asset}}
  \label{type:CuttingToolArchetype}

\FloatBarrier

The CuttingToolArchetype represents the static cutting tool geometries and nominal values as one would expect from a tool catalog.

\begin{table}[ht]
\centering 
  \caption{\texttt{Properties of CuttingToolArchetype}}
  \label{properties:CuttingToolArchetype}
\tabulinesep=3pt
\begin{tabu} to 6in {|l|l|l|} \everyrow{\hline}
\hline
\rowfont\bfseries {Property} & {Type} & {Multiplicity} \\
\tabucline[1.5pt]{}
\texttt{manufacturers} & \texttt{string} & 0..1 \\
\texttt{serialNumber} & \texttt{string} & 1 \\
\texttt{toolId} & \texttt{string} & 1 \\
\texttt{CuttingToolDefinition} & \texttt{CuttingToolDefinition} & 0..1 \\
\texttt{CuttingToolLifeCycle} & \texttt{CuttingToolLifeCycle} & 0..1 \\
\end{tabu}
\end{table}
\FloatBarrier


\paragraph{\texttt{manufacturers}}\mbox{}
\newline\tab Placeholder for documentation!

\paragraph{\texttt{serialNumber}}\mbox{}
\newline\tab The unique identifier for this assembly.

\paragraph{\texttt{toolId}}\mbox{}
\newline\tab The identifier for a class of cutting tools.

\paragraph{\texttt{CuttingToolDefinition}}\mbox{}
\newline\tab Placeholder for documentation!

\paragraph{\texttt{CuttingToolLifeCycle}}\mbox{}
\newline\tab Placeholder for documentation!
\FloatBarrier
\subsubsection{CuttingToolArchetypeReference}
  \label{type:CuttingToolArchetypeReference}

\FloatBarrier

CuttingToolArchetypeReference has reference information about the assetId and/or the URL of the data source of CuttingToolArchetype.

\begin{table}[ht]
\centering 
  \caption{\texttt{Properties of CuttingToolArchetypeReference}}
  \label{properties:CuttingToolArchetypeReference}
\tabulinesep=3pt
\begin{tabu} to 6in {|l|l|l|} \everyrow{\hline}
\hline
\rowfont\bfseries {Property} & {Type} & {Multiplicity} \\
\tabucline[1.5pt]{}
\texttt{source} & \texttt{string} & 0..1 \\
\texttt{value} & \texttt{CuttingToolArchetype} & 0..1 \\
\end{tabu}
\end{table}
\FloatBarrier


\paragraph{\texttt{source}}\mbox{}
\newline\tab The URL of the CuttingToolArchetype Information Model.


\paragraph{\texttt{value}}\mbox{}
\newline\tab Placeholder for documentation!
\FloatBarrier
\subsubsection{CuttingToolDefinition}
  \label{type:CuttingToolDefinition}

\FloatBarrier

Reference to an ISO 13399.

\begin{table}[ht]
\centering 
  \caption{\texttt{Properties of CuttingToolDefinition}}
  \label{properties:CuttingToolDefinition}
\tabulinesep=3pt
\begin{tabu} to 6in {|l|l|l|} \everyrow{\hline}
\hline
\rowfont\bfseries {Property} & {Type} & {Multiplicity} \\
\tabucline[1.5pt]{}
\texttt{format} & \texttt{FormatType} & 0..1 \\
\texttt{value} & \texttt{string} & 0..* \\
\end{tabu}
\end{table}
\FloatBarrier


\paragraph{\texttt{format}}\mbox{}
\newline\tab Identifies the expected representation of the enclosed data.

Enumeration for CuttingToolDefinition format values.

\begin{table}[ht]
\centering 
  \caption{\texttt{FormatType} Enumeration}
  \label{enum:FormatType}
\tabulinesep=3pt
\begin{tabu} to 6in {|l|X|} \everyrow{\hline}
\hline
\rowfont\bfseries {Name} & {Description} \\
\tabucline[1.5pt]{}
\texttt{EXPRESS} & The document will confirm to the ISO 10303 Part 21 standard.
 \\
\texttt{TEXT} & The document will be a text representation of the tool data.
 \\
\texttt{UNDEFINED} & The document will be provided in an undefined format. \\
\texttt{XML} & The default value for the definition. The content will be an XML document. \\
\end{tabu}
\end{table} 
\FloatBarrier

\paragraph{\texttt{value}}\mbox{}
\newline\tab Placeholder for documentation!
\FloatBarrier
\subsubsection{CuttingToolLifeCycle}
  \label{type:CuttingToolLifeCycle}

\FloatBarrier

Data regarding the use of the cutting tool.

\begin{table}[ht]
\centering 
  \caption{\texttt{Properties of CuttingToolLifeCycle}}
  \label{properties:CuttingToolLifeCycle}
\tabulinesep=3pt
\begin{tabu} to 6in {|l|l|l|} \everyrow{\hline}
\hline
\rowfont\bfseries {Property} & {Type} & {Multiplicity} \\
\tabucline[1.5pt]{}
\texttt{ConnectionCodeMachineSide} & \texttt{string} & 0..1 \\
\texttt{ProgramToolGroup} & \texttt{string} & 0..1 \\
\texttt{ProgramToolNumber} & \texttt{integer} & 0..1 \\
\texttt{ProcessFeedRate} & \texttt{ProcessFeedRate} & 0..1 \\
\texttt{ToolLife} & \texttt{ToolLife} & 0..1 \\
\texttt{ToolLife} & \texttt{ToolLife} & 0..1 \\
\texttt{ProcessSpindleSpeed} & \texttt{ProcessSpindleSpeed} & 0..1 \\
\texttt{ToolLife} & \texttt{ToolLife} & 0..1 \\
\texttt{CutterStatus} & \texttt{Status} & 1..* \\
\texttt{CuttingItems} & \texttt{CuttingItem} & 0..* \\
\texttt{Measurements} & \texttt{Measurement} & 0..* \\
\texttt{ReconditionCount} & \texttt{ReconditionCount} & 0..1 \\
\texttt{Location} & \texttt{Location} & 0..1 \\
\end{tabu}
\end{table}
\FloatBarrier


\paragraph{\texttt{ConnectionCodeMachineSide}}\mbox{}
\newline\tab Identifier for the capability to connect any Component of the cutting tool together, except Assembly Items, on the machine side. Code: CCMS

\paragraph{\texttt{ProgramToolGroup}}\mbox{}
\newline\tab The tool group this tool is assigned in the part program.

\paragraph{\texttt{ProgramToolNumber}}\mbox{}
\newline\tab The number of the tool as referenced in the part program.

\paragraph{\texttt{ProcessFeedRate}}\mbox{}
\newline\tab Placeholder for documentation!

\paragraph{\texttt{ToolLife}}\mbox{}
\newline\tab Placeholder for documentation!

\paragraph{\texttt{ToolLife}}\mbox{}
\newline\tab Placeholder for documentation!

\paragraph{\texttt{ProcessSpindleSpeed}}\mbox{}
\newline\tab Placeholder for documentation!

\paragraph{\texttt{ToolLife}}\mbox{}
\newline\tab Placeholder for documentation!

\paragraph{\texttt{CutterStatus}}\mbox{}
\newline\tab Placeholder for documentation!

\paragraph{\texttt{CuttingItems}}\mbox{}
\newline\tab Placeholder for documentation!

\paragraph{\texttt{Measurements}}\mbox{}
\newline\tab Placeholder for documentation!

\paragraph{\texttt{ReconditionCount}}\mbox{}
\newline\tab Placeholder for documentation!

\paragraph{\texttt{Location}}\mbox{}
\newline\tab Placeholder for documentation!
\FloatBarrier
\subsubsection{Location}
  \label{type:Location}

\FloatBarrier

The Pot or Spindle the cutting tool currently resides in.

\begin{table}[ht]
\centering 
  \caption{\texttt{Properties of Location}}
  \label{properties:Location}
\tabulinesep=3pt
\begin{tabu} to 6in {|l|l|l|} \everyrow{\hline}
\hline
\rowfont\bfseries {Property} & {Type} & {Multiplicity} \\
\tabucline[1.5pt]{}
\texttt{negativeOverlap} & \texttt{integer} & 0..1 \\
\texttt{positiveOverlap} & \texttt{integer} & 0..1 \\
\texttt{type} & \texttt{LocationType} & 1 \\
\end{tabu}
\end{table}
\FloatBarrier


\paragraph{\texttt{negativeOverlap}}\mbox{}
\newline\tab The number of location at lower index values from this location.

\paragraph{\texttt{positiveOverlap}}\mbox{}
\newline\tab The number of locations at higher index value from this location.


\paragraph{\texttt{type}}\mbox{}
\newline\tab The type of location being identified. 

Enumeration for Location types

\begin{table}[ht]
\centering 
  \caption{\texttt{LocationType} Enumeration}
  \label{enum:LocationType}
\tabulinesep=3pt
\begin{tabu} to 6in {|l|X|} \everyrow{\hline}
\hline
\rowfont\bfseries {Name} & {Description} \\
\tabucline[1.5pt]{}
\texttt{POT} & The number of the pot in the tool handling system. \\
\texttt{STATION} & The tool location in a horizontal turning machine. \\
\texttt{CRIB} & The location with regard to a tool crib. \\
\end{tabu}
\end{table} 
\FloatBarrier
\FloatBarrier
\subsubsection{Measurement}
  \label{type:Measurement}

\FloatBarrier

A constrained scalar value associated with this cutting tool.

\begin{table}[ht]
\centering 
  \caption{\texttt{Properties of Measurement}}
  \label{properties:Measurement}
\tabulinesep=3pt
\begin{tabu} to 6in {|l|l|l|} \everyrow{\hline}
\hline
\rowfont\bfseries {Property} & {Type} & {Multiplicity} \\
\tabucline[1.5pt]{}
\texttt{code} & \texttt{CodeEnum} & 1 \\
\texttt{maximum} & \texttt{float} & 0..1 \\
\texttt{minimum} & \texttt{float} & 0..1 \\
\texttt{nativeUnits} & \texttt{NativeUnitEnum} & 0..1 \\
\texttt{nominal} & \texttt{float} & 0..1 \\
\texttt{significantDigits} & \texttt{integer} & 0..1 \\
\texttt{units} & \texttt{UnitEnum} & 0..1 \\
\end{tabu}
\end{table}
\FloatBarrier


\paragraph{\texttt{code}}\mbox{}
\newline\tab A shop specific code for this measurement. ISO 13399 codes MAY be used for these codes as well.

Placeholder for documentation!

\begin{table}[ht]
\centering 
  \caption{\texttt{CodeEnum} Enumeration}
  \label{enum:CodeEnum}
\tabulinesep=3pt
\begin{tabu} to 6in {|l|X|} \everyrow{\hline}
\hline
\rowfont\bfseries {Name} & {Description} \\
\tabucline[1.5pt]{}
\texttt{BDX} & The largest diameter of the body of a Tool Item. \\
\texttt{LBX} & The distance measured along the X axis from that point of the item closest to the workpiece, including the Cutting Item for a Tool Item but excluding a protruding locking mechanism for an Adaptive Item, to either the front of the flange on a flanged body or the beginning of the connection interface feature on the machine side for cylindrical or prismatic shanks. \\
\texttt{APMX} & The maximum engagement of the cutting edge or edges with the workpiece measured perpendicular to the feed motion. \\
\texttt{DC} & The maximum diameter of a circle on which the defined point Pk of each of the master inserts is located on a Tool Item. The normal of the machined peripheral surface points towards the axis of the Cutting Tool. \\
\texttt{DF} & The dimension between two parallel tangents on the outside edge of a flange. \\
\texttt{OAL} & The largest length dimension of the Cutting Tool including the master insert where applicable. \\
\texttt{DMM} & The dimension of the diameter of a cylindrical portion of a Tool Item or an Adaptive Item that can participate in a connection. \\
\texttt{H} & The dimension of the height of the shank. \\
\texttt{LS} & The dimension of the length of the shank. \\
\texttt{LUX} & Maximum length of a Cutting Tool that can be used in a particular cutting operation including the non-cutting portions of the tool. \\
\texttt{LPR} & The dimension from the yz-plane to the furthest point of the Tool Item or Adaptive Item measured in the -X direction. \\
\texttt{WT} & The total weight of the Cutting Tool in grams. The force exerted by the mass of the Cutting Tool. \\
\texttt{LF} & The distance from the gauge plane or from the end of the shank to the furthest point on the tool, if a gauge plane does not exist, to the cutting reference point determined by the main function of the tool. The {model:CuttingTool} functional length will be the length of the entire tool, not a single Cutting Item. Each {model:CuttingItem} can have an independent {model:FunctionalLength} represented in its measurements.  \\
\texttt{CRP} & The theoretical sharp point of the Cutting Tool from which the major functional dimensions are taken. \\
\texttt{L} & The theoretical length of the cutting edge of a Cutting Item over sharp corners. \\
\texttt{DRVA} & Angle between the driving mechanism locator on a Tool Item and the main cutting edge. \\
\texttt{WF} & The distance between the cutting reference point and the rear backing surface of a turning tool or the axis of a boring bar. \\
\texttt{IC} & The diameter of a circle to which all edges of a equilateral and round regular insert are tangential. \\
\texttt{SIG} & The angle between the major cutting edge and the same cutting edge rotated by 180 degrees about the tool axis. \\
\texttt{KAPR} & The angle between the tool cutting edge plane and the tool feed plane measured in a plane parallel the xy-plane. \\
\texttt{PSIR} & The angle between the tool cutting edge plane and a plane perpendicular to the tool feed plane measured in a plane parallel the xy-plane. \\
\texttt{N/A} & The angle of the tool with respect to the workpiece for a given process. The value is application specific. \\
\texttt{BS} & The measure of the length of a wiper edge of a Cutting Item. \\
\texttt{SDLx} & The length of a portion of a stepped tool that is related to a corresponding cutting diameter measured from the cutting reference point of that cutting diameter to the point on the next cutting edge at which the diameter starts to change. \\
\texttt{STAx} & The angle between a major edge on a step of a stepped tool and the same cutting edge rotated 180 degrees about its tool axis. \\
\texttt{DCx} & The diameter of a circle on which the defined point Pk located on this Cutting Tool. The normal of the machined peripheral surface points towards the axis of the Cutting Tool. \\
\texttt{HF} & The distance from the basal plane of the Tool Item to the cutting point. \\
\texttt{RE} & The nominal radius of a rounded corner measured in the X Y-plane. \\
\texttt{LFx} & The distance from the gauge plane or from the end of the shank of the Cutting Tool, if a gauge plane does not exist, to the cutting reference point determined by the main function of the tool. This measurement will be with reference to the Cutting Tool and *MUSTNOT* exist without a Cutting Tool. \\
\texttt{BCH} & The flat length of a chamfer. \\
\texttt{CHW} & The width of the chamfer. \\
\texttt{W1} & W1 is used for the insert width when an inscribed circle diameter is not practical. \\
\end{tabu}
\end{table} 
\FloatBarrier

\paragraph{\texttt{maximum}}\mbox{}
\newline\tab The maximum value for this measurement. 

\paragraph{\texttt{minimum}}\mbox{}
\newline\tab The minimum value for this measurement. 

\paragraph{\texttt{nativeUnits}}\mbox{}
\newline\tab The units the measurement was originally recorded in.

Placeholder for documentation!

\begin{table}[ht]
\centering 
  \caption{\texttt{NativeUnitEnum} Enumeration}
\tabulinesep=3pt
\begin{tabu} to 6in {|l|X|} \everyrow{\hline}
\hline
\rowfont\bfseries {Name} & {Description} \\
\tabucline[1.5pt]{}
\texttt{CENTIPOISE} & Placeholder for documentation! \\
\texttt{DEGREE/MINUTE} & Placeholder for documentation! \\
\texttt{FAHRENHEIT} & Placeholder for documentation! \\
\texttt{FOOT} & Placeholder for documentation! \\
\texttt{FOOT/MINUTE} & Placeholder for documentation! \\
\texttt{FOOT/SECOND} & Placeholder for documentation! \\
\texttt{FOOT/SECOND\^2} & Placeholder for documentation! \\
\texttt{FOOT_3D} & Placeholder for documentation! \\
\texttt{GALLON/MINUTE} & Placeholder for documentation! \\
\texttt{HOUR} & Placeholder for documentation! \\
\texttt{INCH} & Placeholder for documentation! \\
\texttt{INCH/MINUTE} & Placeholder for documentation! \\
\texttt{INCH/SECOND} & Placeholder for documentation! \\
\texttt{INCH/SECOND\^2} & Placeholder for documentation! \\
\texttt{INCH_POUND} & Placeholder for documentation! \\
\texttt{INCH_3D} & Placeholder for documentation! \\
\texttt{KELVIN} & Placeholder for documentation! \\
\texttt{KILOWATT} & Placeholder for documentation! \\
\texttt{KILOWATT_HOUR} & Placeholder for documentation! \\
\texttt{LITER} & Placeholder for documentation! \\
\texttt{LITER/MINUTE} & Placeholder for documentation! \\
\texttt{MILLIMETER/MINUTE} & Placeholder for documentation! \\
\texttt{MINUTE} & Placeholder for documentation! \\
\texttt{OTHER} & Placeholder for documentation! \\
\texttt{POUND} & Placeholder for documentation! \\
\texttt{POUND/INCH\^2} & Placeholder for documentation! \\
\texttt{RADIAN} & Placeholder for documentation! \\
\texttt{RADIAN/MINUTE} & Placeholder for documentation! \\
\texttt{RADIAN/SECOND} & Placeholder for documentation! \\
\texttt{RADIAN/SECOND\^2} & Placeholder for documentation! \\
\texttt{REVOLUTION/SECOND} & Placeholder for documentation! \\
\end{tabu}
\end{table} 
\FloatBarrier

\paragraph{\texttt{nominal}}\mbox{}
\newline\tab The as advertised value for this measurement.


\paragraph{\texttt{significantDigits}}\mbox{}
\newline\tab The number of significant digits in the reported value. 

\paragraph{\texttt{units}}\mbox{}
\newline\tab The units for the measurements. 

Placeholder for documentation!

\begin{table}[ht]
\centering 
  \caption{\texttt{UnitEnum} Enumeration}
\tabulinesep=3pt
\begin{tabu} to 6in {|l|X|} \everyrow{\hline}
\hline
\rowfont\bfseries {Name} & {Description} \\
\tabucline[1.5pt]{}
\texttt{AMPERE} & Amps \\
\texttt{CELSIUS} & Degrees Celsius \\
\texttt{COUNT} & A count of something. \\
\texttt{DECIBEL} & Sound Level \\
\texttt{DEGREE} & Angle in degrees \\
\texttt{DEGREE/SECOND} & Angular degrees per second \\
\texttt{DEGREE/SECOND\^2} & Angular acceleration in degrees per second squared \\
\texttt{HERTZ} & Frequency measured in cycles per second \\
\texttt{JOULE} & A measurement of energy. \\
\texttt{KILOGRAM} & Kilograms \\
\texttt{LITER} & Measurement of volume of a fluid \\
\texttt{LITER/SECOND} & Liters per second \\
\texttt{MICRO_RADIAN} & Measurement of Tilt \\
\texttt{MILLIMETER} & Millimeters \\
\texttt{MILLIMETER_3D} & A point in space identified by X, Y, and Z positions and represented by a space-delimited set of numbers each expressed in millimeters. \\
\texttt{MILLIMETER/REVOLUTION} & Millimeters per revolution. \\
\texttt{MILLIMETER/SECOND} & Millimeters per second \\
\texttt{MILLIMETER/SECOND\^2} & Acceleration in millimeters per second squared \\
\texttt{NEWTON} & Force in Newtons \\
\texttt{NEWTON_METER} & Torque, a unit for force times distance. \\
\texttt{OHM} & Measure of Electrical Resistance \\
\texttt{PASCAL} & Pressure in Newtons per square meter \\
\texttt{PASCAL_SECOND} & Measurement of Viscosity \\
\texttt{PERCENT} & Percentage \\
\texttt{PH} & A measure of the acidity or alkalinity of a solution. \\
\texttt{REVOLUTION/MINUTE} & Revolutions per minute \\
\texttt{SECOND} & A measurement of time. \\
\texttt{SIEMENS/METER} & A measurement of Electrical Conductivity \\
\texttt{VOLT} & Volts \\
\texttt{VOLT_AMPERE} & The measurement of the apparent power in an electrical circuit, equal to the product of root-mean-square (RMS) voltage and RMS current (commonly referred to as VA). \\
\texttt{VOLT_AMPERE_REACTIVE} & The measurement of reactive power in an AC electrical circuit (commonly referred to as VAR). \\
\texttt{WATT} & Watts \\
\texttt{WATT_SECOND} & Measurement of electrical energy, equal to one Joule \\
\end{tabu}
\end{table} 
\FloatBarrier
\FloatBarrier
\subsubsection{ProcessFeedRate}
  \label{type:ProcessFeedRate}

\FloatBarrier

The constrained process feed rate for this tool in mm/s.

\begin{table}[ht]
\centering 
  \caption{\texttt{Properties of ProcessFeedRate}}
  \label{properties:ProcessFeedRate}
\tabulinesep=3pt
\begin{tabu} to 6in {|l|l|l|} \everyrow{\hline}
\hline
\rowfont\bfseries {Property} & {Type} & {Multiplicity} \\
\tabucline[1.5pt]{}
\texttt{maximum} & \texttt{float} & 0..1 \\
\texttt{minimum} & \texttt{float} & 0..1 \\
\texttt{nominal} & \texttt{float} & 0..1 \\
\end{tabu}
\end{table}
\FloatBarrier


\paragraph{\texttt{maximum}}\mbox{}
\newline\tab The upper bound for the tool’s process target feedrate.

\paragraph{\texttt{minimum}}\mbox{}
\newline\tab The lower bound for the tools feedrate.

\paragraph{\texttt{nominal}}\mbox{}
\newline\tab The nominal feedrate the tool is designed to operate at.

\FloatBarrier
\subsubsection{ProcessSpindleSpeed}
  \label{type:ProcessSpindleSpeed}

\FloatBarrier

The constrained process spindle speed for this tool.


\begin{table}[ht]
\centering 
  \caption{\texttt{Properties of ProcessSpindleSpeed}}
  \label{properties:ProcessSpindleSpeed}
\tabulinesep=3pt
\begin{tabu} to 6in {|l|l|l|} \everyrow{\hline}
\hline
\rowfont\bfseries {Property} & {Type} & {Multiplicity} \\
\tabucline[1.5pt]{}
\texttt{maximum} & \texttt{float} & 0..1 \\
\texttt{minimum} & \texttt{float} & 0..1 \\
\texttt{nominal} & \texttt{float} & 0..1 \\
\end{tabu}
\end{table}
\FloatBarrier


\paragraph{\texttt{maximum}}\mbox{}
\newline\tab The upper bound for the tool’s target spindle speed.

\paragraph{\texttt{minimum}}\mbox{}
\newline\tab The lower bound for the tools spindle speed.


\paragraph{\texttt{nominal}}\mbox{}
\newline\tab The nominal speed the tool is designed to operate at.
\FloatBarrier
\subsubsection{ReconditionCount}
  \label{type:ReconditionCount}

\FloatBarrier

The number of times this cutter has been reconditioned.


\begin{table}[ht]
\centering 
  \caption{\texttt{Properties of ReconditionCount}}
  \label{properties:ReconditionCount}
\tabulinesep=3pt
\begin{tabu} to 6in {|l|l|l|} \everyrow{\hline}
\hline
\rowfont\bfseries {Property} & {Type} & {Multiplicity} \\
\tabucline[1.5pt]{}
\texttt{maximumCount} & \texttt{integer} & 0..1 \\
\end{tabu}
\end{table}
\FloatBarrier


\paragraph{\texttt{maximumCount}}\mbox{}
\newline\tab The maximum number of times this tool may be reconditioned.

\FloatBarrier
\subsubsection{Status}
  \label{type:Status}

\FloatBarrier

The status of the cutting tool.

\begin{table}[ht]
\centering 
  \caption{\texttt{Properties of Status}}
  \label{properties:Status}
\tabulinesep=3pt
\begin{tabu} to 6in {|l|l|l|} \everyrow{\hline}
\hline
\rowfont\bfseries {Property} & {Type} & {Multiplicity} \\
\tabucline[1.5pt]{}
\texttt{value} & \texttt{CutterStatusType} & 1 \\
\end{tabu}
\end{table}
\FloatBarrier


\paragraph{\texttt{value}}\mbox{}
\newline\tab The status value of the cutting tool.

Enumeration for CutterStatus values.

\begin{table}[ht]
\centering 
  \caption{\texttt{CutterStatusType} Enumeration}
  \label{enum:CutterStatusType}
\tabulinesep=3pt
\begin{tabu} to 6in {|l|X|} \everyrow{\hline}
\hline
\rowfont\bfseries {Name} & {Description} \\
\tabucline[1.5pt]{}
\texttt{NEW} & A new tool that has not been used or first use. Marks the start of the tool history. \\
\texttt{AVAILABLE} & Indicates the tool is available for use. If this is not present, the tool is currently not ready to be used. \\
\texttt{UNAVAILABLE} & Indicates the tool is unavailable for use in metal removal. If this is not present, the tool is currently not ready to be used.
 \\
\texttt{ALLOCATED} & Indicates if this tool is has been committed to a piece of equipment for use and is not available for use in any other piece of equipment. \\
\texttt{UNALLOCATED} & Indicates this cutting tool has not been committed to a process and can be allocated. \\
\texttt{MEASURED} & The tool has been measured.
 \\
\texttt{RECONDITIONED} & The Cutting Tool has been reconditioned. \\
\texttt{USED} & The cutting tool is in process and has remaining tool life. \\
\texttt{EXPIRED} & The cutting tool has reached the end of its useful life. \\
\texttt{BROKEN} & Premature tool failure. \\
\texttt{NOT_REGISTERED} & This cutting tool cannot be used until it is entered into the system. \\
\texttt{UNKNOWN} & The cutting tool is an indeterminate state. This is the default value. \\
\end{tabu}
\end{table} 
\FloatBarrier
\FloatBarrier
\subsubsection{ToolLife}
  \label{type:ToolLife}

\FloatBarrier

The cutting tool life as related to this assembly.

\begin{table}[ht]
\centering 
  \caption{\texttt{Properties of ToolLife}}
  \label{properties:ToolLife}
\tabulinesep=3pt
\begin{tabu} to 6in {|l|l|l|} \everyrow{\hline}
\hline
\rowfont\bfseries {Property} & {Type} & {Multiplicity} \\
\tabucline[1.5pt]{}
\texttt{countDirection} & \texttt{CountDirectionType} & 1 \\
\texttt{initial} & \texttt{float} & 0..1 \\
\texttt{limit} & \texttt{float} & 0..1 \\
\texttt{warning} & \texttt{float} & 0..1 \\
\end{tabu}
\end{table}
\FloatBarrier


\paragraph{\texttt{countDirection}}\mbox{}
\newline\tab Indicates if the tool life counts from zero to maximum or maximum to zero.

Enumeration for countDirection types.

\begin{table}[ht]
\centering 
  \caption{\texttt{CountDirectionType} Enumeration}
  \label{enum:CountDirectionType}
\tabulinesep=3pt
\begin{tabu} to 6in {|l|X|} \everyrow{\hline}
\hline
\rowfont\bfseries {Name} & {Description} \\
\tabucline[1.5pt]{}
\texttt{UP} & The tool life counts up from zero to the maximum.
 \\
\texttt{DOWN} & The tool life counts down from the maximum to zero. \\
\end{tabu}
\end{table} 
\FloatBarrier

\paragraph{\texttt{initial}}\mbox{}
\newline\tab The initial life of the tool when it is new.

\paragraph{\texttt{limit}}\mbox{}
\newline\tab The end of life limit for this tool.

\paragraph{\texttt{warning}}\mbox{}
\newline\tab The point at which a tool life warning will be raised.
\FloatBarrier
