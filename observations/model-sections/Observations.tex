% Generated 2021-02-06 01:28:16 +0530
\subsection{Observations} \label{sec:Observations}


\block{Observations} \glspl{organize} \block{Observation} elements.

Note: See \sect{Observations Schema Diagrams} for XML schema of \block{Observation} types.


\subsubsection{Observation}
\label{sec:Observation}



\block{Observation} is an abstract element that provides telemetry data for a \block{DataItem} at a point in time.


\paragraph{Attributes of Observation}\mbox{}
\label{sec:Attributes of Observation}

\tbl{Attributes of Observation} lists the attributes of \texttt{Observation}.

\begin{table}[ht]
\centering 
  \caption{Attributes of Observation}
  \label{table:Attributes of Observation}
\tabulinesep=3pt
\begin{tabu} to 6in {|l|l|l|} \everyrow{\hline}
\hline
\rowfont\bfseries {Attribute} & {Type} & {Multiplicity} \\
\tabucline[1.5pt]{}

\property{compositionId}[Observation] & \texttt{ID} & 0..1 \\
\property{dataItemId}[Observation] & \texttt{ID} & 1 \\
\property{name}[Observation] & \texttt{string} & 0..1 \\
\property{sequence}[Observation] & \texttt{integer} & 1 \\
\property{subType}[Observation] & \texttt{string} & 0..1 \\
\property{timestamp}[Observation] & \texttt{dateTime} & 1 \\
\property{type}[Observation] & \texttt{string} & 1 \\
\property{category}[Observation] & \texttt{CategoryEnum} & 1 \\
\property{units}[Observation] & \texttt{string} & 0..1 \\
\property{result}[Observation] & \texttt{string} & 0..1 \\
\end{tabu}
\end{table}
\FloatBarrier

Descriptions for attributes of \block{Observation}:

\begin{itemize}

\item \property{compositionId}[Observation] \newline The identifier of the \block{Composition} element defined in the \block{MTConnectDevices} \gls{Response Document} associated with the data reported for the \block{Observation} element.

\item \property{dataItemId}[Observation] \newline The unique identifier for the \block{Observation} element.

\item \property{name}[Observation] \newline The name of the \block{Observation} element.

\item \property{sequence}[Observation] \newline A number representing the sequential position of an occurrence of an \gls{observation} in the data buffer of an \gls{Agent}.

\item \property{subType}[Observation] \newline The \property{subType} of the \block{Observation}.

\item \property{timestamp}[Observation] \newline The most accurate time available to a piece of equipment that represents the point in time that the data reported was measured.

\item \property{type}[Observation] \newline The \property{type} of the \block{Observation} element.

\item \property{category}[Observation] \newline The \property{category} of the \block{Observation} element.

\item \property{units}[Observation] \newline The \property{units} of the \block{Observation} element.

\item \property{result}[Observation] \newline The \gls{observation} of the \block{Observation} element.
\end{itemize}



\subsubsection{Condition}
\label{sec:Condition}



\block{Condition} is an \block{Observation} that provides the information and data reported from a piece of equipment for those \block{DataItem} elements defined with a \property{category}[DataItem] attribute of \texttt{CONDITION} in the \block{MTConnectDevices} \gls{Response Document}.

The following \sect{Attributes of Condition} lists the additional and/or updated attributes for \block{Condition}.


\paragraph{Attributes of Condition}\mbox{}
\label{sec:Attributes of Condition}

\tbl{Attributes of Condition} lists the attributes of \texttt{Condition}.

\begin{table}[ht]
\centering 
  \caption{Attributes of Condition}
  \label{table:Attributes of Condition}
\tabulinesep=3pt
\begin{tabu} to 6in {|l|l|l|} \everyrow{\hline}
\hline
\rowfont\bfseries {Attribute} & {Type} & {Multiplicity} \\
\tabucline[1.5pt]{}

\property{nativeCode}[Condition] & \texttt{string} & 0..1 \\
\property{nativeSeverity}[Condition] & \texttt{string} & 0..1 \\
\property{qualifier}[Condition] & \texttt{QualifierEnum} & 0..1 \\
\property{statistic}[Condition] & \texttt{StatisticEnum} & 0..1 \\
\property{xs:lang}[Condition] & \texttt{xslang} & 0..1 \\
\property{category}[Condition] & \texttt{CONDITION} & 1 \\
\end{tabu}
\end{table}
\FloatBarrier

Descriptions for attributes of \block{Condition}:

\begin{itemize}

\item \property{nativeCode}[Condition] \newline The native code (usually an alpha-numeric value) generated by the controller of a piece of equipment providing a reference identifier for a \block{Condition}.

\item \property{nativeSeverity}[Condition] \newline If the piece of equipment designates a severity level to a fault, \property{nativeSeverity} reports that severity information to a client software application.

\item \property{qualifier}[Condition] \newline \property{qualifier} provides additional information regarding a \gls{Condition State} associated with the measured value of a process variable.

\texttt{QualifierEnum} Enumeration:

\begin{itemize}
\item \texttt{HIGH} \newline When the measured value is greater than the expected value for a process variable, \property{qualifier} \textbf{MUST} report a value of \texttt{HIGH}. 
\item \texttt{LOW} \newline When the measured value is less than the expected value for a process variable, \property{qualifier} \textbf{MUST} report a value of \texttt{LOW}. 
\end{itemize}


\item \property{statistic}[Condition] \newline \property{statistic} provides additional information describing the meaning of the \block{Condition} element.

\item \property{xs:lang}[Condition] \newline An optional attribute that specifies the language of the CDATA returned for the \block{Condition}. 

See \textit{IETF RFC 4646} (http://www.ietf.org/rfc/rfc4646.txt).
\end{itemize}



\subsubsection{Event}
\label{sec:Event}



\block{Event} is an \block{Observation} that provides the information and data reported from a piece of equipment for those \block{DataItem} elements defined with a \property{category}[DataItem] attribute of \texttt{EVENT} in the \block{MTConnectDevices} \gls{Response Document}.

The following \sect{Attributes of Event} lists the additional and/or updated attributes for \block{Event}.


\paragraph{Attributes of Event}\mbox{}
\label{sec:Attributes of Event}

\tbl{Attributes of Event} lists the attributes of \texttt{Event}.

\begin{table}[ht]
\centering 
  \caption{Attributes of Event}
  \label{table:Attributes of Event}
\tabulinesep=3pt
\begin{tabu} to 6in {|l|l|l|} \everyrow{\hline}
\hline
\rowfont\bfseries {Attribute} & {Type} & {Multiplicity} \\
\tabucline[1.5pt]{}

\property{resetTriggered}[Event] & \texttt{ResetTriggeredEnum} & 0..1 \\
\property{category}[Event] & \texttt{EVENT} & 1 \\
\end{tabu}
\end{table}
\FloatBarrier

Descriptions for attributes of \block{Event}:

\begin{itemize}

\item \property{resetTriggered}[Event] \newline For those \block{DataItem} elements that report data that may be periodically reset to an initial value, \property{resetTriggered} identifies when a reported value has been reset and what has caused that reset to occur.

\texttt{ResetTriggeredEnum} Enumeration:

\begin{itemize}
\item \texttt{ACTION\textunderscore COMPLETE} \newline The value of the \block{Observation} that is measuring an action or operation was reset upon completion of that action or operation. 
\item \texttt{ANNUAL} \newline The value of the \block{Observation} was reset at the end of a 12-month period. 
\item \texttt{DAY} \newline The value of the \block{Observation} was reset at the end of a 24-hour period. 
\item \texttt{MAINTENANCE} \newline The value of the \block{Observation} was reset upon completion of a maintenance event. 
\item \texttt{MANUAL} \newline The value of the \block{Observation} was reset based on a physical reset action. 
\item \texttt{MONTH} \newline The value of the \block{Observation} was reset at the end of a monthly period. 
\item \texttt{POWER\textunderscore ON} \newline The value of the \block{Observation} was reset when power was applied to the piece of equipment after a planned or unplanned interruption of power has occurred. 
\item \texttt{SHIFT} \newline The value of the \block{Observation} was reset at the end of a work shift. 
\item \texttt{WEEK} \newline The value of the \block{Observation} was reset at the end of a 7-day period. 
\end{itemize}

\end{itemize}



\subsubsection{Sample}
\label{sec:Sample}



\block{Sample} is an \block{Observation} that provides the information and data reported from a piece of equipment for those \block{DataItem} elements defined with a \property{category}[DataItem] attribute of \texttt{SAMPLE} in the \block{MTConnectDevices} \gls{Response Document}.

The following \sect{Attributes of Sample} lists the additional and/or updated attributes for \block{Sample}.


\paragraph{Attributes of Sample}\mbox{}
\label{sec:Attributes of Sample}

\tbl{Attributes of Sample} lists the attributes of \texttt{Sample}.

\begin{table}[ht]
\centering 
  \caption{Attributes of Sample}
  \label{table:Attributes of Sample}
\tabulinesep=3pt
\begin{tabu} to 6in {|l|l|l|} \everyrow{\hline}
\hline
\rowfont\bfseries {Attribute} & {Type} & {Multiplicity} \\
\tabucline[1.5pt]{}

\property{duration}[Sample] & \texttt{second} & 0..1 \\
\property{resetTriggered}[Sample] & \texttt{ResetTriggeredEnum} & 0..1 \\
\property{sampleRate}[Sample] & \texttt{float} & 0..1 \\
\property{statistic}[Sample] & \texttt{StatisticEnum} & 0..1 \\
\property{category}[Sample] & \texttt{SAMPLE} & 1 \\
\property{result}[Sample] & \texttt{float} & 0..1 \\
\property{units}[Sample] & \texttt{string} & 1 \\
\end{tabu}
\end{table}
\FloatBarrier

Descriptions for attributes of \block{Sample}:

\begin{itemize}

\item \property{duration}[Sample] \newline The time-period over which the data was collected.

\item \property{resetTriggered}[Sample] \newline For those \block{DataItem} elements that report data that may be periodically reset to an initial value, \property{resetTriggered} identifies when a reported value has been reset and what has caused that reset to occur.

\item \property{sampleRate}[Sample] \newline The rate at which successive samples of the value of a data item are recorded.

\property{sampleRate} is expressed in terms of samples per second.

\item \property{statistic}[Sample] \newline The type of statistical calculation defined by the \property{statistic} attribute of the \block{DataItem} element defined in the \block{MTConnectDevices} \gls{Response Document} that this element represents.
\end{itemize}


