% Generated 2021-01-06 15:42:28 +0530
\subsection{Event Types} \label{sec:Event Types}

\subsubsection{ActiveAxes}
\label{sec:ActiveAxes}



The result is a space delimited list of axes names.


The value of \texttt{ActiveAxes} \MUST be one of the following: 

\FloatBarrier

\subsubsection{ActuatorState}
\label{sec:ActuatorState}



Represents the operational state of an apparatus for moving or controlling a mechanism or system.


The value of \texttt{ActuatorState} \MUST be one of the following: 


\texttt{ActuatorStateEnum} Enumeration:

\begin{itemize}
\item \texttt{ACTIVE} \newline The value of the \gls{Data Entity} that is engaging. 
\item \texttt{INACTIVE} \newline The value of the \gls{Data Entity} that is not engaging. 
\end{itemize}

\FloatBarrier

\subsubsection{AdapterSoftwareVersion}
\label{sec:AdapterSoftwareVersion}



The originator’s software version of the \gls{Adapter}.


\subsubsection{AdapterURI}
\label{sec:AdapterURI}



The \gls{URI} of the \gls{Adapter}.


\subsubsection{Alarm}
\label{sec:Alarm}



\textbf{DEPRECATED:} Replaced with \block{CONDITION} category data items in Version 1.1.0.


\subsubsection{AlarmLimit}
\label{sec:AlarmLimit}



A set of limits used to trigger warning or alarm indicators.


The value of \texttt{AlarmLimit} \MUST be one of the following: 

\FloatBarrier

\subsubsection{AlarmLimitResultType}
\label{sec:AlarmLimitResultType}






\subsubsection{Application}
\label{sec:Application}



The application on a component.



\paragraph{Subtypes of Application}\mbox{}
\label{sec:Subtypes of Application}

\begin{itemize}

\item \texttt{INSTALL\textunderscore DATE}


The date the hardware or software was installed.

\item \texttt{LICENSE}


The license code to validate or activate the hardware or software.

\item \texttt{MANUFACTURER}


The corporate identity for the maker of the hardware or software.


\item \texttt{RELEASE\textunderscore DATE}


The date the hardware or software was released for general use.


\item \texttt{VERSION}


The version of the hardware or software.


\end{itemize}

\subsubsection{AssetChanged}
\label{sec:AssetChanged}



The value of the \gls{CDATA} for the event \textbf{MUST} be the \block{assetId} of the asset that has been added or changed. There will not be a separate message for new assets.


\subsubsection{AssetRemoved}
\label{sec:AssetRemoved}



The value of the \gls{CDATA} for the event \textbf{MUST} be the \block{assetId} of the asset that has been removed. The asset will still be visible if requested with the \block{includeRemoved} parameter as described in the protocol section. When assets are removed they are not moved to the beginning of the most recently modified list.


\subsubsection{Availability}
\label{sec:Availability}



Represents the \gls{Agent}'s ability to communicate with the data source.


The value of \texttt{Availability} \MUST be one of the following: 


\texttt{AvailabilityEnum} Enumeration:

\begin{itemize}
\item \texttt{AVAILABLE} \newline The value or status of an XML element when it is available. 
\item \texttt{UNAVAILABLE} \newline The value of the \gls{Data Entity} either when the data is not received or the entity is incapable of providing data. 
\end{itemize}

\FloatBarrier

\subsubsection{AxisCoupling}
\label{sec:AxisCoupling}



Describes the way the axes will be associated to each other. 
  
 This is used in conjunction with \block{COUPLED\textunderscore AXES} to indicate the way they are interacting.


The value of \texttt{AxisCoupling} \MUST be one of the following: 


\texttt{AxisCouplingEnum} Enumeration:

\begin{itemize}
\item \texttt{TANDEM} \newline Elements are physically connected to each other and operate as a single unit. 
\item \texttt{SYNCHRONOUS} \newline Physical or logical parts which are not physically connected to each other but are operating together. 
\item \texttt{MASTER} \newline It provides information or state values that influences the operation of other \block{DataItem} of similar type. 
\item \texttt{SLAVE} \newline The axis is a slave to the \block{COUPLED\textunderscore AXES} 
\end{itemize}

\FloatBarrier

\subsubsection{AxisFeedrateOverride}
\label{sec:AxisFeedrateOverride}



The value of a signal or calculation issued to adjust the feedrate of an individual linear type axis.


The value of \texttt{AxisFeedrateOverride} \MUST be one of the following: 

\FloatBarrier

\paragraph{Subtypes of AxisFeedrateOverride}\mbox{}
\label{sec:Subtypes of AxisFeedrateOverride}

\begin{itemize}

\item \texttt{JOG}


The feedrate specified by a logic or motion program, by a pre-set value, or set by a switch as the feedrate for the \block{Axes}. 

\item \texttt{PROGRAMMED}


The value of a signal or calculation specified by a logic or motion program or set by a switch.

\item \texttt{RAPID}


The value of a signal or calculation issued to adjust the feedrate of a component or composition that is operating in a rapid positioning mode.


\end{itemize}

\subsubsection{AxisInterlock}
\label{sec:AxisInterlock}



An indicator of the state of the axis lockout function when power has been removed and the axis is allowed to move freely.


The value of \texttt{AxisInterlock} \MUST be one of the following: 


\texttt{ActuatorStateEnum} Enumeration:

\begin{itemize}
\item \texttt{ACTIVE} \newline The value of the \gls{Data Entity} that is engaging. 
\item \texttt{INACTIVE} \newline The value of the \gls{Data Entity} that is not engaging. 
\end{itemize}

\FloatBarrier

\subsubsection{AxisState}
\label{sec:AxisState}



An indicator of the controlled state of a \block{Linear} or \block{Rotary} component representing an axis.


The value of \texttt{AxisState} \MUST be one of the following: 


\texttt{AxisStateEnum} Enumeration:

\begin{itemize}
\item \texttt{HOME} \newline The component at its home position. 
\item \texttt{TRAVEL} \newline The component is in motion. 
\item \texttt{PARKED} \newline The component has been moved to a fixed position. 
\item \texttt{STOPPED} \newline The component is stopped. 
\end{itemize}

\FloatBarrier

\subsubsection{Block}
\label{sec:Block}



The line of code or command being executed by a \block{Controller} \gls{Structural Element}.


\subsubsection{BlockCount}
\label{sec:BlockCount}



The total count of the number of blocks of program code that have been executed since execution started.


The value of \texttt{BlockCount} \MUST be one of the following: 

\FloatBarrier

\subsubsection{ChuckInterlock}




An indication of the state of an interlock function or control logic state intended to prevent the associated \block{CHUCK} component from being operated.


The value of \texttt{ChuckInterlock} \MUST be one of the following: 


\texttt{ActuatorStateEnum} Enumeration:

\begin{itemize}
\item \texttt{ACTIVE} \newline The value of the \gls{Data Entity} that is engaging. 
\item \texttt{INACTIVE} \newline The value of the \gls{Data Entity} that is not engaging. 
\end{itemize}

\FloatBarrier

\paragraph{Subtypes of ChuckInterlock}\mbox{}
\label{sec:Subtypes of ChuckInterlock}

\begin{itemize}

\item \texttt{MANUAL\textunderscore UNCLAMP}


An indication of the state of an operator controlled interlock that can inhibit the ability to initiate an unclamp action of an electronically controlled chuck.
 The \gls{Valid Data Value} \textbf{MUST} be \block{ACTIVE} or \block{INACTIVE}. 
 When \block{MANUAL\textunderscore UNCLAMP} is \block{ACTIVE}, it is expected that a chuck cannot be unclamped until \block{MANUAL\textunderscore UNCLAMP} is set to \block{INACTIVE}. 


\end{itemize}

\subsubsection{ChuckState}
\label{sec:ChuckState}



An indication of the operating state of a mechanism that holds a part or stock material during a manufacturing process. It may also represent a mechanism that holds any other mechanism in place within a piece of equipment.


The value of \texttt{ChuckState} \MUST be one of the following: 


\texttt{LatchedStateEnum} Enumeration:

\begin{itemize}
\item \texttt{OPEN} \newline A component is open to the point of a positive confirmation. 
\item \texttt{CLOSED} \newline A component is closed to the point of a positive confirmation. 
\item \texttt{UNLATCHED} \newline An intermediate position. 
\end{itemize}

\FloatBarrier

\subsubsection{CloseChuck}
\label{sec:CloseChuck}



Service to close a chuck.


\subsubsection{CloseDoor}
\label{sec:CloseDoor}



Service to close a door.


\subsubsection{Code}
\label{sec:Code}



\textbf{DEPRECATED} in Version 1.1.


\subsubsection{CompositionState}
\label{sec:CompositionState}



An indication of the operating condition of a mechanism represented by a \block{Composition} type element.


\paragraph{Subtypes of CompositionState}\mbox{}
\label{sec:Subtypes of CompositionState}

\begin{itemize}

\item \texttt{ACTION}


An indication of the operating state of a mechanism represented by a \block{Composition} type component.
 The operating state indicates whether the \block{Composition} element is activated or disabled. 
 The \gls{Valid Data Value} \textbf{MUST} be \block{ACTIVE} or \block{INACTIVE}.

\item \texttt{LATERAL}


An indication of the position of a mechanism that may move in a lateral direction.   The mechanism is represented by a \block{Composition} type component. 
 The position information indicates whether the \block{Composition} element is positioned to the right, to the left, or is in transition.  
 The \gls{Valid Data Value} \textbf{MUST} be \block{RIGHT}, \block{LEFT}, or \block{TRANSITIONING}.

\item \texttt{MOTION}


An indication of the open or closed state of a mechanism.   The mechanism is represented by a \block{Composition} type component. 
 The operating state indicates whether the state of the \block{Composition} element is open, closed, or unlatched.   
 The \gls{Valid Data Value} \textbf{MUST} be \block{OPEN}, \block{UNLATCHED}, or \block{CLOSED}.

\item \texttt{SWITCHED}


An indication of the activation state of a mechanism represented by a \block{Composition} type component.
 The activation state indicates whether the \block{Composition} element is activated or not.
 The \gls{Valid Data Value} \textbf{MUST} be \block{ON} or \block{OFF}.

\item \texttt{VERTICAL}


An indication of the position of a mechanism that may move in a vertical direction. The mechanism is represented by a \block{Composition} type component. 
 The position information indicates whether the \block{Composition} element is positioned to the top, to the bottom, or is in transition.  
 The \gls{Valid Data Value} \textbf{MUST} be \block{UP}, \block{DOWN}, or \block{TRANSITIONING}.


\end{itemize}

\subsubsection{ConnectionStatus}
\label{sec:ConnectionStatus}



The status of the connection between an \gls{Adapter} and an \gls{Agent}.


The value of \texttt{ConnectionStatus} \MUST be one of the following: 


\texttt{ConnectionStatusEnum} Enumeration:

\begin{itemize}
\item \texttt{CLOSED} \newline represents no connection at all. 
\item \texttt{LISTEN} \newline represents the \gls{Agent} waiting for a connection request from an \gls{Adapter}. 
\item \texttt{ESTABLISHED} \newline represents an open connection.

The normal state for the data transfer phase of the connection. 
\end{itemize}

\FloatBarrier

\subsubsection{ControlLimit}
\label{sec:ControlLimit}



A set of limits used to indicate whether a process variable is stable and in control.


The value of \texttt{ControlLimit} \MUST be one of the following: 

\FloatBarrier

\subsubsection{ControlLimitResultType}
\label{sec:ControlLimitResultType}






\subsubsection{ControllerMode}
\label{sec:ControllerMode}



The current operating mode of the \block{Controller} component.


The value of \texttt{ControllerMode} \MUST be one of the following: 


\texttt{ControllerModeEnum} Enumeration:

\begin{itemize}
\item \texttt{AUTOMATIC} \newline The \block{Controller} is configured to automatically execute a program. 
\item \texttt{MANUAL} \newline Operations based on the instructions received from an external source. 
\item \texttt{MANUAL\textunderscore DATA\textunderscore INPUT} \newline The operator can enter a series of operations for the controller to perform. 
\item \texttt{SEMI\textunderscore AUTOMATIC} \newline The controller  executes a single set of instructions from an active program and then stops until given a command to execute the next set of instructions. 
\item \texttt{EDIT} \newline The controller is currently functioning as a programming device and is not capable of executing an active program. 
\end{itemize}

\FloatBarrier

\subsubsection{ControllerModeOverride}
\label{sec:ControllerModeOverride}



A setting or operator selection that changes the behavior of a piece of equipment.


The value of \texttt{ControllerModeOverride} \MUST be one of the following: 


\texttt{OnOffEnum} Enumeration:

\begin{itemize}
\item \texttt{ON} \newline On state or value. 
\item \texttt{OFF} \newline Off state or value. 
\end{itemize}

\FloatBarrier

\paragraph{Subtypes of ControllerModeOverride}\mbox{}
\label{sec:Subtypes of ControllerModeOverride}

\begin{itemize}

\item \texttt{DRY\textunderscore RUN}


A setting or operator selection used to execute a test mode to confirm the execution of machine functions. 
 The \gls{Valid Data Value} \textbf{MUST} be \block{ON} or \block{OFF}. 
 When \block{DRY\textunderscore RUN} is \block{ON}, the equipment performs all of its normal functions, except no part or product is produced.  If the equipment has a spindle, spindle operation is suspended.

\item \texttt{MACHINE\textunderscore AXIS\textunderscore LOCK}


A setting or operator selection that changes the behavior of the controller on a piece of equipment. 
 The \gls{Valid Data Value} \textbf{MUST} be \block{ON} or \block{OFF}. 
 When \block{MACHINE\textunderscore AXIS\textunderscore LOCK} is \block{ON}, program execution continues normally, but no equipment motion occurs 

\item \texttt{OPTIONAL\textunderscore STOP}


A setting or operator selection that changes the behavior of the controller on a piece of equipment. 
 The \gls{Valid Data Value} \textbf{MUST} be \block{ON} or \block{OFF}.
 The program execution is stopped after a specific program block is executed when \block{OPTIONAL\textunderscore STOP} is \block{ON}.    
 In the case of a G-Code program, a program \block{BLOCK} containing a M01 code designates the command for an \block{OPTIONAL\textunderscore STOP}. 
 \block{EXECUTION} \textbf{MUST} change to \block{OPTIONAL\textunderscore STOP} after a program block specifying an optional stop is executed and the \block{OPTIONAL\textunderscore STOP} selection is \block{ON}.

\item \texttt{SINGLE\textunderscore BLOCK}


A setting or operator selection that changes the behavior of the controller on a piece of equipment. 
 The \gls{Valid Data Value} \textbf{MUST} be \block{ON} or \block{OFF}.
 Program execution is paused after each \block{BLOCK} of code is executed when \block{SINGLE\textunderscore BLOCK} is \block{ON}.   
 When \block{SINGLE\textunderscore BLOCK} is \block{ON}, \block{EXECUTION} \textbf{MUST} change to \block{INTERRUPTED} after completion of each \block{BLOCK} of code. 

\item \texttt{TOOL\textunderscore CHANGE\textunderscore STOP}


A setting or operator selection that changes the behavior of the controller on a piece of equipment. 
 The \gls{Valid Data Value} \textbf{MUST} be \block{ON} or \block{OFF}. 
 Program execution is paused when a command is executed requesting a cutting tool to be changed. 
 \block{EXECUTION} \textbf{MUST} change to \block{INTERRUPTED} after completion of the command requesting a cutting tool to be changed and \block{TOOL\textunderscore CHANGE\textunderscore STOP} is \block{ON}.


\end{itemize}

\subsubsection{CoupledAxes}
\label{sec:CoupledAxes}



Refers to the set of associated axes.


The value of \texttt{CoupledAxes} \MUST be one of the following: 

\FloatBarrier

\subsubsection{DateCode}
\label{sec:DateCode}



The time and date code associated with a material or other physical item.
  
 \block{DATE\textunderscore CODE} \textbf{MUST} be reported in ISO 8601 format.


\paragraph{Subtypes of DateCode}\mbox{}
\label{sec:Subtypes of DateCode}

\begin{itemize}

\item \texttt{EXPIRATION}


The time and date code relating to the expiration or end of useful life for a material or other physical item.

\item \texttt{FIRST\textunderscore USE}


The time and date code relating the first use of a material or other physical item.

\item \texttt{MANUFACTURE}


The time and date code relating to the production of a material or other physical item.


\end{itemize}

\subsubsection{DeviceAdded}
\label{sec:DeviceAdded}



An \block{Event} that provides the \gls{UUID} of new device added to an \gls{MTConnect Agent}.


\subsubsection{DeviceChanged}
\label{sec:DeviceChanged}



An \block{Event} that provides the \gls{UUID} of the device whose \gls{Metadata} has changed.


\subsubsection{DeviceRemoved}
\label{sec:DeviceRemoved}



An \block{Event} that provides the \gls{UUID} of a device removed from an \gls{MTConnect Agent}.


\subsubsection{DeviceUuid}
\label{sec:DeviceUuid}



The identifier of another piece of equipment that is temporarily associated with a component of this piece of equipment to perform a particular function.
  
 The \gls{Valid Data Value} \textbf{MUST} be a NMTOKEN XML type.


\subsubsection{Direction}




The direction of motion.


\paragraph{Subtypes of Direction}\mbox{}
\label{sec:Subtypes of Direction}

\begin{itemize}

\item \texttt{LINEAR}


The direction of motion of a linear motion.


The value for \block{Direction} when \property{subType} is \texttt{LINEAR} \MUST be one of the following: 


\texttt{LinearDirectionEnum} Enumeration:

\begin{itemize}
\item \texttt{POSITIVE} \newline Linear position is increasing. 
\item \texttt{NEGATIVE} \newline Linear position is decreasing. 
\item \texttt{NONE} \newline No direction. 
\end{itemize}

\item \texttt{ROTARY}


The rotational direction of a rotary motion using the right hand rule convention.
 The \gls{Valid Data Value} \textbf{MUST} be \block{CLOCKWISE} or \block{COUNTER\textunderscore CLOCKWISE}.


The value for \block{Direction} when \property{subType} is \texttt{ROTARY} \MUST be one of the following: 


\texttt{RotaryDirectionEnum} Enumeration:

\begin{itemize}
\item \texttt{CLOCKWISE} \newline Clockwise rotation using the right-hand rule. 
\item \texttt{COUNTER\textunderscore CLOCKWISE} \newline Counter-clockwise rotation using the right-hand rule. 
\item \texttt{NONE} \newline No direction. 
\end{itemize}


\end{itemize}

\subsubsection{DoorState}
\label{sec:DoorState}



The operational state of a \block{DOOR} type component or composition element.


The value of \texttt{DoorState} \MUST be one of the following: 


\texttt{LatchedStateEnum} Enumeration:

\begin{itemize}
\item \texttt{OPEN} \newline A component is open to the point of a positive confirmation. 
\item \texttt{CLOSED} \newline A component is closed to the point of a positive confirmation. 
\item \texttt{UNLATCHED} \newline An intermediate position. 
\end{itemize}

\FloatBarrier

\subsubsection{EmergencyStop}
\label{sec:EmergencyStop}



The current state of the emergency stop signal for a piece of equipment, controller path, or any other component or subsystem of a piece of equipment.


The value of \texttt{EmergencyStop} \MUST be one of the following: 


\texttt{EmergencyStopEnum} Enumeration:

\begin{itemize}
\item \texttt{ARMED} \newline The emergency stop circuit is complete and the piece of equipment, component, or composition element is allowed to operate.  
\item \texttt{TRIGGERED} \newline The operation of the piece of equipment, component, or composition element is inhibited. 
\end{itemize}

\FloatBarrier

\subsubsection{EndOfBar}




An indication of whether the end of a piece of bar stock being feed by a bar feeder has been reached.


The value of \texttt{EndOfBar} \MUST be one of the following: 


\texttt{YesNoEnum} Enumeration:

\begin{itemize}
\item \texttt{YES} \newline The \block{END\textunderscore OF\textunderscore BAR} has been reached. 
\item \texttt{NO} \newline The \block{END\textunderscore OF\textunderscore BAR} has not been reached. 
\end{itemize}

\FloatBarrier

\paragraph{Subtypes of EndOfBar}\mbox{}
\label{sec:Subtypes of EndOfBar}

\begin{itemize}

\item \texttt{AUXILIARY}


When multiple locations on a piece of bar stock are referenced as the indication for the \block{END\textunderscore OF\textunderscore BAR}, the additional location(s) \textbf{MUST} be designated as \block{AUXILIARY} indication(s) for the \block{END\textunderscore OF\textunderscore BAR}.  

\item \texttt{PRIMARY}


Specific applications \textbf{MAY} reference one or more locations on a piece of bar stock as the indication for the \block{END\textunderscore OF\textunderscore BAR}. 

The main or most important location \textbf{MUST} be designated as the \block{PRIMARY} indication for the \block{END\textunderscore OF\textunderscore BAR}.

If no \block{subType} is specified, \block{PRIMARY} \textbf{MUST} be the default \block{END\textunderscore OF\textunderscore BAR} indication.


\end{itemize}

\subsubsection{EquipmentMode}
\label{sec:EquipmentMode}



An indication that a piece of equipment, or a sub-part of a piece of equipment, is performing specific types of activities.


The value of \texttt{EquipmentMode} \MUST be one of the following: 


\texttt{OnOffEnum} Enumeration:

\begin{itemize}
\item \texttt{ON} \newline On state or value. 
\item \texttt{OFF} \newline Off state or value. 
\end{itemize}

\FloatBarrier

\paragraph{Subtypes of EquipmentMode}\mbox{}
\label{sec:Subtypes of EquipmentMode}

\begin{itemize}

\item \texttt{DELAY}


A piece of equipment waiting for an event or an action to occur.

\item \texttt{LOADED}


Subparts of a piece of equipment are under load.

\item \texttt{OPERATING}


A piece of equipment are powered or performing any activity.

\item \texttt{POWERED}


Primary  power is  applied  to the  piece  of  equipment and,  as  a minimum, the controller or logic portion of the piece of equipment is powered and functioning or components that are required to remain on are powered.

\item \texttt{WORKING}


A piece of equipment performing any activity, the equipment is active and performing a function under load or not.


\end{itemize}

\subsubsection{Execution}
\label{sec:Execution}



The execution status of the \block{Component}.


The value of \texttt{Execution} \MUST be one of the following: 


\texttt{ExecutionEnum} Enumeration:

\begin{itemize}
\item \texttt{READY} \newline A component is ready to engage. 
\item \texttt{ACTIVE} \newline The value of the \gls{Data Entity} that is engaging. 
\item \texttt{INTERRUPTED} \newline The action of a \block{Component} has been suspended due to an external signal. 
\item \texttt{FEED\textunderscore HOLD} \newline Motion of a \block{Component} has been commanded to stop at its current position. 
\item \texttt{STOPPED} \newline The component is stopped. 
\item \texttt{OPTIONAL\textunderscore STOP} \newline The controllers program has been intentionally stopped 
\item \texttt{PROGRAM\textunderscore STOPPED} \newline The execution of the \block{Controller}'s program has been stopped by a command from within the program. 
\item \texttt{PROGRAM\textunderscore COMPLETED} \newline The execution of the controllers program has been stopped by a command from within the program. 
\end{itemize}

\FloatBarrier

\subsubsection{Firmware}
\label{sec:Firmware}



The embedded software of a component.


\paragraph{Subtypes of Firmware}\mbox{}
\label{sec:Subtypes of Firmware}

\begin{itemize}

\item \texttt{INSTALL\textunderscore DATE}


The date the hardware or software was installed.

\item \texttt{LICENSE}


The license code to validate or activate the hardware or software.

\item \texttt{MANUFACTURER}


The corporate identity for the maker of the hardware or software.


\item \texttt{RELEASE\textunderscore DATE}


The date the hardware or software was released for general use.


\item \texttt{VERSION}


The version of the hardware or software.


\end{itemize}

\subsubsection{FunctionalMode}
\label{sec:FunctionalMode}



The current intended production status of the device or component.


The value of \texttt{FunctionalMode} \MUST be one of the following: 


\texttt{FunctionalModeEnum} Enumeration:

\begin{itemize}
\item \texttt{PRODUCTION} \newline A \gls{Structural Element} is currently producing product. 
\item \texttt{SETUP} \newline A \gls{Structural Element} is being prepared or modified to begin production of product. 
\item \texttt{TEARDOWN} \newline Typically, a \gls{Structural Element} has completed the production of a product and is being modified or returned to a neutral state such that it may then be prepared to begin production of a different product. 
\item \texttt{MAINTENANCE} \newline Action related to maintenance on the piece of equipment. 
\item \texttt{PROCESS\textunderscore DEVELOPMENT} \newline A \gls{Structural Element} is being used to prove-out a new process. 
\end{itemize}

\FloatBarrier

\subsubsection{Hardness}
\label{sec:Hardness}



The measurement of the hardness of a material.


The value of \texttt{Hardness} \MUST be one of the following: 

\FloatBarrier

\paragraph{Subtypes of Hardness}\mbox{}
\label{sec:Subtypes of Hardness}

\begin{itemize}

\item \texttt{BRINELL}


A scale to measure the resistance to deformation of a surface.

\item \texttt{LEEB}


A scale to measure the elasticity of a surface.

\item \texttt{MOHS}


A scale to measure the resistance to scratching of a surface.

\item \texttt{ROCKWELL}


A scale to measure the resistance to deformation of a surface.

\item \texttt{SHORE}


A scale to measure the resistance to deformation of a surface.

\item \texttt{VICKERS}


A scale to measure the resistance to deformation of a surface.


\end{itemize}

\subsubsection{Hardware}




The hardware of a component.


\paragraph{Subtypes of Hardware}\mbox{}
\label{sec:Subtypes of Hardware}

\begin{itemize}

\item \texttt{INSTALL\textunderscore DATE}


The date the hardware or software was installed.

\item \texttt{LICENSE}


The license code to validate or activate the hardware or software.

\item \texttt{MANUFACTURER}


The corporate identity for the maker of the hardware or software.


\item \texttt{RELEASE\textunderscore DATE}


The date the hardware or software was released for general use.


\item \texttt{VERSION}


The version of the hardware or software.


\end{itemize}

\subsubsection{InterfaceState}




An indication of the operational state of an \block{Interface} component.


The value of \texttt{InterfaceState} \MUST be one of the following: 


\texttt{EnabledStateEnum} Enumeration:

\begin{itemize}
\item \texttt{ENABLED} \newline A component is currently operational and performing as expected. 
\item \texttt{DISABLED} \newline A component is currently not operational. 
\end{itemize}

\FloatBarrier

\subsubsection{Library}
\label{sec:Library}



The software library on a component.



\paragraph{Subtypes of Library}\mbox{}
\label{sec:Subtypes of Library}

\begin{itemize}

\item \texttt{INSTALL\textunderscore DATE}


The date the hardware or software was installed.

\item \texttt{LICENSE}


The license code to validate or activate the hardware or software.

\item \texttt{MANUFACTURER}


The corporate identity for the maker of the hardware or software.


\item \texttt{RELEASE\textunderscore DATE}


The date the hardware or software was released for general use.


\item \texttt{VERSION}


The version of the hardware or software.


\end{itemize}

\subsubsection{Line}
\label{sec:Line}



\textbf{DEPRECATED} in Version 1.4.0.


\paragraph{Subtypes of Line}\mbox{}
\label{sec:Subtypes of Line}

\begin{itemize}

\item \texttt{MAXIMUM}


Maximum value of a data entity or attribute.

\item \texttt{MINIMUM}


The minimum value of a data entity or attribute.


\end{itemize}

\subsubsection{LineLabel}
\label{sec:LineLabel}



An optional identifier for a \block{BLOCK} of code in a \block{PROGRAM}.


\subsubsection{LineNumber}
\label{sec:LineNumber}



A reference to the position of a block of program code within a control program.


The value of \texttt{LineNumber} \MUST be one of the following: 

\FloatBarrier

\paragraph{Subtypes of LineNumber}\mbox{}
\label{sec:Subtypes of LineNumber}

\begin{itemize}

\item \texttt{ABSOLUTE}


The position of a block of program code relative to the beginning of the control program.

\item \texttt{INCREMENTAL}


The position of a block of program code relative to the occurrence of the last \block{LINE\textunderscore LABEL} encountered in the control program.


\end{itemize}

\subsubsection{MTConnectVersion}
\label{sec:MTConnectVersion}



The reference version of the MTConnect Standard supported by the \gls{Adapter}
.


\subsubsection{Material}




The identifier of a material used or consumed in the manufacturing process.


\subsubsection{MaterialChange}
\label{sec:MaterialChange}



Service to change the type of material or product being loaded or fed to a piece of equipment.


\subsubsection{MaterialFeed}
\label{sec:MaterialFeed}



Service to advance material or feed product to a piece of equipment from a continuous or bulk source.


\subsubsection{MaterialLayer}
\label{sec:MaterialLayer}



Identifies the layers of material applied to a part or product as part of an additive manufacturing process.
  
 The \gls{Valid Data Value} \textbf{MUST} be an integer.


\paragraph{Subtypes of MaterialLayer}\mbox{}
\label{sec:Subtypes of MaterialLayer}

\begin{itemize}

\item \texttt{ACTUAL}


The measured value of the data item type given by a sensor or encoder.

\item \texttt{TARGET}


The desired measure or count for a data item value.


\end{itemize}

\subsubsection{MaterialLoad}
\label{sec:MaterialLoad}



Service to load a piece of material or product.


\subsubsection{MaterialRetract}
\label{sec:MaterialRetract}



Service to remove or retract material or product.


\subsubsection{MaterialUnload}
\label{sec:MaterialUnload}



Service to unload a piece of material or product.


\subsubsection{Message}
\label{sec:Message}



Any text string of information to be transferred from a piece of equipment to a client software application.


\subsubsection{Network}
\label{sec:Network}



Network details of a component.


\paragraph{Subtypes of Network}\mbox{}
\label{sec:Subtypes of Network}

\begin{itemize}

\item \texttt{GATEWAY}


The Gateway for the component network.

\item \texttt{IPV4\textunderscore ADDRESS}


The IPV4 network address of the component.


\item \texttt{IPV6\textunderscore ADDRESS}


The IPV6 network address of the component.


\item \texttt{MAC\textunderscore ADDRESS}


Media Access Control Address. The unique physical address of the network hardware.


\item \texttt{SUBNET\textunderscore MASK}


The SubNet mask for the component network.


\item \texttt{VLAN\textunderscore ID}


The layer2 Virtual Local Network (VLAN) ID for the component network.

\item \texttt{WIRELESS}


Identifies whether the connection type is wireless.


The value for \block{Network} when \property{subType} is \texttt{WIRELESS} \MUST be one of the following: 


\texttt{YesNoEnum} Enumeration:

\begin{itemize}
\item \texttt{YES} \newline The \block{END\textunderscore OF\textunderscore BAR} has been reached. 
\item \texttt{NO} \newline The \block{END\textunderscore OF\textunderscore BAR} has not been reached. 
\end{itemize}


\end{itemize}

\subsubsection{OpenChuck}
\label{sec:OpenChuck}



Service to open a chuck.


\subsubsection{OpenDoor}
\label{sec:OpenDoor}



Service to open a door.


\subsubsection{OperatingSystem}
\label{sec:OperatingSystem}



The Operating System of a component.


\paragraph{Subtypes of OperatingSystem}\mbox{}
\label{sec:Subtypes of OperatingSystem}

\begin{itemize}

\item \texttt{INSTALL\textunderscore DATE}


The date the hardware or software was installed.

\item \texttt{LICENSE}


The license code to validate or activate the hardware or software.

\item \texttt{MANUFACTURER}


The corporate identity for the maker of the hardware or software.


\item \texttt{RELEASE\textunderscore DATE}


The date the hardware or software was released for general use.


\item \texttt{VERSION}


The version of the hardware or software.


\end{itemize}

\subsubsection{OperatorId}
\label{sec:OperatorId}



The identifier of the person currently responsible for operating the piece of equipment.


\subsubsection{PalletId}
\label{sec:PalletId}



The identifier for a pallet.


\subsubsection{PartChange}
\label{sec:PartChange}



Service to change the part or product associated with a piece of equipment to a different part or product.


\subsubsection{PartCount}
\label{sec:PartCount}



The aggregate count of parts.


The value of \texttt{PartCount} \MUST be one of the following: 

\FloatBarrier

\paragraph{Subtypes of PartCount}\mbox{}
\label{sec:Subtypes of PartCount}

\begin{itemize}

\item \texttt{ALL}


The number of parts produced. 

\item \texttt{BAD}


The number of parts produced that do not conform to specification.

\item \texttt{GOOD}


The number of parts produced that conform to specification.


\item \texttt{REMAINING}


The number of remaining or in-stock parts to be produced.

\item \texttt{TARGET}


The number of projected or planned parts to be produced.


\end{itemize}

\subsubsection{PartDetect}
\label{sec:PartDetect}



An indication designating whether a part or work piece has been detected or is present.



The value of \texttt{PartDetect} \MUST be one of the following: 


\texttt{PartDetectEnum} Enumeration:

\begin{itemize}
\item \texttt{PRESENT} \newline If a part or work piece has been detected or is present. 
\item \texttt{NOT\textunderscore PRESENT} \newline If a part or work piece is not detected or is not present. 
\end{itemize}

\FloatBarrier

\subsubsection{PartGroupId}
\label{sec:PartGroupId}



Identifier given to a collection of individual parts. 

If no \property{subType} is specified, \texttt{UUID} is default.


\paragraph{Subtypes of PartGroupId}\mbox{}
\label{sec:Subtypes of PartGroupId}

\begin{itemize}

\item \texttt{BATCH}


An identifier that references a group of parts produced in a batch.

\item \texttt{HEAT\textunderscore TREAT}


An identifier used to reference a material heat number.

\item \texttt{LOT}


An identifier that references a group of parts tracked as a lot.


\item \texttt{RAW\textunderscore MATERIAL}


The unique identifier for a singular piece of material that is used to make multiple parts.


\item \texttt{UUID}


The globally unique identifier as specified in ISO 11578 or RFC 4122.


\end{itemize}

\subsubsection{PartId}
\label{sec:PartId}



An identifier of a part in a manufacturing operation.


\subsubsection{PartKindId}
\label{sec:PartKindId}



Identifier given to link the individual occurrence to a class of parts, typically distinguished by a particular part design. 

If no \property{subType} is specified, \texttt{UUID} is default.



\paragraph{Subtypes of PartKindId}\mbox{}
\label{sec:Subtypes of PartKindId}

\begin{itemize}

\item \texttt{PART\textunderscore FAMILY}


An identifier given to a group of parts having similarities in geometry, manufacturing process, and/or functions.

\item \texttt{PART\textunderscore NAME}


A word or set of words by which a part is known, addressed, or referred to.

\item \texttt{PART\textunderscore NUMBER}


Identifier of a particular part design or model.

\item \texttt{UUID}


The globally unique identifier as specified in ISO 11578 or RFC 4122.


\end{itemize}

\subsubsection{PartNumber}
\label{sec:PartNumber}



An identifier of a part or product moving through the manufacturing process. 
 The \gls{Valid Data Value} \textbf{MUST} be a text string. 


\subsubsection{PartStatus}
\label{sec:PartStatus}



State or condition of a part.

If unique identifier is given, part status is for that individual. If group identifier is given without a unique identifier, then the status is assumed to be for the whole group.


The value of \texttt{PartStatus} \MUST be one of the following: 


\texttt{PartStatusEnum} Enumeration:

\begin{itemize}
\item \texttt{PASS} \newline The part does conform to given requirements. 
\item \texttt{FAIL} \newline The part does not conform to some given requirements. 
\end{itemize}

\FloatBarrier

\subsubsection{PartUniqueId}
\label{sec:PartUniqueId}



Identifier given to a distinguishable, individual part. 

If no \property{subType} is specified, \texttt{UUID} is default.



\paragraph{Subtypes of PartUniqueId}\mbox{}
\label{sec:Subtypes of PartUniqueId}

\begin{itemize}

\item \texttt{RAW\textunderscore MATERIAL}


The unique identifier for a singular piece of material that is used to make a single part.

\item \texttt{SERIAL\textunderscore NUMBER}


A serial number that uniquely identifies a specific part.

\item \texttt{UUID}


The globally unique identifier as specified in ISO 11578 or RFC 4122.


\end{itemize}

\subsubsection{PathFeedrateOverride}
\label{sec:PathFeedrateOverride}



The value of a signal or calculation issued to adjust the feedrate for the axes associated with a \block{Path} component that may represent a single axis or the coordinated movement of multiple axes.


The value of \texttt{PathFeedrateOverride} \MUST be one of the following: 

\FloatBarrier

\paragraph{Subtypes of PathFeedrateOverride}\mbox{}
\label{sec:Subtypes of PathFeedrateOverride}

\begin{itemize}

\item \texttt{JOG}


The feedrate specified by a logic or motion program, by a pre-set value, or set by a switch as the feedrate for the \block{Axes}. 

\item \texttt{PROGRAMMED}


The value of a signal or calculation specified by a logic or motion program or set by a switch.

\item \texttt{RAPID}


The value of a signal or calculation issued to adjust the feedrate of a component or composition that is operating in a rapid positioning mode.


\end{itemize}

\subsubsection{PathMode}
\label{sec:PathMode}



Describes the operational relationship between a \block{Path} \gls{Structural Element} and another \block{Path} \gls{Structural Element} for pieces of equipment comprised of multiple logical groupings of controlled axes or other logical operations.


The value of \texttt{PathMode} \MUST be one of the following: 


\texttt{PathModeEnum} Enumeration:

\begin{itemize}
\item \texttt{INDEPENDENT} \newline The path is operating independently and without the influence of another path. 
\item \texttt{MASTER} \newline It provides information or state values that influences the operation of other \block{DataItem} of similar type. 
\item \texttt{SYNCHRONOUS} \newline Physical or logical parts which are not physically connected to each other but are operating together. 
\item \texttt{MIRROR} \newline The axes associated with the path are mirroring the motion of the \block{MASTER} path. 
\end{itemize}

\FloatBarrier

\subsubsection{PowerState}
\label{sec:PowerState}



The indication of the status of the source of energy for a \gls{Structural Element} to allow it to perform its intended function or the state of an enabling signal providing permission for the \gls{Structural Element} to perform its functions.


The value of \texttt{PowerState} \MUST be one of the following: 


\texttt{OnOffEnum} Enumeration:

\begin{itemize}
\item \texttt{ON} \newline On state or value. 
\item \texttt{OFF} \newline Off state or value. 
\end{itemize}

\FloatBarrier

\paragraph{Subtypes of PowerState}\mbox{}
\label{sec:Subtypes of PowerState}

\begin{itemize}

\item \texttt{CONTROL}


The state of the enabling signal or control logic that enables or disables the function or operation of the \gls{Structural Element}.

\item \texttt{LINE}


The state of the power source for the \gls{Structural Element}.


\end{itemize}

\subsubsection{PowerStatus}
\label{sec:PowerStatus}



\textbf{DEPRECATED} in Version 1.1.0.


\subsubsection{ProcessAggregateId}
\label{sec:ProcessAggregateId}



Identifier given to link the individual occurrence to a group of related occurrences, such as a process step in a process plan.


\paragraph{Subtypes of ProcessAggregateId}\mbox{}
\label{sec:Subtypes of ProcessAggregateId}

\begin{itemize}

\item \texttt{ORDER\textunderscore NUMBER}


Identifier of the authorization of the process occurrence. Synonyms include "job id", "work order".

\item \texttt{PROCESS\textunderscore PLAN}


Identifier of the process plan that this occurrence belongs to. Synonyms include "routing id", "job id".


\item \texttt{PROCESS\textunderscore STEP}


Identifier of the step in the process plan that this occurrence corresponds to. Synonyms include "operation id".


\end{itemize}

\subsubsection{ProcessKindId}
\label{sec:ProcessKindId}



Identifier given to link the individual occurrence to a class of processes or process definition.



\paragraph{Subtypes of ProcessKindId}\mbox{}
\label{sec:Subtypes of ProcessKindId}

\begin{itemize}

\item \texttt{ISO\textunderscore STEP\textunderscore EXECUTABLE}


A reference to a ISO 10303 Executable.

\item \texttt{PROCESS\textunderscore NAME}


A word or set of words by which a process being executed (process occurrence) by the device is known, addressed, or referred to.


\item \texttt{UUID}


The globally unique identifier as specified in ISO 11578 or RFC 4122.


\end{itemize}

\subsubsection{ProcessOccurrenceId}
\label{sec:ProcessOccurrenceId}



An identifier of a process being executed by the device.


\subsubsection{ProcessState}
\label{sec:ProcessState}



The condition of the part or process occurrence.


The value of \texttt{ProcessState} \MUST be one of the following: 


\texttt{ProcessStateEnum} Enumeration:

\begin{itemize}
\item \texttt{RAW} \newline This is a raw material process occurrence, prior to active execution or preparation of raw material for processing. 
\item \texttt{IN\textunderscore PROCESS} \newline The process occurrence is actively executing.
 
\item \texttt{REWORK} \newline Process occurrence for correcting defective, failed, or non-conforming items after inspection. 
\item \texttt{ON\textunderscore HOLD} \newline The process occurrence is currently on hold. 
\item \texttt{SCRAP} \newline The target part has been scrapped and no further processes will be executed. 
\item \texttt{COMPLETE} \newline The process occurrence is now complete.
 
\end{itemize}

\FloatBarrier

\subsubsection{ProcessTime}
\label{sec:ProcessTime}



The time and date associated with an activity or event.
  
 \block{PROCESS\textunderscore TIME} \textbf{MUST} be reported in ISO 8601 format.


\paragraph{Subtypes of ProcessTime}\mbox{}
\label{sec:Subtypes of ProcessTime}

\begin{itemize}

\item \texttt{COMPLETE}


Completion of an action.

\item \texttt{START}


The time and date associated with the beginning of an activity or event.

\item \texttt{TARGET\textunderscore COMPLETION}


The projected time and date associated with the end or completion of an activity or event.


\end{itemize}

\subsubsection{Program}
\label{sec:Program}



The name of the logic or motion program being executed by the \block{Controller} component.


\subsubsection{ProgramComment}
\label{sec:ProgramComment}



A comment or non-executable statement in the control program.
 The \gls{Valid Data Value} \textbf{MUST} be a text string.


\subsubsection{ProgramEdit}
\label{sec:ProgramEdit}



An indication of the status of the \block{Controller} components program editing mode. 
 On many controls, a program can be edited while another program is currently being executed.


The value of \texttt{ProgramEdit} \MUST be one of the following: 


\texttt{ActiveStateEnum} Enumeration:

\begin{itemize}
\item \texttt{ACTIVE} \newline The value of the \gls{Data Entity} that is engaging. 
\item \texttt{READY} \newline A component is ready to engage. 
\item \texttt{NOT\textunderscore READY} \newline A component is not ready to engage. 
\end{itemize}

\FloatBarrier

\subsubsection{ProgramEditName}
\label{sec:ProgramEditName}



The name of the program being edited. 
 This is used in conjunction with \block{PROGRAM\textunderscore EDIT} when in \block{ACTIVE} state. 
 The \gls{Valid Data Value} \textbf{MUST} be a text string.


\subsubsection{ProgramHeader}
\label{sec:ProgramHeader}



The non-executable header section of the control program.


\paragraph{Subtypes of ProgramHeader}\mbox{}
\label{sec:Subtypes of ProgramHeader}

\begin{itemize}

\item \texttt{ACTIVE}


The value of the \gls{Data Entity} that is engaging.

\item \texttt{MAIN}


The identity of the primary logic or motion program currently being executed. It is the starting nest level in a call structure and may contain calls to sub programs.

\item \texttt{SCHEDULE}


The identity of a control program that is used to specify the order of execution of other programs.


\end{itemize}

\subsubsection{ProgramLocation}
\label{sec:ProgramLocation}



The Uniform Resource Identifier (URI) for the source file associated with \block{PROGRAM}.


\paragraph{Subtypes of ProgramLocation}\mbox{}
\label{sec:Subtypes of ProgramLocation}

\begin{itemize}

\item \texttt{ACTIVE}


The value of the \gls{Data Entity} that is engaging.

\item \texttt{MAIN}


The identity of the primary logic or motion program currently being executed. It is the starting nest level in a call structure and may contain calls to sub programs.

\item \texttt{SCHEDULE}


The identity of a control program that is used to specify the order of execution of other programs.


\end{itemize}

\subsubsection{ProgramLocationType}
\label{sec:ProgramLocationType}



Defines whether the logic or motion program defined by \block{PROGRAM} is being executed from the local memory of the controller or from an outside source.
  
 The \gls{Valid Data Value} \textbf{MUST} be \block{LOCAL} or \block{EXTERNAL}.


\paragraph{Subtypes of ProgramLocationType}\mbox{}
\label{sec:Subtypes of ProgramLocationType}

\begin{itemize}

\item \texttt{ACTIVE}


The value of the \gls{Data Entity} that is engaging.

\item \texttt{MAIN}


The identity of the primary logic or motion program currently being executed. It is the starting nest level in a call structure and may contain calls to sub programs.

\item \texttt{SCHEDULE}


The identity of a control program that is used to specify the order of execution of other programs.


\end{itemize}

\subsubsection{ProgramNestLevel}
\label{sec:ProgramNestLevel}



An indication of the nesting level within a control program that is associated with the code or instructions that is currently being executed.
  
 If an Initial Value is not defined, the nesting level associated with the highest or initial nesting level of the program \textbf{MUST} default to zero (0).
  
 The value reported for \block{PROGRAM\textunderscore NEST\textunderscore LEVEL} \textbf{MUST} be an integer.


\subsubsection{RotaryMode}
\label{sec:RotaryMode}



The current operating mode for a \block{Rotary} type axis.


The value of \texttt{RotaryMode} \MUST be one of the following: 


\texttt{RotaryModeEnum} Enumeration:

\begin{itemize}
\item \texttt{SPINDLE} \newline The axis is functioning as a spindle. 
\item \texttt{INDEX} \newline The axis is configured to index. 
\item \texttt{CONTOUR} \newline The position of the axis is being interpolated. 
\end{itemize}

\FloatBarrier

\subsubsection{RotaryVelocityOverride}
\label{sec:RotaryVelocityOverride}



The value of a command issued to adjust the programmed velocity for a \block{Rotary} type axis.
 This command represents a percentage change to the velocity calculated by a logic or motion program or set by a switch for a \block{Rotary} type axis.


The value of \texttt{RotaryVelocityOverride} \MUST be one of the following: 

\FloatBarrier

\subsubsection{Rotation}




A three space angular rotation relative to a coordinate system.


Units: \texttt{DEGREE\textunderscore 3D}.

\subsubsection{SensorAttachment}
\label{sec:SensorAttachment}



The reference version of the MTConnect Standard supported by the \gls{Adapter}
.


The value of \texttt{SensorAttachment} \MUST be one of the following: 

\FloatBarrier

\subsubsection{SensorAttachmentResultType}
\label{sec:SensorAttachmentResultType}






\subsubsection{SerialNumber}
\label{sec:SerialNumber}



The serial number associated with a \block{Component}, \block{Asset}, or \block{Device}. The \gls{Valid Data Value} \textbf{MUST} be a text string.


\subsubsection{SpecificationLimit}
\label{sec:SpecificationLimit}



A set of limits defining a range of values designating acceptable performance for a variable.


The value of \texttt{SpecificationLimit} \MUST be one of the following: 

\FloatBarrier

\subsubsection{SpecificationLimitResultType}
\label{sec:SpecificationLimitResultType}






\subsubsection{SpindleInterlock}
\label{sec:SpindleInterlock}



An indication of the status of the spindle for a piece of equipment when power has been removed and it is free to rotate.


The value of \texttt{SpindleInterlock} \MUST be one of the following: 


\texttt{ActuatorStateEnum} Enumeration:

\begin{itemize}
\item \texttt{ACTIVE} \newline The value of the \gls{Data Entity} that is engaging. 
\item \texttt{INACTIVE} \newline The value of the \gls{Data Entity} that is not engaging. 
\end{itemize}

\FloatBarrier

\subsubsection{ToolAssetId}
\label{sec:ToolAssetId}



The identifier of an individual tool asset.The \gls{Valid Data Value} \textbf{MUST} be a text string.


\subsubsection{ToolGroup}
\label{sec:ToolGroup}



An identifier for the tool group associated with a specific tool. Commonly used to designate spare tools.


\subsubsection{ToolId}
\label{sec:ToolId}



\textbf{DEPRECATED} in Version 1.2.0.   See \block{TOOL\textunderscore ASSET\textunderscore ID}. \textit{DEPRECATED:The identifier of the tool currently in use for a given \block{Path}.}


\subsubsection{ToolNumber}
\label{sec:ToolNumber}



The identifier assigned by the \block{Controller} component to a cutting tool when in use by a piece of equipment. 
 The \gls{Valid Data Value} \textbf{MUST} be a text string.


\subsubsection{ToolOffset}
\label{sec:ToolOffset}



A reference to the tool offset variables applied to the active cutting tool associated with a \block{Path} in a \block{Controller} type component.


The value of \texttt{ToolOffset} \MUST be one of the following: 

\FloatBarrier

\paragraph{Subtypes of ToolOffset}\mbox{}
\label{sec:Subtypes of ToolOffset}

\begin{itemize}

\item \texttt{LENGTH}


A reference to a length type tool offset variable.

\item \texttt{RADIAL}


A reference to a radial type tool offset variable.


\end{itemize}

\subsubsection{Translation}




A three space linear translation relative to a coordinate system.



Units: \texttt{MILLIMETER\textunderscore 3D}.

\subsubsection{User}
\label{sec:User}



The identifier of the person currently responsible for operating the piece of equipment.


\paragraph{Subtypes of User}\mbox{}
\label{sec:Subtypes of User}

\begin{itemize}

\item \texttt{MAINTENANCE}


Action related to maintenance on the piece of equipment.

\item \texttt{OPERATOR}


The identifier of the person currently responsible for operating the piece of equipment.

\item \texttt{SET\textunderscore UP}


The identifier of the person currently responsible for preparing a piece of equipment for production or restoring the piece of equipment to a neutral state after production.


\end{itemize}

\subsubsection{Variable}
\label{sec:Variable}



A data value whose meaning may change over time due to changes in the opertion of a piece of equipment or the process being executed on that piece of equipment.


\subsubsection{WaitState}
\label{sec:WaitState}



An indication of the reason that \block{EXECUTION} is reporting a value of \block{WAIT}.


The value of \texttt{WaitState} \MUST be one of the following: 


\texttt{WaitStateEnum} Enumeration:

\begin{itemize}
\item \texttt{POWERING\textunderscore UP} \newline An indication that execution is waiting while the equipment is powering up and is not currently available to begin producing parts or products. 
\item \texttt{POWERING\textunderscore DOWN} \newline An indication that the execution is waiting while the equipment is powering down but has not fully reached a stopped state. 
\item \texttt{PART\textunderscore LOAD} \newline An indication that the execution is waiting while one or more discrete workpieces are being loaded. 
\item \texttt{PART\textunderscore UNLOAD} \newline An indication that the execution is waiting while one or more discrete workpieces are being unloaded. 
\item \texttt{TOOL\textunderscore LOAD} \newline An indication that the execution is waiting while a tool or tooling is being loaded. 
\item \texttt{TOOL\textunderscore UNLOAD} \newline An indication that the execution is waiting while a tool or tooling is being unloaded. 
\item \texttt{MATERIAL\textunderscore LOAD} \newline An indication that the execution is waiting while material is being loaded. 
\item \texttt{MATERIAL\textunderscore UNLOAD} \newline An indication that the execution is waiting while material is being unloaded. 
\item \texttt{SECONDARY\textunderscore PROCESS} \newline An indication that the execution is waiting while another process is completed before the execution can resume. 
\item \texttt{PAUSING} \newline An indication that the execution is waiting while the equipment is pausing but the piece of equipment has not yet reached a fully paused state. 
\item \texttt{RESUMING} \newline An indication that the execution is waiting while the equipment is resuming the production cycle but has not yet resumed execution. 
\end{itemize}

\FloatBarrier

\subsubsection{Wire}




A string like piece or filament of relatively rigid or flexible material provided in a variety of diameters.


\subsubsection{WorkOffset}
\label{sec:WorkOffset}



A reference to the offset variables for a work piece or part associated with a \block{Path} in a \block{Controller} type component.


The value of \texttt{WorkOffset} \MUST be one of the following: 

\FloatBarrier

\subsubsection{WorkholdingId}
\label{sec:WorkholdingId}



The identifier for the current workholding or part clamp in use by a piece of equipment. 
 The \gls{Valid Data Value} \textbf{MUST} be a text string.

