% Generated 2020-08-21 17:56:43 +0530
\subsection{Event Types} \label{sec:Event Types}

\subsubsection{ActiveAxes}
\label{sec:ActiveAxes}



The result is a space delimited list of axes names.

\FloatBarrier

\subsubsection{ActuatorState}
\label{sec:ActuatorState}



Represents the operational state of an apparatus for moving or controlling a mechanism or system.


The value of \texttt{ActuatorState} \MUST be one of the following: 


\tabulinesep = 5pt
\begin{longtabu} to \textwidth {
    |l|X|}
  \caption{ActuatorStateEnum Enumeration}
  \label{enum:ActuatorStateEnum} \\

\hline
Name & Description \\
\hline
\endfirsthead
\hline
\multicolumn{2}{|c|}{Continuation of Table \texttt{ActuatorStateEnum} Enumeration} \\
\hline
Name & Description \\
\hline
\endhead
\texttt{ACTIVE} & The value of the \gls{Data Entity} that is engaging. \\ \hline
\texttt{INACTIVE} & The value of the \gls{Data Entity} that is not engaging. \\ \hline
\end{longtabu}

\FloatBarrier
\FloatBarrier

\subsubsection{Alarm}
\label{sec:Alarm}



\textbf{DEPRECATED:} Replaced with \block{CONDITION} category data items in Version 1.1.0.

\FloatBarrier

\subsubsection{Application}
\label{sec:Application}



The application on a component.



Subtypes of \texttt{Application} are: \texttt{INSTALL_DATE}, \texttt{LICENSE}, \texttt{MANUFACTURER}, \texttt{RELEASE_DATE}, \texttt{VERSION} and \texttt{VERSION}. 
\FloatBarrier

\paragraph{VersionApplication}\mbox{}
\label{sec:VersionApplication}



The version of the hardware or software.


\paragraph{ReleaseDateApplication}\mbox{}
\label{sec:ReleaseDateApplication}



The date the hardware or software was released for general use.



\paragraph{ManufacturerApplication}\mbox{}
\label{sec:ManufacturerApplication}



The corporate identity for the maker of the hardware or software.



\paragraph{LicenseApplication}\mbox{}
\label{sec:LicenseApplication}



The license code to validate or activate the hardware or software.


\paragraph{InstallDateApplication}\mbox{}
\label{sec:InstallDateApplication}



The date the hardware or software was installed.


\subsubsection{AssetChanged}
\label{sec:AssetChanged}



The value of the \gls{CDATA} for the event \textbf{MUST} be the \block{assetId} of the asset that has been added or changed. There will not be a separate message for new assets.

\FloatBarrier

\subsubsection{AssetRemoved}
\label{sec:AssetRemoved}



The value of the \gls{CDATA} for the event \textbf{MUST} be the \block{assetId} of the asset that has been removed. The asset will still be visible if requested with the \block{includeRemoved} parameter as described in the protocol section. When assets are removed they are not moved to the beginning of the most recently modified list.

\FloatBarrier

\subsubsection{Availability}
\label{sec:Availability}



Represents the \gls{Agent}'s ability to communicate with the data source.


The value of \texttt{Availability} \MUST be one of the following: 


\tabulinesep = 5pt
\begin{longtabu} to \textwidth {
    |l|X|}
  \caption{AvailabilityEnum Enumeration}
  \label{enum:AvailabilityEnum} \\

\hline
Name & Description \\
\hline
\endfirsthead
\hline
\multicolumn{2}{|c|}{Continuation of Table \texttt{AvailabilityEnum} Enumeration} \\
\hline
Name & Description \\
\hline
\endhead
\texttt{AVAILABLE} & The value or status of an XML element when it is available. \\ \hline
\texttt{UNAVAILABLE} & The value of the \gls{Data Entity} either when the data is not received or the entity is incapable of providing data. \\ \hline
\end{longtabu}

\FloatBarrier
\FloatBarrier

\subsubsection{AxisCoupling}
\label{sec:AxisCoupling}



Describes the way the axes will be associated to each other. 
  
 This is used in conjunction with \block{COUPLED\textunderscore AXES} to indicate the way they are interacting.


The value of \texttt{AxisCoupling} \MUST be one of the following: 


\tabulinesep = 5pt
\begin{longtabu} to \textwidth {
    |l|X|}
  \caption{AxisCouplingEnum Enumeration}
  \label{enum:AxisCouplingEnum} \\

\hline
Name & Description \\
\hline
\endfirsthead
\hline
\multicolumn{2}{|c|}{Continuation of Table \texttt{AxisCouplingEnum} Enumeration} \\
\hline
Name & Description \\
\hline
\endhead
\texttt{TANDEM} & Elements are physically connected to each other and operate as a single unit. \\ \hline
\texttt{SYNCHRONOUS} & Physical or logical parts which are not physically connected to each other but are operating together. \\ \hline
\texttt{MASTER} & It provides information or state values that influences the operation of other \block{DataItem} of similar type. \\ \hline
\texttt{SLAVE} & The axis is a slave to the \block{COUPLED\textunderscore AXES} \\ \hline
\end{longtabu}

\FloatBarrier
\FloatBarrier

\subsubsection{AxisFeedrateOverride}
\label{sec:AxisFeedrateOverride}



The value of a signal or calculation issued to adjust the feedrate of an individual linear type axis.


Subtypes of \texttt{AxisFeedrateOverride} are: \texttt{JOG}, \texttt{PROGRAMMED}, \texttt{RAPID} and \texttt{RAPID}. 
\FloatBarrier

\paragraph{JogAxisFeedrateOverride}\mbox{}
\label{sec:JogAxisFeedrateOverride}



The feedrate specified by a logic or motion program, by a pre-set value, or set by a switch as the feedrate for the \block{Axes}. 


\paragraph{ProgrammedAxisFeedrateOverride}\mbox{}
\label{sec:ProgrammedAxisFeedrateOverride}



The value of a signal or calculation specified by a logic or motion program or set by a switch.


\paragraph{RapidAxisFeedrateOverride}\mbox{}
\label{sec:RapidAxisFeedrateOverride}



The value of a signal or calculation issued to adjust the feedrate of a component or composition that is operating in a rapid positioning mode.


\subsubsection{AxisInterlock}
\label{sec:AxisInterlock}



An indicator of the state of the axis lockout function when power has been removed and the axis is allowed to move freely.


The value of \texttt{AxisInterlock} \MUST be one of the following: 


\tabulinesep = 5pt
\begin{longtabu} to \textwidth {
    |l|X|}
  \caption{ActuatorStateEnum Enumeration}
   \\

\hline
Name & Description \\
\hline
\endfirsthead
\hline
\multicolumn{2}{|c|}{Continuation of Table \texttt{ActuatorStateEnum} Enumeration} \\
\hline
Name & Description \\
\hline
\endhead
\texttt{ACTIVE} & The value of the \gls{Data Entity} that is engaging. \\ \hline
\texttt{INACTIVE} & The value of the \gls{Data Entity} that is not engaging. \\ \hline
\end{longtabu}

\FloatBarrier
\FloatBarrier

\subsubsection{AxisState}
\label{sec:AxisState}



An indicator of the controlled state of a \block{Linear} or \block{Rotary} component representing an axis.


The value of \texttt{AxisState} \MUST be one of the following: 


\tabulinesep = 5pt
\begin{longtabu} to \textwidth {
    |l|X|}
  \caption{AxisStateEnum Enumeration}
  \label{enum:AxisStateEnum} \\

\hline
Name & Description \\
\hline
\endfirsthead
\hline
\multicolumn{2}{|c|}{Continuation of Table \texttt{AxisStateEnum} Enumeration} \\
\hline
Name & Description \\
\hline
\endhead
\texttt{HOME} & The component at its home position. \\ \hline
\texttt{TRAVEL} & The component is in motion. \\ \hline
\texttt{PARKED} & The component has been moved to a fixed position. \\ \hline
\texttt{STOPPED} & The component is stopped. \\ \hline
\end{longtabu}

\FloatBarrier
\FloatBarrier

\subsubsection{Block}
\label{sec:Block}



The line of code or command being executed by a \block{Controller} \gls{Structural Element}.

\FloatBarrier

\subsubsection{BlockCount}
\label{sec:BlockCount}



The total count of the number of blocks of program code that have been executed since execution started.

\FloatBarrier

\subsubsection{ChuckInterlock}
\label{sec:ChuckInterlock}



An indication of the state of an interlock function or control logic state intended to prevent the associated \block{CHUCK} component from being operated.


The value of \texttt{ChuckInterlock} \MUST be one of the following: 


\tabulinesep = 5pt
\begin{longtabu} to \textwidth {
    |l|X|}
  \caption{ActuatorStateEnum Enumeration}
   \\

\hline
Name & Description \\
\hline
\endfirsthead
\hline
\multicolumn{2}{|c|}{Continuation of Table \texttt{ActuatorStateEnum} Enumeration} \\
\hline
Name & Description \\
\hline
\endhead
\texttt{ACTIVE} & The value of the \gls{Data Entity} that is engaging. \\ \hline
\texttt{INACTIVE} & The value of the \gls{Data Entity} that is not engaging. \\ \hline
\end{longtabu}

\FloatBarrier

Subtype of \texttt{ChuckInterlock} is: \texttt{MANUAL_UNCLAMP}.
\FloatBarrier

\paragraph{ManualUnclampChuckInterlock}\mbox{}
\label{sec:ManualUnclampChuckInterlock}



An indication of the state of an operator controlled interlock that can inhibit the ability to initiate an unclamp action of an electronically controlled chuck.
 The \gls{Valid Data Value} \textbf{MUST} be \block{ACTIVE} or \block{INACTIVE}. 
 When \block{MANUAL\textunderscore UNCLAMP} is \block{ACTIVE}, it is expected that a chuck cannot be unclamped until \block{MANUAL\textunderscore UNCLAMP} is set to \block{INACTIVE}. 


\subsubsection{ChuckState}
\label{sec:ChuckState}



An indication of the operating state of a mechanism that holds a part or stock material during a manufacturing process. It may also represent a mechanism that holds any other mechanism in place within a piece of equipment.


The value of \texttt{ChuckState} \MUST be one of the following: 


\tabulinesep = 5pt
\begin{longtabu} to \textwidth {
    |l|X|}
  \caption{LatchedStateEnum Enumeration}
  \label{enum:LatchedStateEnum} \\

\hline
Name & Description \\
\hline
\endfirsthead
\hline
\multicolumn{2}{|c|}{Continuation of Table \texttt{LatchedStateEnum} Enumeration} \\
\hline
Name & Description \\
\hline
\endhead
\texttt{OPEN} & A component is open to the point of a positive confirmation. \\ \hline
\texttt{CLOSED} & A component is closed to the point of a positive confirmation. \\ \hline
\texttt{UNLATCHED} & An intermediate position. \\ \hline
\end{longtabu}

\FloatBarrier
\FloatBarrier

\subsubsection{CloseChuck}
\label{sec:CloseChuck}



Service to close a chuck.

\FloatBarrier

\subsubsection{CloseDoor}
\label{sec:CloseDoor}



Service to close a door.

\FloatBarrier

\subsubsection{Code}
\label{sec:Code}



\textbf{DEPRECATED} in Version 1.1.

\FloatBarrier

\subsubsection{CompositionState}
\label{sec:CompositionState}



An indication of the operating condition of a mechanism represented by a \block{Composition} type element.


Subtypes of \texttt{CompositionState} are: \texttt{ACTION}, \texttt{LATERAL}, \texttt{MOTION}, \texttt{SWITCHED}, \texttt{VERTICAL} and \texttt{VERTICAL}. 
\FloatBarrier

\paragraph{ActionCompositionState}\mbox{}
\label{sec:ActionCompositionState}



An indication of the operating state of a mechanism represented by a \block{Composition} type component.
 The operating state indicates whether the \block{Composition} element is activated or disabled. 
 The \gls{Valid Data Value} \textbf{MUST} be \block{ACTIVE} or \block{INACTIVE}.


\paragraph{LateralCompositionState}\mbox{}
\label{sec:LateralCompositionState}



An indication of the position of a mechanism that may move in a lateral direction.   The mechanism is represented by a \block{Composition} type component. 
 The position information indicates whether the \block{Composition} element is positioned to the right, to the left, or is in transition.  
 The \gls{Valid Data Value} \textbf{MUST} be \block{RIGHT}, \block{LEFT}, or \block{TRANSITIONING}.


\paragraph{MotionCompositionState}\mbox{}
\label{sec:MotionCompositionState}



An indication of the open or closed state of a mechanism.   The mechanism is represented by a \block{Composition} type component. 
 The operating state indicates whether the state of the \block{Composition} element is open, closed, or unlatched.   
 The \gls{Valid Data Value} \textbf{MUST} be \block{OPEN}, \block{UNLATCHED}, or \block{CLOSED}.


\paragraph{SwitchedCompositionState}\mbox{}
\label{sec:SwitchedCompositionState}



An indication of the activation state of a mechanism represented by a \block{Composition} type component.
 The activation state indicates whether the \block{Composition} element is activated or not.
 The \gls{Valid Data Value} \textbf{MUST} be \block{ON} or \block{OFF}.


\paragraph{VerticalCompositionState}\mbox{}
\label{sec:VerticalCompositionState}



An indication of the position of a mechanism that may move in a vertical direction. The mechanism is represented by a \block{Composition} type component. 
 The position information indicates whether the \block{Composition} element is positioned to the top, to the bottom, or is in transition.  
 The \gls{Valid Data Value} \textbf{MUST} be \block{UP}, \block{DOWN}, or \block{TRANSITIONING}.


\subsubsection{ControllerMode}
\label{sec:ControllerMode}



The current operating mode of the \block{Controller} component.


The value of \texttt{ControllerMode} \MUST be one of the following: 


\tabulinesep = 5pt
\begin{longtabu} to \textwidth {
    |l|X|}
  \caption{ControllerModeEnum Enumeration}
  \label{enum:ControllerModeEnum} \\

\hline
Name & Description \\
\hline
\endfirsthead
\hline
\multicolumn{2}{|c|}{Continuation of Table \texttt{ControllerModeEnum} Enumeration} \\
\hline
Name & Description \\
\hline
\endhead
\texttt{AUTOMATIC} & The \block{Controller} is configured to automatically execute a program. \\ \hline
\texttt{MANUAL} & Operations based on the instructions received from an external source. \\ \hline
\texttt{MANUAL\textunderscore DATA\textunderscore INPUT} & The operator can enter a series of operations for the controller to perform. \\ \hline
\texttt{SEMI\textunderscore AUTOMATIC} & The controller  executes a single set of instructions from an active program and then stops until given a command to execute the next set of instructions. \\ \hline
\texttt{EDIT} & The controller is currently functioning as a programming device and is not capable of executing an active program. \\ \hline
\end{longtabu}

\FloatBarrier
\FloatBarrier

\subsubsection{ControllerModeOverride}
\label{sec:ControllerModeOverride}



A setting or operator selection that changes the behavior of a piece of equipment.


The value of \texttt{ControllerModeOverride} \MUST be one of the following: 


\tabulinesep = 5pt
\begin{longtabu} to \textwidth {
    |l|X|}
  \caption{OnOffEnum Enumeration}
  \label{enum:OnOffEnum} \\

\hline
Name & Description \\
\hline
\endfirsthead
\hline
\multicolumn{2}{|c|}{Continuation of Table \texttt{OnOffEnum} Enumeration} \\
\hline
Name & Description \\
\hline
\endhead
\texttt{ON} & On state or value. \\ \hline
\texttt{OFF} & Off state or value. \\ \hline
\end{longtabu}

\FloatBarrier

Subtypes of \texttt{ControllerModeOverride} are: \texttt{DRY_RUN}, \texttt{MACHINE_AXIS_LOCK}, \texttt{OPTIONAL_STOP}, \texttt{SINGLE_BLOCK}, \texttt{TOOL_CHANGE_STOP} and \texttt{TOOL_CHANGE_STOP}. 
\FloatBarrier

\paragraph{DryRunControllerModeOverride}\mbox{}
\label{sec:DryRunControllerModeOverride}



A setting or operator selection used to execute a test mode to confirm the execution of machine functions. 
 The \gls{Valid Data Value} \textbf{MUST} be \block{ON} or \block{OFF}. 
 When \block{DRY\textunderscore RUN} is \block{ON}, the equipment performs all of its normal functions, except no part or product is produced.  If the equipment has a spindle, spindle operation is suspended.


\paragraph{SingleBlockControllerModeOverride}\mbox{}
\label{sec:SingleBlockControllerModeOverride}



A setting or operator selection that changes the behavior of the controller on a piece of equipment. 
 The \gls{Valid Data Value} \textbf{MUST} be \block{ON} or \block{OFF}.
 Program execution is paused after each \block{BLOCK} of code is executed when \block{SINGLE\textunderscore BLOCK} is \block{ON}.   
 When \block{SINGLE\textunderscore BLOCK} is \block{ON}, \block{EXECUTION} \textbf{MUST} change to \block{INTERRUPTED} after completion of each \block{BLOCK} of code. 


\paragraph{MachineAxisLockControllerModeOverride}\mbox{}
\label{sec:MachineAxisLockControllerModeOverride}



A setting or operator selection that changes the behavior of the controller on a piece of equipment. 
 The \gls{Valid Data Value} \textbf{MUST} be \block{ON} or \block{OFF}. 
 When \block{MACHINE\textunderscore AXIS\textunderscore LOCK} is \block{ON}, program execution continues normally, but no equipment motion occurs 


\paragraph{OptionalStopControllerModeOverride}\mbox{}
\label{sec:OptionalStopControllerModeOverride}



A setting or operator selection that changes the behavior of the controller on a piece of equipment. 
 The \gls{Valid Data Value} \textbf{MUST} be \block{ON} or \block{OFF}.
 The program execution is stopped after a specific program block is executed when \block{OPTIONAL\textunderscore STOP} is \block{ON}.    
 In the case of a G-Code program, a program \block{BLOCK} containing a M01 code designates the command for an \block{OPTIONAL\textunderscore STOP}. 
 \block{EXECUTION} \textbf{MUST} change to \block{OPTIONAL\textunderscore STOP} after a program block specifying an optional stop is executed and the \block{OPTIONAL\textunderscore STOP} selection is \block{ON}.


\paragraph{ToolChangeStopControllerModeOverride}\mbox{}
\label{sec:ToolChangeStopControllerModeOverride}



A setting or operator selection that changes the behavior of the controller on a piece of equipment. 
 The \gls{Valid Data Value} \textbf{MUST} be \block{ON} or \block{OFF}. 
 Program execution is paused when a command is executed requesting a cutting tool to be changed. 
 \block{EXECUTION} \textbf{MUST} change to \block{INTERRUPTED} after completion of the command requesting a cutting tool to be changed and \block{TOOL\textunderscore CHANGE\textunderscore STOP} is \block{ON}.


\subsubsection{CoupledAxes}
\label{sec:CoupledAxes}



Refers to the set of associated axes.

\FloatBarrier

\subsubsection{DateCode}
\label{sec:DateCode}



The time and date code associated with a material or other physical item.
  
 \block{DATE\textunderscore CODE} \textbf{MUST} be reported in ISO 8601 format.


Subtypes of \texttt{DateCode} are: \texttt{EXPIRATION}, \texttt{FIRST_USE}, \texttt{MANUFACTURE} and \texttt{MANUFACTURE}. 
\FloatBarrier

\paragraph{ManufactureDateCode}\mbox{}
\label{sec:ManufactureDateCode}



The time and date code relating to the production of a material or other physical item.


\paragraph{ExpirationDateCode}\mbox{}
\label{sec:ExpirationDateCode}



The time and date code relating to the expiration or end of useful life for a material or other physical item.


\paragraph{FirstUseDateCode}\mbox{}
\label{sec:FirstUseDateCode}



The time and date code relating the first use of a material or other physical item.


\subsubsection{DeviceUuid}
\label{sec:DeviceUuid}



The identifier of another piece of equipment that is temporarily associated with a component of this piece of equipment to perform a particular function.
  
 The \gls{Valid Data Value} \textbf{MUST} be a NMTOKEN XML type.

\FloatBarrier

\subsubsection{Direction}
\label{sec:Direction}



The direction of motion.


Subtypes of \texttt{Direction} are: \texttt{LINEAR} and \texttt{ROTARY}. 

The value fof \texttt{Direction} when \texttt{subType} is \texttt{LINEAR} \MUST be one of the following: 


\tabulinesep = 5pt
\begin{longtabu} to \textwidth {
    |l|X|}
  \caption{LinearDirectionEnum Enumeration}
  \label{enum:LinearDirectionEnum} \\

\hline
Name & Description \\
\hline
\endfirsthead
\hline
\multicolumn{2}{|c|}{Continuation of Table \texttt{LinearDirectionEnum} Enumeration} \\
\hline
Name & Description \\
\hline
\endhead
\texttt{POSITIVE} &  \\ \hline
\texttt{NEGATIVE} &  \\ \hline
\texttt{NONE} &  \\ \hline
\end{longtabu}

\FloatBarrier

The value fof \texttt{Direction} when \texttt{subType} is \texttt{ROTARY} \MUST be one of the following: 


\tabulinesep = 5pt
\begin{longtabu} to \textwidth {
    |l|X|}
  \caption{RotaryDirectionEnum Enumeration}
  \label{enum:RotaryDirectionEnum} \\

\hline
Name & Description \\
\hline
\endfirsthead
\hline
\multicolumn{2}{|c|}{Continuation of Table \texttt{RotaryDirectionEnum} Enumeration} \\
\hline
Name & Description \\
\hline
\endhead
\texttt{CLOCKWISE} &  \\ \hline
\texttt{COUNTER\textunderscore CLOCKWISE} &  \\ \hline
\texttt{NONE} &  \\ \hline
\end{longtabu}

\FloatBarrier
\FloatBarrier

\paragraph{RotaryDirection}\mbox{}
\label{sec:RotaryDirection}



The rotational direction of a rotary motion using the right hand rule convention.
 The \gls{Valid Data Value} \textbf{MUST} be \block{CLOCKWISE} or \block{COUNTER\textunderscore CLOCKWISE}.


\paragraph{LinearDirection}\mbox{}
\label{sec:LinearDirection}



The direction of motion of a linear motion.


\subsubsection{DoorState}
\label{sec:DoorState}



The operational state of a \block{DOOR} type component or composition element.


The value of \texttt{DoorState} \MUST be one of the following: 


\tabulinesep = 5pt
\begin{longtabu} to \textwidth {
    |l|X|}
  \caption{LatchedStateEnum Enumeration}
   \\

\hline
Name & Description \\
\hline
\endfirsthead
\hline
\multicolumn{2}{|c|}{Continuation of Table \texttt{LatchedStateEnum} Enumeration} \\
\hline
Name & Description \\
\hline
\endhead
\texttt{OPEN} & A component is open to the point of a positive confirmation. \\ \hline
\texttt{CLOSED} & A component is closed to the point of a positive confirmation. \\ \hline
\texttt{UNLATCHED} & An intermediate position. \\ \hline
\end{longtabu}

\FloatBarrier
\FloatBarrier

\subsubsection{EmergencyStop}
\label{sec:EmergencyStop}



The current state of the emergency stop signal for a piece of equipment, controller path, or any other component or subsystem of a piece of equipment.


The value of \texttt{EmergencyStop} \MUST be one of the following: 


\tabulinesep = 5pt
\begin{longtabu} to \textwidth {
    |l|X|}
  \caption{EmergencyStopEnum Enumeration}
  \label{enum:EmergencyStopEnum} \\

\hline
Name & Description \\
\hline
\endfirsthead
\hline
\multicolumn{2}{|c|}{Continuation of Table \texttt{EmergencyStopEnum} Enumeration} \\
\hline
Name & Description \\
\hline
\endhead
\texttt{ARMED} & The emergency stop circuit is complete and the piece of equipment, component, or composition element is allowed to operate.  \\ \hline
\texttt{TRIGGERED} & The operation of the piece of equipment, component, or composition element is inhibited. \\ \hline
\end{longtabu}

\FloatBarrier
\FloatBarrier

\subsubsection{EndOfBar}
\label{sec:EndOfBar}



An indication of whether the end of a piece of bar stock being feed by a bar feeder has been reached.


The value of \texttt{EndOfBar} \MUST be one of the following: 


\tabulinesep = 5pt
\begin{longtabu} to \textwidth {
    |l|X|}
  \caption{YesNoEnum Enumeration}
  \label{enum:YesNoEnum} \\

\hline
Name & Description \\
\hline
\endfirsthead
\hline
\multicolumn{2}{|c|}{Continuation of Table \texttt{YesNoEnum} Enumeration} \\
\hline
Name & Description \\
\hline
\endhead
\texttt{YES} & The \block{END\textunderscore OF\textunderscore BAR} has been reached. \\ \hline
\texttt{NO} & The \block{END\textunderscore OF\textunderscore BAR} has not been reached. \\ \hline
\end{longtabu}

\FloatBarrier

Subtypes of \texttt{EndOfBar} are: \texttt{AUXILIARY} and \texttt{PRIMARY}. 
\FloatBarrier

\paragraph{PrimaryEndOfBar}\mbox{}
\label{sec:PrimaryEndOfBar}



Specific applications \textbf{MAY} reference one or more locations on a piece of bar stock as the indication for the \block{END\textunderscore OF\textunderscore BAR}. 

The main or most important location \textbf{MUST} be designated as the \block{PRIMARY} indication for the \block{END\textunderscore OF\textunderscore BAR}.

If no \block{subType} is specified, \block{PRIMARY} \textbf{MUST} be the default \block{END\textunderscore OF\textunderscore BAR} indication.


\paragraph{AuxiliaryEndOfBar}\mbox{}
\label{sec:AuxiliaryEndOfBar}



When multiple locations on a piece of bar stock are referenced as the indication for the \block{END\textunderscore OF\textunderscore BAR}, the additional location(s) \textbf{MUST} be designated as \block{AUXILIARY} indication(s) for the \block{END\textunderscore OF\textunderscore BAR}.  


\subsubsection{EquipmentMode}
\label{sec:EquipmentMode}



An indication that a piece of equipment, or a sub-part of a piece of equipment, is performing specific types of activities.


The value of \texttt{EquipmentMode} \MUST be one of the following: 


\tabulinesep = 5pt
\begin{longtabu} to \textwidth {
    |l|X|}
  \caption{OnOffEnum Enumeration}
   \\

\hline
Name & Description \\
\hline
\endfirsthead
\hline
\multicolumn{2}{|c|}{Continuation of Table \texttt{OnOffEnum} Enumeration} \\
\hline
Name & Description \\
\hline
\endhead
\texttt{ON} & On state or value. \\ \hline
\texttt{OFF} & Off state or value. \\ \hline
\end{longtabu}

\FloatBarrier

Subtypes of \texttt{EquipmentMode} are: \texttt{DELAY}, \texttt{LOADED}, \texttt{OPERATING}, \texttt{POWERED}, \texttt{WORKING} and \texttt{WORKING}. 
\FloatBarrier

\paragraph{LoadedEquipmentMode}\mbox{}
\label{sec:LoadedEquipmentMode}



Subparts of a piece of equipment are under load.


\paragraph{WorkingEquipmentMode}\mbox{}
\label{sec:WorkingEquipmentMode}



A piece of equipment performing any activity, the equipment is active and performing a function under load or not.


\paragraph{OperatingEquipmentMode}\mbox{}
\label{sec:OperatingEquipmentMode}



A piece of equipment are powered or performing any activity.


\paragraph{PoweredEquipmentMode}\mbox{}
\label{sec:PoweredEquipmentMode}



Primary  power is  applied  to the  piece  of  equipment and,  as  a minimum, the controller or logic portion of the piece of equipment is powered and functioning or components that are required to remain on are powered.


\paragraph{DelayEquipmentMode}\mbox{}
\label{sec:DelayEquipmentMode}



A piece of equipment waiting for an event or an action to occur.


\subsubsection{Execution}
\label{sec:Execution}



The execution status of the \block{Component}.


The value of \texttt{Execution} \MUST be one of the following: 


\tabulinesep = 5pt
\begin{longtabu} to \textwidth {
    |l|X|}
  \caption{ExecutionEnum Enumeration}
  \label{enum:ExecutionEnum} \\

\hline
Name & Description \\
\hline
\endfirsthead
\hline
\multicolumn{2}{|c|}{Continuation of Table \texttt{ExecutionEnum} Enumeration} \\
\hline
Name & Description \\
\hline
\endhead
\texttt{READY} & A component is ready to engage. \\ \hline
\texttt{ACTIVE} & The value of the \gls{Data Entity} that is engaging. \\ \hline
\texttt{INTERRUPTED} & The action of a \block{Component} has been suspended due to an external signal. \\ \hline
\texttt{FEED\textunderscore HOLD} & Motion of a \block{Component} has been commanded to stop at its current position. \\ \hline
\texttt{STOPPED} & The component is stopped. \\ \hline
\texttt{OPTIONAL\textunderscore STOP} & The controllers program has been intentionally stopped \\ \hline
\texttt{PROGRAM\textunderscore STOPPED} & The execution of the \block{Controller}'s program has been stopped by a command from within the program. \\ \hline
\texttt{PROGRAM\textunderscore COMPLETED} & The execution of the controllers program has been stopped by a command from within the program. \\ \hline
\end{longtabu}

\FloatBarrier
\FloatBarrier

\subsubsection{Firmware}
\label{sec:Firmware}



The embedded software of a component.


Subtypes of \texttt{Firmware} are: \texttt{INSTALL_DATE}, \texttt{LICENSE}, \texttt{MANUFACTURER}, \texttt{RELEASE_DATE}, \texttt{VERSION} and \texttt{VERSION}. 
\FloatBarrier

\paragraph{VersionFirmware}\mbox{}
\label{sec:VersionFirmware}



The version of the hardware or software.


\paragraph{ReleaseDateFirmware}\mbox{}
\label{sec:ReleaseDateFirmware}



The date the hardware or software was released for general use.



\paragraph{ManufacturerFirmware}\mbox{}
\label{sec:ManufacturerFirmware}



The corporate identity for the maker of the hardware or software.



\paragraph{LicenseFirmware}\mbox{}
\label{sec:LicenseFirmware}



The license code to validate or activate the hardware or software.


\paragraph{InstallDateFirmware}\mbox{}
\label{sec:InstallDateFirmware}



The date the hardware or software was installed.


\subsubsection{FunctionalMode}
\label{sec:FunctionalMode}



The current intended production status of the device or component.


The value of \texttt{FunctionalMode} \MUST be one of the following: 


\tabulinesep = 5pt
\begin{longtabu} to \textwidth {
    |l|X|}
  \caption{FunctionalModeEnum Enumeration}
  \label{enum:FunctionalModeEnum} \\

\hline
Name & Description \\
\hline
\endfirsthead
\hline
\multicolumn{2}{|c|}{Continuation of Table \texttt{FunctionalModeEnum} Enumeration} \\
\hline
Name & Description \\
\hline
\endhead
\texttt{PRODUCTION} & A \gls{Structural Element} is currently producing product. \\ \hline
\texttt{SETUP} & A \gls{Structural Element} is being prepared or modified to begin production of product. \\ \hline
\texttt{TEARDOWN} & Typically, a \gls{Structural Element} has completed the production of a product and is being modified or returned to a neutral state such that it may then be prepared to begin production of a different product. \\ \hline
\texttt{MAINTENANCE} & Action related to maintenance on the piece of equipment. \\ \hline
\texttt{PROCESS\textunderscore DEVELOPMENT} & A \gls{Structural Element} is being used to prove-out a new process. \\ \hline
\end{longtabu}

\FloatBarrier
\FloatBarrier

\subsubsection{Hardness}
\label{sec:Hardness}



The measurement of the hardness of a material.


Subtypes of \texttt{Hardness} are: \texttt{BRINELL}, \texttt{LEEB}, \texttt{MOHS}, \texttt{ROCKWELL}, \texttt{SHORE}, \texttt{VICKERS} and \texttt{VICKERS}. 
\FloatBarrier

\paragraph{RockwellHardness}\mbox{}
\label{sec:RockwellHardness}



A scale to measure the resistance to deformation of a surface.


\paragraph{VickersHardness}\mbox{}
\label{sec:VickersHardness}



A scale to measure the resistance to deformation of a surface.


\paragraph{ShoreHardness}\mbox{}
\label{sec:ShoreHardness}



A scale to measure the resistance to deformation of a surface.


\paragraph{BrinellHardness}\mbox{}
\label{sec:BrinellHardness}



A scale to measure the resistance to deformation of a surface.


\paragraph{LeebHardness}\mbox{}
\label{sec:LeebHardness}



A scale to measure the elasticity of a surface.


\paragraph{MohsHardness}\mbox{}
\label{sec:MohsHardness}



A scale to measure the resistance to scratching of a surface.


\subsubsection{Hardware}
\label{sec:Hardware}



The hardware of a component.


Subtypes of \texttt{Hardware} are: \texttt{INSTALL_DATE}, \texttt{LICENSE}, \texttt{MANUFACTURER}, \texttt{RELEASE_DATE}, \texttt{VERSION} and \texttt{VERSION}. 
\FloatBarrier

\paragraph{VersionHardware}\mbox{}
\label{sec:VersionHardware}



The version of the hardware or software.


\paragraph{ReleaseDateHardware}\mbox{}
\label{sec:ReleaseDateHardware}



The date the hardware or software was released for general use.



\paragraph{ManufacturerHardware}\mbox{}
\label{sec:ManufacturerHardware}



The corporate identity for the maker of the hardware or software.



\paragraph{LicenseHardware}\mbox{}
\label{sec:LicenseHardware}



The license code to validate or activate the hardware or software.


\paragraph{InstallDateHardware}\mbox{}
\label{sec:InstallDateHardware}



The date the hardware or software was installed.


\subsubsection{InterfaceState}
\label{sec:InterfaceState}



An indication of the operational state of an \block{Interface} component.


The value of \texttt{InterfaceState} \MUST be one of the following: 


\tabulinesep = 5pt
\begin{longtabu} to \textwidth {
    |l|X|}
  \caption{EnabledStateEnum Enumeration}
  \label{enum:EnabledStateEnum} \\

\hline
Name & Description \\
\hline
\endfirsthead
\hline
\multicolumn{2}{|c|}{Continuation of Table \texttt{EnabledStateEnum} Enumeration} \\
\hline
Name & Description \\
\hline
\endhead
\texttt{ENABLED} & A component is currently operational and performing as expected. \\ \hline
\texttt{DISABLED} & A component is currently not operational. \\ \hline
\end{longtabu}

\FloatBarrier
\FloatBarrier

\subsubsection{Library}
\label{sec:Library}



The software library on a component.



Subtypes of \texttt{Library} are: \texttt{INSTALL_DATE}, \texttt{LICENSE}, \texttt{MANUFACTURER}, \texttt{RELEASE_DATE}, \texttt{VERSION} and \texttt{VERSION}. 
\FloatBarrier

\paragraph{VersionLibrary}\mbox{}
\label{sec:VersionLibrary}



The version of the hardware or software.


\paragraph{ReleaseDateLibrary}\mbox{}
\label{sec:ReleaseDateLibrary}



The date the hardware or software was released for general use.



\paragraph{ManufacturerLibrary}\mbox{}
\label{sec:ManufacturerLibrary}



The corporate identity for the maker of the hardware or software.



\paragraph{LicenseLibrary}\mbox{}
\label{sec:LicenseLibrary}



The license code to validate or activate the hardware or software.


\paragraph{InstallDateLibrary}\mbox{}
\label{sec:InstallDateLibrary}



The date the hardware or software was installed.


\subsubsection{Line}
\label{sec:Line}



\textbf{DEPRECATED} in Version 1.4.0.


Subtypes of \texttt{Line} are: \texttt{MAXIMUM} and \texttt{MINIMUM}. 
\FloatBarrier

\paragraph{MaximumLine}\mbox{}
\label{sec:MaximumLine}



Maximum value of a data entity or attribute.


\paragraph{MinimumLine}\mbox{}
\label{sec:MinimumLine}



The minimum value of a data entity or attribute.


\subsubsection{LineLabel}
\label{sec:LineLabel}



An optional identifier for a \block{BLOCK} of code in a \block{PROGRAM}.

\FloatBarrier

\subsubsection{LineNumber}
\label{sec:LineNumber}



A reference to the position of a block of program code within a control program.


Subtypes of \texttt{LineNumber} are: \texttt{ABSOLUTE} and \texttt{INCREMENTAL}. 
\FloatBarrier

\paragraph{AbsoluteLineNumber}\mbox{}
\label{sec:AbsoluteLineNumber}



The position of a block of program code relative to the beginning of the control program.


\paragraph{IncrementalLineNumber}\mbox{}
\label{sec:IncrementalLineNumber}



The position of a block of program code relative to the occurrence of the last \block{LINE\textunderscore LABEL} encountered in the control program.


\subsubsection{Material}
\label{sec:Material}



The identifier of a material used or consumed in the manufacturing process.

\FloatBarrier

\subsubsection{MaterialChange}
\label{sec:MaterialChange}



Service to change the type of material or product being loaded or fed to a piece of equipment.

\FloatBarrier

\subsubsection{MaterialFeed}
\label{sec:MaterialFeed}



Service to advance material or feed product to a piece of equipment from a continuous or bulk source.

\FloatBarrier

\subsubsection{MaterialLayer}
\label{sec:MaterialLayer}



Identifies the layers of material applied to a part or product as part of an additive manufacturing process.
  
 The \gls{Valid Data Value} \textbf{MUST} be an integer.


Subtypes of \texttt{MaterialLayer} are: \texttt{ACTUAL} and \texttt{TARGET}. 
\FloatBarrier

\paragraph{ActualMaterialLayer}\mbox{}
\label{sec:ActualMaterialLayer}



The measured value of the data item type given by a sensor or encoder.


\paragraph{TargetMaterialLayer}\mbox{}
\label{sec:TargetMaterialLayer}



The desired measure or count for a data item value.


\subsubsection{MaterialLoad}
\label{sec:MaterialLoad}



Service to load a piece of material or product.

\FloatBarrier

\subsubsection{MaterialRetract}
\label{sec:MaterialRetract}



Service to remove or retract material or product.

\FloatBarrier

\subsubsection{MaterialUnload}
\label{sec:MaterialUnload}



Service to unload a piece of material or product.

\FloatBarrier

\subsubsection{Message}
\label{sec:Message}



Any text string of information to be transferred from a piece of equipment to a client software application.

\FloatBarrier

\subsubsection{Network}
\label{sec:Network}



Network details of a component.


Subtypes of \texttt{Network} are: \texttt{GATEWAY}, \texttt{IPV4_ADDRESS}, \texttt{IPV6_ADDRESS}, \texttt{MAC_ADDRESS}, \texttt{SUBNET_MASK}, \texttt{VLAN_ID}, \texttt{WIRELESS} and \texttt{WIRELESS}. 

The value fof \texttt{Network} when \texttt{subType} is \texttt{WIRELESS} \MUST be one of the following: 


\tabulinesep = 5pt
\begin{longtabu} to \textwidth {
    |l|X|}
  \caption{YesNoEnum Enumeration}
   \\

\hline
Name & Description \\
\hline
\endfirsthead
\hline
\multicolumn{2}{|c|}{Continuation of Table \texttt{YesNoEnum} Enumeration} \\
\hline
Name & Description \\
\hline
\endhead
\texttt{YES} & The \block{END\textunderscore OF\textunderscore BAR} has been reached. \\ \hline
\texttt{NO} & The \block{END\textunderscore OF\textunderscore BAR} has not been reached. \\ \hline
\end{longtabu}

\FloatBarrier
\FloatBarrier

\paragraph{IPv4AddressNetwork}\mbox{}
\label{sec:IPv4AddressNetwork}



The IPV4 network address of the component.



\paragraph{IPv6AddressNetwork}\mbox{}
\label{sec:IPv6AddressNetwork}



The IPV6 network address of the component.



\paragraph{GatewayNetwork}\mbox{}
\label{sec:GatewayNetwork}



The Gateway for the component network.


\paragraph{SubnetMaskNetwork}\mbox{}
\label{sec:SubnetMaskNetwork}



The SubNet mask for the component network.



\paragraph{VLanIdNetwork}\mbox{}
\label{sec:VLanIdNetwork}



The layer2 Virtual Local Network (VLAN) ID for the component network.


\paragraph{MacAddressNetwork}\mbox{}
\label{sec:MacAddressNetwork}



Media Access Control Address. The unique physical address of the network hardware.



\paragraph{WirelessNetwork}\mbox{}
\label{sec:WirelessNetwork}



Identifies whether the connection type is wireless.


\subsubsection{OpenChuck}
\label{sec:OpenChuck}



Service to open a chuck.

\FloatBarrier

\subsubsection{OpenDoor}
\label{sec:OpenDoor}



Service to open a door.

\FloatBarrier

\subsubsection{OperatingSystem}
\label{sec:OperatingSystem}



The Operating System of a component.


Subtypes of \texttt{OperatingSystem} are: \texttt{INSTALL_DATE}, \texttt{LICENSE}, \texttt{MANUFACTURER}, \texttt{RELEASE_DATE}, \texttt{VERSION} and \texttt{VERSION}. 
\FloatBarrier

\paragraph{LicenseOperatingSystem}\mbox{}
\label{sec:LicenseOperatingSystem}



The license code to validate or activate the hardware or software.


\paragraph{VersionOperatingSystem}\mbox{}
\label{sec:VersionOperatingSystem}



The version of the hardware or software.


\paragraph{ReleaseDateOperatingSystem}\mbox{}
\label{sec:ReleaseDateOperatingSystem}



The date the hardware or software was released for general use.



\paragraph{InstallDateOperatingSystem}\mbox{}
\label{sec:InstallDateOperatingSystem}



The date the hardware or software was installed.


\paragraph{ManufacturerOperatingSystem}\mbox{}
\label{sec:ManufacturerOperatingSystem}



The corporate identity for the maker of the hardware or software.



\subsubsection{OperatorId}
\label{sec:OperatorId}



The identifier of the person currently responsible for operating the piece of equipment.

\FloatBarrier

\subsubsection{PalletId}
\label{sec:PalletId}



The identifier for a pallet.

\FloatBarrier

\subsubsection{PartChange}
\label{sec:PartChange}



Service to change the part or product associated with a piece of equipment to a different part or product.

\FloatBarrier

\subsubsection{PartCount}
\label{sec:PartCount}



The aggregate count of parts.


Subtypes of \texttt{PartCount} are: \texttt{ALL}, \texttt{BAD}, \texttt{GOOD}, \texttt{REMAINING}, \texttt{TARGET} and \texttt{TARGET}. 
\FloatBarrier

\paragraph{AllPartCount}\mbox{}
\label{sec:AllPartCount}



The number of parts produced. 


\paragraph{GoodPartCount}\mbox{}
\label{sec:GoodPartCount}



The number of parts produced that conform to specification.



\paragraph{BadPartCount}\mbox{}
\label{sec:BadPartCount}



The number of parts produced that do not conform to specification.


\paragraph{TargetPartCount}\mbox{}
\label{sec:TargetPartCount}



The number of projected or planned parts to be produced.


\paragraph{RemainingPartCount}\mbox{}
\label{sec:RemainingPartCount}



The number of remaining or in-stock parts to be produced.


\subsubsection{PartDetect}
\label{sec:PartDetect}



An indication designating whether a part or work piece has been detected or is present.



The value of \texttt{PartDetect} \MUST be one of the following: 


\tabulinesep = 5pt
\begin{longtabu} to \textwidth {
    |l|X|}
  \caption{PartDetectEnum Enumeration}
  \label{enum:PartDetectEnum} \\

\hline
Name & Description \\
\hline
\endfirsthead
\hline
\multicolumn{2}{|c|}{Continuation of Table \texttt{PartDetectEnum} Enumeration} \\
\hline
Name & Description \\
\hline
\endhead
\texttt{PRESENT} & If a part or work piece has been detected or is present. \\ \hline
\texttt{NOT\textunderscore PRESENT} & If a part or work piece is not detected or is not present. \\ \hline
\end{longtabu}

\FloatBarrier
\FloatBarrier

\subsubsection{PartId}
\label{sec:PartId}



An identifier of a part in a manufacturing operation.

\FloatBarrier

\subsubsection{PartNumber}
\label{sec:PartNumber}



An identifier of a part or product moving through the manufacturing process. 
 The \gls{Valid Data Value} \textbf{MUST} be a text string. 

\FloatBarrier

\subsubsection{PathFeedrateOverride}
\label{sec:PathFeedrateOverride}



The value of a signal or calculation issued to adjust the feedrate for the axes associated with a \block{Path} component that may represent a single axis or the coordinated movement of multiple axes.


Subtypes of \texttt{PathFeedrateOverride} are: \texttt{JOG}, \texttt{PROGRAMMED}, \texttt{RAPID} and \texttt{RAPID}. 
\FloatBarrier

\paragraph{JogPathFeedrateOverride}\mbox{}
\label{sec:JogPathFeedrateOverride}



The feedrate specified by a logic or motion program, by a pre-set value, or set by a switch as the feedrate for the \block{Axes}. 


\paragraph{ProgrammedPathFeedrateOverride}\mbox{}
\label{sec:ProgrammedPathFeedrateOverride}



The value of a signal or calculation specified by a logic or motion program or set by a switch.


\paragraph{RapidPathFeedrateOverride}\mbox{}
\label{sec:RapidPathFeedrateOverride}



The value of a signal or calculation issued to adjust the feedrate of a component or composition that is operating in a rapid positioning mode.


\subsubsection{PathMode}
\label{sec:PathMode}



Describes the operational relationship between a \block{Path} \gls{Structural Element} and another \block{Path} \gls{Structural Element} for pieces of equipment comprised of multiple logical groupings of controlled axes or other logical operations.


The value of \texttt{PathMode} \MUST be one of the following: 


\tabulinesep = 5pt
\begin{longtabu} to \textwidth {
    |l|X|}
  \caption{PathModeEnum Enumeration}
  \label{enum:PathModeEnum} \\

\hline
Name & Description \\
\hline
\endfirsthead
\hline
\multicolumn{2}{|c|}{Continuation of Table \texttt{PathModeEnum} Enumeration} \\
\hline
Name & Description \\
\hline
\endhead
\texttt{INDEPENDENT} & The path is operating independently and without the influence of another path. \\ \hline
\texttt{MASTER} & It provides information or state values that influences the operation of other \block{DataItem} of similar type. \\ \hline
\texttt{SYNCHRONOUS} & Physical or logical parts which are not physically connected to each other but are operating together. \\ \hline
\texttt{MIRROR} & The axes associated with the path are mirroring the motion of the \block{MASTER} path. \\ \hline
\end{longtabu}

\FloatBarrier
\FloatBarrier

\subsubsection{PowerState}
\label{sec:PowerState}



The indication of the status of the source of energy for a \gls{Structural Element} to allow it to perform its intended function or the state of an enabling signal providing permission for the \gls{Structural Element} to perform its functions.


The value of \texttt{PowerState} \MUST be one of the following: 


\tabulinesep = 5pt
\begin{longtabu} to \textwidth {
    |l|X|}
  \caption{OnOffEnum Enumeration}
   \\

\hline
Name & Description \\
\hline
\endfirsthead
\hline
\multicolumn{2}{|c|}{Continuation of Table \texttt{OnOffEnum} Enumeration} \\
\hline
Name & Description \\
\hline
\endhead
\texttt{ON} & On state or value. \\ \hline
\texttt{OFF} & Off state or value. \\ \hline
\end{longtabu}

\FloatBarrier

Subtypes of \texttt{PowerState} are: \texttt{CONTROL} and \texttt{LINE}. 
\FloatBarrier

\paragraph{LinePowerState}\mbox{}
\label{sec:LinePowerState}



The state of the power source for the \gls{Structural Element}.


\paragraph{ControlPowerState}\mbox{}
\label{sec:ControlPowerState}



The state of the enabling signal or control logic that enables or disables the function or operation of the \gls{Structural Element}.


\subsubsection{PowerStatus}
\label{sec:PowerStatus}



\textbf{DEPRECATED} in Version 1.1.0.

\FloatBarrier

\subsubsection{ProcessTime}
\label{sec:ProcessTime}



The time and date associated with an activity or event.
  
 \block{PROCESS\textunderscore TIME} \textbf{MUST} be reported in ISO 8601 format.


Subtypes of \texttt{ProcessTime} are: \texttt{COMPLETE}, \texttt{START}, \texttt{TARGET_COMPLETION} and \texttt{TARGET_COMPLETION}. 
\FloatBarrier

\paragraph{StartProcessTime}\mbox{}
\label{sec:StartProcessTime}



The time and date associated with the beginning of an activity or event.


\paragraph{CompleteProcessTime}\mbox{}
\label{sec:CompleteProcessTime}



Completion of an action.


\paragraph{TargetCompletionProcessTime}\mbox{}
\label{sec:TargetCompletionProcessTime}



The projected time and date associated with the end or completion of an activity or event.


\subsubsection{Program}
\label{sec:Program}



The name of the logic or motion program being executed by the \block{Controller} component.

\FloatBarrier

\subsubsection{ProgramComment}
\label{sec:ProgramComment}



A comment or non-executable statement in the control program.
 The \gls{Valid Data Value} \textbf{MUST} be a text string.

\FloatBarrier

\subsubsection{ProgramEdit}
\label{sec:ProgramEdit}



An indication of the status of the \block{Controller} components program editing mode. 
 On many controls, a program can be edited while another program is currently being executed.


The value of \texttt{ProgramEdit} \MUST be one of the following: 


\tabulinesep = 5pt
\begin{longtabu} to \textwidth {
    |l|X|}
  \caption{ActiveStateEnum Enumeration}
  \label{enum:ActiveStateEnum} \\

\hline
Name & Description \\
\hline
\endfirsthead
\hline
\multicolumn{2}{|c|}{Continuation of Table \texttt{ActiveStateEnum} Enumeration} \\
\hline
Name & Description \\
\hline
\endhead
\texttt{ACTIVE} & The value of the \gls{Data Entity} that is engaging. \\ \hline
\texttt{READY} & A component is ready to engage. \\ \hline
\texttt{NOT\textunderscore READY} & A component is not ready to engage. \\ \hline
\end{longtabu}

\FloatBarrier
\FloatBarrier

\subsubsection{ProgramEditName}
\label{sec:ProgramEditName}



The name of the program being edited. 
 This is used in conjunction with \block{PROGRAM\textunderscore EDIT} when in \block{ACTIVE} state. 
 The \gls{Valid Data Value} \textbf{MUST} be a text string.

\FloatBarrier

\subsubsection{ProgramHeader}
\label{sec:ProgramHeader}



The non-executable header section of the control program.


Subtypes of \texttt{ProgramHeader} are: \texttt{ACTIVE}, \texttt{MAIN}, \texttt{SCHEDULE} and \texttt{SCHEDULE}. 
\FloatBarrier

\paragraph{MainProgramHeader}\mbox{}
\label{sec:MainProgramHeader}



The identity of the primary logic or motion program currently being executed. It is the starting nest level in a call structure and may contain calls to sub programs.


\paragraph{ScheduleProgramHeader}\mbox{}
\label{sec:ScheduleProgramHeader}



The identity of a control program that is used to specify the order of execution of other programs.


\paragraph{ActiveProgramHeader}\mbox{}
\label{sec:ActiveProgramHeader}



The value of the \gls{Data Entity} that is engaging.


\subsubsection{ProgramLocation}
\label{sec:ProgramLocation}



The Uniform Resource Identifier (URI) for the source file associated with \block{PROGRAM}.


Subtypes of \texttt{ProgramLocation} are: \texttt{ACTIVE}, \texttt{MAIN}, \texttt{SCHEDULE} and \texttt{SCHEDULE}. 
\FloatBarrier

\paragraph{ScheduleProgramLocation}\mbox{}
\label{sec:ScheduleProgramLocation}



The identity of a control program that is used to specify the order of execution of other programs.


\paragraph{MainProgramLocation}\mbox{}
\label{sec:MainProgramLocation}



The identity of the primary logic or motion program currently being executed. It is the starting nest level in a call structure and may contain calls to sub programs.


\paragraph{ActiveProgramLocation}\mbox{}
\label{sec:ActiveProgramLocation}



The value of the \gls{Data Entity} that is engaging.


\subsubsection{ProgramLocationType}
\label{sec:ProgramLocationType}



Defines whether the logic or motion program defined by \block{PROGRAM} is being executed from the local memory of the controller or from an outside source.
  
 The \gls{Valid Data Value} \textbf{MUST} be \block{LOCAL} or \block{EXTERNAL}.


Subtypes of \texttt{ProgramLocationType} are: \texttt{ACTIVE}, \texttt{MAIN}, \texttt{SCHEDULE} and \texttt{SCHEDULE}. 
\FloatBarrier

\paragraph{ScheduleProgramLocationType}\mbox{}
\label{sec:ScheduleProgramLocationType}



The identity of a control program that is used to specify the order of execution of other programs.


\paragraph{MainProgramLocationType}\mbox{}
\label{sec:MainProgramLocationType}



The identity of the primary logic or motion program currently being executed. It is the starting nest level in a call structure and may contain calls to sub programs.


\paragraph{ActiveProgramLocationType}\mbox{}
\label{sec:ActiveProgramLocationType}



The value of the \gls{Data Entity} that is engaging.


\subsubsection{ProgramNestLevel}
\label{sec:ProgramNestLevel}



An indication of the nesting level within a control program that is associated with the code or instructions that is currently being executed.
  
 If an Initial Value is not defined, the nesting level associated with the highest or initial nesting level of the program \textbf{MUST} default to zero (0).
  
 The value reported for \block{PROGRAM\textunderscore NEST\textunderscore LEVEL} \textbf{MUST} be an integer.

\FloatBarrier

\subsubsection{RotaryMode}
\label{sec:RotaryMode}



The current operating mode for a \block{Rotary} type axis.


The value of \texttt{RotaryMode} \MUST be one of the following: 


\tabulinesep = 5pt
\begin{longtabu} to \textwidth {
    |l|X|}
  \caption{RotaryModeEnum Enumeration}
  \label{enum:RotaryModeEnum} \\

\hline
Name & Description \\
\hline
\endfirsthead
\hline
\multicolumn{2}{|c|}{Continuation of Table \texttt{RotaryModeEnum} Enumeration} \\
\hline
Name & Description \\
\hline
\endhead
\texttt{SPINDLE} & The axis is functioning as a spindle. \\ \hline
\texttt{INDEX} & The axis is configured to index. \\ \hline
\texttt{CONTOUR} & The position of the axis is being interpolated. \\ \hline
\end{longtabu}

\FloatBarrier
\FloatBarrier

\subsubsection{RotaryVelocityOverride}
\label{sec:RotaryVelocityOverride}



The value of a command issued to adjust the programmed velocity for a \block{Rotary} type axis.
 This command represents a percentage change to the velocity calculated by a logic or motion program or set by a switch for a \block{Rotary} type axis.

\FloatBarrier

\subsubsection{Rotation}




A three space angular rotation relative to a coordinate system.


Units for \texttt{Rotation} is: \texttt{DEGREE_3D}.

\FloatBarrier

\subsubsection{SerialNumber}
\label{sec:SerialNumber}



The serial number associated with a \block{Component}, \block{Asset}, or \block{Device}. The \gls{Valid Data Value} \textbf{MUST} be a text string.

\FloatBarrier

\subsubsection{SpindleInterlock}
\label{sec:SpindleInterlock}



An indication of the status of the spindle for a piece of equipment when power has been removed and it is free to rotate.


The value of \texttt{SpindleInterlock} \MUST be one of the following: 


\tabulinesep = 5pt
\begin{longtabu} to \textwidth {
    |l|X|}
  \caption{ActuatorStateEnum Enumeration}
   \\

\hline
Name & Description \\
\hline
\endfirsthead
\hline
\multicolumn{2}{|c|}{Continuation of Table \texttt{ActuatorStateEnum} Enumeration} \\
\hline
Name & Description \\
\hline
\endhead
\texttt{ACTIVE} & The value of the \gls{Data Entity} that is engaging. \\ \hline
\texttt{INACTIVE} & The value of the \gls{Data Entity} that is not engaging. \\ \hline
\end{longtabu}

\FloatBarrier
\FloatBarrier

\subsubsection{ToolAssetId}
\label{sec:ToolAssetId}



The identifier of an individual tool asset.The \gls{Valid Data Value} \textbf{MUST} be a text string.

\FloatBarrier

\subsubsection{ToolGroup}
\label{sec:ToolGroup}



An identifier for the tool group associated with a specific tool. Commonly used to designate spare tools.

\FloatBarrier

\subsubsection{ToolId}
\label{sec:ToolId}



\textbf{DEPRECATED} in Version 1.2.0.   See \block{TOOL\textunderscore ASSET\textunderscore ID}. \textit{DEPRECATED:The identifier of the tool currently in use for a given \block{Path}.}

\FloatBarrier

\subsubsection{ToolNumber}
\label{sec:ToolNumber}



The identifier assigned by the \block{Controller} component to a cutting tool when in use by a piece of equipment. 
 The \gls{Valid Data Value} \textbf{MUST} be a text string.

\FloatBarrier

\subsubsection{ToolOffset}
\label{sec:ToolOffset}



A reference to the tool offset variables applied to the active cutting tool associated with a \block{Path} in a \block{Controller} type component.


Subtypes of \texttt{ToolOffset} are: \texttt{LENGTH} and \texttt{RADIAL}. 
\FloatBarrier

\paragraph{RadialToolOffset}\mbox{}
\label{sec:RadialToolOffset}



A reference to a radial type tool offset variable.


\paragraph{LengthToolOffset}\mbox{}
\label{sec:LengthToolOffset}



A reference to a length type tool offset variable.


\subsubsection{Translation}




A three space linear translation relative to a coordinate system.



Units for \texttt{Translation} is: \texttt{MILLIMETER_3D}.

\FloatBarrier

\subsubsection{User}
\label{sec:User}



The identifier of the person currently responsible for operating the piece of equipment.


Subtypes of \texttt{User} are: \texttt{MAINTENANCE}, \texttt{OPERATOR}, \texttt{SET_UP} and \texttt{SET_UP}. 
\FloatBarrier

\paragraph{OperatorUser}\mbox{}
\label{sec:OperatorUser}



The identifier of the person currently responsible for operating the piece of equipment.


\paragraph{MaintenanceUser}\mbox{}
\label{sec:MaintenanceUser}



Action related to maintenance on the piece of equipment.


\paragraph{SetUpUser}\mbox{}
\label{sec:SetUpUser}



The identifier of the person currently responsible for preparing a piece of equipment for production or restoring the piece of equipment to a neutral state after production.


\subsubsection{Variable}
\label{sec:Variable}



A data value whose meaning may change over time due to changes in the opertion of a piece of equipment or the process being executed on that piece of equipment.

\FloatBarrier

\subsubsection{WaitState}
\label{sec:WaitState}



An indication of the reason that \block{EXECUTION} is reporting a value of \block{WAIT}.


The value of \texttt{WaitState} \MUST be one of the following: 


\tabulinesep = 5pt
\begin{longtabu} to \textwidth {
    |l|X|}
  \caption{WaitStateEnum Enumeration}
  \label{enum:WaitStateEnum} \\

\hline
Name & Description \\
\hline
\endfirsthead
\hline
\multicolumn{2}{|c|}{Continuation of Table \texttt{WaitStateEnum} Enumeration} \\
\hline
Name & Description \\
\hline
\endhead
\texttt{POWERING\textunderscore UP} & An indication that execution is waiting while the equipment is powering up and is not currently available to begin producing parts or products. \\ \hline
\texttt{POWERING\textunderscore DOWN} & An indication that the execution is waiting while the equipment is powering down but has not fully reached a stopped state. \\ \hline
\texttt{PART\textunderscore LOAD} & An indication that the execution is waiting while one or more discrete workpieces are being loaded. \\ \hline
\texttt{PART\textunderscore UNLOAD} & An indication that the execution is waiting while one or more discrete workpieces are being unloaded. \\ \hline
\texttt{TOOL\textunderscore LOAD} & An indication that the execution is waiting while a tool or tooling is being loaded. \\ \hline
\texttt{TOOL\textunderscore UNLOAD} & An indication that the execution is waiting while a tool or tooling is being unloaded. \\ \hline
\texttt{MATERIAL\textunderscore LOAD} & An indication that the execution is waiting while material is being loaded. \\ \hline
\texttt{MATERIAL\textunderscore UNLOAD} & An indication that the execution is waiting while material is being unloaded. \\ \hline
\texttt{SECONDARY\textunderscore PROCESS} & An indication that the execution is waiting while another process is completed before the execution can resume. \\ \hline
\texttt{PAUSING} & An indication that the execution is waiting while the equipment is pausing but the piece of equipment has not yet reached a fully paused state. \\ \hline
\texttt{RESUMING} & An indication that the execution is waiting while the equipment is resuming the production cycle but has not yet resumed execution. \\ \hline
\end{longtabu}

\FloatBarrier
\FloatBarrier

\subsubsection{Wire}




A string like piece or filament of relatively rigid or flexible material provided in a variety of diameters.

\FloatBarrier

\subsubsection{WorkOffset}
\label{sec:WorkOffset}



A reference to the offset variables for a work piece or part associated with a \block{Path} in a \block{Controller} type component.

\FloatBarrier

\subsubsection{WorkholdingId}
\label{sec:WorkholdingId}



The identifier for the current workholding or part clamp in use by a piece of equipment. 
 The \gls{Valid Data Value} \textbf{MUST} be a text string.

\FloatBarrier
