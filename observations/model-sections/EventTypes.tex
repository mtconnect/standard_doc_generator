% Generated 2020-02-08 16:28:54 -0800
\subsection{EventTypes} \label{model:EventTypes}
\subsubsection[ActiveAxes]{ActiveAxes \\ {\small Subtype of Event}}
  \label{type:ActiveAxes}

\FloatBarrier

The result is a space delimited list of axes names.

\begin{table}[ht]
\centering 
  \caption{\texttt{Property of ActiveAxes}}
  \label{properties:ActiveAxes}
\tabulinesep=3pt
\begin{tabu} to 6in {|l|l|} \everyrow{\hline}
\hline
\rowfont\bfseries {Property} & {Value} \\
\tabucline[1.5pt]{}
\texttt{type} & \texttt{ACTIVE_AXES} \\
\end{tabu}
\end{table}
\FloatBarrier

\FloatBarrier
\subsubsection[ActuatorState]{ActuatorState \\ {\small Subtype of Event}}
  \label{type:ActuatorState}

\FloatBarrier

Represents the operational state of an apparatus for moving or controlling a mechanism or system.

\begin{table}[ht]
\centering 
  \caption{\texttt{Property of ActuatorState}}
  \label{properties:ActuatorState}
\tabulinesep=3pt
\begin{tabu} to 6in {|l|l|} \everyrow{\hline}
\hline
\rowfont\bfseries {Property} & {Value} \\
\tabucline[1.5pt]{}
\texttt{type} & \texttt{ACTUATOR_STATE} \\
\texttt{result} & \texttt{ActuatorStateEnum} \\
\end{tabu}
\end{table}
\FloatBarrier


 Enumerated \texttt{result} values for \texttt{ActuatorState} are:
\begin{table}[ht]
\centering 
  \caption{\texttt{ActuatorStateEnum} Enumeration}
  \label{enum:ActuatorStateEnum}
\tabulinesep=3pt
\begin{tabu} to 6in {|l|X|} \everyrow{\hline}
\hline
\rowfont\bfseries {Name} & {Description} \\
\tabucline[1.5pt]{}
\texttt{ACTIVE} & The value of the {term:Data Entity} that is engaging. \\
\texttt{INACTIVE} & The value of the {term:Data Entity} that is not engaging. \\
\end{tabu}
\end{table} 
\FloatBarrier
\FloatBarrier
\subsubsection[Alarm]{Alarm \\ {\small Subtype of Event}}
  \label{type:Alarm}

\FloatBarrier

*DEPRECATED:* Replaced with {model:CONDITION} category data items in Version 1.1.0.

\begin{table}[ht]
\centering 
  \caption{\texttt{Property of Alarm}}
  \label{properties:Alarm}
\tabulinesep=3pt
\begin{tabu} to 6in {|l|l|} \everyrow{\hline}
\hline
\rowfont\bfseries {Property} & {Value} \\
\tabucline[1.5pt]{}
\texttt{type} & \texttt{ALARM} \\
\end{tabu}
\end{table}
\FloatBarrier

\FloatBarrier
\subsubsection[AssetChanged]{AssetChanged \\ {\small Subtype of Event}}
  \label{type:AssetChanged}

\FloatBarrier

The value of the {term:CDATA} for the event *MUST* be the {model:assetId} of the asset that has been added or changed. There will not be a separate message for new assets.

\begin{table}[ht]
\centering 
  \caption{\texttt{Property of AssetChanged}}
  \label{properties:AssetChanged}
\tabulinesep=3pt
\begin{tabu} to 6in {|l|l|} \everyrow{\hline}
\hline
\rowfont\bfseries {Property} & {Value} \\
\tabucline[1.5pt]{}
\texttt{type} & \texttt{ASSET_CHANGED} \\
\end{tabu}
\end{table}
\FloatBarrier

\FloatBarrier
\subsubsection[AssetRemoved]{AssetRemoved \\ {\small Subtype of Event}}
  \label{type:AssetRemoved}

\FloatBarrier

The value of the {term:CDATA} for the event *MUST* be the {model:assetId} of the asset that has been removed. The asset will still be visible if requested with the {model:includeRemoved} parameter as described in the protocol section. When assets are removed they are not moved to the beginning of the most recently modified list.

\begin{table}[ht]
\centering 
  \caption{\texttt{Property of AssetRemoved}}
  \label{properties:AssetRemoved}
\tabulinesep=3pt
\begin{tabu} to 6in {|l|l|} \everyrow{\hline}
\hline
\rowfont\bfseries {Property} & {Value} \\
\tabucline[1.5pt]{}
\texttt{type} & \texttt{ASSET_REMOVED} \\
\end{tabu}
\end{table}
\FloatBarrier

\FloatBarrier
\subsubsection[Availability]{Availability \\ {\small Subtype of Event}}
  \label{type:Availability}

\FloatBarrier

Represents the {term:Agent}'s ability to communicate with the data source.

\begin{table}[ht]
\centering 
  \caption{\texttt{Property of Availability}}
  \label{properties:Availability}
\tabulinesep=3pt
\begin{tabu} to 6in {|l|l|} \everyrow{\hline}
\hline
\rowfont\bfseries {Property} & {Value} \\
\tabucline[1.5pt]{}
\texttt{type} & \texttt{AVAILABILITY} \\
\texttt{result} & \texttt{AvailabilityEnum} \\
\end{tabu}
\end{table}
\FloatBarrier


 Enumerated \texttt{result} values for \texttt{Availability} are:
\begin{table}[ht]
\centering 
  \caption{\texttt{AvailabilityEnum} Enumeration}
  \label{enum:AvailabilityEnum}
\tabulinesep=3pt
\begin{tabu} to 6in {|l|X|} \everyrow{\hline}
\hline
\rowfont\bfseries {Name} & {Description} \\
\tabucline[1.5pt]{}
\texttt{AVAILABLE} & The value or status of an XML element when it is available. \\
\texttt{UNAVAILABLE} & The value of the {term:Data Entity} either when the data is not received or the entity is incapable of providing data. \\
\end{tabu}
\end{table} 
\FloatBarrier
\FloatBarrier
\subsubsection[AxisCoupling]{AxisCoupling \\ {\small Subtype of Event}}
  \label{type:AxisCoupling}

\FloatBarrier

Describes the way the axes will be associated to each other. 
  
 This is used in conjunction with {model:COUPLED_AXES} to indicate the way they are interacting.

\begin{table}[ht]
\centering 
  \caption{\texttt{Property of AxisCoupling}}
  \label{properties:AxisCoupling}
\tabulinesep=3pt
\begin{tabu} to 6in {|l|l|} \everyrow{\hline}
\hline
\rowfont\bfseries {Property} & {Value} \\
\tabucline[1.5pt]{}
\texttt{type} & \texttt{AXIS_COUPLING} \\
\texttt{result} & \texttt{AxisCouplingEnum} \\
\end{tabu}
\end{table}
\FloatBarrier


 Enumerated \texttt{result} values for \texttt{AxisCoupling} are:
\begin{table}[ht]
\centering 
  \caption{\texttt{AxisCouplingEnum} Enumeration}
  \label{enum:AxisCouplingEnum}
\tabulinesep=3pt
\begin{tabu} to 6in {|l|X|} \everyrow{\hline}
\hline
\rowfont\bfseries {Name} & {Description} \\
\tabucline[1.5pt]{}
\texttt{TANDEM} & Elements are physically connected to each other and operate as a single unit. \\
\texttt{SYNCHRONOUS} & Physical or logical parts which are not physically connected to each other but are operating together. \\
\texttt{MASTER} & It provides information or state values that influences the operation of other {model:DataItem} of similar type. \\
\texttt{SLAVE} & The axis is a slave to the {model:COUPLED_AXES} \\
\end{tabu}
\end{table} 
\FloatBarrier
\FloatBarrier
\subsubsection[AxisFeedrateOverride]{AxisFeedrateOverride \\ {\small Subtype of Event}}
  \label{type:AxisFeedrateOverride}

\FloatBarrier

The value of a signal or calculation issued to adjust the feedrate of an individual linear type axis.

\begin{table}[ht]
\centering 
  \caption{\texttt{Property of AxisFeedrateOverride}}
  \label{properties:AxisFeedrateOverride}
\tabulinesep=3pt
\begin{tabu} to 6in {|l|l|} \everyrow{\hline}
\hline
\rowfont\bfseries {Property} & {Value} \\
\tabucline[1.5pt]{}
\texttt{type} & \texttt{AXIS_FEEDRATE_OVERRIDE} \\
\end{tabu}
\end{table}
\FloatBarrier

Subtypes of \texttt{AxisFeedrateOverride} are :

\begin{itemize}
\item \texttt{JOG} : The feedrate specified by a logic or motion program, by a pre-set value, or set by a switch as the feedrate for the {model:Axes}. 

\item \texttt{PROGRAMMED} : The value of a signal or calculation specified by a logic or motion program or set by a switch.

\item \texttt{RAPID} : The value of a signal or calculation issued to adjust the feedrate of a component or composition that is operating in a rapid positioning mode.

\end{itemize}

\FloatBarrier
\subsubsection[AxisInterlock]{AxisInterlock \\ {\small Subtype of Event}}
  \label{type:AxisInterlock}

\FloatBarrier

An indicator of the state of the axis lockout function when power has been removed and the axis is allowed to move freely.

\begin{table}[ht]
\centering 
  \caption{\texttt{Property of AxisInterlock}}
  \label{properties:AxisInterlock}
\tabulinesep=3pt
\begin{tabu} to 6in {|l|l|} \everyrow{\hline}
\hline
\rowfont\bfseries {Property} & {Value} \\
\tabucline[1.5pt]{}
\texttt{type} & \texttt{AXIS_INTERLOCK} \\
\texttt{result} & \texttt{ActuatorStateEnum} \\
\end{tabu}
\end{table}
\FloatBarrier


 Enumerated \texttt{result} values for \texttt{AxisInterlock} are:
\begin{table}[ht]
\centering 
  \caption{\texttt{ActuatorStateEnum} Enumeration}
\tabulinesep=3pt
\begin{tabu} to 6in {|l|X|} \everyrow{\hline}
\hline
\rowfont\bfseries {Name} & {Description} \\
\tabucline[1.5pt]{}
\texttt{ACTIVE} & The value of the {term:Data Entity} that is engaging. \\
\texttt{INACTIVE} & The value of the {term:Data Entity} that is not engaging. \\
\end{tabu}
\end{table} 
\FloatBarrier
\FloatBarrier
\subsubsection[AxisState]{AxisState \\ {\small Subtype of Event}}
  \label{type:AxisState}

\FloatBarrier

An indicator of the controlled state of a {model:Linear} or {model:Rotary} component representing an axis.

\begin{table}[ht]
\centering 
  \caption{\texttt{Property of AxisState}}
  \label{properties:AxisState}
\tabulinesep=3pt
\begin{tabu} to 6in {|l|l|} \everyrow{\hline}
\hline
\rowfont\bfseries {Property} & {Value} \\
\tabucline[1.5pt]{}
\texttt{type} & \texttt{AXIS_STATE} \\
\texttt{result} & \texttt{AxisStateEnum} \\
\end{tabu}
\end{table}
\FloatBarrier


 Enumerated \texttt{result} values for \texttt{AxisState} are:
\begin{table}[ht]
\centering 
  \caption{\texttt{AxisStateEnum} Enumeration}
  \label{enum:AxisStateEnum}
\tabulinesep=3pt
\begin{tabu} to 6in {|l|X|} \everyrow{\hline}
\hline
\rowfont\bfseries {Name} & {Description} \\
\tabucline[1.5pt]{}
\texttt{HOME} & The component at its home position. \\
\texttt{TRAVEL} & The component is in motion. \\
\texttt{PARKED} & The component has been moved to a fixed position. \\
\texttt{STOPPED} & The component is stopped. \\
\end{tabu}
\end{table} 
\FloatBarrier
\FloatBarrier
\subsubsection[Block]{Block \\ {\small Subtype of Event}}
  \label{type:Block}

\FloatBarrier

The line of code or command being executed by a {model:Controller} {term:Structural Element}.

\begin{table}[ht]
\centering 
  \caption{\texttt{Property of Block}}
  \label{properties:Block}
\tabulinesep=3pt
\begin{tabu} to 6in {|l|l|} \everyrow{\hline}
\hline
\rowfont\bfseries {Property} & {Value} \\
\tabucline[1.5pt]{}
\texttt{type} & \texttt{BLOCK} \\
\end{tabu}
\end{table}
\FloatBarrier

\FloatBarrier
\subsubsection[BlockCount]{BlockCount \\ {\small Subtype of Event}}
  \label{type:BlockCount}

\FloatBarrier

The total count of the number of blocks of program code that have been executed since execution started.

\begin{table}[ht]
\centering 
  \caption{\texttt{Property of BlockCount}}
  \label{properties:BlockCount}
\tabulinesep=3pt
\begin{tabu} to 6in {|l|l|} \everyrow{\hline}
\hline
\rowfont\bfseries {Property} & {Value} \\
\tabucline[1.5pt]{}
\texttt{type} & \texttt{BLOCK_COUNT} \\
\end{tabu}
\end{table}
\FloatBarrier

\FloatBarrier
\subsubsection[ChuckInterlock]{ChuckInterlock \\ {\small Subtype of Event}}
  \label{type:ChuckInterlock}

\FloatBarrier

An indication of the state of an interlock function or control logic state intended to prevent the associated {model:CHUCK} component from being operated.

\begin{table}[ht]
\centering 
  \caption{\texttt{Property of ChuckInterlock}}
  \label{properties:ChuckInterlock}
\tabulinesep=3pt
\begin{tabu} to 6in {|l|l|} \everyrow{\hline}
\hline
\rowfont\bfseries {Property} & {Value} \\
\tabucline[1.5pt]{}
\texttt{type} & \texttt{CHUCK_INTERLOCK} \\
\texttt{result} & \texttt{ActuatorStateEnum} \\
\end{tabu}
\end{table}
\FloatBarrier


 Enumerated \texttt{result} values for \texttt{ChuckInterlock} are:
\begin{table}[ht]
\centering 
  \caption{\texttt{ActuatorStateEnum} Enumeration}
\tabulinesep=3pt
\begin{tabu} to 6in {|l|X|} \everyrow{\hline}
\hline
\rowfont\bfseries {Name} & {Description} \\
\tabucline[1.5pt]{}
\texttt{ACTIVE} & The value of the {term:Data Entity} that is engaging. \\
\texttt{INACTIVE} & The value of the {term:Data Entity} that is not engaging. \\
\end{tabu}
\end{table} 
\FloatBarrier
Subtypes of \texttt{ChuckInterlock} are :

\begin{itemize}
\item \texttt{MANUAL_UNCLAMP} : An indication of the state of an operator controlled interlock that can inhibit the ability to initiate an unclamp action of an electronically controlled chuck.
 The {term:Valid Data Value} *MUST* be {model:ACTIVE} or {model:INACTIVE}. 
 When {model:MANUAL_UNCLAMP} is {model:ACTIVE}, it is expected that a chuck cannot be unclamped until {model:MANUAL_UNCLAMP} is set to {model:INACTIVE}. 

\end{itemize}

\FloatBarrier
\subsubsection[ChuckState]{ChuckState \\ {\small Subtype of Event}}
  \label{type:ChuckState}

\FloatBarrier

An indication of the operating state of a mechanism that holds a part or stock material during a manufacturing process. It may also represent a mechanism that holds any other mechanism in place within a piece of equipment.

\begin{table}[ht]
\centering 
  \caption{\texttt{Property of ChuckState}}
  \label{properties:ChuckState}
\tabulinesep=3pt
\begin{tabu} to 6in {|l|l|} \everyrow{\hline}
\hline
\rowfont\bfseries {Property} & {Value} \\
\tabucline[1.5pt]{}
\texttt{type} & \texttt{CHUCK_STATE} \\
\texttt{result} & \texttt{LatchedStateEnum} \\
\end{tabu}
\end{table}
\FloatBarrier


 Enumerated \texttt{result} values for \texttt{ChuckState} are:
\begin{table}[ht]
\centering 
  \caption{\texttt{LatchedStateEnum} Enumeration}
  \label{enum:LatchedStateEnum}
\tabulinesep=3pt
\begin{tabu} to 6in {|l|X|} \everyrow{\hline}
\hline
\rowfont\bfseries {Name} & {Description} \\
\tabucline[1.5pt]{}
\texttt{OPEN} & A component is open to the point of a positive confirmation. \\
\texttt{CLOSED} & A component is closed to the point of a positive confirmation. \\
\texttt{UNLATCHED} & An intermediate position. \\
\end{tabu}
\end{table} 
\FloatBarrier
\FloatBarrier
\subsubsection[CloseChuck]{CloseChuck \\ {\small Subtype of Event}}
  \label{type:CloseChuck}

\FloatBarrier

Service to close a chuck.

\begin{table}[ht]
\centering 
  \caption{\texttt{Property of CloseChuck}}
  \label{properties:CloseChuck}
\tabulinesep=3pt
\begin{tabu} to 6in {|l|l|} \everyrow{\hline}
\hline
\rowfont\bfseries {Property} & {Value} \\
\tabucline[1.5pt]{}
\texttt{type} & \texttt{CLOSE_CHUCK} \\
\end{tabu}
\end{table}
\FloatBarrier

\FloatBarrier
\subsubsection[CloseDoor]{CloseDoor \\ {\small Subtype of Event}}
  \label{type:CloseDoor}

\FloatBarrier

Service to close a door.

\begin{table}[ht]
\centering 
  \caption{\texttt{Property of CloseDoor}}
  \label{properties:CloseDoor}
\tabulinesep=3pt
\begin{tabu} to 6in {|l|l|} \everyrow{\hline}
\hline
\rowfont\bfseries {Property} & {Value} \\
\tabucline[1.5pt]{}
\texttt{type} & \texttt{CLOSE_DOOR} \\
\end{tabu}
\end{table}
\FloatBarrier

\FloatBarrier
\subsubsection[Code]{Code \\ {\small Subtype of Event}}
  \label{type:Code}

\FloatBarrier

*DEPRECATED* in Version 1.1.

\begin{table}[ht]
\centering 
  \caption{\texttt{Property of Code}}
  \label{properties:Code}
\tabulinesep=3pt
\begin{tabu} to 6in {|l|l|} \everyrow{\hline}
\hline
\rowfont\bfseries {Property} & {Value} \\
\tabucline[1.5pt]{}
\texttt{type} & \texttt{CODE} \\
\end{tabu}
\end{table}
\FloatBarrier

\FloatBarrier
\subsubsection[CompositionState]{CompositionState \\ {\small Subtype of Event}}
  \label{type:CompositionState}

\FloatBarrier

An indication of the operating condition of a mechanism represented by a {model:Composition} type element.

\begin{table}[ht]
\centering 
  \caption{\texttt{Property of CompositionState}}
  \label{properties:CompositionState}
\tabulinesep=3pt
\begin{tabu} to 6in {|l|l|} \everyrow{\hline}
\hline
\rowfont\bfseries {Property} & {Value} \\
\tabucline[1.5pt]{}
\texttt{type} & \texttt{COMPOSITION_STATE} \\
\end{tabu}
\end{table}
\FloatBarrier

Subtypes of \texttt{CompositionState} are :

\begin{itemize}
\item \texttt{ACTION} : An indication of the operating state of a mechanism represented by a {model:Composition} type component.
 The operating state indicates whether the {model:Composition} element is activated or disabled. 
 The {term:Valid Data Value} *MUST* be {model:ACTIVE} or {model:INACTIVE}.

\item \texttt{LATERAL} : An indication of the position of a mechanism that may move in a lateral direction.   The mechanism is represented by a {model:Composition} type component. 
 The position information indicates whether the {model:Composition} element is positioned to the right, to the left, or is in transition.  
 The {term:Valid Data Value} *MUST* be {model:RIGHT}, {model:LEFT}, or {model:TRANSITIONING}.

\item \texttt{MOTION} : An indication of the open or closed state of a mechanism.   The mechanism is represented by a {model:Composition} type component. 
 The operating state indicates whether the state of the {model:Composition} element is open, closed, or unlatched.   
 The {term:Valid Data Value} *MUST* be {model:OPEN}, {model:UNLATCHED}, or {model:CLOSED}.

\item \texttt{SWITCHED} : An indication of the activation state of a mechanism represented by a {model:Composition} type component.
 The activation state indicates whether the {model:Composition} element is activated or not.
 The {term:Valid Data Value} *MUST* be {model:ON} or {model:OFF}.

\item \texttt{VERTICAL} : An indication of the position of a mechanism that may move in a vertical direction. The mechanism is represented by a {model:Composition} type component. 
 The position information indicates whether the {model:Composition} element is positioned to the top, to the bottom, or is in transition.  
 The {term:Valid Data Value} *MUST* be {model:UP}, {model:DOWN}, or {model:TRANSITIONING}.

\end{itemize}

\FloatBarrier
\subsubsection[ControllerMode]{ControllerMode \\ {\small Subtype of Event}}
  \label{type:ControllerMode}

\FloatBarrier

The current operating mode of the {model:Controller} component.

\begin{table}[ht]
\centering 
  \caption{\texttt{Property of ControllerMode}}
  \label{properties:ControllerMode}
\tabulinesep=3pt
\begin{tabu} to 6in {|l|l|} \everyrow{\hline}
\hline
\rowfont\bfseries {Property} & {Value} \\
\tabucline[1.5pt]{}
\texttt{type} & \texttt{CONTROLLER_MODE} \\
\texttt{result} & \texttt{ControllerModeEnum} \\
\end{tabu}
\end{table}
\FloatBarrier


 Enumerated \texttt{result} values for \texttt{ControllerMode} are:
\begin{table}[ht]
\centering 
  \caption{\texttt{ControllerModeEnum} Enumeration}
  \label{enum:ControllerModeEnum}
\tabulinesep=3pt
\begin{tabu} to 6in {|l|X|} \everyrow{\hline}
\hline
\rowfont\bfseries {Name} & {Description} \\
\tabucline[1.5pt]{}
\texttt{AUTOMATIC} & The {model:Controller} is configured to automatically execute a program. \\
\texttt{MANUAL} & Operations based on the instructions received from an external source. \\
\texttt{MANUAL_DATA_INPUT} & The operator can enter a series of operations for the controller to perform. \\
\texttt{SEMI_AUTOMATIC} & The controller  executes a single set of instructions from an active program and then stops until given a command to execute the next set of instructions. \\
\texttt{EDIT} & The controller is currently functioning as a programming device and is not capable of executing an active program. \\
\end{tabu}
\end{table} 
\FloatBarrier
\FloatBarrier
\subsubsection[ControllerModeOverride]{ControllerModeOverride \\ {\small Subtype of Event}}
  \label{type:ControllerModeOverride}

\FloatBarrier

A setting or operator selection that changes the behavior of a piece of equipment.

\begin{table}[ht]
\centering 
  \caption{\texttt{Property of ControllerModeOverride}}
  \label{properties:ControllerModeOverride}
\tabulinesep=3pt
\begin{tabu} to 6in {|l|l|} \everyrow{\hline}
\hline
\rowfont\bfseries {Property} & {Value} \\
\tabucline[1.5pt]{}
\texttt{type} & \texttt{CONTROLLER_MODE_OVERRIDE} \\
\texttt{result} & \texttt{OnOffEnum} \\
\end{tabu}
\end{table}
\FloatBarrier


 Enumerated \texttt{result} values for \texttt{ControllerModeOverride} are:
\begin{table}[ht]
\centering 
  \caption{\texttt{OnOffEnum} Enumeration}
  \label{enum:OnOffEnum}
\tabulinesep=3pt
\begin{tabu} to 6in {|l|X|} \everyrow{\hline}
\hline
\rowfont\bfseries {Name} & {Description} \\
\tabucline[1.5pt]{}
\texttt{ON} & On state or value. \\
\texttt{OFF} & Off state or value. \\
\end{tabu}
\end{table} 
\FloatBarrier
Subtypes of \texttt{ControllerModeOverride} are :

\begin{itemize}
\item \texttt{DRY_RUN} : A setting or operator selection used to execute a test mode to confirm the execution of machine functions. 
 The {term:Valid Data Value} *MUST* be {model:ON} or {model:OFF}. 
 When {model:DRY_RUN} is {model:ON}, the equipment performs all of its normal functions, except no part or product is produced.  If the equipment has a spindle, spindle operation is suspended.

\item \texttt{MACHINE_AXIS_LOCK} : A setting or operator selection that changes the behavior of the controller on a piece of equipment. 
 The {term:Valid Data Value} *MUST* be {model:ON} or {model:OFF}. 
 When {model:MACHINE_AXIS_LOCK} is {model:ON}, program execution continues normally, but no equipment motion occurs 

\item \texttt{OPTIONAL_STOP} : A setting or operator selection that changes the behavior of the controller on a piece of equipment. 
 The {term:Valid Data Value} *MUST* be {model:ON} or {model:OFF}.
 The program execution is stopped after a specific program block is executed when {model:OPTIONAL_STOP} is {model:ON}.    
 In the case of a G-Code program, a program {model:BLOCK} containing a M01 code designates the command for an {model:OPTIONAL_STOP}. 
 {model:EXECUTION} *MUST* change to {model:OPTIONAL_STOP} after a program block specifying an optional stop is executed and the {model:OPTIONAL_STOP} selection is {model:ON}.

\item \texttt{SINGLE_BLOCK} : A setting or operator selection that changes the behavior of the controller on a piece of equipment. 
 The {term:Valid Data Value} *MUST* be {model:ON} or {model:OFF}.
 Program execution is paused after each {model:BLOCK} of code is executed when {model:SINGLE_BLOCK} is {model:ON}.   
 When {model:SINGLE_BLOCK} is {model:ON}, {model:EXECUTION} *MUST* change to {model:INTERRUPTED} after completion of each {model:BLOCK} of code. 

\item \texttt{TOOL_CHANGE_STOP} : A setting or operator selection that changes the behavior of the controller on a piece of equipment. 
 The {term:Valid Data Value} *MUST* be {model:ON} or {model:OFF}. 
 Program execution is paused when a command is executed requesting a cutting tool to be changed. 
 {model:EXECUTION} *MUST* change to {model:INTERRUPTED} after completion of the command requesting a cutting tool to be changed and {model:TOOL_CHANGE_STOP} is {model:ON}.

\end{itemize}

\FloatBarrier
\subsubsection[CoupledAxes]{CoupledAxes \\ {\small Subtype of Event}}
  \label{type:CoupledAxes}

\FloatBarrier

Refers to the set of associated axes.

\begin{table}[ht]
\centering 
  \caption{\texttt{Property of CoupledAxes}}
  \label{properties:CoupledAxes}
\tabulinesep=3pt
\begin{tabu} to 6in {|l|l|} \everyrow{\hline}
\hline
\rowfont\bfseries {Property} & {Value} \\
\tabucline[1.5pt]{}
\texttt{type} & \texttt{COUPLED_AXES} \\
\end{tabu}
\end{table}
\FloatBarrier

\FloatBarrier
\subsubsection[DateCode]{DateCode \\ {\small Subtype of Event}}
  \label{type:DateCode}

\FloatBarrier

The time and date code associated with a material or other physical item.
  
 {model:DATE_CODE} *MUST* be reported in ISO 8601 format.

\begin{table}[ht]
\centering 
  \caption{\texttt{Property of DateCode}}
  \label{properties:DateCode}
\tabulinesep=3pt
\begin{tabu} to 6in {|l|l|} \everyrow{\hline}
\hline
\rowfont\bfseries {Property} & {Value} \\
\tabucline[1.5pt]{}
\texttt{type} & \texttt{DATE_CODE} \\
\end{tabu}
\end{table}
\FloatBarrier

Subtypes of \texttt{DateCode} are :

\begin{itemize}
\item \texttt{EXPIRATION} : The time and date code relating to the expiration or end of useful life for a material or other physical item.

\item \texttt{FIRST_USE} : The time and date code relating the first use of a material or other physical item.

\item \texttt{MANUFACTURE} : The time and date code relating to the production of a material or other physical item.

\end{itemize}

\FloatBarrier
\subsubsection[DeviceUuid]{DeviceUuid \\ {\small Subtype of Event}}
  \label{type:DeviceUuid}

\FloatBarrier

The identifier of another piece of equipment that is temporarily associated with a component of this piece of equipment to perform a particular function.
  
 The {term:Valid Data Value} *MUST* be a NMTOKEN XML type.

\begin{table}[ht]
\centering 
  \caption{\texttt{Property of DeviceUuid}}
  \label{properties:DeviceUuid}
\tabulinesep=3pt
\begin{tabu} to 6in {|l|l|} \everyrow{\hline}
\hline
\rowfont\bfseries {Property} & {Value} \\
\tabucline[1.5pt]{}
\texttt{type} & \texttt{DEVICE_UUID} \\
\end{tabu}
\end{table}
\FloatBarrier

\FloatBarrier
\subsubsection[Direction]{Direction \\ {\small Subtype of Event}}
  \label{type:Direction}

\FloatBarrier

The direction of motion.

\begin{table}[ht]
\centering 
  \caption{\texttt{Property of Direction}}
  \label{properties:Direction}
\tabulinesep=3pt
\begin{tabu} to 6in {|l|l|} \everyrow{\hline}
\hline
\rowfont\bfseries {Property} & {Value} \\
\tabucline[1.5pt]{}
\texttt{type} & \texttt{DIRECTION} \\
\end{tabu}
\end{table}
\FloatBarrier

Subtypes of \texttt{Direction} are :

\begin{itemize}
\item \texttt{LINEAR} : The direction of motion of a linear motion.

\item \texttt{ROTARY} : The rotational direction of a rotary motion using the right hand rule convention.
 The {term:Valid Data Value} *MUST* be {model:CLOCKWISE} or {model:COUNTER_CLOCKWISE}.

\end{itemize}

\FloatBarrier
\subsubsection[DoorState]{DoorState \\ {\small Subtype of Event}}
  \label{type:DoorState}

\FloatBarrier

The operational state of a {model:DOOR} type component or composition element.

\begin{table}[ht]
\centering 
  \caption{\texttt{Property of DoorState}}
  \label{properties:DoorState}
\tabulinesep=3pt
\begin{tabu} to 6in {|l|l|} \everyrow{\hline}
\hline
\rowfont\bfseries {Property} & {Value} \\
\tabucline[1.5pt]{}
\texttt{type} & \texttt{DOOR_STATE} \\
\texttt{result} & \texttt{LatchedStateEnum} \\
\end{tabu}
\end{table}
\FloatBarrier


 Enumerated \texttt{result} values for \texttt{DoorState} are:
\begin{table}[ht]
\centering 
  \caption{\texttt{LatchedStateEnum} Enumeration}
\tabulinesep=3pt
\begin{tabu} to 6in {|l|X|} \everyrow{\hline}
\hline
\rowfont\bfseries {Name} & {Description} \\
\tabucline[1.5pt]{}
\texttt{OPEN} & A component is open to the point of a positive confirmation. \\
\texttt{CLOSED} & A component is closed to the point of a positive confirmation. \\
\texttt{UNLATCHED} & An intermediate position. \\
\end{tabu}
\end{table} 
\FloatBarrier
\FloatBarrier
\subsubsection[EmergencyStop]{EmergencyStop \\ {\small Subtype of Event}}
  \label{type:EmergencyStop}

\FloatBarrier

The current state of the emergency stop signal for a piece of equipment, controller path, or any other component or subsystem of a piece of equipment.

\begin{table}[ht]
\centering 
  \caption{\texttt{Property of EmergencyStop}}
  \label{properties:EmergencyStop}
\tabulinesep=3pt
\begin{tabu} to 6in {|l|l|} \everyrow{\hline}
\hline
\rowfont\bfseries {Property} & {Value} \\
\tabucline[1.5pt]{}
\texttt{type} & \texttt{EMERGENCY_STOP} \\
\texttt{result} & \texttt{EmergencyStopEnum} \\
\end{tabu}
\end{table}
\FloatBarrier


 Enumerated \texttt{result} values for \texttt{EmergencyStop} are:
\begin{table}[ht]
\centering 
  \caption{\texttt{EmergencyStopEnum} Enumeration}
  \label{enum:EmergencyStopEnum}
\tabulinesep=3pt
\begin{tabu} to 6in {|l|X|} \everyrow{\hline}
\hline
\rowfont\bfseries {Name} & {Description} \\
\tabucline[1.5pt]{}
\texttt{ARMED} & The emergency stop circuit is complete and the piece of equipment, component, or composition element is allowed to operate.  \\
\texttt{TRIGGERED} & The operation of the piece of equipment, component, or composition element is inhibited. \\
\end{tabu}
\end{table} 
\FloatBarrier
\FloatBarrier
\subsubsection[EndOfBar]{EndOfBar \\ {\small Subtype of Event}}
  \label{type:EndOfBar}

\FloatBarrier

An indication of whether the end of a piece of bar stock being feed by a bar feeder has been reached.

\begin{table}[ht]
\centering 
  \caption{\texttt{Property of EndOfBar}}
  \label{properties:EndOfBar}
\tabulinesep=3pt
\begin{tabu} to 6in {|l|l|} \everyrow{\hline}
\hline
\rowfont\bfseries {Property} & {Value} \\
\tabucline[1.5pt]{}
\texttt{type} & \texttt{END_OF_BAR} \\
\texttt{result} & \texttt{YesNoEnum} \\
\end{tabu}
\end{table}
\FloatBarrier


 Enumerated \texttt{result} values for \texttt{EndOfBar} are:
\begin{table}[ht]
\centering 
  \caption{\texttt{YesNoEnum} Enumeration}
  \label{enum:YesNoEnum}
\tabulinesep=3pt
\begin{tabu} to 6in {|l|X|} \everyrow{\hline}
\hline
\rowfont\bfseries {Name} & {Description} \\
\tabucline[1.5pt]{}
\texttt{YES} & The {model:END_OF_BAR} has been reached. \\
\texttt{NO} & The {model:END_OF_BAR} has not been reached. \\
\end{tabu}
\end{table} 
\FloatBarrier
Subtypes of \texttt{EndOfBar} are :

\begin{itemize}
\item \texttt{AUXILIARY} : When multiple locations on a piece of bar stock are referenced as the indication for the {model:END_OF_BAR}, the additional location(s) *MUST* be designated as {model:AUXILIARY} indication(s) for the {model:END_OF_BAR}.  

\item \texttt{PRIMARY} : Specific applications *MAY* reference one or more locations on a piece of bar stock as the indication for the {model:END_OF_BAR}. 

The main or most important location *MUST* be designated as the {model:PRIMARY} indication for the {model:END_OF_BAR}.

If no {model:subType} is specified, {model:PRIMARY} *MUST* be the default {model:END_OF_BAR} indication.

\end{itemize}

\FloatBarrier
\subsubsection[EquipmentMode]{EquipmentMode \\ {\small Subtype of Event}}
  \label{type:EquipmentMode}

\FloatBarrier

An indication that a piece of equipment, or a sub-part of a piece of equipment, is performing specific types of activities.

\begin{table}[ht]
\centering 
  \caption{\texttt{Property of EquipmentMode}}
  \label{properties:EquipmentMode}
\tabulinesep=3pt
\begin{tabu} to 6in {|l|l|} \everyrow{\hline}
\hline
\rowfont\bfseries {Property} & {Value} \\
\tabucline[1.5pt]{}
\texttt{type} & \texttt{EQUIPMENT_MODE} \\
\texttt{result} & \texttt{OnOffEnum} \\
\end{tabu}
\end{table}
\FloatBarrier


 Enumerated \texttt{result} values for \texttt{EquipmentMode} are:
\begin{table}[ht]
\centering 
  \caption{\texttt{OnOffEnum} Enumeration}
\tabulinesep=3pt
\begin{tabu} to 6in {|l|X|} \everyrow{\hline}
\hline
\rowfont\bfseries {Name} & {Description} \\
\tabucline[1.5pt]{}
\texttt{ON} & On state or value. \\
\texttt{OFF} & Off state or value. \\
\end{tabu}
\end{table} 
\FloatBarrier
Subtypes of \texttt{EquipmentMode} are :

\begin{itemize}
\item \texttt{DELAY} : A piece of equipment waiting for an event or an action to occur.

\item \texttt{LOADED} : Subparts of a piece of equipment are under load.

\item \texttt{OPERATING} : A piece of equipment are powered or performing any activity.

\item \texttt{POWERED} : Primary  power is  applied  to the  piece  of  equipment and,  as  a minimum, the controller or logic portion of the piece of equipment is powered and functioning or components that are required to remain on are powered.

\item \texttt{WORKING} : A piece of equipment performing any activity, the equipment is active and performing a function under load or not.

\end{itemize}

\FloatBarrier
\subsubsection[Execution]{Execution \\ {\small Subtype of Event}}
  \label{type:Execution}

\FloatBarrier

The execution status of the {model:Controller}.

\begin{table}[ht]
\centering 
  \caption{\texttt{Property of Execution}}
  \label{properties:Execution}
\tabulinesep=3pt
\begin{tabu} to 6in {|l|l|} \everyrow{\hline}
\hline
\rowfont\bfseries {Property} & {Value} \\
\tabucline[1.5pt]{}
\texttt{type} & \texttt{EXECUTION} \\
\texttt{result} & \texttt{ExecutionEnum} \\
\end{tabu}
\end{table}
\FloatBarrier


 Enumerated \texttt{result} values for \texttt{Execution} are:
\begin{table}[ht]
\centering 
  \caption{\texttt{ExecutionEnum} Enumeration}
  \label{enum:ExecutionEnum}
\tabulinesep=3pt
\begin{tabu} to 6in {|l|X|} \everyrow{\hline}
\hline
\rowfont\bfseries {Name} & {Description} \\
\tabucline[1.5pt]{}
\texttt{READY} & A component is ready to engage. \\
\texttt{ACTIVE} & The value of the {term:Data Entity} that is engaging. \\
\texttt{INTERRUPTED} & The action of a {model:Component} has been suspended due to an external signal. \\
\texttt{FEED_HOLD} & Motion of a {model:Component} has been commanded to stop at its current position. \\
\texttt{STOPPED} & The component is stopped. \\
\texttt{OPTIONAL_STOP} & The controllers program has been intentionally stopped \\
\texttt{PROGRAM_STOPPED} & The execution of the {model:Controller}'s program has been stopped by a command from within the program. \\
\texttt{PROGRAM_COMPLETED} & The execution of the controllers program has been stopped by a command from within the program. \\
\end{tabu}
\end{table} 
\FloatBarrier
\FloatBarrier
\subsubsection[FunctionalMode]{FunctionalMode \\ {\small Subtype of Event}}
  \label{type:FunctionalMode}

\FloatBarrier

The current intended production status of the device or component.

\begin{table}[ht]
\centering 
  \caption{\texttt{Property of FunctionalMode}}
  \label{properties:FunctionalMode}
\tabulinesep=3pt
\begin{tabu} to 6in {|l|l|} \everyrow{\hline}
\hline
\rowfont\bfseries {Property} & {Value} \\
\tabucline[1.5pt]{}
\texttt{type} & \texttt{FUNCTIONAL_MODE} \\
\texttt{result} & \texttt{FunctionalModeEnum} \\
\end{tabu}
\end{table}
\FloatBarrier


 Enumerated \texttt{result} values for \texttt{FunctionalMode} are:
\begin{table}[ht]
\centering 
  \caption{\texttt{FunctionalModeEnum} Enumeration}
  \label{enum:FunctionalModeEnum}
\tabulinesep=3pt
\begin{tabu} to 6in {|l|X|} \everyrow{\hline}
\hline
\rowfont\bfseries {Name} & {Description} \\
\tabucline[1.5pt]{}
\texttt{PRODUCTION} & A {term:Structural Element} is currently producing product. \\
\texttt{SETUP} & A {term:Structural Element} is being prepared or modified to begin production of product. \\
\texttt{TEARDOWN} & Typically, a {term:Structural Element} has completed the production of a product and is being modified or returned to a neutral state such that it may then be prepared to begin production of a different product. \\
\texttt{MAINTENANCE} & Action related to maintenance on the piece of equipment. \\
\texttt{PROCESS_DEVELOPMENT} & A {term:Structural Element} is being used to prove-out a new process. \\
\end{tabu}
\end{table} 
\FloatBarrier
\FloatBarrier
\subsubsection[Hardness]{Hardness \\ {\small Subtype of Event}}
  \label{type:Hardness}

\FloatBarrier

The measurement of the hardness of a material.

\begin{table}[ht]
\centering 
  \caption{\texttt{Property of Hardness}}
  \label{properties:Hardness}
\tabulinesep=3pt
\begin{tabu} to 6in {|l|l|} \everyrow{\hline}
\hline
\rowfont\bfseries {Property} & {Value} \\
\tabucline[1.5pt]{}
\texttt{type} & \texttt{HARDNESS} \\
\end{tabu}
\end{table}
\FloatBarrier

Subtypes of \texttt{Hardness} are :

\begin{itemize}
\item \texttt{BRINELL} : A scale to measure the resistance to deformation of a surface.

\item \texttt{LEEB} : A scale to measure the elasticity of a surface.

\item \texttt{MOHS} : A scale to measure the resistance to scratching of a surface.

\item \texttt{ROCKWELL} : A scale to measure the resistance to deformation of a surface.

\item \texttt{SHORE} : A scale to measure the resistance to deformation of a surface.

\item \texttt{VICKERS} : A scale to measure the resistance to deformation of a surface.

\end{itemize}

\FloatBarrier
\subsubsection[InterfaceState]{InterfaceState \\ {\small Subtype of Event}}
  \label{type:InterfaceState}

\FloatBarrier

An indication of the operational state of an {model:Interface} component.

\begin{table}[ht]
\centering 
  \caption{\texttt{Property of InterfaceState}}
  \label{properties:InterfaceState}
\tabulinesep=3pt
\begin{tabu} to 6in {|l|l|} \everyrow{\hline}
\hline
\rowfont\bfseries {Property} & {Value} \\
\tabucline[1.5pt]{}
\texttt{type} & \texttt{INTERFACE_STATE} \\
\texttt{result} & \texttt{EnabledStateEnum} \\
\end{tabu}
\end{table}
\FloatBarrier


 Enumerated \texttt{result} values for \texttt{InterfaceState} are:
\begin{table}[ht]
\centering 
  \caption{\texttt{EnabledStateEnum} Enumeration}
  \label{enum:EnabledStateEnum}
\tabulinesep=3pt
\begin{tabu} to 6in {|l|X|} \everyrow{\hline}
\hline
\rowfont\bfseries {Name} & {Description} \\
\tabucline[1.5pt]{}
\texttt{ENABLED} & A component is currently operational and performing as expected. \\
\texttt{DISABLED} & A component is currently not operational. \\
\end{tabu}
\end{table} 
\FloatBarrier
\FloatBarrier
\subsubsection[Line]{Line \\ {\small Subtype of Event}}
  \label{type:Line}

\FloatBarrier

*DEPRECATED* in Version 1.4.0.

\begin{table}[ht]
\centering 
  \caption{\texttt{Property of Line}}
  \label{properties:Line}
\tabulinesep=3pt
\begin{tabu} to 6in {|l|l|} \everyrow{\hline}
\hline
\rowfont\bfseries {Property} & {Value} \\
\tabucline[1.5pt]{}
\texttt{type} & \texttt{LINE} \\
\end{tabu}
\end{table}
\FloatBarrier

Subtypes of \texttt{Line} are :

\begin{itemize}
\item \texttt{MAXIMUM} : Maximum value of a data entity or attribute.

\item \texttt{MINIMUM} : The minimum value of a data entity or attribute.

\end{itemize}

\FloatBarrier
\subsubsection[LineLabel]{LineLabel \\ {\small Subtype of Event}}
  \label{type:LineLabel}

\FloatBarrier

An optional identifier for a {model:BLOCK} of code in a {model:PROGRAM}.

\begin{table}[ht]
\centering 
  \caption{\texttt{Property of LineLabel}}
  \label{properties:LineLabel}
\tabulinesep=3pt
\begin{tabu} to 6in {|l|l|} \everyrow{\hline}
\hline
\rowfont\bfseries {Property} & {Value} \\
\tabucline[1.5pt]{}
\texttt{type} & \texttt{LINE_LABEL} \\
\end{tabu}
\end{table}
\FloatBarrier

\FloatBarrier
\subsubsection[LineNumber]{LineNumber \\ {\small Subtype of Event}}
  \label{type:LineNumber}

\FloatBarrier

A reference to the position of a block of program code within a control program.

\begin{table}[ht]
\centering 
  \caption{\texttt{Property of LineNumber}}
  \label{properties:LineNumber}
\tabulinesep=3pt
\begin{tabu} to 6in {|l|l|} \everyrow{\hline}
\hline
\rowfont\bfseries {Property} & {Value} \\
\tabucline[1.5pt]{}
\texttt{type} & \texttt{LINE_NUMBER} \\
\end{tabu}
\end{table}
\FloatBarrier

Subtypes of \texttt{LineNumber} are :

\begin{itemize}
\item \texttt{ABSOLUTE} : The position of a block of program code relative to the beginning of the control program.

\item \texttt{INCREMENTAL} : The position of a block of program code relative to the occurrence of the last {model:LINE_LABEL} encountered in the control program.

\end{itemize}

\FloatBarrier
\subsubsection[Material]{Material \\ {\small Subtype of Event}}
  \label{type:Material}

\FloatBarrier

The identifier of a material used or consumed in the manufacturing process.

\begin{table}[ht]
\centering 
  \caption{\texttt{Property of Material}}
  \label{properties:Material}
\tabulinesep=3pt
\begin{tabu} to 6in {|l|l|} \everyrow{\hline}
\hline
\rowfont\bfseries {Property} & {Value} \\
\tabucline[1.5pt]{}
\texttt{type} & \texttt{MATERIAL} \\
\end{tabu}
\end{table}
\FloatBarrier

\FloatBarrier
\subsubsection[MaterialChange]{MaterialChange \\ {\small Subtype of Event}}
  \label{type:MaterialChange}

\FloatBarrier

Service to change the type of material or product being loaded or fed to a piece of equipment.

\begin{table}[ht]
\centering 
  \caption{\texttt{Property of MaterialChange}}
  \label{properties:MaterialChange}
\tabulinesep=3pt
\begin{tabu} to 6in {|l|l|} \everyrow{\hline}
\hline
\rowfont\bfseries {Property} & {Value} \\
\tabucline[1.5pt]{}
\texttt{type} & \texttt{MATERIAL_CHANGE} \\
\end{tabu}
\end{table}
\FloatBarrier

\FloatBarrier
\subsubsection[MaterialFeed]{MaterialFeed \\ {\small Subtype of Event}}
  \label{type:MaterialFeed}

\FloatBarrier

Service to advance material or feed product to a piece of equipment from a continuous or bulk source.

\begin{table}[ht]
\centering 
  \caption{\texttt{Property of MaterialFeed}}
  \label{properties:MaterialFeed}
\tabulinesep=3pt
\begin{tabu} to 6in {|l|l|} \everyrow{\hline}
\hline
\rowfont\bfseries {Property} & {Value} \\
\tabucline[1.5pt]{}
\texttt{type} & \texttt{MATERIAL_FEED} \\
\end{tabu}
\end{table}
\FloatBarrier

\FloatBarrier
\subsubsection[MaterialLayer]{MaterialLayer \\ {\small Subtype of Event}}
  \label{type:MaterialLayer}

\FloatBarrier

Identifies the layers of material applied to a part or product as part of an additive manufacturing process.
  
 The {term:Valid Data Value} *MUST* be an integer.

\begin{table}[ht]
\centering 
  \caption{\texttt{Property of MaterialLayer}}
  \label{properties:MaterialLayer}
\tabulinesep=3pt
\begin{tabu} to 6in {|l|l|} \everyrow{\hline}
\hline
\rowfont\bfseries {Property} & {Value} \\
\tabucline[1.5pt]{}
\texttt{type} & \texttt{MATERIAL_LAYER} \\
\end{tabu}
\end{table}
\FloatBarrier

Subtypes of \texttt{MaterialLayer} are :

\begin{itemize}
\item \texttt{ACTUAL} : The measured value of the data item type given by a sensor or encoder.

\item \texttt{TARGET} : The desired measure or count for a data item value.

\end{itemize}

\FloatBarrier
\subsubsection[MaterialLoad]{MaterialLoad \\ {\small Subtype of Event}}
  \label{type:MaterialLoad}

\FloatBarrier

Service to load a piece of material or product.

\begin{table}[ht]
\centering 
  \caption{\texttt{Property of MaterialLoad}}
  \label{properties:MaterialLoad}
\tabulinesep=3pt
\begin{tabu} to 6in {|l|l|} \everyrow{\hline}
\hline
\rowfont\bfseries {Property} & {Value} \\
\tabucline[1.5pt]{}
\texttt{type} & \texttt{MATERIAL_LOAD} \\
\end{tabu}
\end{table}
\FloatBarrier

\FloatBarrier
\subsubsection[MaterialRetract]{MaterialRetract \\ {\small Subtype of Event}}
  \label{type:MaterialRetract}

\FloatBarrier

Service to remove or retract material or product.

\begin{table}[ht]
\centering 
  \caption{\texttt{Property of MaterialRetract}}
  \label{properties:MaterialRetract}
\tabulinesep=3pt
\begin{tabu} to 6in {|l|l|} \everyrow{\hline}
\hline
\rowfont\bfseries {Property} & {Value} \\
\tabucline[1.5pt]{}
\texttt{type} & \texttt{MATERIAL_RETRACT} \\
\end{tabu}
\end{table}
\FloatBarrier

\FloatBarrier
\subsubsection[MaterialUnload]{MaterialUnload \\ {\small Subtype of Event}}
  \label{type:MaterialUnload}

\FloatBarrier

Service to unload a piece of material or product.

\begin{table}[ht]
\centering 
  \caption{\texttt{Property of MaterialUnload}}
  \label{properties:MaterialUnload}
\tabulinesep=3pt
\begin{tabu} to 6in {|l|l|} \everyrow{\hline}
\hline
\rowfont\bfseries {Property} & {Value} \\
\tabucline[1.5pt]{}
\texttt{type} & \texttt{MATERIAL_UNLOAD} \\
\end{tabu}
\end{table}
\FloatBarrier

\FloatBarrier
\subsubsection[Message]{Message \\ {\small Subtype of Event}}
  \label{type:Message}

\FloatBarrier

Any text string of information to be transferred from a piece of equipment to a client software application.

\begin{table}[ht]
\centering 
  \caption{\texttt{Property of Message}}
  \label{properties:Message}
\tabulinesep=3pt
\begin{tabu} to 6in {|l|l|} \everyrow{\hline}
\hline
\rowfont\bfseries {Property} & {Value} \\
\tabucline[1.5pt]{}
\texttt{type} & \texttt{MESSAGE} \\
\end{tabu}
\end{table}
\FloatBarrier

\FloatBarrier
\subsubsection[OpenChuck]{OpenChuck \\ {\small Subtype of Event}}
  \label{type:OpenChuck}

\FloatBarrier

Service to open a chuck.

\begin{table}[ht]
\centering 
  \caption{\texttt{Property of OpenChuck}}
  \label{properties:OpenChuck}
\tabulinesep=3pt
\begin{tabu} to 6in {|l|l|} \everyrow{\hline}
\hline
\rowfont\bfseries {Property} & {Value} \\
\tabucline[1.5pt]{}
\texttt{type} & \texttt{OPEN_CHUCK} \\
\end{tabu}
\end{table}
\FloatBarrier

\FloatBarrier
\subsubsection[OpenDoor]{OpenDoor \\ {\small Subtype of Event}}
  \label{type:OpenDoor}

\FloatBarrier

Service to open a door.

\begin{table}[ht]
\centering 
  \caption{\texttt{Property of OpenDoor}}
  \label{properties:OpenDoor}
\tabulinesep=3pt
\begin{tabu} to 6in {|l|l|} \everyrow{\hline}
\hline
\rowfont\bfseries {Property} & {Value} \\
\tabucline[1.5pt]{}
\texttt{type} & \texttt{OPEN_DOOR} \\
\end{tabu}
\end{table}
\FloatBarrier

\FloatBarrier
\subsubsection[OperatorId]{OperatorId \\ {\small Subtype of Event}}
  \label{type:OperatorId}

\FloatBarrier

The identifier of the person currently responsible for operating the piece of equipment.

\begin{table}[ht]
\centering 
  \caption{\texttt{Property of OperatorId}}
  \label{properties:OperatorId}
\tabulinesep=3pt
\begin{tabu} to 6in {|l|l|} \everyrow{\hline}
\hline
\rowfont\bfseries {Property} & {Value} \\
\tabucline[1.5pt]{}
\texttt{type} & \texttt{OPERATOR_ID} \\
\end{tabu}
\end{table}
\FloatBarrier

\FloatBarrier
\subsubsection[PalletId]{PalletId \\ {\small Subtype of Event}}
  \label{type:PalletId}

\FloatBarrier

The identifier for a pallet.

\begin{table}[ht]
\centering 
  \caption{\texttt{Property of PalletId}}
  \label{properties:PalletId}
\tabulinesep=3pt
\begin{tabu} to 6in {|l|l|} \everyrow{\hline}
\hline
\rowfont\bfseries {Property} & {Value} \\
\tabucline[1.5pt]{}
\texttt{type} & \texttt{PALLET_ID} \\
\end{tabu}
\end{table}
\FloatBarrier

\FloatBarrier
\subsubsection[PartChange]{PartChange \\ {\small Subtype of Event}}
  \label{type:PartChange}

\FloatBarrier

Service to change the part or product associated with a piece of equipment to a different part or product.

\begin{table}[ht]
\centering 
  \caption{\texttt{Property of PartChange}}
  \label{properties:PartChange}
\tabulinesep=3pt
\begin{tabu} to 6in {|l|l|} \everyrow{\hline}
\hline
\rowfont\bfseries {Property} & {Value} \\
\tabucline[1.5pt]{}
\texttt{type} & \texttt{PART_CHANGE} \\
\end{tabu}
\end{table}
\FloatBarrier

\FloatBarrier
\subsubsection[PartCount]{PartCount \\ {\small Subtype of Event}}
  \label{type:PartCount}

\FloatBarrier

The count of parts produced.

\begin{table}[ht]
\centering 
  \caption{\texttt{Property of PartCount}}
  \label{properties:PartCount}
\tabulinesep=3pt
\begin{tabu} to 6in {|l|l|} \everyrow{\hline}
\hline
\rowfont\bfseries {Property} & {Value} \\
\tabucline[1.5pt]{}
\texttt{type} & \texttt{PART_COUNT} \\
\end{tabu}
\end{table}
\FloatBarrier

Subtypes of \texttt{PartCount} are :

\begin{itemize}
\item \texttt{ALL} : The count of all the parts produced.  If the subtype is not given, this is the default.

\item \texttt{BAD} : Indicates the count of incorrect parts produced.

\item \texttt{GOOD} : Indicates the count of correct parts made.

\item \texttt{REMAINING} : Remaining measure of an object or an action.

\item \texttt{TARGET} : The desired measure or count for a data item value.

\end{itemize}

\FloatBarrier
\subsubsection[PartDetect]{PartDetect \\ {\small Subtype of Event}}
  \label{type:PartDetect}

\FloatBarrier

An indication designating whether a part or work piece has been detected or is present.


\begin{table}[ht]
\centering 
  \caption{\texttt{Property of PartDetect}}
  \label{properties:PartDetect}
\tabulinesep=3pt
\begin{tabu} to 6in {|l|l|} \everyrow{\hline}
\hline
\rowfont\bfseries {Property} & {Value} \\
\tabucline[1.5pt]{}
\texttt{type} & \texttt{PART_DETECT} \\
\texttt{result} & \texttt{PartDetectEnum} \\
\end{tabu}
\end{table}
\FloatBarrier


 Enumerated \texttt{result} values for \texttt{PartDetect} are:
\begin{table}[ht]
\centering 
  \caption{\texttt{PartDetectEnum} Enumeration}
  \label{enum:PartDetectEnum}
\tabulinesep=3pt
\begin{tabu} to 6in {|l|X|} \everyrow{\hline}
\hline
\rowfont\bfseries {Name} & {Description} \\
\tabucline[1.5pt]{}
\texttt{PRESENT} & If a part or work piece has been detected or is present. \\
\texttt{NOT_PRESENT} & If a part or work piece is not detected or is not present. \\
\end{tabu}
\end{table} 
\FloatBarrier
\FloatBarrier
\subsubsection[PartId]{PartId \\ {\small Subtype of Event}}
  \label{type:PartId}

\FloatBarrier

An identifier of a part in a manufacturing operation.

\begin{table}[ht]
\centering 
  \caption{\texttt{Property of PartId}}
  \label{properties:PartId}
\tabulinesep=3pt
\begin{tabu} to 6in {|l|l|} \everyrow{\hline}
\hline
\rowfont\bfseries {Property} & {Value} \\
\tabucline[1.5pt]{}
\texttt{type} & \texttt{PART_ID} \\
\end{tabu}
\end{table}
\FloatBarrier

\FloatBarrier
\subsubsection[PartNumber]{PartNumber \\ {\small Subtype of Event}}
  \label{type:PartNumber}

\FloatBarrier

An identifier of a part or product moving through the manufacturing process. 
 The {term:Valid Data Value} *MUST* be a text string. 

\begin{table}[ht]
\centering 
  \caption{\texttt{Property of PartNumber}}
  \label{properties:PartNumber}
\tabulinesep=3pt
\begin{tabu} to 6in {|l|l|} \everyrow{\hline}
\hline
\rowfont\bfseries {Property} & {Value} \\
\tabucline[1.5pt]{}
\texttt{type} & \texttt{PART_NUMBER} \\
\end{tabu}
\end{table}
\FloatBarrier

\FloatBarrier
\subsubsection[PathFeedrateOverride]{PathFeedrateOverride \\ {\small Subtype of Event}}
  \label{type:PathFeedrateOverride}

\FloatBarrier

The value of a signal or calculation issued to adjust the feedrate for the axes associated with a {model:Path} component that may represent a single axis or the coordinated movement of multiple axes.

\begin{table}[ht]
\centering 
  \caption{\texttt{Property of PathFeedrateOverride}}
  \label{properties:PathFeedrateOverride}
\tabulinesep=3pt
\begin{tabu} to 6in {|l|l|} \everyrow{\hline}
\hline
\rowfont\bfseries {Property} & {Value} \\
\tabucline[1.5pt]{}
\texttt{type} & \texttt{PATH_FEEDRATE_OVERRIDE} \\
\end{tabu}
\end{table}
\FloatBarrier

Subtypes of \texttt{PathFeedrateOverride} are :

\begin{itemize}
\item \texttt{JOG} : The feedrate specified by a logic or motion program, by a pre-set value, or set by a switch as the feedrate for the {model:Axes}. 

\item \texttt{PROGRAMMED} : The value of a signal or calculation specified by a logic or motion program or set by a switch.

\item \texttt{RAPID} : The value of a signal or calculation issued to adjust the feedrate of a component or composition that is operating in a rapid positioning mode.

\end{itemize}

\FloatBarrier
\subsubsection[PathMode]{PathMode \\ {\small Subtype of Event}}
  \label{type:PathMode}

\FloatBarrier

Describes the operational relationship between a {model:Path} {term:Structural Element} and another {model:Path} {term:Structural Element} for pieces of equipment comprised of multiple logical groupings of controlled axes or other logical operations.

\begin{table}[ht]
\centering 
  \caption{\texttt{Property of PathMode}}
  \label{properties:PathMode}
\tabulinesep=3pt
\begin{tabu} to 6in {|l|l|} \everyrow{\hline}
\hline
\rowfont\bfseries {Property} & {Value} \\
\tabucline[1.5pt]{}
\texttt{type} & \texttt{PATH_MODE} \\
\texttt{result} & \texttt{PathModeEnum} \\
\end{tabu}
\end{table}
\FloatBarrier


 Enumerated \texttt{result} values for \texttt{PathMode} are:
\begin{table}[ht]
\centering 
  \caption{\texttt{PathModeEnum} Enumeration}
  \label{enum:PathModeEnum}
\tabulinesep=3pt
\begin{tabu} to 6in {|l|X|} \everyrow{\hline}
\hline
\rowfont\bfseries {Name} & {Description} \\
\tabucline[1.5pt]{}
\texttt{INDEPENDENT} & The path is operating independently and without the influence of another path. \\
\texttt{MASTER} & It provides information or state values that influences the operation of other {model:DataItem} of similar type. \\
\texttt{SYNCHRONOUS} & Physical or logical parts which are not physically connected to each other but are operating together. \\
\texttt{MIRROR} & The axes associated with the path are mirroring the motion of the {model:MASTER} path. \\
\end{tabu}
\end{table} 
\FloatBarrier
\FloatBarrier
\subsubsection[PowerState]{PowerState \\ {\small Subtype of Event}}
  \label{type:PowerState}

\FloatBarrier

The indication of the status of the source of energy for a {term:Structural Element} to allow it to perform its intended function or the state of an enabling signal providing permission for the {term:Structural Element} to perform its functions.

\begin{table}[ht]
\centering 
  \caption{\texttt{Property of PowerState}}
  \label{properties:PowerState}
\tabulinesep=3pt
\begin{tabu} to 6in {|l|l|} \everyrow{\hline}
\hline
\rowfont\bfseries {Property} & {Value} \\
\tabucline[1.5pt]{}
\texttt{type} & \texttt{POWER_STATE} \\
\texttt{result} & \texttt{OnOffEnum} \\
\end{tabu}
\end{table}
\FloatBarrier


 Enumerated \texttt{result} values for \texttt{PowerState} are:
\begin{table}[ht]
\centering 
  \caption{\texttt{OnOffEnum} Enumeration}
\tabulinesep=3pt
\begin{tabu} to 6in {|l|X|} \everyrow{\hline}
\hline
\rowfont\bfseries {Name} & {Description} \\
\tabucline[1.5pt]{}
\texttt{ON} & On state or value. \\
\texttt{OFF} & Off state or value. \\
\end{tabu}
\end{table} 
\FloatBarrier
Subtypes of \texttt{PowerState} are :

\begin{itemize}
\item \texttt{CONTROL} : The state of the enabling signal or control logic that enables or disables the function or operation of the {term:Structural Element}.

\item \texttt{LINE} : The state of the power source for the {term:Structural Element}.

\end{itemize}

\FloatBarrier
\subsubsection[PowerStatus]{PowerStatus \\ {\small Subtype of Event}}
  \label{type:PowerStatus}

\FloatBarrier

*DEPRECATED* in Version 1.1.0.

\begin{table}[ht]
\centering 
  \caption{\texttt{Property of PowerStatus}}
  \label{properties:PowerStatus}
\tabulinesep=3pt
\begin{tabu} to 6in {|l|l|} \everyrow{\hline}
\hline
\rowfont\bfseries {Property} & {Value} \\
\tabucline[1.5pt]{}
\texttt{type} & \texttt{POWER_STATUS} \\
\end{tabu}
\end{table}
\FloatBarrier

\FloatBarrier
\subsubsection[ProcessTime]{ProcessTime \\ {\small Subtype of Event}}
  \label{type:ProcessTime}

\FloatBarrier

The time and date associated with an activity or event.
  
 {model:PROCESS_TIME} *MUST* be reported in ISO 8601 format.

\begin{table}[ht]
\centering 
  \caption{\texttt{Property of ProcessTime}}
  \label{properties:ProcessTime}
\tabulinesep=3pt
\begin{tabu} to 6in {|l|l|} \everyrow{\hline}
\hline
\rowfont\bfseries {Property} & {Value} \\
\tabucline[1.5pt]{}
\texttt{type} & \texttt{PROCESS_TIME} \\
\end{tabu}
\end{table}
\FloatBarrier

Subtypes of \texttt{ProcessTime} are :

\begin{itemize}
\item \texttt{COMPLETE} : Completion of an action.

\item \texttt{START} : The time and date associated with the beginning of an activity or event.

\item \texttt{TARGET_COMPLETION} : The projected time and date associated with the end or completion of an activity or event.

\end{itemize}

\FloatBarrier
\subsubsection[Program]{Program \\ {\small Subtype of Event}}
  \label{type:Program}

\FloatBarrier

The name of the logic or motion program being executed by the {model:Controller} component.

\begin{table}[ht]
\centering 
  \caption{\texttt{Property of Program}}
  \label{properties:Program}
\tabulinesep=3pt
\begin{tabu} to 6in {|l|l|} \everyrow{\hline}
\hline
\rowfont\bfseries {Property} & {Value} \\
\tabucline[1.5pt]{}
\texttt{type} & \texttt{PROGRAM} \\
\end{tabu}
\end{table}
\FloatBarrier

\FloatBarrier
\subsubsection[ProgramComment]{ProgramComment \\ {\small Subtype of Event}}
  \label{type:ProgramComment}

\FloatBarrier

A comment or non-executable statement in the control program.
 The {term:Valid Data Value} *MUST* be a text string.

\begin{table}[ht]
\centering 
  \caption{\texttt{Property of ProgramComment}}
  \label{properties:ProgramComment}
\tabulinesep=3pt
\begin{tabu} to 6in {|l|l|} \everyrow{\hline}
\hline
\rowfont\bfseries {Property} & {Value} \\
\tabucline[1.5pt]{}
\texttt{type} & \texttt{PROGRAM_COMMENT} \\
\end{tabu}
\end{table}
\FloatBarrier

\FloatBarrier
\subsubsection[ProgramEdit]{ProgramEdit \\ {\small Subtype of Event}}
  \label{type:ProgramEdit}

\FloatBarrier

An indication of the status of the {model:Controller} components program editing mode. 
 On many controls, a program can be edited while another program is currently being executed.

\begin{table}[ht]
\centering 
  \caption{\texttt{Property of ProgramEdit}}
  \label{properties:ProgramEdit}
\tabulinesep=3pt
\begin{tabu} to 6in {|l|l|} \everyrow{\hline}
\hline
\rowfont\bfseries {Property} & {Value} \\
\tabucline[1.5pt]{}
\texttt{type} & \texttt{PROGRAM_EDIT} \\
\texttt{result} & \texttt{ActiveStateEnum} \\
\end{tabu}
\end{table}
\FloatBarrier


 Enumerated \texttt{result} values for \texttt{ProgramEdit} are:
\begin{table}[ht]
\centering 
  \caption{\texttt{ActiveStateEnum} Enumeration}
  \label{enum:ActiveStateEnum}
\tabulinesep=3pt
\begin{tabu} to 6in {|l|X|} \everyrow{\hline}
\hline
\rowfont\bfseries {Name} & {Description} \\
\tabucline[1.5pt]{}
\texttt{ACTIVE} & The value of the {term:Data Entity} that is engaging. \\
\texttt{READY} & A component is ready to engage. \\
\texttt{NOT_READY} & A component is not ready to engage. \\
\end{tabu}
\end{table} 
\FloatBarrier
\FloatBarrier
\subsubsection[ProgramEditName]{ProgramEditName \\ {\small Subtype of Event}}
  \label{type:ProgramEditName}

\FloatBarrier

The name of the program being edited. 
 This is used in conjunction with {model:PROGRAM_EDIT} when in {model:ACTIVE} state. 
 The {term:Valid Data Value} *MUST* be a text string.

\begin{table}[ht]
\centering 
  \caption{\texttt{Property of ProgramEditName}}
  \label{properties:ProgramEditName}
\tabulinesep=3pt
\begin{tabu} to 6in {|l|l|} \everyrow{\hline}
\hline
\rowfont\bfseries {Property} & {Value} \\
\tabucline[1.5pt]{}
\texttt{type} & \texttt{PROGRAM_EDIT_NAME} \\
\end{tabu}
\end{table}
\FloatBarrier

\FloatBarrier
\subsubsection[ProgramHeader]{ProgramHeader \\ {\small Subtype of Event}}
  \label{type:ProgramHeader}

\FloatBarrier

The non-executable header section of the control program.

\begin{table}[ht]
\centering 
  \caption{\texttt{Property of ProgramHeader}}
  \label{properties:ProgramHeader}
\tabulinesep=3pt
\begin{tabu} to 6in {|l|l|} \everyrow{\hline}
\hline
\rowfont\bfseries {Property} & {Value} \\
\tabucline[1.5pt]{}
\texttt{type} & \texttt{PROGRAM_HEADER} \\
\end{tabu}
\end{table}
\FloatBarrier

\FloatBarrier
\subsubsection[ProgramLocation]{ProgramLocation \\ {\small Subtype of Event}}
  \label{type:ProgramLocation}

\FloatBarrier

The Uniform Resource Identifier (URI) for the source file associated with {model:PROGRAM}.

\begin{table}[ht]
\centering 
  \caption{\texttt{Property of ProgramLocation}}
  \label{properties:ProgramLocation}
\tabulinesep=3pt
\begin{tabu} to 6in {|l|l|} \everyrow{\hline}
\hline
\rowfont\bfseries {Property} & {Value} \\
\tabucline[1.5pt]{}
\texttt{type} & \texttt{PROGRAM_LOCATION} \\
\end{tabu}
\end{table}
\FloatBarrier

Subtypes of \texttt{ProgramLocation} are :

\begin{itemize}
\item \texttt{ACTIVE} : The value of the {term:Data Entity} that is engaging.

\item \texttt{MAIN} : The identity of the primary logic or motion program currently being executed. It is the starting nest level in a call structure and may contain calls to sub programs.

\item \texttt{SCHEDULE} : The identity of a control program that is used to specify the order of execution of other programs.

\end{itemize}

\FloatBarrier
\subsubsection[ProgramLocationType]{ProgramLocationType \\ {\small Subtype of Event}}
  \label{type:ProgramLocationType}

\FloatBarrier

Defines whether the logic or motion program defined by {model:PROGRAM} is being executed from the local memory of the controller or from an outside source.
  
 The {term:Valid Data Value} *MUST* be {model:LOCAL} or {model:EXTERNAL}.

\begin{table}[ht]
\centering 
  \caption{\texttt{Property of ProgramLocationType}}
  \label{properties:ProgramLocationType}
\tabulinesep=3pt
\begin{tabu} to 6in {|l|l|} \everyrow{\hline}
\hline
\rowfont\bfseries {Property} & {Value} \\
\tabucline[1.5pt]{}
\texttt{type} & \texttt{PROGRAM_LOCATION_TYPE} \\
\end{tabu}
\end{table}
\FloatBarrier

Subtypes of \texttt{ProgramLocationType} are :

\begin{itemize}
\item \texttt{ACTIVE} : The value of the {term:Data Entity} that is engaging.

\item \texttt{MAIN} : The identity of the primary logic or motion program currently being executed. It is the starting nest level in a call structure and may contain calls to sub programs.

\item \texttt{SCHEDULE} : The identity of a control program that is used to specify the order of execution of other programs.

\end{itemize}

\FloatBarrier
\subsubsection[ProgramNestLevel]{ProgramNestLevel \\ {\small Subtype of Event}}
  \label{type:ProgramNestLevel}

\FloatBarrier

An indication of the nesting level within a control program that is associated with the code or instructions that is currently being executed.
  
 If an Initial Value is not defined, the nesting level associated with the highest or initial nesting level of the program *MUST* default to zero (0).
  
 The value reported for {model:PROGRAM_NEST_LEVEL} *MUST* be an integer.

\begin{table}[ht]
\centering 
  \caption{\texttt{Property of ProgramNestLevel}}
  \label{properties:ProgramNestLevel}
\tabulinesep=3pt
\begin{tabu} to 6in {|l|l|} \everyrow{\hline}
\hline
\rowfont\bfseries {Property} & {Value} \\
\tabucline[1.5pt]{}
\texttt{type} & \texttt{PROGRAM_NEST_LEVEL} \\
\end{tabu}
\end{table}
\FloatBarrier

\FloatBarrier
\subsubsection[RotaryMode]{RotaryMode \\ {\small Subtype of Event}}
  \label{type:RotaryMode}

\FloatBarrier

The current operating mode for a {model:Rotary} type axis.

\begin{table}[ht]
\centering 
  \caption{\texttt{Property of RotaryMode}}
  \label{properties:RotaryMode}
\tabulinesep=3pt
\begin{tabu} to 6in {|l|l|} \everyrow{\hline}
\hline
\rowfont\bfseries {Property} & {Value} \\
\tabucline[1.5pt]{}
\texttt{type} & \texttt{ROTARY_MODE} \\
\texttt{result} & \texttt{RotaryModeEnum} \\
\end{tabu}
\end{table}
\FloatBarrier


 Enumerated \texttt{result} values for \texttt{RotaryMode} are:
\begin{table}[ht]
\centering 
  \caption{\texttt{RotaryModeEnum} Enumeration}
  \label{enum:RotaryModeEnum}
\tabulinesep=3pt
\begin{tabu} to 6in {|l|X|} \everyrow{\hline}
\hline
\rowfont\bfseries {Name} & {Description} \\
\tabucline[1.5pt]{}
\texttt{SPINDLE} & The axis is functioning as a spindle. \\
\texttt{INDEX} & The axis is configured to index. \\
\texttt{CONTOUR} & The position of the axis is being interpolated. \\
\end{tabu}
\end{table} 
\FloatBarrier
\FloatBarrier
\subsubsection[RotaryVelocityOverride]{RotaryVelocityOverride \\ {\small Subtype of Event}}
  \label{type:RotaryVelocityOverride}

\FloatBarrier

The value of a command issued to adjust the programmed velocity for a {model:Rotary} type axis.
 This command represents a percentage change to the velocity calculated by a logic or motion program or set by a switch for a {model:Rotary} type axis.

\begin{table}[ht]
\centering 
  \caption{\texttt{Property of RotaryVelocityOverride}}
  \label{properties:RotaryVelocityOverride}
\tabulinesep=3pt
\begin{tabu} to 6in {|l|l|} \everyrow{\hline}
\hline
\rowfont\bfseries {Property} & {Value} \\
\tabucline[1.5pt]{}
\texttt{type} & \texttt{ROTARY_VELOCITY_OVERRIDE} \\
\end{tabu}
\end{table}
\FloatBarrier

\FloatBarrier
\subsubsection[SerialNumber]{SerialNumber \\ {\small Subtype of Event}}
  \label{type:SerialNumber}

\FloatBarrier

The serial number associated with a {model:Component}, {model:Asset}, or {model:Device}. The {term:Valid Data Value} *MUST* be a text string.

\begin{table}[ht]
\centering 
  \caption{\texttt{Property of SerialNumber}}
  \label{properties:SerialNumber}
\tabulinesep=3pt
\begin{tabu} to 6in {|l|l|} \everyrow{\hline}
\hline
\rowfont\bfseries {Property} & {Value} \\
\tabucline[1.5pt]{}
\texttt{type} & \texttt{SERIAL_NUMBER} \\
\end{tabu}
\end{table}
\FloatBarrier

\FloatBarrier
\subsubsection[SpindleInterlock]{SpindleInterlock \\ {\small Subtype of Event}}
  \label{type:SpindleInterlock}

\FloatBarrier

An indication of the status of the spindle for a piece of equipment when power has been removed and it is free to rotate.

\begin{table}[ht]
\centering 
  \caption{\texttt{Property of SpindleInterlock}}
  \label{properties:SpindleInterlock}
\tabulinesep=3pt
\begin{tabu} to 6in {|l|l|} \everyrow{\hline}
\hline
\rowfont\bfseries {Property} & {Value} \\
\tabucline[1.5pt]{}
\texttt{type} & \texttt{SPINDLE_INTERLOCK} \\
\texttt{result} & \texttt{ActuatorStateEnum} \\
\end{tabu}
\end{table}
\FloatBarrier


 Enumerated \texttt{result} values for \texttt{SpindleInterlock} are:
\begin{table}[ht]
\centering 
  \caption{\texttt{ActuatorStateEnum} Enumeration}
\tabulinesep=3pt
\begin{tabu} to 6in {|l|X|} \everyrow{\hline}
\hline
\rowfont\bfseries {Name} & {Description} \\
\tabucline[1.5pt]{}
\texttt{ACTIVE} & The value of the {term:Data Entity} that is engaging. \\
\texttt{INACTIVE} & The value of the {term:Data Entity} that is not engaging. \\
\end{tabu}
\end{table} 
\FloatBarrier
\FloatBarrier
\subsubsection[ToolAssetId]{ToolAssetId \\ {\small Subtype of Event}}
  \label{type:ToolAssetId}

\FloatBarrier

The identifier of an individual tool asset.The {term:Valid Data Value} *MUST* be a text string.

\begin{table}[ht]
\centering 
  \caption{\texttt{Property of ToolAssetId}}
  \label{properties:ToolAssetId}
\tabulinesep=3pt
\begin{tabu} to 6in {|l|l|} \everyrow{\hline}
\hline
\rowfont\bfseries {Property} & {Value} \\
\tabucline[1.5pt]{}
\texttt{type} & \texttt{TOOL_ASSET_ID} \\
\end{tabu}
\end{table}
\FloatBarrier

\FloatBarrier
\subsubsection[ToolGroup]{ToolGroup \\ {\small Subtype of Event}}
  \label{type:ToolGroup}

\FloatBarrier

An identifier for the tool group associated with a specific tool. Commonly used to designate spare tools.

\begin{table}[ht]
\centering 
  \caption{\texttt{Property of ToolGroup}}
  \label{properties:ToolGroup}
\tabulinesep=3pt
\begin{tabu} to 6in {|l|l|} \everyrow{\hline}
\hline
\rowfont\bfseries {Property} & {Value} \\
\tabucline[1.5pt]{}
\texttt{type} & \texttt{TOOL_GROUP} \\
\end{tabu}
\end{table}
\FloatBarrier

\FloatBarrier
\subsubsection[ToolId]{ToolId \\ {\small Subtype of Event}}
  \label{type:ToolId}

\FloatBarrier

*DEPRECATED* in Version 1.2.0.   See {model:TOOL_ASSET_ID}. *DEPRECATED:The identifier of the tool currently in use for a given {model:Path}.*

\begin{table}[ht]
\centering 
  \caption{\texttt{Property of ToolId}}
  \label{properties:ToolId}
\tabulinesep=3pt
\begin{tabu} to 6in {|l|l|} \everyrow{\hline}
\hline
\rowfont\bfseries {Property} & {Value} \\
\tabucline[1.5pt]{}
\texttt{type} & \texttt{TOOL_ID} \\
\end{tabu}
\end{table}
\FloatBarrier

\FloatBarrier
\subsubsection[ToolNumber]{ToolNumber \\ {\small Subtype of Event}}
  \label{type:ToolNumber}

\FloatBarrier

The identifier assigned by the {model:Controller} component to a cutting tool when in use by a piece of equipment. 
 The {term:Valid Data Value} *MUST* be a text string.

\begin{table}[ht]
\centering 
  \caption{\texttt{Property of ToolNumber}}
  \label{properties:ToolNumber}
\tabulinesep=3pt
\begin{tabu} to 6in {|l|l|} \everyrow{\hline}
\hline
\rowfont\bfseries {Property} & {Value} \\
\tabucline[1.5pt]{}
\texttt{type} & \texttt{TOOL_NUMBER} \\
\end{tabu}
\end{table}
\FloatBarrier

\FloatBarrier
\subsubsection[ToolOffset]{ToolOffset \\ {\small Subtype of Event}}
  \label{type:ToolOffset}

\FloatBarrier

A reference to the tool offset variables applied to the active cutting tool associated with a {model:Path} in a {model:Controller} type component.

\begin{table}[ht]
\centering 
  \caption{\texttt{Property of ToolOffset}}
  \label{properties:ToolOffset}
\tabulinesep=3pt
\begin{tabu} to 6in {|l|l|} \everyrow{\hline}
\hline
\rowfont\bfseries {Property} & {Value} \\
\tabucline[1.5pt]{}
\texttt{type} & \texttt{TOOL_OFFSET} \\
\end{tabu}
\end{table}
\FloatBarrier

Subtypes of \texttt{ToolOffset} are :

\begin{itemize}
\item \texttt{LENGTH} : A reference to a length type tool offset variable.

\item \texttt{RADIAL} : A reference to a radial type tool offset variable.

\end{itemize}

\FloatBarrier
\subsubsection[User]{User \\ {\small Subtype of Event}}
  \label{type:User}

\FloatBarrier

The identifier of the person currently responsible for operating the piece of equipment.

\begin{table}[ht]
\centering 
  \caption{\texttt{Property of User}}
  \label{properties:User}
\tabulinesep=3pt
\begin{tabu} to 6in {|l|l|} \everyrow{\hline}
\hline
\rowfont\bfseries {Property} & {Value} \\
\tabucline[1.5pt]{}
\texttt{type} & \texttt{USER} \\
\end{tabu}
\end{table}
\FloatBarrier

Subtypes of \texttt{User} are :

\begin{itemize}
\item \texttt{MAINTENANCE} : Action related to maintenance on the piece of equipment.

\item \texttt{OPERATOR} : The identifier of the person currently responsible for operating the piece of equipment.

\item \texttt{SET_UP} : The identifier of the person currently responsible for preparing a piece of equipment for production or restoring the piece of equipment to a neutral state after production.

\end{itemize}

\FloatBarrier
\subsubsection[Variable]{Variable \\ {\small Subtype of Event}}
  \label{type:Variable}

\FloatBarrier

A data value whose meaning may change over time due to changes in the opertion of a piece of equipment or the process being executed on that piece of equipment.

\begin{table}[ht]
\centering 
  \caption{\texttt{Property of Variable}}
  \label{properties:Variable}
\tabulinesep=3pt
\begin{tabu} to 6in {|l|l|} \everyrow{\hline}
\hline
\rowfont\bfseries {Property} & {Value} \\
\tabucline[1.5pt]{}
\texttt{type} & \texttt{VARIABLE} \\
\end{tabu}
\end{table}
\FloatBarrier

\FloatBarrier
\subsubsection[WaitState]{WaitState \\ {\small Subtype of Event}}
  \label{type:WaitState}

\FloatBarrier

An indication of the reason that {model:EXECUTION} is reporting a value of {model:WAIT}.

\begin{table}[ht]
\centering 
  \caption{\texttt{Property of WaitState}}
  \label{properties:WaitState}
\tabulinesep=3pt
\begin{tabu} to 6in {|l|l|} \everyrow{\hline}
\hline
\rowfont\bfseries {Property} & {Value} \\
\tabucline[1.5pt]{}
\texttt{type} & \texttt{WAIT_STATE} \\
\texttt{result} & \texttt{WaitStateEnum} \\
\end{tabu}
\end{table}
\FloatBarrier


 Enumerated \texttt{result} values for \texttt{WaitState} are:
\begin{table}[ht]
\centering 
  \caption{\texttt{WaitStateEnum} Enumeration}
  \label{enum:WaitStateEnum}
\tabulinesep=3pt
\begin{tabu} to 6in {|l|X|} \everyrow{\hline}
\hline
\rowfont\bfseries {Name} & {Description} \\
\tabucline[1.5pt]{}
\texttt{POWERING_UP} & An indication that execution is waiting while the equipment is powering up and is not currently available to begin producing parts or products. \\
\texttt{POWERING_DOWN} & An indication that the execution is waiting while the equipment is powering down but has not fully reached a stopped state. \\
\texttt{PART_LOAD} & An indication that the execution is waiting while one or more discrete workpieces are being loaded. \\
\texttt{PART_UNLOAD} & An indication that the execution is waiting while one or more discrete workpieces are being unloaded. \\
\texttt{TOOL_LOAD} & An indication that the execution is waiting while a tool or tooling is being loaded. \\
\texttt{TOOL_UNLOAD} & An indication that the execution is waiting while a tool or tooling is being unloaded. \\
\texttt{MATERIAL_LOAD} & An indication that the execution is waiting while material is being loaded. \\
\texttt{MATERIAL_UNLOAD} & An indication that the execution is waiting while material is being unloaded. \\
\texttt{SECONDARY_PROCESS} & An indication that the execution is waiting while another process is completed before the execution can resume. \\
\texttt{PAUSING} & An indication that the execution is waiting while the equipment is pausing but the piece of equipment has not yet reached a fully paused state. \\
\texttt{RESUMING} & An indication that the execution is waiting while the equipment is resuming the production cycle but has not yet resumed execution. \\
\end{tabu}
\end{table} 
\FloatBarrier
\FloatBarrier
\subsubsection[Wire]{Wire \\ {\small Subtype of Event}}
  \label{type:Wire}

\FloatBarrier

A string like piece or filament of relatively rigid or flexible material provided in a variety of diameters.

\begin{table}[ht]
\centering 
  \caption{\texttt{Property of Wire}}
  \label{properties:Wire}
\tabulinesep=3pt
\begin{tabu} to 6in {|l|l|} \everyrow{\hline}
\hline
\rowfont\bfseries {Property} & {Value} \\
\tabucline[1.5pt]{}
\texttt{type} & \texttt{WIRE} \\
\end{tabu}
\end{table}
\FloatBarrier

\FloatBarrier
\subsubsection[WorkOffset]{WorkOffset \\ {\small Subtype of Event}}
  \label{type:WorkOffset}

\FloatBarrier

A reference to the offset variables for a work piece or part associated with a {model:Path} in a {model:Controller} type component.

\begin{table}[ht]
\centering 
  \caption{\texttt{Property of WorkOffset}}
  \label{properties:WorkOffset}
\tabulinesep=3pt
\begin{tabu} to 6in {|l|l|} \everyrow{\hline}
\hline
\rowfont\bfseries {Property} & {Value} \\
\tabucline[1.5pt]{}
\texttt{type} & \texttt{WORK_OFFSET} \\
\end{tabu}
\end{table}
\FloatBarrier

\FloatBarrier
\subsubsection[WorkholdingId]{WorkholdingId \\ {\small Subtype of Event}}
  \label{type:WorkholdingId}

\FloatBarrier

The identifier for the current workholding or part clamp in use by a piece of equipment. 
 The {term:Valid Data Value} *MUST* be a text string.

\begin{table}[ht]
\centering 
  \caption{\texttt{Property of WorkholdingId}}
  \label{properties:WorkholdingId}
\tabulinesep=3pt
\begin{tabu} to 6in {|l|l|} \everyrow{\hline}
\hline
\rowfont\bfseries {Property} & {Value} \\
\tabucline[1.5pt]{}
\texttt{type} & \texttt{WORKHOLDING_ID} \\
\end{tabu}
\end{table}
\FloatBarrier

\FloatBarrier
