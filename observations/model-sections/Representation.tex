% Generated 2020-08-21 17:56:42 +0530
\subsection{Representation} \label{sec:Representation}

\subsubsection{Cell}
\label{sec:Cell}






\paragraph{Attributes of Cell}\mbox{}
\label{sec:Attributes of Cell}

\tbl{Attributes of Cell} lists the attributes of \texttt{Cell}.

\begin{table}[ht]
\centering 
  \caption{Attributes of Cell}
  \label{table:Attributes of Cell}
\tabulinesep=3pt
\begin{tabu} to 6in {|l|l|l|} \everyrow{\hline}
\hline
\rowfont\bfseries {Attribute} & {Type} & {Multiplicity} \\
\tabucline[1.5pt]{}
\property{key}[Cell] & \texttt{ID} & 1 \\
\property{value}[Cell] & \texttt{T} & 1 \\
\end{tabu}
\end{table}
\FloatBarrier


Descriptions for attributes of \block{Cell}:

\begin{itemize}
\item \property{key}[Cell] : 
\item \property{value}[Cell] : 
\end{itemize}
\FloatBarrier

\subsubsection{DataSet}
\label{sec:DataSet}






\paragraph{Attributes of DataSet}\mbox{}
\label{sec:Attributes of DataSet}

\tbl{Attributes of DataSet} lists the attributes of \texttt{DataSet}.

\begin{table}[ht]
\centering 
  \caption{Attributes of DataSet}
  \label{table:Attributes of DataSet}
\tabulinesep=3pt
\begin{tabu} to 6in {|l|l|l|} \everyrow{\hline}
\hline
\rowfont\bfseries {Attribute} & {Type} & {Multiplicity} \\
\tabucline[1.5pt]{}
\property{count}[DataSet] & \texttt{integer} & 1 \\
\end{tabu}
\end{table}
\FloatBarrier


Descriptions for attributes of \block{DataSet}:

\begin{itemize}
\item \property{count}[DataSet] : 
\end{itemize}

\paragraph{Elements of DataSet}\mbox{}
\label{sec:Elements of DataSet}

\tbl{Elements of DataSet} lists the elements of \texttt{DataSet}.

\begin{table}[ht]
\centering 
  \caption{Elements of DataSet}
  \label{table:Elements of DataSet}
\tabulinesep=3pt
\begin{tabu} to 6in {|l|l|l|} \everyrow{\hline}
\hline
\rowfont\bfseries {Element Name} & {Type} & {Multiplicity} \\
\tabucline[1.5pt]{}
\block{Entry} & \texttt{Entry} & 0..* \\
\end{tabu}
\end{table}
\FloatBarrier


Descriptions for elements of \block{DataSet}:

\begin{itemize}
\item \block{Entry} : 
\end{itemize}
\FloatBarrier

\subsubsection{Discrete}
\label{sec:Discrete}






\paragraph{Attributes of Discrete}\mbox{}
\label{sec:Attributes of Discrete}

\tbl{Attributes of Discrete} lists the attributes of \texttt{Discrete}.

\begin{table}[ht]
\centering 
  \caption{Attributes of Discrete}
  \label{table:Attributes of Discrete}
\tabulinesep=3pt
\begin{tabu} to 6in {|l|l|l|} \everyrow{\hline}
\hline
\rowfont\bfseries {Attribute} & {Type} & {Multiplicity} \\
\tabucline[1.5pt]{}
\property{observationDiscrete}[Discrete] & \texttt{} & 1 \\
\end{tabu}
\end{table}
\FloatBarrier


Descriptions for attributes of \block{Discrete}:

\begin{itemize}
\item \property{observationDiscrete}[Discrete] : 
\end{itemize}
\FloatBarrier

\subsubsection{Entry}
\label{sec:Entry}






\paragraph{Attributes of Entry}\mbox{}
\label{sec:Attributes of Entry}

\tbl{Attributes of Entry} lists the attributes of \texttt{Entry}.

\begin{table}[ht]
\centering 
  \caption{Attributes of Entry}
  \label{table:Attributes of Entry}
\tabulinesep=3pt
\begin{tabu} to 6in {|l|l|l|} \everyrow{\hline}
\hline
\rowfont\bfseries {Attribute} & {Type} & {Multiplicity} \\
\tabucline[1.5pt]{}
\property{key}[Entry] & \texttt{ID} & 1 \\
\property{removed}[Entry] & \texttt{boolean} & 0..1 \\
\property{hasCell}[Entry] & \texttt{Cell} & 0..* \\
\property{value}[Entry] & \texttt{T} & 0..1 \\
\end{tabu}
\end{table}
\FloatBarrier


Descriptions for attributes of \block{Entry}:

\begin{itemize}
\item \property{key}[Entry] : 
\item \property{removed}[Entry] : 
\item \property{hasCell}[Entry] : 
\item \property{value}[Entry] : 
\end{itemize}

\paragraph{Elements of Entry}\mbox{}
\label{sec:Elements of Entry}

\tbl{Elements of Entry} lists the elements of \texttt{Entry}.

\begin{table}[ht]
\centering 
  \caption{Elements of Entry}
  \label{table:Elements of Entry}
\tabulinesep=3pt
\begin{tabu} to 6in {|l|l|l|} \everyrow{\hline}
\hline
\rowfont\bfseries {Element Name} & {Type} & {Multiplicity} \\
\tabucline[1.5pt]{}
\block{Entry} & \texttt{DataSet} & 1 \\
\end{tabu}
\end{table}
\FloatBarrier


Descriptions for elements of \block{Entry}:

\begin{itemize}
\item \block{Entry} : 
\end{itemize}
\FloatBarrier

\subsubsection{Table}







\paragraph{Attributes of Table}\mbox{}
\label{sec:Attributes of Table}

\tbl{Attributes of Table} lists the attributes of \texttt{Table}.

\begin{table}[ht]
\centering 
  \caption{Attributes of Table}
  \label{table:Attributes of Table}
\tabulinesep=3pt
\begin{tabu} to 6in {|l|l|l|} \everyrow{\hline}
\hline
\rowfont\bfseries {Attribute} & {Type} & {Multiplicity} \\
\tabucline[1.5pt]{}
\property{hasEntry}[Table] & \texttt{Entry} & 0..* \\
\property{count}[Table] & \texttt{integer} & 1 \\
\end{tabu}
\end{table}
\FloatBarrier


Descriptions for attributes of \block{Table}:

\begin{itemize}
\item \property{hasEntry}[Table] : 
\item \property{count}[Table] : 
\end{itemize}
\FloatBarrier

\subsubsection{TimeSeries}
\label{sec:TimeSeries}






\paragraph{Attributes of TimeSeries}\mbox{}
\label{sec:Attributes of TimeSeries}

\tbl{Attributes of TimeSeries} lists the attributes of \texttt{TimeSeries}.

\begin{table}[ht]
\centering 
  \caption{Attributes of TimeSeries}
  \label{table:Attributes of TimeSeries}
\tabulinesep=3pt
\begin{tabu} to 6in {|l|l|l|} \everyrow{\hline}
\hline
\rowfont\bfseries {Attribute} & {Type} & {Multiplicity} \\
\tabucline[1.5pt]{}
\property{sampleCount}[TimeSeries] & \texttt{integer} & 1 \\
\property{observationTimeSeries}[TimeSeries] & \texttt{} & 1..* \\
\end{tabu}
\end{table}
\FloatBarrier


Descriptions for attributes of \block{TimeSeries}:

\begin{itemize}
\item \property{sampleCount}[TimeSeries] : 
\item \property{observationTimeSeries}[TimeSeries] : 
\end{itemize}
\FloatBarrier
