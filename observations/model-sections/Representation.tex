% Generated 2020-05-20 17:19:31 -0400
\subsection{Representation} \label{sec:Representation}

\subsubsection{Cell}
  \label{sec:Cell}





\paragraph{Attributes of Cell}\mbox{}
\label{sec:Attributes of Cell}

\tbl{attributes of Cell} lists the attributes of \texttt{Cell}.

\begin{table}[ht]
\centering 
  \caption{Attributes of Cell}
  \label{table:attributes of Cell}
\tabulinesep=3pt
\begin{tabu} to 6in {|l|l|l|} \everyrow{\hline}
\hline
\rowfont\bfseries {Attribute} & {Type} & {Multiplicity} \\
\tabucline[1.5pt]{}
\texttt{key} & \texttt{ID} & 1 \\
\texttt{value} & \texttt{T} & 1 \\
\end{tabu}
\end{table}
\FloatBarrier


Descriptions for attributes of \texttt{Cell}:

\begin{itemize}
\item \texttt{key} : 
\item \texttt{value} : 
\end{itemize}
\FloatBarrier

\subsubsection{DataSet}
  \label{sec:DataSet}





\paragraph{Attributes of DataSet}\mbox{}
\label{sec:Attributes of DataSet}

\tbl{attributes of DataSet} lists the attributes of \texttt{DataSet}.

\begin{table}[ht]
\centering 
  \caption{Attributes of DataSet}
  \label{table:attributes of DataSet}
\tabulinesep=3pt
\begin{tabu} to 6in {|l|l|l|} \everyrow{\hline}
\hline
\rowfont\bfseries {Attribute} & {Type} & {Multiplicity} \\
\tabucline[1.5pt]{}
\texttt{count} & \texttt{integer} & 1 \\
\end{tabu}
\end{table}
\FloatBarrier


Descriptions for attributes of \texttt{DataSet}:

\begin{itemize}
\item \texttt{count} : 
\end{itemize}

\paragraph{Elements of DataSet}\mbox{}
\label{sec:Elements of DataSet}

\tbl{elements of DataSet} lists the elements of \texttt{DataSet}.

\begin{table}[ht]
\centering 
  \caption{Elements of DataSet}
  \label{table:elements of DataSet}
\tabulinesep=3pt
\begin{tabu} to 6in {|l|l|l|} \everyrow{\hline}
\hline
\rowfont\bfseries {Association Name} & {Element} & {Multiplicity} \\
\tabucline[1.5pt]{}
\texttt{Entry} & \texttt{Entry} & 0..* \\
\end{tabu}
\end{table}
\FloatBarrier


Descriptions for elements of \texttt{DataSet}:

\begin{itemize}
\item \texttt{Entry} : 
\end{itemize}
\FloatBarrier

\subsubsection{Discrete}
  \label{sec:Discrete}





\paragraph{Attributes of Discrete}\mbox{}
\label{sec:Attributes of Discrete}

\tbl{attributes of Discrete} lists the attributes of \texttt{Discrete}.

\begin{table}[ht]
\centering 
  \caption{Attributes of Discrete}
  \label{table:attributes of Discrete}
\tabulinesep=3pt
\begin{tabu} to 6in {|l|l|l|} \everyrow{\hline}
\hline
\rowfont\bfseries {Attribute} & {Type} & {Multiplicity} \\
\tabucline[1.5pt]{}
\texttt{observationDiscrete} & \texttt{} & 1 \\
\end{tabu}
\end{table}
\FloatBarrier


Descriptions for attributes of \texttt{Discrete}:

\begin{itemize}
\item \texttt{observationDiscrete} : 
\end{itemize}
\FloatBarrier

\subsubsection{Entry}
  \label{sec:Entry}





\paragraph{Attributes of Entry}\mbox{}
\label{sec:Attributes of Entry}

\tbl{attributes of Entry} lists the attributes of \texttt{Entry}.

\begin{table}[ht]
\centering 
  \caption{Attributes of Entry}
  \label{table:attributes of Entry}
\tabulinesep=3pt
\begin{tabu} to 6in {|l|l|l|} \everyrow{\hline}
\hline
\rowfont\bfseries {Attribute} & {Type} & {Multiplicity} \\
\tabucline[1.5pt]{}
\texttt{key} & \texttt{ID} & 1 \\
\texttt{removed} & \texttt{boolean} & 0..1 \\
\texttt{hasCell} & \texttt{Cell} & 0..* \\
\texttt{value} & \texttt{T} & 0..1 \\
\end{tabu}
\end{table}
\FloatBarrier


Descriptions for attributes of \texttt{Entry}:

\begin{itemize}
\item \texttt{key} : 
\item \texttt{removed} : 
\item \texttt{hasCell} : 
\item \texttt{value} : 
\end{itemize}

\paragraph{Elements of Entry}\mbox{}
\label{sec:Elements of Entry}

\tbl{elements of Entry} lists the elements of \texttt{Entry}.

\begin{table}[ht]
\centering 
  \caption{Elements of Entry}
  \label{table:elements of Entry}
\tabulinesep=3pt
\begin{tabu} to 6in {|l|l|l|} \everyrow{\hline}
\hline
\rowfont\bfseries {Association Name} & {Element} & {Multiplicity} \\
\tabucline[1.5pt]{}
\texttt{Entry} & \texttt{DataSet} & 1 \\
\end{tabu}
\end{table}
\FloatBarrier


Descriptions for elements of \texttt{Entry}:

\begin{itemize}
\item \texttt{Entry} : 
\end{itemize}
\FloatBarrier

\subsubsection{Table}
  \label{sec:Table}





\paragraph{Attributes of Table}\mbox{}
\label{sec:Attributes of Table}

\tbl{attributes of Table} lists the attributes of \texttt{Table}.

\begin{table}[ht]
\centering 
  \caption{Attributes of Table}
  \label{table:attributes of Table}
\tabulinesep=3pt
\begin{tabu} to 6in {|l|l|l|} \everyrow{\hline}
\hline
\rowfont\bfseries {Attribute} & {Type} & {Multiplicity} \\
\tabucline[1.5pt]{}
\texttt{hasEntry} & \texttt{Entry} & 0..* \\
\texttt{count} & \texttt{integer} & 1 \\
\end{tabu}
\end{table}
\FloatBarrier


Descriptions for attributes of \texttt{Table}:

\begin{itemize}
\item \texttt{hasEntry} : 
\item \texttt{count} : 
\end{itemize}
\FloatBarrier

\subsubsection{TimeSeries}
  \label{sec:TimeSeries}





\paragraph{Attributes of TimeSeries}\mbox{}
\label{sec:Attributes of TimeSeries}

\tbl{attributes of TimeSeries} lists the attributes of \texttt{TimeSeries}.

\begin{table}[ht]
\centering 
  \caption{Attributes of TimeSeries}
  \label{table:attributes of TimeSeries}
\tabulinesep=3pt
\begin{tabu} to 6in {|l|l|l|} \everyrow{\hline}
\hline
\rowfont\bfseries {Attribute} & {Type} & {Multiplicity} \\
\tabucline[1.5pt]{}
\texttt{sampleCount} & \texttt{integer} & 1 \\
\texttt{observationTimeSeries} & \texttt{} & 1..* \\
\end{tabu}
\end{table}
\FloatBarrier


Descriptions for attributes of \texttt{TimeSeries}:

\begin{itemize}
\item \texttt{sampleCount} : 
\item \texttt{observationTimeSeries} : 
\end{itemize}
\FloatBarrier
