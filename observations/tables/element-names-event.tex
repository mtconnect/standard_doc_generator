
\tabulinesep = 5pt
\begin{longtabu} to \textwidth {
    |X[2l]|X[3.3l]|X[3.2l]|}
\caption{Element Names for Event} 
\label{table:element-names-event} \\

\hline
DataItem Type & Element Name & Description\\
\hline
\endfirsthead

\hline
\multicolumn{3}{|c|}{Continuation of Table \ref{table:element-names-event}: \nameref{table:element-names-event}}\\
\hline
DataItem Type & Element Name & Description\\
\hline
\endhead

\gls{activeaxes event}
&
\glselementname{activeaxes event}
&
\glsentrydesc{activeaxes event}
\newline The \gls{valid data value} reported \SHOULD be a space-delimited set of axes names. The names returned \SHOULD match the \gls{name} attribute of the \gls{linear} or \gls{rotary} \glspl{structural element} defined in the \gls{mtconnectdevices} document that this \gls{event} element represents. If \gls{name} is not available, \gls{nativename} \MUST be returned to identify the \gls{linear} or \gls{rotary} \glspl{structural element}.
\newline For example:
\newline \tab \cfont{<ActiveAxes ...>X Y Z W S</ActiveAxes>}
\newline where X, Y, Z, W, and S are the \gls{nativename} attributes of the \glspl{structural element}. 
\newline If it is not specified elsewhere in the \gls{mtconnectdevices} document, it \MUST be assumed that all of the axes are associated with the \gls{path} component.
\\ \hline 

\gls{actuatorstate event}
&
\glselementname{actuatorstate event}
&
\glsentrydesc{actuatorstate event}
\newline \glspl{valid data value}:
\newline \tab \gls{active value}: The actuator is operating
\newline \tab \gls{inactive value}: The actuator is not operating 
\\ \hline 

\gls{alarm event} & \glselementname{alarm event} & \glsentrydesc{alarm event} \\ \hline 

\gls{availability event}
&
\glselementname{availability event}
&
\glsentrydesc{availability event}
\newline \glselementname{availability event} \MUST be provided for each \gls{device} \gls{structural element} and \MAY be provided for any other \gls{structural element}.
\newline \glspl{valid data value}:
\newline \tab \gls{available value}: The \gls{structural element} is active and capable of providing data.
\newline \tab \gls{available value}: The \gls{structural element} is either inactive or not capable of providing data.
\\ \hline 

\gls{axiscoupling event}
&
\glselementname{axiscoupling event}
&
\glsentrydesc{axiscoupling event}
\newline The coupling of the axes \MUST be viewed from the
perspective of a specified axis. Therefore, a
\gls{master value} coupling indicates that this axis is the
master for the \gls{coupledaxes event}.
\newline \glselementname{axiscoupling event} \MUST be provided for each axis
element associated with a set of axes defined by the
\gls{coupledaxes event} data item element defined in the
\gls{mtconnectdevices} document.
\newline \glspl{valid data value}:
\newline \tab \gls{tandem value}: The axes are physically connected to
each other and operate as a single unit.
\newline \tab \gls{synchronous value}: The axes are not physically
connected to each other but are operating together in
lockstep.
\newline \tab \gls{master value}: The axis is the master of the
\glselementname{coupledaxes event}
\newline \tab \gls{slave value}: The axis is a slave to the
\glselementname{coupledaxes event}
\\ \hline 

\gls{axisfeedrateoverride event}
&
\glselementname{axisfeedrateoverride event}
&
\glsentrydesc{axisfeedrateoverride event}
\newline The value provided for
\glselementname{axisfeedrateoverride event} is expressed as a
percentage of the designated feedrate for the axis.
\newline Subtypes of \glselementname{axisfeedrateoverride event} are \gls{jog subtype},
\gls{programmed subtype}, and \gls{rapid subtype}.
\newline If a \gls{subtype} is not specified, the reported value
for the data \MUST default to the \gls{subtype} of
\gls{programmed subtype}.
\newline The \gls{valid data value} \MUST be a floating-point
number.
\\ \hline 

\gls{axisinterlock event}
&
\glselementname{axisinterlock event}
&
\glsentrydesc{axisinterlock event}
\newline \glspl{valid data value}:
\newline \gls{active value}: The axis lockout function is activated,
power has been removed from the axis, and the axis
is allowed to move freely.
\newline \gls{inactive value}: The axis lockout function has not
been activated, the axis may be powered, and the
axis is capable of being controlled by another
component.
\\ \hline 

\gls{axisstate event}
&
\glselementname{axisstate event}
&
\glsentrydesc{axisstate event}
\newline \glspl{valid data value}:
\newline \tab \gls{home value}: The axis is in its home position.
\newline \tab \gls{travel value}: The axis is in motion
\newline \tab \gls{parked value}: The axis has been moved to a fixed
position and is being maintained in that position
either electrically or mechanically. Action is
required to release the axis from this position.
\newline \tab \gls{stopped value}: The axis is stopped
\\ \hline 

\gls{block event}
&
\glselementname{block event}
&
\glsentrydesc{block event}
\newline \glselementname{block event} \MUST include the entire expression for a line of program code, including all parameters
\newline The \gls{valid data value} \MUST be a text string.
\\ \hline 

\gls{blockcount event}
&
\glselementname{blockcount event}
&
\glsentrydesc{blockcount event}
\newline The \gls{valid data value} \MUST be an integer.
\\ \hline 

\gls{chuckinterlock event}
&
\glselementname{chuckinterlock event}
&
\glsentrydesc{chuckinterlock event}
\newline A \gls{chuck} component or composition element may
be controlled by more than one type of
\glselementname{chuckinterlock event} function. When the
\newline \glselementname{chuckinterlock event} function is provided by an
operator controlled interlock that can inhibit the
ability to initiate an unclamp action of an
electronically controlled chuck, this
\newline \glselementname{chuckinterlock event} function \SHOULD be further
characterized by specifying a \gls{subtype} of
\gls{manualunclamp subtype}.
\newline \glspl{valid data value}:
\newline \tab \gls{active value}: The chuck cannot be unclamped
\newline \tab \gls{inactive value}: The chuck can be unclamped.
\\ \hline 

\gls{chuckstate event}
&
\glselementname{chuckstate event}
&
\glsentrydesc{chuckstate event}
\newline \glspl{valid data value}:
\newline \tab \gls{open value}: The \gls{chuck} component or composition
element is open to the point of a positive
confirmation
\newline \tab \gls{closed value}: The \gls{chuck} component or
composition element is closed to the point of a
positive confirmation
\newline \tab \gls{unlatched value}: The \gls{chuck} component or
composition element is not closed to the point of a
positive confirmation and not open to the point of a
positive confirmation. It is in an intermediate
position.
\\ \hline 

\gls{code event} & \glselementname{code event} & \glsentrydesc{code event} \\ \hline 

\gls{compositionstate event}
&
\glselementname{compositionstate event}
&
\glsentrydesc{compositionstate event}
\newline Subtypes of \glselementname{compositionstate event} are \gls{action subtype}, \gls{lateral subtype}, \gls{motion subtype}, \gls{switched subtype}, and \gls{vertical subtype}.
\newline A \gls{subtype} \MUST be provided.
\newline \glspl{valid data value} for \gls{subtype} \gls{action subtype} are:
\newline \tab \gls{active value}: The \gls{composition} element is
operating
\newline \tab \gls{inactive value}: The \gls{composition} element is not
operating.
\newline \glspl{valid data value} for \gls{subtype} \gls{lateral subtype} are:
\newline \tab \gls{right value} : The position of the \gls{composition} 
element is oriented to the right to the point of a
positive confirmation
\newline \tab \gls{left value} : The position of the \gls{composition} 
element is oriented to the left to the point of a
positive confirmation
\\ \hline

\gls{compositionstate event}
\newline (Continued)
&
\glselementname{compositionstate event}
&
\glspl{valid data value} for \gls{subtype} \gls{switched subtype} are:
\newline \tab \gls{on value} : The activation state of the \gls{composition} 
element is in an \gls{on value}  condition, it is operating, or it is
powered.
\newline \tab \gls{off value} : The activation state of the \gls{composition}
element is in an \gls{off value}  condition, it is not operating,
or it is not powered.
\glspl{valid data value} for \gls{subtype} \gls{vertical subtype} are:
\newline \tab \gls{up value} : The position of the \gls{composition} element
is oriented in an upward direction to the point of a
positive confirmation
\newline \tab \gls{down value} : The position of the \gls{composition} 
element is oriented in a downward direction to the
point of a positive confirmation
\newline \tab \gls{transitioning value} : The position of the
\gls{composition} element is not oriented in an
upward direction to the point of a positive
confirmation and is not oriented in a downward
direction to the point of a positive confirmation. It
is in an intermediate position.
\\ \hline 

\gls{compositionstate event}
\newline (Continued)
&
\glselementname{compositionstate event}
&
\tab \gls{transitioning value} : The position of the
\gls{composition} element is not oriented to the right
to the point of a positive confirmation and is not
oriented to the left to the point of a positive
confirmation. It is in an intermediate position.
\newline \glspl{valid data value} for \gls{subtype} \gls{motion subtype} are:
\newline \tab \gls{open value}: The position of the \gls{composition} 
element is open to the point of a positive
confirmation
\newline \tab \gls{closed value}: The position of the \gls{composition} 
element is closed to the point of a positive
confirmation
\newline \tab \gls{unlatched value}: The position of the
\gls{composition} element is not open to the point of
a positive confirmation and is not closed to the point
of a positive confirmation. It is in an intermediate
position.
\\ \hline

\gls{controllermode event}
&
\glselementname{controllermode event}
&
The current operating mode of the \gls{controller} component.
\newline \glspl{valid data value}:
\newline \tab \gls{automatic value}: The controller is configured to automatically execute a program. 
\newline \tab \gls{manual value}: The controller is not executing an active program. It is capable of receiving instructions from an external source – typically an operator. The controller executes operations based on the instructions received from the external source. 
\newline \tab \gls{manualdatainput value}: The operator can enter a series of operations for the controller to perform. The controller will execute this specific series of operations and then stop. 
\newline \tab \gls{semiautomatic value}: The controller is operating in a mode that restricts the active program from processing its next process step without operator intervention. 
\newline \tab \gls{edit value}: The controller is currently functioning as a programming device and is not capable of executing an active program. \\
\hline 

\gls{controllermodeoverride event}
&
\glselementname{controllermodeoverride event}
&
\glsentrydesc{controllermodeoverride event}
\newline Subtypes of \glselementname{controllermodeoverride event} are \gls{dryrun subtype}, \gls{singleblock subtype}, \gls{machineaxislock subtype},
\gls{optionalstop subtype}, and \gls{toolchangestop subtype}.
\newline A \gls{subtype} \MUST always be specified.
\newline \glspl{valid data value}:
\newline \tab \gls{on value} : The indicator of the
\glselementname{controllermodeoverride event} is in the \gls{on value}  state
and the mode override is active.
\newline \tab \gls{off value} : The indicator of the
\glselementname{controllermodeoverride event} is in the \gls{off value}  state and the mode override is inactive
\\ \hline 

\gls{coupledaxes event}
&
\glselementname{coupledaxes event}
&
\glsentrydesc{coupledaxes event}
\newline Used in conjunction with \glselementname{axiscoupling event} to
describe how the \glselementname{coupledaxes event} relate to each
other.
\newline The \gls{valid data value} reported \SHOULD be a
space-delimited set of axes names. The names
returned \SHOULD match the name attribute of the
\gls{linear}  or \gls{rotary}  \glspl{structural element} defined in
the \gls{mtconnectdevices} document that this
\gls{event} element represents. If name is not
available, \gls{nativename} \MUST be returned to
identify the \gls{linear}  or \gls{rotary}  \glspl{structural element}.
\newline Example:
\newline \cfont{<CoupledAxes ...>Y1 Y2</CoupledAxes>}
\\ \hline 

\gls{datecode event}
&
\glselementname{datecode event}
&
The time and date code associated with a material or other physical item.
\newline Subtypes of \glselementname{datecode event} are \gls{manufacture subtype}, \gls{expiration subtype}, and \gls{firstuse subtype}.
\newline A \gls{subtype} \MUST always be specified.
\newline \glselementname{datecode event} \MUST be reported in ISO 8601 format. \\
\hline

\gls{deviceuuid event}
&
\glselementname{deviceuuid event}
&
The identifier of another piece of equipment that is temporarily associated with a component of this piece of equipment to perform a particular function.
\newline \glspl{valid data value} are the value of the UUID attribute of the associated device - a \gls{nmtoken} XML type. \\
\hline

\gls{direction event}
&
\glselementname{direction event}
&
\glsentrydesc{direction event}
\newline Subtypes of \glselementname{direction event} are \gls{rotary subtype} and \gls{linear subtype}.
\newline A \gls{subtype} \MUST always be specified.
\glspl{valid data value} for \gls{subtype} \gls{rotary subtype} are:
\newline \tab \gls{clockwise value} : A \gls{rotary} type component is rotating in a clockwise fashion using the right-hand
rule.
\newline \tab \gls{counterclockwise value} : A \gls{rotary} type component is rotating in a counter clockwise
fashion using the right-hand rule.
\glspl{valid data value} for \gls{subtype} \gls{linear subtype} are:
\newline \tab \gls{positive value} : A \gls{linear} type component is moving in the direction of increasing position value
\newline \tab \gls{negative value} : A \gls{linear} type component is moving in the direction of decreasing position value
\\ \hline 

\gls{doorstate event}
&
\glselementname{doorstate event}
&
\glsentrydesc{doorstate event}
\newline \glspl{valid data value}:
\newline \tab \gls{open value}: The \gls{door}  is open to the point of a
positive confirmation
\newline \tab \gls{closed value}: The \gls{door}  is closed to the point of a
positive confirmation
\newline \tab \gls{unlatched value}: The \gls{door} is not closed to the
point of a positive confirmation and is not open to
the point of a positive confirmation. It is in an
intermediate position.
\\ \hline 

\gls{emergencystop event}
&
\glselementname{emergencystop event}
&
\glsentrydesc{emergencystop event}
\newline \glspl{valid data value}:
\newline \tab \gls{armed value} : The emergency stop circuit is complete
and the piece of equipment, component, or
composition element is allowed to operate.
\newline \tab \gls{triggered value} : The emergency stop circuit is open
and the operation of the piece of equipment,
component, or composition element is inhibited.
\\ \hline 

\gls{endofbar event}
&
\glselementname{endofbar event}
&
\glsentrydesc{endofbar event}
\newline Subtypes of \glselementname{endofbar event} are \gls{primary subtype} and \gls{auxiliary subtype}.
\newline If a \gls{subtype} is not specified, the reported value
for the data \MUST default to the \gls{subtype} of
\gls{primary subtype}.
\newline \glspl{valid data value}:
\newline \tab \gls{yes value} : The \glselementname{endofbar event} has been reached.
\newline \tab \gls{no value} : The \glselementname{endofbar event} has not been reached.
\\ \hline 

\gls{equipmentmode event}
&
\glselementname{equipmentmode event}
&
\glsentrydesc{equipmentmode event}
\newline Subtypes of \glselementname{equipmentmode event} are \gls{loaded subtype}, \gls{working subtype}, \gls{operating subtype}, and \gls{powered subtype}.
\newline A \gls{subtype} \MUST always be specified.
\newline \glspl{valid data value}:
\newline \tab \gls{on value} : The equipment is functioning in the mode
designated by the \gls{subtype}.
\newline \tab \gls{off value} : The equipment is not functioning in the
mode designated by the \gls{subtype}.
\\ \hline 

\gls{execution event}
&
\glselementname{execution event}
&
The execution status of the \gls{controller} component.
\newline \glspl{valid data value}:
\newline \tab \gls{ready value}:  The controller is ready to execute instructions. It is currently idle.
\newline \tab \gls{active value}:  The controller is actively executing an instruction.
\newline \tab \gls{interrupted value}:  The execution of the controller’s program has been suspended due to an external signal.  Action is required to resume execution.
\newline \tab \gls{wait}:  The execution of the controller's program is suspended while a secondary operation is executing or completing.  Execution will resume automatically once the secondary operation is completed.
\newline \tab \gls{feedhold value}:  Motion of the device has been commanded to stop at its current position.  The controller remains able to execute instructions but cannot complete the current set of instructions until after motion resumes.   The command to stop the motion must be removed before execution can resume.\\
\hline 

\gls{execution event}
(Continued)
&
\glselementname{execution event}
&
\tab \gls{stopped value}:  The execution of the controller’s program has been stopped in an unplanned manner and execution of the program cannot be resumed without intervention by an operator or external signal.
\newline \tab \gls{optionalstop value}:  The controller’s program has been intentionally stopped using an M01 or similar command.  The program may be stopped at the designated location based upon the state of a secondary indication provided to the controller indicating whether the program execution must be stopped at this location or program execution should continue.
\newline \tab \gls{programstopped value}:  The execution of the controller’s program has been stopped by a command from within the program.   Action is required to resume execution.
\newline \tab \gls{programcompleted value}:  The program has completed execution.
\\ \hline

\gls{functionalmode event}
&
\glselementname{functionalmode event}
&
\glsentrydesc{functionalmode event}
\newline Typically, the \glselementname{functionalmode event} \SHOULD be
associated with the \gls{device} \gls{structural element}, but
it \MAY be associated with any \gls{structural element}
in the XML document.
\newline \glspl{valid data value}:
\newline \tab \gls{production value} : The \gls{device} element or another
\gls{structural element} is currently producing product,
ready to produce product, or its current intended use
is to be producing product.
\newline \tab \gls{setup value} : The \gls{device} element or another
\gls{structural element} is not currently producing
product. It is being prepared or modified to begin
production of product.
\newline \tab \gls{teardown value} : The \gls{device} element or another
\gls{structural element} is not currently producing
product. Typically, it has completed the production
of a product and is being modified or returned to a
neutral state such that it may then be prepared to
begin production of a different product.
\\ \hline 

\gls{functionalmode event}
\newline (Continued)
&
\glselementname{functionalmode event}
&
\tab \gls{maintenance} : The \gls{device} element or
another \gls{structural element} is not currently
producing product. It is currently being repaired,
waiting to be repaired, or has not yet been returned
to a normal production status after maintenance has
been performed.
\newline \tab \gls{processdevelopment value} : The \gls{device}
element or another \gls{structural element} is being used
to prove-out a new process, testing of equipment or
processes, or any other active use that does not
result in the production of product.
\\ \hline 

\gls{hardness event}
&
\glselementname{hardness event}
&
\glsentrydesc{hardness event}
\newline Subtypes of \glselementname{hardness event} are \gls{rockwell subtype}, \gls{vickers subtype}, \gls{shore subtype}, \gls{brinell subtype}, \gls{leeb subtype}, and \gls{mohs subtype}.
\newline A \gls{subtype} \MUST always be specified.
\newline The \gls{valid data value} \MUST be a floating-point
number.
\\ \hline 

\gls{interfacestate event}
&
\glselementname{interfacestate event}
&
The current functional or operational state of an \gls{interface component} type element indicating whether the \gls{interface} is active or not currently functioning.
\newline \glspl{valid data value}:
\newline \tab \gls{enabled value}: The \gls{interface} is currently operational and performing as expected.
\newline \tab \gls{disabled value}: The Interface is currently not operational.
\newline When the \gls{interfacestate event} is \gls{disabled value}, the state of all data items that are specific for the \gls{interaction model} associated with that \gls{interface} \MUST be set to \gls{notready value}.
\\ \hline 

\gls{line event} & \glselementname{line event} & \glsentrydesc{line event} \\ \hline 

\gls{linelabel event}
&
\glselementname{linelabel event}
&
\glsentrydesc{linelabel event}
\newline The \gls{valid data value} \MUST be any text string.
\\ \hline 

\gls{linenumber event}
&
\glselementname{linenumber event}
&
\glsentrydesc{linenumber event}
\newline Subtypes of \glselementname{linenumber event} are \gls{absolute subtype} and \gls{incremental subtype}.
\newline A \gls{subtype} \MUST always be specified.
\newline The \gls{valid data value} \MUST be an integer.
\\ \hline 

\gls{material event}
&
\glselementname{material event}
&
\glsentrydesc{material event}
\newline The \gls{valid data value} \MUST be any text string.
\\ \hline 

\gls{materiallayer event}
&
\glselementname{materiallayer event}
&
Designates the layers of material applied to a part or product as part of an additive manufacturing process.
\newline Subtypes of \glselementname{materiallayer event} are \gls{actual subtype} and \gls{target subtype}.
\newline If a \gls{subtype} is not specified, the reported value for the data \MUST default to the subtype of \gls{actual subtype}.
\newline The \gls{valid data value} \MUST be an integer. \\
\hline

\gls{message event}
&
\glselementname{message event}
&
\glsentrydesc{message event}
\newline The \gls{valid data value} \MUST be any text string.
\\ \hline 

\gls{operatorid event}
&
\glselementname{operatorid event}
&
\glsentrydesc{operatorid event}
\newline The \gls{valid data value} \MAY be any text string.
\newline \DEPRECATIONWARNING: May be
deprecated in the future. See USER below.
\\ \hline 

\gls{palletid event}
&
\glselementname{palletid event}
&
\glsentrydesc{palletid event}
\newline The \gls{valid data value} \MAY be any text string.
\\ \hline 

\gls{partcount}
&
\glselementname{partcount}
&
The current count of parts produced as represented by the \gls{controller} component.
\newline Subtypes of \glselementname{partcount} are \gls{all subtype}, \gls{good subtype}, \gls{bad subtype}, \gls{target subtype}, and \gls{remaining subtype}.
\newline \glselementname{partcount} will not be accumulated by an
\gls{agent} and \MUST only be supplied if
the \gls{controller}  provides the count.

\newline The \gls{valid data value} \MUST be a floating-point
number, usually an integer.
\\ \hline 

\gls{partdetect event}
&
\glselementname{partdetect event}
&
An indication designating whether a part or work piece has been detected or is present.
\newline The \gls{valid data value} \MUST be:
\newline \tab \gls{present}: if a part or work piece has been detected or is present.
\newline \tab \gls{notpresent}: if a part or work piece is not detected or is not present. \\
\hline

\gls{partid event}
&
\glselementname{partid event}
&
\glsentrydesc{partid event}
\newline The \gls{valid data value} \MAY be any text string.
\\ \hline

\gls{partnumber event}
&
\glselementname{partnumber event}
&
An identifier of a part or product moving through the manufacturing process.
\newline The \gls{valid data value} \MUST be a text string. 
\newline \DEPRECATIONWARNING: May be deprecated in the future. \\
\hline 

\gls{pathfeedrateoverride event}
&
\glselementname{pathfeedrateoverride event}
&
\glsentrydesc{pathfeedrateoverride event}
\newline The value provided for
\glselementname{pathfeedrateoverride event} is expressed as a
percentage of the designated feedrate for the path.
\newline Sub-types of \glselementname{pathfeedrateoverride event} are \gls{jog subtype}, \gls{programmed subtype}, and \gls{rapid subtype}.
\newline If a \gls{subtype} is not specified, the reported value
for the data \MUST default to the \gls{subtype} of
\gls{programmed subtype}.
\newline The \gls{valid data value} \MUST be a floating-point
number.
\\ \hline 

\gls{pathmode event}
&
\glselementname{pathmode event}
&
\glsentrydesc{pathmode event}
\newline \glspl{valid data value}:
\newline \tab \gls{independent value} : The path is operating
independently and without the influence of another
path.
\newline  \tab \gls{master value}: The path provides the reference motion
for a \gls{synchronous value} or \gls{mirror value}  type path to
follow. For non-motion type paths, the \gls{master value}
provides information or state values that influences
the operation of other paths
\newline  \tab \gls{synchronous value}: The axes associated with the
path are following the motion of the \gls{master value} type
path.
\newline  \tab \gls{mirror value} : The axes associated with the path are
mirroring the motion of the \gls{master value} path.
When \glselementname{pathmode event} is not specified, the operational
mode of the path \MUST be interpreted as
\gls{independent value} .
\\ \hline 

\gls{powerstate event}
&
\glselementname{powerstate event}
&
\glsentrydesc{powerstate event}
\newline Subtypes of \glselementname{powerstate event} are LINE and
CONTROL.
\newline When the \gls{subtype} is \gls{line subtype}, \glselementname{powerstate event}
represents the primary source of energy for a \gls{structural element}.
\newline When the \gls{subtype} is \gls{control subtype}, \glselementname{powerstate event} represents an enabling signal providing permission for the \gls{structural element} to perform its function(s).
\newline If a \gls{subtype} is not specified, the reported value
for the data \MUST default to the \gls{subtype} of \gls{line subtype}.
\\ \hline 

\gls{powerstate event}
\newline (Continued)
&
\glselementname{powerstate event}
&
\glspl{valid data value}:
\newline \tab \gls{on value} : The source of energy for a \gls{structural element} or the enabling signal providing permission for the
\gls{structural element} to perform its function(s) is
present and active.
\newline \tab \gls{off value} : The source of energy for a \gls{structural element} or the enabling signal providing permission
for the \gls{structural element} to perform its function(s)
is not present or is disconnected.
\newline \DEPRECATIONWARNING: \glselementname{powerstate event} may be deprecated in the future.
\\ \hline 

\gls{powerstatus event}
&
\glselementname{powerstatus event}
&
\glsentrydesc{powerstatus event}
\\ \hline

\gls{processtime event}
&
\glselementname{processtime event}
&
The time and date associated with an activity or event.
\newline Subtypes of \glselementname{processtime event} are \gls{start subtype}, \gls{complete value}, and \gls{targetcompletion subtype}.
\newline A \gls{subtype} \MUST always be specified.
\newline \glselementname{processtime event} \MUST be reported in ISO 8601 format. \\
\hline

\gls{program event}
&
\glselementname{program event}
&
The identity of the logic or motion program being executed.
\newline The \gls{valid data value} \MUST be any text string.
\newline Subtypes of \gls{program event} are \gls{schedule subtype}, \gls{main subtype} and \gls{active value}.
\newline If a \gls{subtype} is not specified, it is assumed to be \gls{main subtype}. \\
\hline

\gls{programcomment event}
&
\glselementname{programcomment event}
&
A comment or non-executable statement in the control program.
\newline The \gls{valid data value} \MUST be any text string.
\newline Subtypes of \gls{programcomment event} are \gls{schedule subtype}, \gls{main subtype} and \gls{active value}.
\newline If a \gls{subtype} is not specified, it is assumed to be \gls{main subtype}. \\
\hline 

\gls{programedit event}
&
\glselementname{programedit event}
&
\glsentrydesc{programedit event}
\newline \glselementname{programedit event} provides an indication of whether
the controller is being used to edit programs in
either case.
\newline \glspl{valid data value}:
\newline  \gls{active value}: The controller is in the program edit
mode.
\newline  \gls{ready value} : The controller is capable of entering the
program edit mode and no function is inhibiting a
change to that mode.
\newline  \gls{notready value} : A function is inhibiting the
controller from entering the program edit mode.
\\ \hline 

\gls{programeditname event} & \glselementname{programeditname event} & \glsentrydesc{programeditname event} \\ \hline 

\gls{programheader event}
&
\glselementname{programheader event}
&
\glsentrydesc{programheader event}
\newline The content \SHOULD be limited to 512 bytes.
\newline The \gls{valid data value} \MUST be any text string.
\\ \hline 

\gls{programlocation event}
&
\glselementname{programlocation event}
&
The Uniform Resource Identifier (URI) for the source file associated with \gls{program event}.
\newline The \gls{valid data value} \MUST be any text string.
\newline A \gls{subtype} \MUST always be specified.
\newline Subtypes of \gls{programlocation event} are \gls{schedule subtype}, \gls{main subtype}, and \gls{active value}. \\
\hline

\gls{programlocationtype event}
&
\glselementname{programlocationtype event}
&
Defines whether the logic or motion program defined by \gls{program event} is being executed from the local memory of the controller or from an outside source.
\newline A \gls{subtype} \MUST always be specified.
\newline Subtypes of \gls{programlocationtype event} are \gls{schedule subtype}, \gls{main subtype}, and \gls{active value}.
\newline \glspl{valid data value} are:
\newline \tab \gls{local}: Managed by the controller.
\newline \tab \gls{external}: Not managed by the controller. \\
\hline

\gls{programnestlevel event}
&
\glselementname{programnestlevel event}
&
An indication of the nesting level within a control program that is associated with the code or instructions that is currently being executed.
\newline If an initial value is not defined, the nesting level associated with the highest or initial nesting level of the program \MUST default to zero (0).
\newline The value reported for \glselementname{programnestlevel event} \MUST be an integer. \\
\hline

\gls{rotarymode event}
&
\glselementname{rotarymode event}
&
\glsentrydesc{rotarymode event}
\newline \glspl{valid data value}:
\newline \tab \gls{spindle value}: The axis is functioning as a spindle.
Generally, it is configured to rotate at a defined
speed.
\newline \tab \gls{index value}: The axis is configured to index to a set of
fixed positions or to incrementally index by a fixed
amount.
\newline \tab \gls{contour value}: The position of the axis is being
interpolated as part of the \glselementname{pathposition sample} defined
by the \gls{controller}  \gls{structural element}.
\\ \hline 

\gls{rotaryvelocityoverride event}
&
\glselementname{rotaryvelocityoverride event}
&
\glsentrydesc{rotaryvelocityoverride event}
\newline \hspace{0pt}\glselementname{rotaryvelocityoverride event} is expressed as a
percentage of the programmed \glselementname{rotaryvelocity sample}.
\newline The \gls{valid data value} \MUST be a floating-point
number.
\\ \hline 

\gls{serialnumber event} & \glselementname{serialnumber event} & \glsentrydesc{serialnumber event} \\ \hline 

\gls{spindleinterlock event}
&
\glselementname{spindleinterlock event}
&
\glsentrydesc{spindleinterlock event}
\newline \glspl{valid data value}:
\newline  \gls{active value}: Power has been removed and the
spindle cannot be operated.
\newline  \gls{inactive value}: Spindle has not been deactivated.
\\ \hline 

\gls{toolassetid event} & \glselementname{toolassetid event} & \glsentrydesc{toolassetid event} \\ \hline 

\gls{toolgroup event}
&
\glselementname{toolgroup event}
&
An identifier for the tool group associated with a specific tool. Commonly used to designate spare tools.
\newline The \gls{valid data value} \MUST be any text string. \\
\hline

\deprecated{\mbox{\gls{toolid event}}} & \deprecated{\glselementname{toolid event}} & \glsentrydesc{toolid event} \\ \hline 

\gls{toolnumber event} & \glselementname{toolnumber event} & \glsentrydesc{toolnumber event} \\ \hline

\gls{tooloffset event}
&
\glselementname{tooloffset event}
&
A reference to the tool offset variables applied to the active cutting tool.
\newline Subtypes of \glselementname{tooloffset event} are \gls{radial subtype} and \gls{length subtype}.
\newline \DEPRECATED in V1.5 \deprecated{A subType \MUST always be specified.}
\newline The \gls{valid data value} \MUST be a text string. \\
\hline 

\gls{user event}
&
\glselementname{user event}
&
\glsentrydesc{user event}
\newline Subtypes of \glselementname{user event} are \gls{operator subtype}, \gls{maintenance}, and \gls{setup subtype}.
\newline A \gls{subtype} \MUST always be specified.
\newline The \gls{valid data value} \MUST be any text string.
\\ \hline 

\gls{variable event}
&
\glselementname{variable event}
&
A data value whose meaning may change over time due to changes in the operation of a piece of equipment or the process being executed on that piece of equipment.
\newline The \gls{valid data value} \MUST be a string. \\
\hline

\gls{waitstate event}
&
\glselementname{waitstate event}
&
An indication of the reason that \gls{execution event} is reporting a value of \gls{wait}.
\newline \glspl{valid data value} are:
\newline \tab \gls{poweringup}: An indication that execution is waiting while the equipment is powering up and is not currently available to begin producing parts or products.
\newline \tab \gls{poweringdown}: An indication that the execution is waiting while the equipment is powering down but has not fully reached a stopped state.
\newline \tab \gls{partload}: An indication that the execution is waiting while one or more discrete workpieces are being loaded.
\newline \tab \gls{partunload}: An indication that the execution is waiting while one or more discrete workpieces are being unloaded.
\newline \tab \gls{toolload}: An indication that the execution is waiting while a tool or tooling is being loaded.
\newline \tab \gls{toolunload}: An indication that the execution is waiting while a tool or tooling is being unloaded. \\
\hline

\gls{waitstate event}
(Continued)
&
\glselementname{waitstate event}
&
\tab \gls{materialload event}: An indication that the execution is waiting while bulk material or the container for bulk material used in the production process is being loaded.  Bulk material includes those materials from which multiple workpieces may be created.
\newline \tab \gls{materialunload event}: An indication that the execution is waiting while bulk material or the container for bulk material used in the production process is being unloaded.  Bulk material includes those materials from which multiple workpieces may be created.
\newline \tab \gls{secondaryprocess}: An indication that the execution is waiting while another process is completed before the execution can resume.
\newline \tab \gls{pausing}: An indication that the execution is waiting while the equipment is pausing but the piece of equipment has not yet reached a fully paused state.
\newline \tab \gls{resuming}: An indication that the execution is waiting while the equipment is resuming the production cycle but has not yet resumed execution. \\
\hline

\gls{wire}
&
\glselementname{wire}
&
The identifier for the type of wire used as the cutting mechanism in Electrical Discharge Machining or similar processes.
\newline The \gls{valid data value} \MUST be any text string. \\ \hline 

\gls{workholdingid event} & \glselementname{workholdingid event} & \glsentrydesc{workholdingid event} \\ \hline 




\gls{workoffset event}
&
\glselementname{workoffset event}
&
\glsentrydesc{workoffset event}
\newline The \gls{valid data value} \MUST be a text string.
\\ \hline 










\end{longtabu}