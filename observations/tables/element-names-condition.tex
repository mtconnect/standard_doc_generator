
\tabulinesep = 5pt
\begin{longtabu} to \textwidth {
    |l|X[3l]|}
\caption{Element Names for Condition} 
\label{table:element-names-condition} \\

\hline
DataItem Type & Description\\
\hline
\endfirsthead

\hline
\multicolumn{2}{|c|}{Continuation of Table \ref{table:element-names-condition}}\\
\hline
DataItem Type & Description\\
\hline
\endhead

\gls{actuator type}
&
An indication of a fault associated with an actuator.
\\ \hline 

\gls{chuckinterlock event}
&
An indication of the operational condition of the interlock function for an electronically controller chuck.
\\ \hline 

\gls{communications condition} & \glsentrydesc{communications condition} \\ \hline 

\gls{datarange condition} & \glsentrydesc{datarange condition} \\ \hline 

\gls{direction event}
&
An indication of a fault associated with the direction of motion of a \gls{structural element}.
\\ \hline

\gls{endofbar event}
&
An indication that the end of a piece of bar stock has been reached.
\\ \hline 

\gls{hardware condition} & \glsentrydesc{hardware condition} \\ \hline 

\gls{interfacestate event}
&
An indication of the operation condition of an \gls{interface component} component.
\\ \hline 

\gls{logicprogram condition} & \glsentrydesc{logicprogram condition} \\ \hline 

\gls{motionprogram condition} & \glsentrydesc{motionprogram condition} \\ \hline

\gls{system condition}
&
An indication of a fault associated with a piece of equipment or component that cannot be classified as a specific type. \\
\hline 


\end{longtabu}