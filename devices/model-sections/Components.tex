% Generated 2020-02-05 14:03:21 -0500
\subsection{Components} \label{model:Components}
\subsubsection{Component}
  \label{type:Component}

\FloatBarrier

Placeholder for documentation!

\begin{table}[ht]
\centering 
  \caption{\texttt{Properties of Component}}
  \label{properties:Component}
\tabulinesep=3pt
\begin{tabu} to 6in {|l|l|l|} \everyrow{\hline}
\hline
\rowfont\bfseries {Properties} & {Value} & {Multiplicity} \\
\tabucline[1.5pt]{}
\texttt{id} & \texttt{ID} & 1 \\
\texttt{name} & \texttt{string} & 0..1 \\
\texttt{nativeName} & \texttt{string} & 0..1 \\
\texttt{sampleInterval} & \texttt{integer} & 0..1 \\
\texttt{<<deprecated>> sampleRate} & \texttt{float} & 0..1 \\
\texttt{uuid} & \texttt{NMTOKEN} & 0..1 \\
\texttt{Description} & \texttt{Description} & 0..1 \\
\texttt{Compositions} & \texttt{Composition} & 0..* \\
\texttt{Components} & \texttt{Component} & 0..* \\
\texttt{Configuration} & \texttt{Configuration} & 0..* \\
\texttt{DataItems} & \texttt{DataItem} & 0..* \\
\texttt{DataItemRef} & \texttt{DataItem} & 0..1 \\
\texttt{ComponentRef} & \texttt{Component} & 0..1 \\
\end{tabu}
\end{table}
\FloatBarrier


\paragraph{\texttt{id}}\mbox{}
\newline\tab The unique identifier for this element.

\paragraph{\texttt{name}}\mbox{}
\newline\tab The name of an element or a piece of equipment.

\paragraph{\texttt{nativeName}}\mbox{}
\newline\tab The common name normally associated with a piece of equipment or an element.

\paragraph{\texttt{sampleInterval}}\mbox{}
\newline\tab An optional attribute that is an indication provided by a piece of equipment describing the interval in milliseconds between the completion of the reading of the data associated with the {model:Device} element until the beginning of the next sampling of that data.

\paragraph{\texttt{sampleRate}}\mbox{}
\newline\tab Placeholder for documentation!

\paragraph{\texttt{uuid}}\mbox{}
\newline\tab The unique identifier for an XML element.

\paragraph{\texttt{Description}}\mbox{}
\newline\tab Placeholder for documentation!

\paragraph{\texttt{Compositions}}\mbox{}
\newline\tab Placeholder for documentation!

\paragraph{\texttt{Components}}\mbox{}
\newline\tab Placeholder for documentation!

\paragraph{\texttt{Configuration}}\mbox{}
\newline\tab Placeholder for documentation!

\paragraph{\texttt{DataItems}}\mbox{}
\newline\tab Placeholder for documentation!

\paragraph{\texttt{DataItemRef}}\mbox{}
\newline\tab Placeholder for documentation!

\paragraph{\texttt{ComponentRef}}\mbox{}
\newline\tab Placeholder for documentation!
\FloatBarrier
\subsubsection[Device]{Device \\ {\small Subtype of Component}}
  \label{type:Device}

\FloatBarrier

Placeholder for documentation!

\begin{table}[ht]
\centering 
  \caption{\texttt{Properties of Device}}
  \label{properties:Device}
\tabulinesep=3pt
\begin{tabu} to 6in {|l|l|l|} \everyrow{\hline}
\hline
\rowfont\bfseries {Properties} & {Value} & {Multiplicity} \\
\tabucline[1.5pt]{}
\texttt{<<deprecated>> iso841Class} & \texttt{string} & 0..1 \\
\texttt{name} & \texttt{string} & 1 \\
\texttt{uuid} & \texttt{NMTOKEN} & 1 \\
\end{tabu}
\end{table}
\FloatBarrier


\paragraph{\texttt{iso841Class}}\mbox{}
\newline\tab Placeholder for documentation!
\FloatBarrier
\subsubsection{Description}
  \label{type:Description}

\FloatBarrier

Placeholder for documentation!

\begin{table}[ht]
\centering 
  \caption{\texttt{Properties of Description}}
  \label{properties:Description}
\tabulinesep=3pt
\begin{tabu} to 6in {|l|l|l|} \everyrow{\hline}
\hline
\rowfont\bfseries {Properties} & {Value} & {Multiplicity} \\
\tabucline[1.5pt]{}
\texttt{manufacturer} & \texttt{string} & 0..1 \\
\texttt{model} & \texttt{string} & 0..1 \\
\texttt{serialNumber} & \texttt{string} & 0..1 \\
\texttt{station} & \texttt{string} & 0..1 \\
\texttt{value} & \texttt{string} & 0..1 \\
\end{tabu}
\end{table}
\FloatBarrier


\paragraph{\texttt{manufacturer}}\mbox{}
\newline\tab The name of the manufacturer of the physical or logical part of a piece of equipment represented by an XML element.

\paragraph{\texttt{model}}\mbox{}
\newline\tab The model description of the physical part or logical function of a piece of equipment represented by this XML element.

\paragraph{\texttt{serialNumber}}\mbox{}
\newline\tab The serial number associated with a piece of equipment.

\paragraph{\texttt{station}}\mbox{}
\newline\tab The station where the physical part or logical function of a piece of equipment is located when it is part of a manufacturing unit or cell with multiple stations.

\paragraph{\texttt{value}}\mbox{}
\newline\tab Placeholder for documentation!
\FloatBarrier
