% Generated 2020-07-29 16:29:18 +0530
\subsection{Components} \label{sec:Components}

{block:Components} groups information describing physical parts or logical functions of a piece of equipment.   {block:Components} {term:organizes} one or more {block:Component} elements. 


\subsubsection{Component}
  \label{sec:Component}


A \block{Component} organizes information describing a physical part or logical function of a piece of equipment.


\paragraph{Attributes of Component}\mbox{}
\label{sec:Attributes of Component}

\tbl{attributes of Component} lists the attributes of \texttt{Component}.

\begin{table}[ht]
\centering 
  \caption{Attributes of Component}
  \label{table:attributes of Component}
\tabulinesep=3pt
\begin{tabu} to 6in {|l|l|l|} \everyrow{\hline}
\hline
\rowfont\bfseries {Attribute} & {Type} & {Multiplicity} \\
\tabucline[1.5pt]{}
\texttt{id} & \texttt{ID} & 1 \\
\texttt{name} & \texttt{string} & 0..1 \\
\texttt{nativeName} & \texttt{string} & 0..1 \\
\texttt{sampleInterval} & \texttt{integer} & 0..1 \\
\texttt{<<deprecated>> sampleRate} & \texttt{float} & 0..1 \\
\texttt{uuid} & \texttt{NMTOKEN} & 0..1 \\
\end{tabu}
\end{table}
\FloatBarrier


Descriptions for attributes of \texttt{Component}:

\begin{itemize}
\item \texttt{id} : The unique identifier for this element.
\item \texttt{name} : The name of an element or a piece of equipment.
\item \texttt{nativeName} : The common name normally associated with a piece of equipment or an element.
\item \texttt{sampleInterval} : An optional attribute that is an indication provided by a piece of equipment describing the interval in milliseconds between the completion of the reading of the data associated with the \block{Device} element until the beginning of the next sampling of that data.
\item \texttt{sampleRate} : 
\item \texttt{uuid} : The unique identifier for an XML element.
\end{itemize}

\paragraph{Elements of Component}\mbox{}
\label{sec:Elements of Component}

\tbl{elements of Component} lists the elements of \texttt{Component}.

\begin{table}[ht]
\centering 
  \caption{Elements of Component}
  \label{table:elements of Component}
\tabulinesep=3pt
\begin{tabu} to 6in {|l|l|l|} \everyrow{\hline}
\hline
\rowfont\bfseries {Association Name} & {Element} & {Multiplicity} \\
\tabucline[1.5pt]{}
\texttt{Description} & \texttt{Description} & 0..1 \\
\texttt{Compositions} & \texttt{Composition} & 0..* \\
\texttt{Components} & \texttt{Component} & 0..* \\
\texttt{Configuration} & \texttt{Configuration} & 0..* \\
\texttt{DataItems} & \texttt{DataItem} & 0..* \\
\texttt{References} & \texttt{Reference} & 0..* \\
\end{tabu}
\end{table}
\FloatBarrier


Descriptions for elements of \texttt{Component}:

\begin{itemize}
\item \texttt{Description} : An element that can contain any descriptive content.
\item \texttt{Compositions} : \block{Compositions} \gls{organizes} \block{Composition} elements.
\item \texttt{Components} : \block{Components} \gls{organizes} \block{Component} elements.
\item \texttt{Configuration} : \block{Configuration} contains technical information about a piece of equipment describing its physical layout or functional characteristics.
\item \texttt{DataItems} : \block{DataItems} \gls{organizes} \block{DataItem} elements. See \ref{DataItems}
\item \texttt{References} : \block{References} \gls{organizes} \block{Reference} elements associated with this \block{Component} element.
\end{itemize}
\FloatBarrier

\subsubsection{Description}
  \label{sec:Description}


An element that can contain any descriptive content.


\paragraph{Attributes of Description}\mbox{}
\label{sec:Attributes of Description}

\tbl{attributes of Description} lists the attributes of \texttt{Description}.

\begin{table}[ht]
\centering 
  \caption{Attributes of Description}
  \label{table:attributes of Description}
\tabulinesep=3pt
\begin{tabu} to 6in {|l|l|l|} \everyrow{\hline}
\hline
\rowfont\bfseries {Attribute} & {Type} & {Multiplicity} \\
\tabucline[1.5pt]{}
\texttt{manufacturer} & \texttt{string} & 0..1 \\
\texttt{model} & \texttt{string} & 0..1 \\
\texttt{serialNumber} & \texttt{string} & 0..1 \\
\texttt{station} & \texttt{string} & 0..1 \\
\texttt{value} & \texttt{string} & 0..1 \\
\end{tabu}
\end{table}
\FloatBarrier


Descriptions for attributes of \texttt{Description}:

\begin{itemize}
\item \texttt{manufacturer} : The name of the manufacturer of the physical or logical part of a piece of equipment represented by an XML element.
\item \texttt{model} : The model description of the physical part or logical function of a piece of equipment represented by this XML element.
\item \texttt{serialNumber} : The serial number associated with a piece of equipment.
\item \texttt{station} : The station where the physical part or logical function of a piece of equipment is located when it is part of a manufacturing unit or cell with multiple stations.
\item \texttt{value} : 
\end{itemize}
\FloatBarrier
