% Generated 2020-10-22 16:25:25 +0530
\subsection{DataItem Types} \label{sec:DataItem Types}


In the MTConnect Standard, \block{DataItem} elements are defined and organized based upon the \property{category} and \property{type} attributes.  The \property{category} attribute provides a high level grouping for \block{DataItem} elements based on the kind of information that is reported by the data item.

These categories are:

\begin{itemize}

\item \block{SAMPLE}

A \block{SAMPLE} reports a continuously variable or analog data value. 

\item \block{EVENT}

An \block{EVENT} reports information representing a functional state, with two or more discrete values, associated with a component or it contains a message.  The data provided may be a numeric value or text.

\item \block{CONDITION}

A \block{CONDITION} reports information about the health of a piece of equipment and its ability to function.
\end{itemize}

The \property{type} attribute specifies the specific kind of data that is reported.   For some types of data items, a \property{subType} attribute may also be used to differentiate between multiple data items of the same \property{type} where the information reported by the data item has a different, but related, meaning.

Many types of data items provide two forms of data: a value (reported as either a \block{SAMPLE} or \block{EVENT}) and a health status (reported as a \block{CONDITION}).  These \block{DataItem} types \textbf{MAY} be defined in more than one \property{category} based on the data that they report.



\subsubsection{Condition}
\label{sec:Condition}



\block{CONDITION} category data items report data representing a \gls{Structural Element}’s status regarding its ability to operate or it provides an indication whether the data reported for the \gls{Structural Element} is within an expected range.

\block{CONDITION} is reported differently than \block{SAMPLE} or \block{EVENT}.  \block{CONDITION} \textbf{MUST} be reported as \block{Normal}, \block{Warning}, or \block{Fault}.

All \block{DataItem} types in the \block{SAMPLE} category \textbf{MAY} have associated \block{CONDITION} states.  \block{CONDITION} states indicate whether the value for the data is within an expected range and \textbf{MUST} be reported as \block{Normal}, or the value is unexpected or out of tolerance for the data and a \block{Warning} or \block{Fault} \textbf{MUST} be provided.


The value of \property{type}{DataItem} with \property{category}{DataItem} \texttt{Condition} \MUST be one of the following:
\begin{itemize}


\item \texttt{COMMUNICATIONS}  

An indication that the piece of equipment has experienced a communications failure.


\item \texttt{DATA\textunderscore RANGE}  

An indication that the value of the data associated with a measured value or a calculation is outside of an expected range.


\item \texttt{HARDWARE}  

An indication of a fault associated with the hardware subsystem of the \gls{Structural Element}.


\item \texttt{LOGIC\textunderscore PROGRAM}  

An indication that an error occurred in the logic program or programmable logic controller (PLC) associated with a piece of equipment.


\item \texttt{MOTION\textunderscore PROGRAM}  

An indication that an error occurred in the motion program associated with a piece of equipment.


\item \texttt{SYSTEM}  

A general purpose indication associated with an electronic component of a piece of equipment or a controller that represents a fault that is not associated with the operator, program, or hardware.

\end{itemize}


\subsubsection{Event}
\label{sec:Event}



\block{DataItem} types in the \block{EVENT} category represent a discrete piece of information from a piece of equipment.  \block{EVENT} does not have intermediate values that vary over time.

An \block{EVENT} is information that, when provided at any specific point in time, represents the current state of the piece of equipment.


The value of \property{type}{DataItem} with \property{category}{DataItem} \texttt{Event} \MUST be one of the following:
\begin{itemize}


\item \texttt{ACTIVE\textunderscore AXES}  

The set of axes currently associated with a \block{Path} or \block{Controller} \gls{Structural Element}.


\item \texttt{ACTUATOR\textunderscore STATE}  

Represents the operational state of an apparatus for moving or controlling a mechanism or system.


\item \texttt{ALARM}  

\textbf{DEPRECATED:} Replaced with \block{CONDITION} category data items in Version 1.1.0.


\item \texttt{ASSET\textunderscore CHANGED}  

The \block{assetId} of the asset that has been added or changed.


\item \texttt{ASSET\textunderscore REMOVED}  

The \block{assetId} of the asset that has been removed.


\item \texttt{AVAILABILITY}  

Represents the \gls{Agent}'s ability to communicate with the data source.


\item \texttt{AXIS\textunderscore COUPLING}  

Describes the way the axes will be associated to each other. 
  
This is used in conjunction with \block{COUPLED\textunderscore AXES} to indicate the way they are interacting.


\item \texttt{AXIS\textunderscore FEEDRATE\textunderscore OVERRIDE}  

The value of a signal or calculation issued to adjust the feedrate of an individual linear type axis.


\item \texttt{AXIS\textunderscore INTERLOCK}  

An indicator of the state of the axis lockout function when power has been removed and the axis is allowed to move freely.


\item \texttt{AXIS\textunderscore STATE}  

An indicator of the controlled state of a \block{Linear} or \block{Rotary} component representing an axis.


\item \texttt{BLOCK}  

The line of code or command being executed by a \block{Controller} \gls{Structural Element}.


\item \texttt{BLOCK\textunderscore COUNT}  

The total count of the number of blocks of program code that have been executed since execution started.


\item \texttt{CHUCK\textunderscore INTERLOCK}  

An indication of the state of an interlock function or control logic state intended to prevent the associated \block{Chuck} component from being operated.


\item \texttt{CHUCK\textunderscore STATE}  

An indication of the operating state of a mechanism that holds a part or stock material during a manufacturing process. It may also represent a mechanism that holds any other mechanism in place within a piece of equipment.


\item \texttt{CLOSE\textunderscore CHUCK}  

Service to close a chuck.


\item \texttt{CLOSE\textunderscore DOOR}  

Service to close a door.


\item \texttt{CODE}  

\textbf{DEPRECATED} in Version 1.1.


\item \texttt{COMPOSITION\textunderscore STATE}  

An indication of the operating condition of a mechanism represented by a \block{Composition} type element.


\item \texttt{CONTROLLER\textunderscore MODE}  

The current operating mode of the \block{Controller} component.


\item \texttt{CONTROLLER\textunderscore MODE\textunderscore OVERRIDE}  

A setting or operator selection that changes the behavior of a piece of equipment.


\item \texttt{COUPLED\textunderscore AXES}  

Refers to the set of associated axes.


\item \texttt{DATE\textunderscore CODE}  

The time and date code associated with a material or other physical item.


\item \texttt{DEVICE\textunderscore UUID}  

The identifier of another piece of equipment that is temporarily associated with a component of this piece of equipment to perform a particular function.


\item \texttt{DIRECTION}  

The direction of motion.


\item \texttt{DOOR\textunderscore STATE}  

The operational state of a \block{Door} component or composition element.


\item \texttt{EMERGENCY\textunderscore STOP}  

The current state of the emergency stop signal for a piece of equipment, controller path, or any other component or subsystem of a piece of equipment.


\item \texttt{END\textunderscore OF\textunderscore BAR}  

An indication of whether the end of a piece of bar stock being feed by a bar feeder has been reached.


\item \texttt{EQUIPMENT\textunderscore MODE}  

An indication that a piece of equipment, or a sub-part of a piece of equipment, is performing specific types of activities.


\item \texttt{EXECUTION}  

The execution status of the \block{Component}.


\item \texttt{FUNCTIONAL\textunderscore MODE}  

The current intended production status of the device or component.


\item \texttt{HARDNESS}  

The measurement of the hardness of a material.


\item \texttt{INTERFACE\textunderscore STATE}  

An indication of the operational state of an \block{Interface} component.


\item \texttt{LINE}  

\textbf{DEPRECATED} in Version 1.4.0.


\item \texttt{LINE\textunderscore LABEL}  

An optional identifier for a \block{BLOCK} of code in a \block{PROGRAM}.


\item \texttt{LINE\textunderscore NUMBER}  

A reference to the position of a block of program code within a control program.


\item \texttt{MATERIAL}  

The identifier of a material used or consumed in the manufacturing process.


\item \texttt{MATERIAL\textunderscore CHANGE}  

Service to change the type of material or product being loaded or fed to a piece of equipment.


\item \texttt{MATERIAL\textunderscore FEED}  

Service to advance material or feed product to a piece of equipment from a continuous or bulk source.


\item \texttt{MATERIAL\textunderscore LAYER}  

Identifies the layers of material applied to a part or product as part of an additive manufacturing process.


\item \texttt{MATERIAL\textunderscore LOAD}  

Service to load a piece of material or product.


\item \texttt{MATERIAL\textunderscore RETRACT}  

Service to remove or retract material or product.


\item \texttt{MATERIAL\textunderscore UNLOAD}  

Service to unload a piece of material or product.


\item \texttt{MESSAGE}  

Any text string of information to be transferred from a piece of equipment to a client software application.


\item \texttt{OPEN\textunderscore CHUCK}  

Service to open a chuck.


\item \texttt{OPEN\textunderscore DOOR}  

Service to open a door.


\item \texttt{OPERATOR\textunderscore ID}  

The identifier of the person currently responsible for operating the piece of equipment.


\item \texttt{PALLET\textunderscore ID}  

The identifier for a pallet.


\item \texttt{PART\textunderscore CHANGE}  

Service to change the part or product associated with a piece of equipment to a different part or product.


\item \texttt{PART\textunderscore COUNT}  

The aggregate count of parts.


\item \texttt{PART\textunderscore DETECT}  

An indication designating whether a part or work piece has been detected or is present.


\item \texttt{PART\textunderscore ID}  

An identifier of a part in a manufacturing operation.


\item \texttt{PART\textunderscore NUMBER}  

An identifier of a part or product moving through the manufacturing process.


\item \texttt{PATH\textunderscore FEEDRATE\textunderscore OVERRIDE}  

The value of a signal or calculation issued to adjust the feedrate for the axes associated with a \block{Path} component that may represent a single axis or the coordinated movement of multiple axes.


\item \texttt{PATH\textunderscore MODE}  

Describes the operational relationship between a \block{Path} \gls{Structural Element} and another \block{Path} \gls{Structural Element} for pieces of equipment comprised of multiple logical groupings of controlled axes or other logical operations.


\item \texttt{POWER\textunderscore STATE}  

The indication of the status of the source of energy for a \gls{Structural Element} to allow it to perform its intended function or the state of an enabling signal providing permission for the \gls{Structural Element} to perform its functions.


\item \texttt{POWER\textunderscore STATUS}  

\textbf{DEPRECATED} in Version 1.1.0.


\item \texttt{PROCESS\textunderscore TIME}  

The time and date associated with an activity or event.


\item \texttt{PROGRAM}  

The name of the logic or motion program being executed by the \block{Controller} component.


\item \texttt{PROGRAM\textunderscore COMMENT}  

A comment or non-executable statement in the control program.


\item \texttt{PROGRAM\textunderscore EDIT}  

An indication of the status of the \block{Controller} components program editing mode. A program may be edited while another is executed.


\item \texttt{PROGRAM\textunderscore EDIT\textunderscore NAME}  

The name of the program being edited. 
 This is used in conjunction with \block{PROGRAM\textunderscore EDIT} when in \block{ACTIVE} state. 


\item \texttt{PROGRAM\textunderscore HEADER}  

The non-executable header section of the control program.


\item \texttt{PROGRAM\textunderscore LOCATION}  

The Uniform Resource Identifier (URI) for the source file associated with \block{PROGRAM}.


\item \texttt{PROGRAM\textunderscore LOCATION\textunderscore TYPE}  

Defines whether the logic or motion program defined by \block{PROGRAM} is being executed from the local memory of the controller or from an outside source.


\item \texttt{PROGRAM\textunderscore NEST\textunderscore LEVEL}  

An indication of the nesting level within a control program that is associated with the code or instructions that is currently being executed.


\item \texttt{ROTARY\textunderscore MODE}  

The current operating mode for a \block{Rotary} type axis.


\item \texttt{ROTARY\textunderscore VELOCITY\textunderscore OVERRIDE}  

The percentage change to the velocity of the programmed velocity for a \block{Rotary} type axis.


\item \texttt{SERIAL\textunderscore NUMBER}  

The serial number associated with a \block{Component}, \block{Asset}, or \block{Device}.


\item \texttt{SPINDLE\textunderscore INTERLOCK}  

An indication of the status of the spindle for a piece of equipment when power has been removed and it is free to rotate.


\item \texttt{TOOL\textunderscore ASSET\textunderscore ID}  

The identifier of an individual tool asset.


\item \texttt{TOOL\textunderscore GROUP}  

An identifier for the tool group associated with a specific tool. Commonly used to designate spare tools.


\item \texttt{TOOL\textunderscore ID}  

\textbf{DEPRECATED} in Version 1.2.0.   See \block{TOOL\textunderscore ASSET\textunderscore ID}. \textit{DEPRECATED:The identifier of the tool currently in use for a given \block{Path}.}


\item \texttt{TOOL\textunderscore NUMBER}  

The identifier assigned by the \block{Controller} component to a cutting tool when in use by a piece of equipment.


\item \texttt{TOOL\textunderscore OFFSET}  

A reference to the tool offset variables applied to the active cutting tool associated with a \block{Path} in a \block{Controller} type component.


\item \texttt{USER}  

The identifier of the person currently responsible for operating the piece of equipment.


\item \texttt{VARIABLE}  

A data value whose meaning may change over time due to changes in the opertion of a piece of equipment or the process being executed on that piece of equipment.


\item \texttt{WAIT\textunderscore STATE}  

An indication of the reason that \block{EXECUTION} is reporting a value of \block{WAIT}.


\item \texttt{WIRE}  

A string like piece or filament of relatively rigid or flexible material provided in a variety of diameters.


\item \texttt{WORKHOLDING\textunderscore ID}  

The identifier for the current workholding or part clamp in use by a piece of equipment.


\item \texttt{WORK\textunderscore OFFSET}  

A reference to the offset variables for a work piece or part associated with a \block{Path} in a \block{Controller} type component.


\item \texttt{OPERATING\textunderscore SYSTEM}  

The Operating System of a component.


\item \texttt{FIRMWARE}  

The embedded software of a component.



\item \texttt{APPLICATION}  

The application on a component.


\item \texttt{LIBRARY}  

The software library on a component


\item \texttt{HARDWARE}  

The hardware of a component.



\item \texttt{NETWORK}  

Network details of a component.


\item \texttt{ROTATION}  

A three space angular rotation relative to a coordinate system.


\item \texttt{TRANSLATION}  

A three space linear translation relative to a coordinate system.

\end{itemize}


\subsubsection{Sample}
\label{sec:Sample}



The types of \block{DataItem} elements in the \block{SAMPLE} category report data representing a continuously changing or analog data value.

This data can be measured at any point-in-time and will always produce a result.


The value of \property{type}{DataItem} with \property{category}{DataItem} \texttt{Sample} \MUST be one of the following:
\begin{itemize}


\item \texttt{ACCELERATION}  

The measurement of the rate of change of velocity.


\item \texttt{ACCUMULATED\textunderscore TIME}  

The measurement of accumulated time for an activity or event.


\item \texttt{AMPERAGE}  

The measurement of electrical current.


\item \texttt{ANGLE}  

The measurement of angular position.


\item \texttt{ANGULAR\textunderscore ACCELERATION}  

The measurement rate of change of angular velocity.


\item \texttt{ANGULAR\textunderscore VELOCITY}  

The measurement of the rate of change of angular position.


\item \texttt{AXIS\textunderscore FEEDRATE}  

The measurement of the feedrate of a linear axis.


\item \texttt{CAPACITY\textunderscore FLUID}  

The fluid capacity of an object or container.


\item \texttt{CAPACITY\textunderscore SPATIAL}  

The geometric capacity of an object or container.


\item \texttt{CLOCK\textunderscore TIME}  

The value provided by a timing device at a specific point in time.


\item \texttt{CONCENTRATION}  

The measurement of the percentage of one component within a mixture of components


\item \texttt{CONDUCTIVITY}  

The measurement of the ability of a material to conduct electricity.


\item \texttt{CUTTING\textunderscore SPEED}  

The speed difference (relative velocity) between the cutting mechanism and the surface of the workpiece it is operating on.


\item \texttt{DENSITY}  

The volumetric mass of a material per unit volume of that material.


\item \texttt{DEPOSITION\textunderscore ACCELERATION\textunderscore VOLUMETRIC}  

The rate of change in spatial volume of material deposited in an additive manufacturing process.


\item \texttt{DEPOSITION\textunderscore DENSITY}  

The density of the material deposited in an additive manufacturing process per unit of volume.


\item \texttt{DEPOSITION\textunderscore MASS}  

The mass of the material deposited in an additive manufacturing process.


\item \texttt{DEPOSITION\textunderscore RATE\textunderscore VOLUMETRIC}  

The rate at which a spatial volume of material is deposited in an additive manufacturing process.


\item \texttt{DEPOSITION\textunderscore VOLUME}  

The spatial volume of material to be deposited in an additive manufacturing process.


\item \texttt{DISPLACEMENT}  

The measurement of the change in position of an object.


\item \texttt{ELECTRICAL\textunderscore ENERGY}  

The measurement of electrical energy consumption by a component.


\item \texttt{EQUIPMENT\textunderscore TIMER}  

The measurement of the amount of time a piece of equipment or a sub-part of a piece of equipment has performed specific activities.


\item \texttt{FILL\textunderscore LEVEL}  

The measurement of the amount of a substance remaining compared to the planned maximum amount of that substance.


\item \texttt{FLOW}  

The measurement of the rate of flow of a fluid.


\item \texttt{FREQUENCY}  

The measurement of the number of occurrences of a repeating event per unit time.


\item \texttt{GLOBAL\textunderscore POSITION}  

\textbf{DEPRECATED} in Version 1.1


\item \texttt{LENGTH}  

The measurement of the length of an object.


\item \texttt{LEVEL}  

\textbf{DEPRECATED} in Version 1.2.  See \block{FILL\textunderscore LEVEL}


\item \texttt{LINEAR\textunderscore FORCE}  

The measurement of the push or pull introduced by an actuator or exerted on an object.


\item \texttt{LOAD}  

The measurement of the actual versus the standard rating of a piece of equipment.


\item \texttt{MASS}  

The measurement of the mass of an object(s) or an amount of material.


\item \texttt{PATH\textunderscore FEEDRATE}  

The measurement of the feedrate for the axes, or a single axis, associated with a \block{Path} component-a vector.


\item \texttt{PATH\textunderscore FEEDRATE\textunderscore PER\textunderscore REVOLUTION}  

The feedrate for the axes, or a single axis.


\item \texttt{PATH\textunderscore POSITION}  

A measured or calculated position of a control point associated with a \block{Controller} element, or \block{Path} element if provided, of a piece of equipment.


\item \texttt{PH}  

A measure of the acidity or alkalinity of a solution.


\item \texttt{POSITION}  

A measured or calculated position of a \block{Component} element as reported by a piece of equipment.


\item \texttt{POWER\textunderscore FACTOR}  

The measurement of the ratio of real power flowing to a load to the apparent power in that AC circuit.


\item \texttt{PRESSURE}  

The measurement of force per unit area exerted by a gas or liquid.


\item \texttt{PROCESS\textunderscore TIMER}  

The measurement of the amount of time a piece of equipment has performed different types of activities associated with the process being performed at that piece of equipment.


\item \texttt{RESISTANCE}  

The measurement of the degree to which a substance opposes the passage of an electric current.


\item \texttt{ROTARY\textunderscore VELOCITY}  

The measurement of the rotational speed of a rotary axis.


\item \texttt{SOUND\textunderscore LEVEL}  

The measurement of a sound level or sound pressure level relative to atmospheric pressure.


\item \texttt{SPINDLE\textunderscore SPEED}  

\textbf{DEPRECATED} in Version 1.2.  Replaced by \block{ROTARY\textunderscore VELOCITY}


\item \texttt{STRAIN}  

The measurement of the amount of deformation per unit length of an object when a load is applied.


\item \texttt{TEMPERATURE}  

The measurement of temperature.


\item \texttt{TENSION}  

The measurement of a force that stretches or elongates an object.


\item \texttt{TILT}  

The measurement of angular displacement.


\item \texttt{TORQUE}  

The measurement of the turning force exerted on an object or by an object.


\item \texttt{VELOCITY}  

The measurement of the rate of change of position of a \block{Component}.


\item \texttt{VISCOSITY}  

The measurement of a fluids resistance to flow.


\item \texttt{VOLTAGE}  

The measurement of electrical potential between two points.


\item \texttt{VOLT\textunderscore AMPERE}  

The measurement of the apparent power in an electrical circuit, equal to the product of root-mean-square (RMS) voltage and RMS current (commonly referred to as VA).


\item \texttt{VOLT\textunderscore AMPERE\textunderscore REACTIVE}  

The measurement of reactive power in an AC electrical circuit (commonly referred to as VAR).


\item \texttt{VOLUME\textunderscore FLUID}  

The fluid volume of an object or container.


\item \texttt{VOLUME\textunderscore SPATIAL}  

The geometric volume of an object or container.


\item \texttt{WATTAGE}  

The measurement of power flowing through or dissipated by an electrical circuit or piece of equipment.


\item \texttt{AMPERAGE\textunderscore AC}  

The measurement of an electrical current that reverses direction at regular short intervals.


\item \texttt{AMPERAGE\textunderscore DC}  

The measurement of an electric current flowing in one direction only.


\item \texttt{VOLTAGE\textunderscore AC}  

The measurement of the electrical potential between two points in an electrical circuit in which the current periodically reverses direction.


\item \texttt{VOLTAGE\textunderscore DC}  

The measurement of the electrical potential between two points in an electrical circuit in which the current is unidirectional.


\item \texttt{X\textunderscore DIMENSION}  

Measured dimension of an entity relative to the X direction of the referenced coordinate system.



\item \texttt{Y\textunderscore DIMENSION}  

Measured dimension of an entity relative to the Y direction of the referenced coordinate system.


\item \texttt{Z\textunderscore DIMENSION}  

Measured dimension of an entity relative to the Z direction of the referenced coordinate system.


\item \texttt{DIAMETER}  

The measured dimension of a diameter.


\item \texttt{ORIENTATION}  

A measured or calculated orientation of a plane or vector relative to a cartesian coordinate system.


\item \texttt{HUMIDITY\textunderscore RELATIVE}  

The amount of water vapor present expressed as a percent to reach saturation at the same temperature.


\item \texttt{HUMIDITY\textunderscore ABSOLUTE}  

The amount of water vapor expressed in grams per cubic meter.


\item \texttt{HUMIDITY\textunderscore SPECIFIC}  

The ratio of the water vapor present over the total weight of the water vapor and air present expressed as a percent.

\end{itemize}

