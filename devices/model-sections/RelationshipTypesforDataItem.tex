% Generated 2021-02-25 22:49:46 +0530
\subsection{Relationship Types for DataItem} \label{sec:Relationship Types for DataItem}


See \sect{Relationships} for details on the \block{Relationship} model.


\subsubsection{SpecificationRelationship}
\label{sec:SpecificationRelationship}



A \block{Relationship} providing a semantic reference to another \block{Specification} described by the \property{type} and \property{idRef} property.


\paragraph{Attributes of SpecificationRelationship}\mbox{}
\label{sec:Attributes of SpecificationRelationship}

\tbl{Attributes of SpecificationRelationship} lists the attributes of \texttt{SpecificationRelationship}.

\begin{table}[ht]
\centering 
  \caption{Attributes of SpecificationRelationship}
  \label{table:Attributes of SpecificationRelationship}
\tabulinesep=3pt
\begin{tabu} to 6in {|l|l|l|} \everyrow{\hline}
\hline
\rowfont\bfseries {Attribute} & {Type} & {Multiplicity} \\
\tabucline[1.5pt]{}

\property{type}[SpecificationRelationship] & \texttt{SpecificationRelationshipTypeEnum} & 1 \\
\property{idRef}[SpecificationRelationship] & \texttt{NMTOKEN} & 1 \\
\end{tabu}
\end{table}
\FloatBarrier

Descriptions for attributes of \block{SpecificationRelationship}:

\begin{itemize}

\item \property{type}[SpecificationRelationship] \newline Specifies how the \block{Specification} is related.

\texttt{SpecificationRelationshipTypeEnum} Enumeration:

\begin{itemize}
\item \texttt{LIMIT} \newline The referenced \block{Specification} provides process limits.
 
\end{itemize}


\item \property{idRef}[SpecificationRelationship] \newline A reference to the related \block{Specification} \property{id}.

\end{itemize}



\subsubsection{DataItemRelationship}
\label{sec:DataItemRelationship}



A \block{Relationship} providing a semantic reference to another \block{DataItem} described by the \property{type} property.


\paragraph{Attributes of DataItemRelationship}\mbox{}
\label{sec:Attributes of DataItemRelationship}

\tbl{Attributes of DataItemRelationship} lists the attributes of \texttt{DataItemRelationship}.

\begin{table}[ht]
\centering 
  \caption{Attributes of DataItemRelationship}
  \label{table:Attributes of DataItemRelationship}
\tabulinesep=3pt
\begin{tabu} to 6in {|l|l|l|} \everyrow{\hline}
\hline
\rowfont\bfseries {Attribute} & {Type} & {Multiplicity} \\
\tabucline[1.5pt]{}

\property{type}[DataItemRelationship] & \texttt{DataItemRelationshipTypeEnum} & 1 \\
\property{idRef}[DataItemRelationship] & \texttt{NMTOKEN} & 1 \\
\end{tabu}
\end{table}
\FloatBarrier

Descriptions for attributes of \block{DataItemRelationship}:

\begin{itemize}

\item \property{type}[DataItemRelationship] \newline Specifies how the \block{DataItem} is related.

\texttt{DataItemRelationshipTypeEnum} Enumeration:

\begin{itemize}
\item \texttt{ATTACHMENT} \newline A reference to a \block{DataItem} that associates the values with an external entity.
 
\item \texttt{COORDINATE\textunderscore SYSTEM} \newline  The referenced \block{DataItem} provides the \property{id} of the effective Coordinate System.
 
\item \texttt{LIMIT} \newline The referenced \block{DataItem} provides process limits.
 
\item \texttt{OBSERVATION} \newline The referenced \block{DataItem} provides the observed values.
 
\end{itemize}


\item \property{idRef}[DataItemRelationship] \newline A reference to the related \block{DataItem} \property{id}.

\end{itemize}


