% Generated 2020-08-19 14:44:34 +0530
\subsection{Relationships} \label{sec:Relationships}

\subsubsection{Relationship}
  \label{sec:Relationship}



\block{Relationship} describes the association between two pieces of equipment that function independently but together perform a manufacturing operation.


\paragraph{Attributes of Relationship}\mbox{}
\label{sec:Attributes of Relationship}

\tbl{attributes of Relationship} lists the attributes of \texttt{Relationship}.

\begin{table}[ht]
\centering 
  \caption{Attributes of Relationship}
  \label{table:attributes of Relationship}
\tabulinesep=3pt
\begin{tabu} to 6in {|l|l|l|} \everyrow{\hline}
\hline
\rowfont\bfseries {Attribute} & {Type} & {Multiplicity} \\
\tabucline[1.5pt]{}
\texttt{criticality} & \texttt{criticalityType} & 0..1 \\
\texttt{id} & \texttt{ID} & 1 \\
\texttt{name} & \texttt{string} & 0..1 \\
\texttt{type} & \texttt{RelationshipType} & 1 \\
\end{tabu}
\end{table}
\FloatBarrier


Descriptions for attributes of \texttt{Relationship}:

\begin{itemize}
\item \texttt{criticality} : Defines whether the services or functions provided by the associated piece of equipment is required for the operation of this piece of equipment.
\tabulinesep = 5pt
\begin{longtabu} to \textwidth {
    |l|X|}
  \caption{criticalityType Enumeration}
  \label{enum:criticalityType} \\
\hline
Name & Description \\
\hline
\endfirsthead
\hline
\multicolumn{2}{|c|}{Continuation of Table \texttt{criticalityType} Enumeration} \\
\hline
Name & Description \\
\hline
\endhead
\texttt{CRITICAL} & The services or functions provided by the associated element is required for the operation of this element. \\ \hline
\texttt{NONCRITICAL} & The services or functions provided by the associated element is not required for the operation of this element. \\ \hline
\end{longtabu}
\FloatBarrier
\item \texttt{id} : The unique identifier for this \block{Relationship}.
\item \texttt{name} : The name associated with this \block{Relationship}.
\item \texttt{type} : Defines the authority that this piece of equipment has relative to the associated piece of equipment.
\end{itemize}
\FloatBarrier

\subsubsection{ComponentRelationship}
  \label{sec:ComponentRelationship}



\block{ComponentRelationship} describes the association between two components within a piece of equipment that function independently but together perform a capability or service within a piece of equipment.


\paragraph{Attributes of ComponentRelationship}\mbox{}
\label{sec:Attributes of ComponentRelationship}

\tbl{attributes of ComponentRelationship} lists the attributes of \texttt{ComponentRelationship}.

\begin{table}[ht]
\centering 
  \caption{Attributes of ComponentRelationship}
  \label{table:attributes of ComponentRelationship}
\tabulinesep=3pt
\begin{tabu} to 6in {|l|l|l|} \everyrow{\hline}
\hline
\rowfont\bfseries {Attribute} & {Type} & {Multiplicity} \\
\tabucline[1.5pt]{}
\texttt{idRef} & \texttt{ID} & 1 \\
\end{tabu}
\end{table}
\FloatBarrier


Descriptions for attributes of \texttt{ComponentRelationship}:

\begin{itemize}
\item \texttt{idRef} : A reference to the associated component element.
\end{itemize}
\FloatBarrier

\subsubsection{DeviceRelationship}
  \label{sec:DeviceRelationship}



\block{DeviceRelationship} describes the association between two pieces of equipment that function independently but together perform a manufacturing operation.


\paragraph{Attributes of DeviceRelationship}\mbox{}
\label{sec:Attributes of DeviceRelationship}

\tbl{attributes of DeviceRelationship} lists the attributes of \texttt{DeviceRelationship}.

\begin{table}[ht]
\centering 
  \caption{Attributes of DeviceRelationship}
  \label{table:attributes of DeviceRelationship}
\tabulinesep=3pt
\begin{tabu} to 6in {|l|l|l|} \everyrow{\hline}
\hline
\rowfont\bfseries {Attribute} & {Type} & {Multiplicity} \\
\tabucline[1.5pt]{}
\texttt{deviceUuidRef} & \texttt{NMTOKEN} & 1 \\
\texttt{href} & \texttt{xlinkhref} & 0..1 \\
\texttt{role} & \texttt{roleType} & 0..1 \\
\texttt{xlink:type} & \texttt{xlinktype} & 0..1 \\
\end{tabu}
\end{table}
\FloatBarrier


Descriptions for attributes of \texttt{DeviceRelationship}:

\begin{itemize}
\item \texttt{deviceUuidRef} : A reference to the associated piece of equipment. 

\item \texttt{href} : A URI identifying the \gls{Agent} that is publishing information for the associated piece of equipment. \property{href} \textbf{MUST} also include the UUID for that specific piece of equipment.

\property{href} is of type \texttt{xlink:href} from the W3C XLink specification: (https://www.w3.org/TR/xlink11/).
\item \texttt{role} : Defines the services or capabilities that the referenced piece of equipment provides relative to this piece of equipment.
\tabulinesep = 5pt
\begin{longtabu} to \textwidth {
    |l|X|}
  \caption{roleType Enumeration}
  \label{enum:roleType} \\
\hline
Name & Description \\
\hline
\endfirsthead
\hline
\multicolumn{2}{|c|}{Continuation of Table \texttt{roleType} Enumeration} \\
\hline
Name & Description \\
\hline
\endhead
\texttt{SYSTEM} & The associated element performs the functions of a \block{System} for this element. \\ \hline
\texttt{AUXILIARY} & The associated element performs the functions as an \texttt{Auxiliary} for this element. \\ \hline
\end{longtabu}
\FloatBarrier
\item \texttt{xlink:type} : The XLink \property{type} attribute \textbf{MUST} have a fixed value of \texttt{locator} as defined in W3C XLink 1.1 https://www.w3.org/TR/xlink11/ \textit{section 5.4 Locator Attribute (\property{href})}.

If the \property{href} attribute is provided, it \textbf{MUST} conform to the URI syntactic rules as defined in IETF RFC 3986 for Uniform Resource Identifiers. (https://www.ietf.org/rfc/rfc3986.txt)
\end{itemize}
\FloatBarrier

\subsubsection{Relationships}
  \label{sec:Relationships}



\block{Relationships} \glspl{organize} \block{Relationship} elements.


\paragraph{Attributes of Relationships}\mbox{}
\label{sec:Attributes of Relationships}

\tbl{attributes of Relationships} lists the attributes of \texttt{Relationships}.

\begin{table}[ht]
\centering 
  \caption{Attributes of Relationships}
  \label{table:attributes of Relationships}
\tabulinesep=3pt
\begin{tabu} to 6in {|l|l|l|} \everyrow{\hline}
\hline
\rowfont\bfseries {Attribute} & {Type} & {Multiplicity} \\
\tabucline[1.5pt]{}
\texttt{isRelatedTo} & \texttt{Relationship} & 1..* \\
\end{tabu}
\end{table}
\FloatBarrier


Descriptions for attributes of \texttt{Relationships}:

\begin{itemize}
\item \texttt{isRelatedTo} : 
\end{itemize}
\FloatBarrier
