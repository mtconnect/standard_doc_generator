% Generated 2021-02-25 22:49:46 +0530
\subsection{DataItem SubTypes} \label{sec:DataItem SubTypes}


\property{subType} is an attribute of \block{DataItem} that provides a sub-categorization for the \property{type} attribute for a piece of information.


\subsubsection{DataItemSubType}
\label{sec:DataItemSubType}






The value of \property{subType}[DataItem] for \block{DataItem} \MUST be one of the following:

\begin{itemize}


\item \texttt{ABSOLUTE}  

Relating to or derived in the simplest manner from the fundamental units or measurements.


\item \texttt{ACTION}  

An indication of the operating state of a mechanism.


\item \texttt{ACTUAL}  

The measured value.


\item \texttt{ALL}  

The number of parts produced.


\item \texttt{ALTERNATING}  

The measurement of alternating voltage or current.   If not specified further in statistic, defaults to RMS voltage. 


\item \texttt{AUXILIARY}  

When multiple locations on a piece of bar stock being feed by a bar feeder are referenced as the indication of whether the end of that piece of bar stock has been reached.


\item \texttt{A\textunderscore SCALE}  

A Scale weighting factor.


\item \texttt{BAD}  

The number of parts produced that do not conform to specification.


\item \texttt{BATCH}  

A group of parts produced in a batch.


\item \texttt{BRINELL}  

A scale to measure the resistance to deformation of a surface.


\item \texttt{B\textunderscore SCALE}  

B Scale weighting factor.


\item \texttt{COMMANDED}  

The commanded value.


\item \texttt{COMPLETE}  

Associated with the completion of an activity or event.


\item \texttt{CONSUMED}  

The amount of material consumed from an object or container during a manufacturing process.


\item \texttt{CONTROL}  

The state of the enabling signal or control logic that enables or disables the function or operation of the structural element.


\item \texttt{C\textunderscore SCALE}  

C Scale weighting factor.


\item \texttt{DELAY}  

A piece of equipment or process waiting for an event or an action to occur.


\item \texttt{DIRECT}  

The measurement of DC current or voltage.


\item \texttt{DRY\textunderscore RUN}  

A setting or operator selection used to execute a test mode to confirm the execution of machine functions.


\item \texttt{D\textunderscore SCALE}  

D Scale weighting factor.


\item \texttt{EXPIRATION}  

The time and date code relating to the expiration or end of useful life for a material or other physical item.


\item \texttt{FIRST\textunderscore USE}  

The time and date code relating the first use of a material or other physical item.


\item \texttt{GATEWAY}  

The Gateway for the component network.


\item \texttt{GOOD}  

The number of parts produced that conform to specification.


\item \texttt{HEAT\textunderscore TREAT}  

A material heat number.


\item \texttt{INCREMENTAL}  

Relating to or derived from the last \gls{observation}.


\item \texttt{INSTALL\textunderscore DATE}  

The date the hardware or software was installed.


\item \texttt{IPV4\textunderscore ADDRESS}  

The IPV4 network address of the component.


\item \texttt{IPV6\textunderscore ADDRESS}  

The IPV6 network address of the component.


\item \texttt{ISO\textunderscore STEP\textunderscore EXECUTABLE}  

A reference to a ISO 10303 Executable.


\item \texttt{JOG}  

The feedrate specified by a logic or motion program when operating in a manual state or method (jogging).


\item \texttt{LATERAL}  

An indication of the position of a mechanism that may move in a lateral direction.


\item \texttt{LEEB}  

A scale to measure the elasticity of a surface.


\item \texttt{LENGTH}  

A reference to a length type tool offset variable.

Subtypes of \texttt{LENGTH}: \texttt{STANDARD}, \texttt{REMAINING}, \texttt{USEABLE}.

\item \texttt{LICENSE}  

The license code to validate or activate the hardware or software.


\item \texttt{LINE}  

The state of the power source.

Subtypes of \texttt{LINE}: \texttt{MAXIMUM}, \texttt{MINIMUM}.

\item \texttt{LINEAR}  

The direction of motion of a linear motion.


\item \texttt{LOADED}  

Subparts of a piece of equipment are under load.


\item \texttt{LOT}  

A group of parts tracked as a lot.


\item \texttt{MACHINE\textunderscore AXIS\textunderscore LOCK}  

A setting or operator selection that changes the behavior of the controller on a piece of equipment.


\item \texttt{MAC\textunderscore ADDRESS}  

Media Access Control Address. The unique physical address of the network hardware.


\item \texttt{MAIN}  

The identity of the primary logic or motion program currently being executed.


\item \texttt{MAINTENANCE}  

Relating to maintenance on the piece of equipment.


\item \texttt{MANUAL\textunderscore UNCLAMP}  

An indication of the state of an operator controlled interlock that can inhibit the ability to initiate an unclamp action of an electronically controlled chuck.


\item \texttt{MANUFACTURE}  

Related to the production of a material or other physical item.


\item \texttt{MANUFACTURER}  

The corporate identity for the maker of the hardware or software.


\item \texttt{MAXIMUM}  

The maximum value.


\item \texttt{MINIMUM}  

The minimum value.


\item \texttt{MOHS}  

A scale to measure the resistance to scratching of a surface.


\item \texttt{MOTION}  

An indication of the open or closed state of a mechanism.


\item \texttt{NO\textunderscore SCALE}  

No weighting factor on the frequency scale.


\item \texttt{OPERATING}  

A piece of equipment that is powered or performing any activity.


\item \texttt{OPERATOR}  

Relating to the person currently responsible for operating the piece of equipment.


\item \texttt{OPTIONAL\textunderscore STOP}  

A setting or operator selection that changes the behavior of the controller on a piece of equipment. 


\item \texttt{ORDER\textunderscore NUMBER}  

The authorization of a process occurrence.


\item \texttt{OVERRIDE}  

The overridden value.


\item \texttt{PART\textunderscore FAMILY}  

A group of parts having similarities in geometry, manufacturing process, and/or functions.


\item \texttt{PART\textunderscore NAME}  

A word or set of words by which a part is known, addressed, or referred to.


\item \texttt{PART\textunderscore NUMBER}  

A particular part design or model.


\item \texttt{POWERED}  

A piece of equipment is powered and functioning or \glspl{Component} that are required to remain on are powered.


\item \texttt{PRIMARY}  

The main or most important location \textbf{MUST} be designated as the end of a piece of bar stock.


\item \texttt{PROBE}  

The value provided by a measurement probe.


\item \texttt{PROCESS}  

Relating to production of a part or product on a piece of equipment.


\item \texttt{PROCESS\textunderscore NAME}  

A word or set of words by which a process being executed (process occurrence) by the device is known, addressed, or referred to.



\item \texttt{PROCESS\textunderscore PLAN}  

A process plan that a process occurrence belongs to.


\item \texttt{PROCESS\textunderscore STEP}  

A step in the process plan that this occurrence corresponds to. 


\item \texttt{PROGRAMMED}  

The programmed value.


\item \texttt{RADIAL}  

A reference to a radial type tool offset variable.


\item \texttt{RAPID}  

The feedrate specified by a logic or motion program when operating in a rapid positioning mode.


\item \texttt{RAW\textunderscore MATERIAL}  

A singular piece of material.


\item \texttt{RELEASE\textunderscore DATE}  

The date the hardware or software was released for general use.


\item \texttt{REMAINING}  

The remaining measure of an object or an action.


\item \texttt{ROCKWELL}  

A scale to measure the resistance to deformation of a surface.


\item \texttt{ROTARY}  

The direction of a rotary motion using the right hand rule convention.


\item \texttt{SCHEDULE}  

The identity of a control program that is used to specify the order of execution of other programs.


\item \texttt{SERIAL\textunderscore NUMBER}  

A serial number that uniquely identifies a specific part.


\item \texttt{SET\textunderscore UP}  

Relating to the preparation of a piece of equipment for production or restoring the piece of equipment to a neutral state after production.


\item \texttt{SHORE}  

A scale to measure the resistance to deformation of a surface.


\item \texttt{SINGLE\textunderscore BLOCK}  

A setting or operator selection that changes the behavior of the controller on a piece of equipment. 


\item \texttt{STANDARD}  

The standard measure of an object or an action.


\item \texttt{START}  

Relating to the beginning of an activity or event.


\item \texttt{SUBNET\textunderscore MASK}  

The SubNet mask for the component network.



\item \texttt{SWITCHED}  

An indication of the activation state of a mechanism represented by a \gls{Composition}.


\item \texttt{TARGET}  

The targeted or desired value.


\item \texttt{TARGET\textunderscore COMPLETION}  

Relating to the end or completion of an activity or event.


\item \texttt{TOOL\textunderscore CHANGE\textunderscore STOP}  

A setting or operator selection that changes the behavior of the controller on a piece of equipment.


\item \texttt{USEABLE}  

The remaining usable measure of an object or action.


\item \texttt{UUID}  

The globally unique identifier as specified in ISO 11578 or RFC 4122.


\item \texttt{VERSION}  

The version of the hardware or software.



\item \texttt{VERTICAL}  

An indication of the position of a mechanism that may move in a vertical direction.


\item \texttt{VICKERS}  

A scale to measure the resistance to deformation of a surface.


\item \texttt{VLAN\textunderscore ID}  

The layer2 Virtual Local Network (VLAN) ID for the component network.


\item \texttt{WIRELESS}  

Identifies whether the connection type is wireless.


\item \texttt{WORKING}  

A piece of equipment performing any activity, the equipment is active and performing a function under load or not.

\end{itemize}


