% Generated 2020-07-27 15:21:27 +0530
\subsection{CoordinateSystems} \label{sec:CoordinateSystems}

\subsubsection{CoordinateSystem}
  \label{sec:CoordinateSystem}


A {block:CoordinateSystem} is a reference system that associates a unique set of n parameters with each point in an n-dimensional space. {cite:Ref: ISO 10303-218:2004}


\paragraph{Attributes of CoordinateSystem}\mbox{}
\label{sec:Attributes of CoordinateSystem}

\tbl{attributes of CoordinateSystem} lists the attributes of \texttt{CoordinateSystem}.

\begin{table}[ht]
\centering 
  \caption{Attributes of CoordinateSystem}
  \label{table:attributes of CoordinateSystem}
\tabulinesep=3pt
\begin{tabu} to 6in {|l|l|l|} \everyrow{\hline}
\hline
\rowfont\bfseries {Attribute} & {Type} & {Multiplicity} \\
\tabucline[1.5pt]{}
\texttt{id} & \texttt{ID} & 1 \\
\texttt{name} & \texttt{string} & 0..1 \\
\texttt{nativeName} & \texttt{string} & 0..1 \\
\texttt{parentIdRef} & \texttt{IDREF} & 0..1 \\
\texttt{type} & \texttt{CoordinateSystemTypeEnum} & 1 \\
\end{tabu}
\end{table}
\FloatBarrier


Descriptions for attributes of \texttt{CoordinateSystem}:

\begin{itemize}
\item \texttt{id} : The unique identifier for this element.
\item \texttt{name} : The name of the coordinate system.
\item \texttt{nativeName} : The manufacturer's name or users name for the coordinate system.
\item \texttt{parentIdRef} : A pointer to the {property:id} attribute of the parent {block:CoordinateSystem}.
\item \texttt{type} : The type of coordinate system.
\end{itemize}

\paragraph{Elements of CoordinateSystem}\mbox{}
\label{sec:Elements of CoordinateSystem}

\tbl{elements of CoordinateSystem} lists the elements of \texttt{CoordinateSystem}.

\begin{table}[ht]
\centering 
  \caption{Elements of CoordinateSystem}
  \label{table:elements of CoordinateSystem}
\tabulinesep=3pt
\begin{tabu} to 6in {|l|l|l|} \everyrow{\hline}
\hline
\rowfont\bfseries {Association Name} & {Element} & {Multiplicity} \\
\tabucline[1.5pt]{}
\texttt{Origin} & \texttt{Origin} & 0..1 \\
\texttt{Transformation} & \texttt{Transformation} & 0..1 \\
\end{tabu}
\end{table}
\FloatBarrier


Descriptions for elements of \texttt{CoordinateSystem}:

\begin{itemize}
\item \texttt{Origin} : The coordinates of the origin position of a coordinate system.
\item \texttt{Transformation} :  The process of transforming to the origin position of the coordinate system from a parent coordinate system using \glselementname{translation event} and \glselementname{rotation event}.
\end{itemize}
\FloatBarrier

\subsubsection{CoordinateSystems}
  \label{sec:CoordinateSystems}


{block:CoordinateSystems} {term:organizes} {block:CoordinateSystem} elements for a {block:Component} and its children.


\paragraph{Elements of CoordinateSystems}\mbox{}
\label{sec:Elements of CoordinateSystems}

\tbl{elements of CoordinateSystems} lists the elements of \texttt{CoordinateSystems}.

\begin{table}[ht]
\centering 
  \caption{Elements of CoordinateSystems}
  \label{table:elements of CoordinateSystems}
\tabulinesep=3pt
\begin{tabu} to 6in {|l|l|l|} \everyrow{\hline}
\hline
\rowfont\bfseries {Association Name} & {Element} & {Multiplicity} \\
\tabucline[1.5pt]{}
\texttt{CoordinateSystem} & \texttt{CoordinateSystem} & 1..* \\
\end{tabu}
\end{table}
\FloatBarrier


Descriptions for elements of \texttt{CoordinateSystems}:

\begin{itemize}
\item \texttt{CoordinateSystem} : 
\end{itemize}
\FloatBarrier

\subsubsection{Origin}
  \label{sec:Origin}


The coordinates of the origin position of a coordinate system.

\FloatBarrier

\subsubsection{Rotation}
  \label{sec:Rotation}


Rotations about X, Y, and Z axes are expressed in A, B, and C respectively within a 3-dimensional vector. 



\paragraph{Attributes of Rotation}\mbox{}
\label{sec:Attributes of Rotation}

\tbl{attributes of Rotation} lists the attributes of \texttt{Rotation}.

\begin{table}[ht]
\centering 
  \caption{Attributes of Rotation}
  \label{table:attributes of Rotation}
\tabulinesep=3pt
\begin{tabu} to 6in {|l|l|l|} \everyrow{\hline}
\hline
\rowfont\bfseries {Attribute} & {Type} & {Multiplicity} \\
\tabucline[1.5pt]{}
\texttt{value} & \texttt{Rotation} & 1 \\
\end{tabu}
\end{table}
\FloatBarrier


Descriptions for attributes of \texttt{Rotation}:

\begin{itemize}
\item \texttt{value} : 
\end{itemize}
\FloatBarrier

\subsubsection{Transformation}
  \label{sec:Transformation}


 The process of transforming to the origin position of the coordinate system from a parent coordinate system using \glselementname{translation event} and \glselementname{rotation event}.


\paragraph{Elements of Transformation}\mbox{}
\label{sec:Elements of Transformation}

\tbl{elements of Transformation} lists the elements of \texttt{Transformation}.

\begin{table}[ht]
\centering 
  \caption{Elements of Transformation}
  \label{table:elements of Transformation}
\tabulinesep=3pt
\begin{tabu} to 6in {|l|l|l|} \everyrow{\hline}
\hline
\rowfont\bfseries {Association Name} & {Element} & {Multiplicity} \\
\tabucline[1.5pt]{}
\texttt{Translation} & \texttt{Translation} & 0..1 \\
\texttt{Rotation} & \texttt{Rotation} & 0..1 \\
\end{tabu}
\end{table}
\FloatBarrier


Descriptions for elements of \texttt{Transformation}:

\begin{itemize}
\item \texttt{Translation} : Translations along X, Y, and Z axes are expressed as x,y, and z respectively within a 3-dimensional vector. 
\item \texttt{Rotation} : Rotations about X, Y, and Z axes are expressed in A, B, and C respectively within a 3-dimensional vector. 

\end{itemize}
\FloatBarrier

\subsubsection{Translation}
  \label{sec:Translation}


Translations along X, Y, and Z axes are expressed as x,y, and z respectively within a 3-dimensional vector. 


\paragraph{Attributes of Translation}\mbox{}
\label{sec:Attributes of Translation}

\tbl{attributes of Translation} lists the attributes of \texttt{Translation}.

\begin{table}[ht]
\centering 
  \caption{Attributes of Translation}
  \label{table:attributes of Translation}
\tabulinesep=3pt
\begin{tabu} to 6in {|l|l|l|} \everyrow{\hline}
\hline
\rowfont\bfseries {Attribute} & {Type} & {Multiplicity} \\
\tabucline[1.5pt]{}
\texttt{value} & \texttt{Translation} & 1 \\
\end{tabu}
\end{table}
\FloatBarrier


Descriptions for attributes of \texttt{Translation}:

\begin{itemize}
\item \texttt{value} : 
\end{itemize}
\FloatBarrier
