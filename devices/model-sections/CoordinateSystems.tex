% Generated 2020-05-14 16:42:03 -0400
\subsection{CoordinateSystems} \label{sec:CoordinateSystems}

\subsubsection{CoordinateSystem}
  \label{sec:CoordinateSystem}





\paragraph{Attributes of CoordinateSystem}\mbox{}
\label{sec:Attributes of CoordinateSystem}

\tbl{attributes of CoordinateSystem} lists the attributes of \texttt{CoordinateSystem}.

\begin{table}[ht]
\centering 
  \caption{Attributes of CoordinateSystem}}
  \label{table:attributes of CoordinateSystem}
\tabulinesep=3pt
\begin{tabu} to 6in {|l|l|l|} \everyrow{\hline}
\hline
\rowfont\bfseries {Attribute} & {Type} & {Multiplicity} \\
\tabucline[1.5pt]{}
\texttt{id} & \texttt{ID} & 1 \\
\texttt{name} & \texttt{string} & 0..1 \\
\texttt{nativeName} & \texttt{string} & 0..1 \\
\texttt{parentIdRef} & \texttt{IDREF} & 0..1 \\
\texttt{type} & \texttt{CoordinateSystemTypeEnum} & 1 \\
\end{tabu}
\end{table}
\FloatBarrier


Descriptions for attributes of \texttt{CoordinateSystem}:

\begin{itemize}
\item \texttt{id} : 
\item \texttt{name} : 
\item \texttt{nativeName} : 
\item \texttt{parentIdRef} : 
\item \texttt{type} : 
\end{itemize}

\paragraph{Elements of CoordinateSystem}\mbox{}
\label{sec:Elements of CoordinateSystem}

\tbl{elements of CoordinateSystem} lists the elements of \texttt{CoordinateSystem}.

\begin{table}[ht]
\centering 
  \caption{Elements of CoordinateSystem}}
  \label{table:elements of CoordinateSystem}
\tabulinesep=3pt
\begin{tabu} to 6in {|l|l|l|} \everyrow{\hline}
\hline
\rowfont\bfseries {Association Name} & {Element} & {Multiplicity} \\
\tabucline[1.5pt]{}
\texttt{Location} & \texttt{Origin} & 0..1 \\
\texttt{Transformation} & \texttt{Transformation} & 0..1 \\
\end{tabu}
\end{table}
\FloatBarrier


Descriptions for elements of \texttt{CoordinateSystem}:

\begin{itemize}
\item \texttt{Location} : 
\item \texttt{Transformation} : All coordinate systems described in this International Standard are defined by the orthogonal right-hand rule.

[ISO 9787:2013(E)]
\end{itemize}
\FloatBarrier

\subsubsection{CoordinateSystems}
  \label{sec:CoordinateSystems}





\paragraph{Attributes of CoordinateSystems}\mbox{}
\label{sec:Attributes of CoordinateSystems}

\tbl{attributes of CoordinateSystems} lists the attributes of \texttt{CoordinateSystems}.

\begin{table}[ht]
\centering 
  \caption{Attributes of CoordinateSystems}}
  \label{table:attributes of CoordinateSystems}
\tabulinesep=3pt
\begin{tabu} to 6in {|l|l|l|} \everyrow{\hline}
\hline
\rowfont\bfseries {Attribute} & {Type} & {Multiplicity} \\
\tabucline[1.5pt]{}
\texttt{coordinateSystem} & \texttt{CoordinateSystem} & 1..* \\
\end{tabu}
\end{table}
\FloatBarrier


Descriptions for attributes of \texttt{CoordinateSystems}:

\begin{itemize}
\item \texttt{coordinateSystem} : 
\end{itemize}
\FloatBarrier

\subsubsection{Origin}
  \label{sec:Origin}




\FloatBarrier

\subsubsection{Rotation}
  \label{sec:Rotation}


Rotations about X, Y, and Z axes are expressed in the following way:

  + or – A about X axis;
  + or – B about Y axis;
  + or – C about Z axis.

Positive A, B and C are in the directions to advance right-hand screws in the positive X, Y and Z directions,
respectively.
General rotations are expressed by the combination of individual rotations.

[ISO 9787:2013(E)]

\FloatBarrier

\subsubsection{Rotation}
  \label{sec:Rotation}





\paragraph{Attributes of Rotation}\mbox{}
\label{sec:Attributes of Rotation}

\tbl{attributes of Rotation} lists the attributes of \texttt{Rotation}.

\begin{table}[ht]
\centering 
  \caption{Attributes of Rotation}}
  \label{table:attributes of Rotation}
\tabulinesep=3pt
\begin{tabu} to 6in {|l|l|l|} \everyrow{\hline}
\hline
\rowfont\bfseries {Attribute} & {Type} & {Multiplicity} \\
\tabucline[1.5pt]{}
\texttt{value} & \texttt{Rotation} & 1 \\
\end{tabu}
\end{table}
\FloatBarrier


Descriptions for attributes of \texttt{Rotation}:

\begin{itemize}
\item \texttt{value} : 
\end{itemize}
\FloatBarrier

\subsubsection{Transformation}
  \label{sec:Transformation}


All coordinate systems described in this International Standard are defined by the orthogonal right-hand rule.

[ISO 9787:2013(E)]


\paragraph{Elements of Transformation}\mbox{}
\label{sec:Elements of Transformation}

\tbl{elements of Transformation} lists the elements of \texttt{Transformation}.

\begin{table}[ht]
\centering 
  \caption{Elements of Transformation}}
  \label{table:elements of Transformation}
\tabulinesep=3pt
\begin{tabu} to 6in {|l|l|l|} \everyrow{\hline}
\hline
\rowfont\bfseries {Association Name} & {Element} & {Multiplicity} \\
\tabucline[1.5pt]{}
\texttt{Translation} & \texttt{Translation} & 0..1 \\
\texttt{Rotation} & \texttt{Rotation} & 0..1 \\
\end{tabu}
\end{table}
\FloatBarrier


Descriptions for elements of \texttt{Transformation}:

\begin{itemize}
\item \texttt{Translation} : 
\item \texttt{Rotation} : 
\end{itemize}
\FloatBarrier

\subsubsection{Translation}
  \label{sec:Translation}





\paragraph{Attributes of Translation}\mbox{}
\label{sec:Attributes of Translation}

\tbl{attributes of Translation} lists the attributes of \texttt{Translation}.

\begin{table}[ht]
\centering 
  \caption{Attributes of Translation}}
  \label{table:attributes of Translation}
\tabulinesep=3pt
\begin{tabu} to 6in {|l|l|l|} \everyrow{\hline}
\hline
\rowfont\bfseries {Attribute} & {Type} & {Multiplicity} \\
\tabucline[1.5pt]{}
\texttt{value} & \texttt{Translation} & 1 \\
\end{tabu}
\end{table}
\FloatBarrier


Descriptions for attributes of \texttt{Translation}:

\begin{itemize}
\item \texttt{value} : 
\end{itemize}
\FloatBarrier

\subsubsection{Translation}
  \label{sec:Translation}


Translations along X, Y, and Z axes are expressed in the following way: 
  + or – x along X axis;
  + or – y along Y axis;
  + or – z along Z axis.

[ISO 9787:2013(E)]

\FloatBarrier
