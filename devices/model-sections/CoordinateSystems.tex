% Generated 2020-02-09 16:19:45 -0800
\subsection{CoordinateSystems} \label{model:CoordinateSystems}
\subsubsection[CoordinateSystem]{CoordinateSystem \\ {\small Subtype of Configuration}}
  \label{type:CoordinateSystem}

\FloatBarrier

Placeholder for documentation!

\begin{table}[ht]
\centering 
  \caption{\texttt{Properties of CoordinateSystem}}
  \label{properties:CoordinateSystem}
\tabulinesep=3pt
\begin{tabu} to 6in {|l|l|l|} \everyrow{\hline}
\hline
\rowfont\bfseries {Property} & {Type} & {Multiplicity} \\
\tabucline[1.5pt]{}
\texttt{Transformation} & \texttt{Transformation} & 0..1 \\
\texttt{id} & \texttt{ID} & 1 \\
\texttt{name} & \texttt{string} & 0..1 \\
\texttt{parentId} & \texttt{IDREF} & 0..1 \\
\texttt{type} & \texttt{CoordinateSystemTypeEnum} & 1 \\
\texttt{Location} & \texttt{Location} & 0..1 \\
\texttt{kinematicsIdRef} & \texttt{IDREF} & 0..1 \\
\end{tabu}
\end{table}
\FloatBarrier


\paragraph{\texttt{Transformation}}\mbox{}
\newline\tab Placeholder for documentation!

\paragraph{\texttt{id}}\mbox{}
\newline\tab Placeholder for documentation!

\paragraph{\texttt{name}}\mbox{}
\newline\tab Placeholder for documentation!

\paragraph{\texttt{parentId}}\mbox{}
\newline\tab Placeholder for documentation!

\paragraph{\texttt{type}}\mbox{}
\newline\tab Placeholder for documentation!

Placeholder for documentation!

\begin{table}[ht]
\centering 
  \caption{\texttt{CoordinateSystemTypeEnum} Enumeration}
  \label{enum:CoordinateSystemTypeEnum}
\tabulinesep=3pt
\begin{tabu} to 6in {|l|X|} \everyrow{\hline}
\hline
\rowfont\bfseries {Name} & {Description} \\
\tabucline[1.5pt]{}
\texttt{WORLD} & Stationary coordinate system referenced to earth, which is independent of the robot motion.

[SOURCE: ISO 8373:2012, 4.7.1]

The origin of the world coordinate system, O0, shall be defined by the users in accordance with their requirements. The +Z0 axis is collinear but in the opposite direction to the acceleration of gravity vector. The +X0 axis shall be defined by the users in accordance with their requirements

[ISO 9787:2013(E) 5.1] \\
\texttt{BASE} & coordinate system referenced to the base mounting surface

[SOURCE: ISO 8373:2012, 4.7.2] \\
\texttt{WORKPIECE} & Placeholder for documentation! \\
\texttt{OBJECT} & Placeholder for documentation! \\
\texttt{TASK} & coordinate system referenced to the site of the task, denoted by Ok - Xk - Yk - Zk

[SOURCE: ISO 14539:2000, 3.3.5] \\
\texttt{MECHANICAL_INTERFACE} & coordinate system referenced to the mechanical interface

[SOURCE: ISO 8373:2012, 4.7.3] \\
\texttt{TOOL} & coordinate system referenced to the tool or to the end effector attached to the mechanical interface

[SOURCE: ISO 8373:2012, 4.7.5] \\
\texttt{GRIPPER} & Placeholder for documentation! \\
\texttt{PLATFORM} & Placeholder for documentation! \\
\texttt{MACHINE} & Placeholder for documentation! \\
\texttt{JOINT} & coordinate system referenced to the joint axes (4.3), the joint coordinates of which are defined relative to the preceding joint coordinates or to some other coordinate system

[ISO 8373:2012(E/F) 4.7.4] \\
\end{tabu}
\end{table} 
\FloatBarrier

\paragraph{\texttt{Location}}\mbox{}
\newline\tab Placeholder for documentation!

\paragraph{\texttt{kinematicsIdRef}}\mbox{}
\newline\tab Placeholder for documentation!
\FloatBarrier
\subsubsection{Location}
  \label{type:Location}

\FloatBarrier

Placeholder for documentation!

\FloatBarrier
\subsubsection{Rotation}
  \label{type:Rotation}

\FloatBarrier

Rotations about X, Y, and Z axes are expressed in the following way:

  + or – A about X axis;
  + or – B about Y axis;
  + or – C about Z axis.

Positive A, B and C are in the directions to advance right-hand screws in the positive X, Y and Z directions,
respectively.
General rotations are expressed by the combination of individual rotations.

[ISO 9787:2013(E)]

\FloatBarrier
\subsubsection{Transformation}
  \label{type:Transformation}

\FloatBarrier

All coordinate systems described in this International Standard are defined by the orthogonal right-hand rule.

[ISO 9787:2013(E)]

\begin{table}[ht]
\centering 
  \caption{\texttt{Properties of Transformation}}
  \label{properties:Transformation}
\tabulinesep=3pt
\begin{tabu} to 6in {|l|l|l|} \everyrow{\hline}
\hline
\rowfont\bfseries {Property} & {Type} & {Multiplicity} \\
\tabucline[1.5pt]{}
\texttt{translation} & \texttt{Translation} & 0..1 \\
\texttt{rottion} & \texttt{Rotation} & 0..1 \\
\end{tabu}
\end{table}
\FloatBarrier


\paragraph{\texttt{translation}}\mbox{}
\newline\tab Placeholder for documentation!

\paragraph{\texttt{rottion}}\mbox{}
\newline\tab Placeholder for documentation!
\FloatBarrier
\subsubsection{Translation}
  \label{type:Translation}

\FloatBarrier

Translations along X, Y, and Z axes are expressed in the following way: 
  + or – x along X axis;
  + or – y along Y axis;
  + or – z along Z axis.

[ISO 9787:2013(E)]

\FloatBarrier
