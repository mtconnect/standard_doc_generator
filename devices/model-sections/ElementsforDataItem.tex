% Generated 2020-07-27 15:21:27 +0530
\subsection{Elements for DataItem} \label{sec:ElementsforDataItem}

This section provides detailed descriptions for the elements of a {block:DataItem}.


\subsubsection{Constraints}
  \label{sec:Constraints}


{block:Constraints} {term:organizes} a set of expected values that can be reported for this {block:DataItem}.


\paragraph{Elements of Constraints}\mbox{}
\label{sec:Elements of Constraints}

\tbl{elements of Constraints} lists the elements of \texttt{Constraints}.

\begin{table}[ht]
\centering 
  \caption{Elements of Constraints}
  \label{table:elements of Constraints}
\tabulinesep=3pt
\begin{tabu} to 6in {|l|l|l|} \everyrow{\hline}
\hline
\rowfont\bfseries {Association Name} & {Element} & {Multiplicity} \\
\tabucline[1.5pt]{}
\texttt{<<deprecated>> Filter} & \texttt{FilterEnum} & 0..1 \\
\texttt{Maximum} & \texttt{float} & 0..1 \\
\texttt{Minimum} & \texttt{float} & 0..1 \\
\texttt{Nominal} & \texttt{float} & 0..1 \\
\texttt{Value} & \texttt{float} & 0..* \\
\end{tabu}
\end{table}
\FloatBarrier


Descriptions for elements of \texttt{Constraints}:

\begin{itemize}
\item \texttt{Filter} : 
\tabulinesep = 5pt
\begin{longtabu} to \textwidth {
    |l|X|}
  \caption{FilterEnum Enumeration}
  \label{enum:FilterEnum} \\
\hline
Name & Description \\
\hline
\endfirsthead
\hline
\multicolumn{2}{|c|}{Continuation of Table \texttt{FilterEnum} Enumeration} \\
\hline
Name & Description \\
\hline
\endhead
\texttt{MINIMUM_DELTA} &  \\ \hline
\texttt{PERIOD} &  \\ \hline
\end{longtabu}
\FloatBarrier
\item \texttt{Maximum} : 
\item \texttt{Minimum} : 
\item \texttt{Nominal} : 
\item \texttt{Value} : 
\end{itemize}
\FloatBarrier

\subsubsection{Definition}
  \label{sec:Definition}


The {block:Definition} defines the meaning of {block:Entry} and {block:Cell} elements associated with the {block:DataItem} when the {property:representation} is either {block:DATA} or {block:TABLE}.


\paragraph{Elements of Definition}\mbox{}
\label{sec:Elements of Definition}

\tbl{elements of Definition} lists the elements of \texttt{Definition}.

\begin{table}[ht]
\centering 
  \caption{Elements of Definition}
  \label{table:elements of Definition}
\tabulinesep=3pt
\begin{tabu} to 6in {|l|l|l|} \everyrow{\hline}
\hline
\rowfont\bfseries {Association Name} & {Element} & {Multiplicity} \\
\tabucline[1.5pt]{}
\texttt{CellDefinitions} & \texttt{CellDefinition} & 0..* \\
\texttt{Description} & \texttt{Description} & 0..1 \\
\texttt{EntryDefinitions} & \texttt{EntryDefinition} & 0..* \\
\end{tabu}
\end{table}
\FloatBarrier


Descriptions for elements of \texttt{Definition}:

\begin{itemize}
\item \texttt{CellDefinitions} : {block:CellDefinitions} {term:organizes} {block:CellDefinition} elements.
\item \texttt{Description} : An element that can contain any descriptive content.
\item \texttt{EntryDefinitions} : {block:EntryDefinitions} {term:organizes} {block:EntryDefinition} elements.
\end{itemize}
\FloatBarrier

\subsubsection{Filter}
  \label{sec:Filter}


{block:Filter} provides a means to control when an {term:Agent} records updated information for a data item. 


\paragraph{Attributes of Filter}\mbox{}
\label{sec:Attributes of Filter}

\tbl{attributes of Filter} lists the attributes of \texttt{Filter}.

\begin{table}[ht]
\centering 
  \caption{Attributes of Filter}
  \label{table:attributes of Filter}
\tabulinesep=3pt
\begin{tabu} to 6in {|l|l|l|} \everyrow{\hline}
\hline
\rowfont\bfseries {Attribute} & {Type} & {Multiplicity} \\
\tabucline[1.5pt]{}
\texttt{type} & \texttt{FilterEnum} & 1 \\
\end{tabu}
\end{table}
\FloatBarrier


Descriptions for attributes of \texttt{Filter}:

\begin{itemize}
\item \texttt{type} : 
\end{itemize}
\FloatBarrier

\subsubsection{InitialValue}
  \label{sec:InitialValue}


{block:InitialValue} defines the starting value for a data item as well as the value to be set for the data item after a reset event.


\paragraph{Attributes of InitialValue}\mbox{}
\label{sec:Attributes of InitialValue}

\tbl{attributes of InitialValue} lists the attributes of \texttt{InitialValue}.

\begin{table}[ht]
\centering 
  \caption{Attributes of InitialValue}
  \label{table:attributes of InitialValue}
\tabulinesep=3pt
\begin{tabu} to 6in {|l|l|l|} \everyrow{\hline}
\hline
\rowfont\bfseries {Attribute} & {Type} & {Multiplicity} \\
\tabucline[1.5pt]{}
\texttt{value} & \texttt{string} & 1 \\
\end{tabu}
\end{table}
\FloatBarrier


Descriptions for attributes of \texttt{InitialValue}:

\begin{itemize}
\item \texttt{value} : 
\end{itemize}
\FloatBarrier

\subsubsection{ResetTrigger}
  \label{sec:ResetTrigger}


{block:ResetTrigger} identifies the type of event that may cause a reset to occur.


\paragraph{Attributes of ResetTrigger}\mbox{}
\label{sec:Attributes of ResetTrigger}

\tbl{attributes of ResetTrigger} lists the attributes of \texttt{ResetTrigger}.

\begin{table}[ht]
\centering 
  \caption{Attributes of ResetTrigger}
  \label{table:attributes of ResetTrigger}
\tabulinesep=3pt
\begin{tabu} to 6in {|l|l|l|} \everyrow{\hline}
\hline
\rowfont\bfseries {Attribute} & {Type} & {Multiplicity} \\
\tabucline[1.5pt]{}
\texttt{type} & \texttt{ResetTriggerEnum} & 1 \\
\end{tabu}
\end{table}
\FloatBarrier


Descriptions for attributes of \texttt{ResetTrigger}:

\begin{itemize}
\item \texttt{type} : 
\end{itemize}
\FloatBarrier

\subsubsection{Source}
  \label{sec:Source}


{block:Source} identifies the {block:Component}, {block:DataItem}, or {block:Composition} representing the area of the piece of equipment from which a measured value originates.


\paragraph{Attributes of Source}\mbox{}
\label{sec:Attributes of Source}

\tbl{attributes of Source} lists the attributes of \texttt{Source}.

\begin{table}[ht]
\centering 
  \caption{Attributes of Source}
  \label{table:attributes of Source}
\tabulinesep=3pt
\begin{tabu} to 6in {|l|l|l|} \everyrow{\hline}
\hline
\rowfont\bfseries {Attribute} & {Type} & {Multiplicity} \\
\tabucline[1.5pt]{}
\texttt{componentId} & \texttt{ID} & 0..1 \\
\texttt{compositionId} & \texttt{ID} & 0..1 \\
\texttt{dataItemId} & \texttt{ID} & 0..1 \\
\end{tabu}
\end{table}
\FloatBarrier


Descriptions for attributes of \texttt{Source}:

\begin{itemize}
\item \texttt{componentId} : The identifier attribute of the {block:Component} element that represents the physical part of a piece of equipment where the data represented by the {block:DataItem} element originated.
\item \texttt{compositionId} : The identifier attribute of the {block:Composition} element that the reported data is most closely associated.
\item \texttt{dataItemId} : The identifier attribute of the {block:DataItem} that represents the originally measured value of the data referenced by this data item.
\end{itemize}

\paragraph{Elements of Source}\mbox{}
\label{sec:Elements of Source}

\tbl{elements of Source} lists the elements of \texttt{Source}.

\begin{table}[ht]
\centering 
  \caption{Elements of Source}
  \label{table:elements of Source}
\tabulinesep=3pt
\begin{tabu} to 6in {|l|l|l|} \everyrow{\hline}
\hline
\rowfont\bfseries {Association Name} & {Element} & {Multiplicity} \\
\tabucline[1.5pt]{}
\texttt{Value} & \texttt{string} & 0..1 \\
\end{tabu}
\end{table}
\FloatBarrier


Descriptions for elements of \texttt{Source}:

\begin{itemize}
\item \texttt{Value} : 
\end{itemize}
\FloatBarrier
