% Generated 2021-02-25 22:49:46 +0530
\subsection{Elements for DataItem} \label{sec:Elements for DataItem}


This section provides detailed descriptions for the elements of a \block{DataItem}.

Note: See \sect{DataItems Schema Diagrams} for XML schema of the elements for \block{DataItem}.


\subsubsection{ResetTrigger}
\label{sec:ResetTrigger}



\block{ResetTrigger} identifies the type of event that may cause a reset to occur.


\paragraph{Attributes of ResetTrigger}\mbox{}
\label{sec:Attributes of ResetTrigger}

\tbl{Attributes of ResetTrigger} lists the attributes of \texttt{ResetTrigger}.

\begin{table}[ht]
\centering 
  \caption{Attributes of ResetTrigger}
  \label{table:Attributes of ResetTrigger}
\tabulinesep=3pt
\begin{tabu} to 6in {|l|l|l|} \everyrow{\hline}
\hline
\rowfont\bfseries {Attribute} & {Type} & {Multiplicity} \\
\tabucline[1.5pt]{}

\property{type}[ResetTrigger] & \texttt{ResetTriggerEnum} & 1 \\
\end{tabu}
\end{table}
\FloatBarrier

Descriptions for attributes of \block{ResetTrigger}:

\begin{itemize}

\item \property{type}[ResetTrigger] \newline The type of \block{ResetTrigger}

\texttt{ResetTriggerEnum} Enumeration:

\begin{itemize}
\item \texttt{ACTION\textunderscore COMPLETE} \newline The value of the \gls{Data Item} that is measuring an action or operation is to be reset upon completion of that action or operation. 
\item \texttt{ANNUAL} \newline The value of the \gls{Data Item} is to be reset at the end of a 12-month period. 
\item \texttt{DAY} \newline The value of the \gls{Data Item} is to be reset at the end of a 24-hour period. 
\item \texttt{LIFE} \newline The value of the \gls{Data Item} is not reset and accumulates for the entire life of the piece of equipment. 
\item \texttt{MAINTENANCE} \newline The value of the \gls{Data Item} is to be reset upon completion of a maintenance event. 
\item \texttt{MONTH} \newline The value of the \gls{Data Item} is to be reset at the end of a monthly period. 
\item \texttt{POWER\textunderscore ON} \newline The value of the \gls{Data Item} is to be reset when power was applied to the piece of equipment after a planned or unplanned interruption of power has occurred. 
\item \texttt{SHIFT} \newline The value of the \gls{Data Item} is to be reset at the end of a work shift. 
\item \texttt{WEEK} \newline The value of the \gls{Data Item} is to be reset at the end of a 7-day period. 
\end{itemize}

\end{itemize}



\subsubsection{Source}
\label{sec:Source}



\block{Source} identifies the \block{Component}, \block{DataItem}, or \block{Composition} representing the area of the piece of equipment from which a measured value originates.


The value of \texttt{Source} \MUST be \texttt{string}.


\paragraph{Attributes of Source}\mbox{}
\label{sec:Attributes of Source}

\tbl{Attributes of Source} lists the attributes of \texttt{Source}.

\begin{table}[ht]
\centering 
  \caption{Attributes of Source}
  \label{table:Attributes of Source}
\tabulinesep=3pt
\begin{tabu} to 6in {|l|l|l|} \everyrow{\hline}
\hline
\rowfont\bfseries {Attribute} & {Type} & {Multiplicity} \\
\tabucline[1.5pt]{}

\property{componentId}[Source] & \texttt{ID} & 0..1 \\
\property{compositionId}[Source] & \texttt{ID} & 0..1 \\
\property{dataItemId}[Source] & \texttt{ID} & 0..1 \\
\end{tabu}
\end{table}
\FloatBarrier

Descriptions for attributes of \block{Source}:

\begin{itemize}

\item \property{componentId}[Source] \newline The identifier attribute of the \block{Component} element that represents the physical part of a piece of equipment where the data represented by the \block{DataItem} element originated.

\item \property{compositionId}[Source] \newline The identifier attribute of the \block{Composition} element that the reported data is most closely associated.

\item \property{dataItemId}[Source] \newline The identifier attribute of the \block{DataItem} that represents the originally measured value of the data referenced by this data item.
\end{itemize}



\subsubsection{InitialValue}
\label{sec:InitialValue}



\block{InitialValue} defines the starting value for a data item as well as the value to be set for the data item after a reset event.


The value of \texttt{InitialValue} \MUST be \texttt{string}.



\subsubsection{Filter}




\block{Filter} provides a means to control when an \gls{Agent} records updated information for a data item. 


\paragraph{Attributes of Filter}\mbox{}
\label{sec:Attributes of Filter}

\tbl{Attributes of Filter} lists the attributes of \texttt{Filter}.

\begin{table}[ht]
\centering 
  \caption{Attributes of Filter}
  \label{table:Attributes of Filter}
\tabulinesep=3pt
\begin{tabu} to 6in {|l|l|l|} \everyrow{\hline}
\hline
\rowfont\bfseries {Attribute} & {Type} & {Multiplicity} \\
\tabucline[1.5pt]{}

\property{type}[Filter] & \texttt{FilterEnum} & 1 \\
\end{tabu}
\end{table}
\FloatBarrier

Descriptions for attributes of \block{Filter}:

\begin{itemize}

\item \property{type}[Filter] \newline The type of \block{Filter}

\texttt{FilterEnum} Enumeration:

\begin{itemize}
\item \texttt{MINIMUM\textunderscore DELTA} \newline For a \texttt{MINIMUM\textunderscore DELTA} type \block{Filter}, a new value \textbf{MUST NOT} be reported for a data item unless the measured value has changed from the
last reported value by at least the delta given as the value of this element.

The value of \block{Filter} \textbf{MUST} be an absolute value using the same units as the reported data. 
\item \texttt{PERIOD} \newline For a \texttt{PERIOD} type \block{Filter}, the data reported for a data item is provided on a periodic basis. The \texttt{PERIOD} for reporting data is defined in the
value of the \block{Filter}.

The value of \block{Filter} \textbf{MUST} be an absolute value reported in seconds representing the time between reported samples of the value of the data item. 
\end{itemize}

\end{itemize}



\subsubsection{Constraints}
\label{sec:Constraints}



\block{Constraints} \glspl{organize} a set of expected values that can be reported for this \block{DataItem}.


\paragraph{Elements of Constraints}\mbox{}
\label{sec:Elements of Constraints}

\tbl{Elements of Constraints} lists the elements of \texttt{Constraints}.

\begin{table}[ht]
\centering 
  \caption{Elements of Constraints}
  \label{table:Elements of Constraints}
\tabulinesep=3pt
\begin{tabu} to 6in {|l|l|} \everyrow{\hline}
\hline
\rowfont\bfseries {Element} & {Multiplicity} \\
\tabucline[1.5pt]{}
\texttt{Filter} & 0..1 \\
\texttt{Maximum} & 0..1 \\
\texttt{Minimum} & 0..1 \\
\texttt{Nominal} & 0..1 \\
\texttt{Value} & 0..* \\
\end{tabu}
\end{table}
\FloatBarrier


Descriptions for elements of \block{Constraints}:

\begin{itemize}

\item \block{Filter} \newline \textbf{DEPRECATED} in \textit{MTConnect Version 1.4} – Moved to the
\block{Filters} element of \block{DataItem}.

\item \block{Maximum} \newline If the data reported for a data item is a range of numeric values, the expected value reported \textbf{MAY} be described with an upper limit defined by this constraint.

The value of \block{Maximum} \MUST be \texttt{float}.

\item \block{Minimum} \newline If the data reported for a data item is a range of numeric values, the expected value reported \textbf{MAY} be described with a lower limit defined by this constraint.

The value of \block{Minimum} \MUST be \texttt{float}.

\item \block{Nominal} \newline The target or expected value for this data item.

The value of \block{Nominal} \MUST be \texttt{float}.

\item \block{Value} \newline \property{Value} represents a single data value that is expected to be reported for a \block{DataItem} element.

\property{Value} \textbf{MUST NOT} be used in conjunction with any other \block{Constraint} elements.

The value of \block{Value} \MUST be \texttt{float}.
\end{itemize}



\subsubsection{Definition}
\label{sec:Definition}



The \block{Definition} defines the meaning of \block{Entry} and \block{Cell} elements associated with the \block{DataItem} when the \property{representation} is either \block{DATA} or \block{TABLE}.


\paragraph{Elements of Definition}\mbox{}
\label{sec:Elements of Definition}

\tbl{Elements of Definition} lists the elements of \texttt{Definition}.

\begin{table}[ht]
\centering 
  \caption{Elements of Definition}
  \label{table:Elements of Definition}
\tabulinesep=3pt
\begin{tabu} to 6in {|l|l|} \everyrow{\hline}
\hline
\rowfont\bfseries {Element} & {Multiplicity} \\
\tabucline[1.5pt]{}
\texttt{CellDefinition} (organized by \block{CellDefinitions}) & 0..* \\
\texttt{Description} & 0..1 \\
\texttt{EntryDefinition} (organized by \block{EntryDefinitions}) & 0..* \\
\end{tabu}
\end{table}
\FloatBarrier


Descriptions for elements of \block{Definition}:

\begin{itemize}

\item \block{CellDefinitions} \newline \block{CellDefinitions} \glspl{organize} \block{CellDefinition} elements.

\item \block{Description} \newline The \block{Description} of the \block{Definition}. See \block{Component} \block{Description}.

\item \block{EntryDefinitions} \newline \block{EntryDefinitions} \glspl{organize} \block{EntryDefinition} elements.
\end{itemize}


