% Generated 2020-10-22 16:25:25 +0530
\subsection{Compositions} \label{sec:Compositions}


\block{Compositions} \glspl{organize} the \block{Composition} elements within a \block{Component} element. See \fig{Components} for details on the semantics of \block{Composition}.


\subsubsection{Composition}
\label{sec:Composition}



\block{Composition} is used to describe the lowest level structural building blocks contained within a \block{Component} element.

There can be multiple types of \block{Composition} elements defined for a \block{Component} element.


\paragraph{Attributes of Composition}\mbox{}
\label{sec:Attributes of Composition}

\tbl{Attributes of Composition} lists the attributes of \texttt{Composition}.

\begin{table}[ht]
\centering 
  \caption{Attributes of Composition}
  \label{table:Attributes of Composition}
\tabulinesep=3pt
\begin{tabu} to 6in {|l|l|l|} \everyrow{\hline}
\hline
\rowfont\bfseries {Attribute} & {Type} & {Multiplicity} \\
\tabucline[1.5pt]{}
\property{type}[Composition] & \texttt{CompositionTypeEnum} & 1 \\
\property{id}[Composition] & \texttt{ID} & 1 \\
\property{name}[Composition] & \texttt{string} & 0..1 \\
\property{uuid}[Composition] & \texttt{NMTOKEN} & 0..1 \\
\property{configuration}[Composition] & \texttt{Configuration} & 0..1 \\
\end{tabu}
\end{table}
\FloatBarrier


Descriptions for attributes of \block{Composition}:

\begin{itemize}

\item \property{type}[Composition] : The type of either a \gls{Structural Element} or a \block{DataItem} being measured.

\item \property{id}[Composition] : The unique identifier for this element.

\item \property{name}[Composition] : The name of this element.

\item \property{uuid}[Composition] : A unique identifier for this element.

\item \property{configuration}[Composition] : 
\end{itemize}

\paragraph{Elements of Composition}\mbox{}
\label{sec:Elements of Composition}

\tbl{Elements of Composition} lists the elements of \texttt{Composition}.

\begin{table}[ht]
\centering 
  \caption{Elements of Composition}
  \label{table:Elements of Composition}
\tabulinesep=3pt
\begin{tabu} to 6in {|l|l|l|} \everyrow{\hline}
\hline
\rowfont\bfseries {Element Name} & {Type} & {Multiplicity} \\
\tabucline[1.5pt]{}
\block{Description} & \texttt{Description} & 0..1 \\
\end{tabu}
\end{table}
\FloatBarrier


Descriptions for elements of \block{Composition}:

\begin{itemize}
\item \block{Description} : The descriptive content.
\end{itemize}
