% Generated 2020-07-29 16:29:20 +0530
\subsection{Specifications} \label{sec:Specifications}

\subsubsection{Characteristic}
  \label{sec:Characteristic}





\paragraph{Attributes of Characteristic}\mbox{}
\label{sec:Attributes of Characteristic}

\tbl{attributes of Characteristic} lists the attributes of \texttt{Characteristic}.

\begin{table}[ht]
\centering 
  \caption{Attributes of Characteristic}
  \label{table:attributes of Characteristic}
\tabulinesep=3pt
\begin{tabu} to 6in {|l|l|l|} \everyrow{\hline}
\hline
\rowfont\bfseries {Attribute} & {Type} & {Multiplicity} \\
\tabucline[1.5pt]{}
\texttt{value} & \texttt{float} & 1 \\
\end{tabu}
\end{table}
\FloatBarrier


Descriptions for attributes of \texttt{Characteristic}:

\begin{itemize}
\item \texttt{value} : 
\end{itemize}
\FloatBarrier

\subsubsection{Rated}
  \label{sec:Rated}




\FloatBarrier

\subsubsection{Minimum}
  \label{sec:Minimum}


A numeric lower limit constraint.

\FloatBarrier

\subsubsection{Maximum}
  \label{sec:Maximum}


A numeric upper limit constraint.

\FloatBarrier

\subsubsection{GearRatio}
  \label{sec:GearRatio}





\paragraph{Attributes of GearRatio}\mbox{}
\label{sec:Attributes of GearRatio}

\tbl{attributes of GearRatio} lists the attributes of \texttt{GearRatio}.

\begin{table}[ht]
\centering 
  \caption{Attributes of GearRatio}
  \label{table:attributes of GearRatio}
\tabulinesep=3pt
\begin{tabu} to 6in {|l|l|l|} \everyrow{\hline}
\hline
\rowfont\bfseries {Attribute} & {Type} & {Multiplicity} \\
\tabucline[1.5pt]{}
\texttt{number} & \texttt{integer} & 1 \\
\end{tabu}
\end{table}
\FloatBarrier


Descriptions for attributes of \texttt{GearRatio}:

\begin{itemize}
\item \texttt{number} : 
\end{itemize}
\FloatBarrier

\subsubsection{Nominal}
  \label{sec:Nominal}


The numeric target or expected value.

\FloatBarrier

\subsubsection{DutyCycle}
  \label{sec:DutyCycle}





\paragraph{Attributes of DutyCycle}\mbox{}
\label{sec:Attributes of DutyCycle}

\tbl{attributes of DutyCycle} lists the attributes of \texttt{DutyCycle}.

\begin{table}[ht]
\centering 
  \caption{Attributes of DutyCycle}
  \label{table:attributes of DutyCycle}
\tabulinesep=3pt
\begin{tabu} to 6in {|l|l|l|} \everyrow{\hline}
\hline
\rowfont\bfseries {Attribute} & {Type} & {Multiplicity} \\
\tabucline[1.5pt]{}
\texttt{duration} & \texttt{float} & 1 \\
\texttt{peak} & \texttt{float} & 1 \\
\end{tabu}
\end{table}
\FloatBarrier


Descriptions for attributes of \texttt{DutyCycle}:

\begin{itemize}
\item \texttt{duration} : 
\item \texttt{peak} : 
\end{itemize}
\FloatBarrier

\subsubsection{Specification}
  \label{sec:Specification}


\block{Specification} elements define information describing the design characteristics for a piece of equipment.



\paragraph{Attributes of Specification}\mbox{}
\label{sec:Attributes of Specification}

\tbl{attributes of Specification} lists the attributes of \texttt{Specification}.

\begin{table}[ht]
\centering 
  \caption{Attributes of Specification}
  \label{table:attributes of Specification}
\tabulinesep=3pt
\begin{tabu} to 6in {|l|l|l|} \everyrow{\hline}
\hline
\rowfont\bfseries {Attribute} & {Type} & {Multiplicity} \\
\tabucline[1.5pt]{}
\texttt{type} & \texttt{DataItemTypeEnum} & 1 \\
\texttt{subType} & \texttt{DataItemSubTypeEnum} & 0..1 \\
\texttt{dataItemIdRef} & \texttt{IDREF} & 0..1 \\
\texttt{units} & \texttt{UnitEnum} & 0..1 \\
\texttt{compositionIdRef} & \texttt{IDREF} & 0..1 \\
\texttt{hasCharacteristic} & \texttt{Characteristic} & 0..* \\
\texttt{name} & \texttt{string} & 0..1 \\
\texttt{coordinateSystemIdRef} & \texttt{IDREF} & 0..1 \\
\end{tabu}
\end{table}
\FloatBarrier


Descriptions for attributes of \texttt{Specification}:

\begin{itemize}
\item \texttt{type} : Same as \block{DataItem} type. See \ref{DataItemTypes}.
\item \texttt{subType} : Same as \block{DataItem} subtypes. See \ref{Attributes of DataItem}.
\item \texttt{dataItemIdRef} : A reference to the {property:id} attribute of the \block{DataItem} associated with this element.
\item \texttt{units} : Same as \block{DataItem} units. See \ref{Attribute of DataItem}.
\item \texttt{compositionIdRef} : A reference to the {property:id} attribute of the \block{Composition} associated with this element.
\item \texttt{hasCharacteristic} : 
\item \texttt{name} : The {property:name} provides additional meaning and differentiates between \block{Specifications}.
\item \texttt{coordinateSystemIdRef} : References the \block{CoordinateSystem} for geometric \block{Specification} elements.
\end{itemize}
\FloatBarrier

\subsubsection{Specifications}
  \label{sec:Specifications}


\block{Specifications} \gls{organizes} \block{Specification} elements for a \block{Component}.


\paragraph{Attributes of Specifications}\mbox{}
\label{sec:Attributes of Specifications}

\tbl{attributes of Specifications} lists the attributes of \texttt{Specifications}.

\begin{table}[ht]
\centering 
  \caption{Attributes of Specifications}
  \label{table:attributes of Specifications}
\tabulinesep=3pt
\begin{tabu} to 6in {|l|l|l|} \everyrow{\hline}
\hline
\rowfont\bfseries {Attribute} & {Type} & {Multiplicity} \\
\tabucline[1.5pt]{}
\texttt{specification} & \texttt{Specification} & 1..* \\
\end{tabu}
\end{table}
\FloatBarrier


Descriptions for attributes of \texttt{Specifications}:

\begin{itemize}
\item \texttt{specification} : 
\end{itemize}
\FloatBarrier
