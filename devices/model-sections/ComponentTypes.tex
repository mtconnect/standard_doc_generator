% Generated 2021-01-06 15:42:25 +0530
\subsection{Component Types} \label{sec:Component Types}


\block{Component} \glspl{Structural Element} are used to represent physical parts or logical functions of a piece of equipment.

\block{Component} \glspl{Structural Element} are defined into two major categories:

\begin{itemize}

\item \gls{Top Level} \block{Component} elements are used to group the \glspl{Structural Element} representing the most significant physical or logical functions of a piece of equipment.  The \gls{Top Level} \block{Component} elements provided in an \block{MTConnectDevices} document \textbf{SHOULD} be restricted to those defined in this section.  However, these \gls{Top Level} \block{Component} elements \textbf{MAY} also be used as \gls{Lower Level} \block{Component} elements; as required.

\item \gls{Lower Level} \block{Component} elements are used to describe the sub-parts of the parent \block{Component} to provide more clarity and granularity to the physical or logical structure of the \gls{Top Level} \block{Component} elements.
\end{itemize}

This section of the \gls{Devices Information Model} provides guidance for the most common relationships between \gls{Top Level} \block{Component} elements and \gls{Lower Level} child components.  However, all \block{Component} elements \textbf{MAY} be used in any configuration, as required, to fully describe a piece of equipment.

As described in \sect{Structural Elements for MTConnectDevices}, \block{Component} is an abstract type \gls{Structural Element} within the \gls{Devices Information Model} and will never appear directly in the \block{MTConnectDevices} document.  As abstract elements, \block{Component} will be replaced in the document by a specific \block{Component} type.


\subsubsection{Actuator}
\label{sec:Actuator}



\block{Actuator} represents the information for an apparatus for moving or controlling a mechanism or system. It takes energy usually provided by air, electric current, or liquid and converts the energy into some kind of motion.


\subsubsection{Adapter}
\label{sec:Adapter}



{{(block(Adapter)}} is a \block{Component} that represents the connectivity state of a data source for the \gls{MTConnect Agent}.


\subsubsection{Auxiliary}
\label{sec:Auxiliary}



\block{Auxiliary} is a \block{Component} that represents functional sub-system(s) that provides supplementary or extended capabilities for a piece of equipment, but they are not required for the basic operation of the equipment.


\paragraph{Loader}\mbox{}
\label{sec:Loader}


\block{Loader} represents the information for a unit comprised of all the parts involved in moving and distributing materials, parts, tooling, and other items to or from a piece of equipment.


\paragraph{BarFeeder}\mbox{}
\label{sec:BarFeeder}


\block{BarFeeder} represents the information for a unit involved in delivering bar stock to a piece of equipment.


\paragraph{Sensor}\mbox{}
\label{sec:Sensor}


\block{Sensor} represents the information for a piece of equipment that responds to a physical stimulus and transmits a resulting impulse or value from a sensing unit.


\paragraph{ToolingDelivery}\mbox{}
\label{sec:ToolingDelivery}


\block{ToolingDelivery} represents the information for a unit involved in managing, positioning, storing, and delivering tooling within a piece of equipment.



\paragraph{GangToolBar}\mbox{}
\label{sec:GangToolBar}


A tool mounting mechanism that holds any number of tools. Tools are located in \block{Station}s. Tools are positioned for use in the manufacturing process by linearly positioning the \block{GangToolBar}.


\paragraph{AutomaticToolChanger}\mbox{}
\label{sec:AutomaticToolChanger}


A tool delivery mechanism that moves tools between a \block{ToolMagazine} and a \gls{Spindle} a \block{Turret}. An \block{AutomaticToolChanger} may also transfer tools between a location outside of a piece of equipment and a \block{ToolMagazine} or \block{Turret}.


\paragraph{ToolMagazine}\mbox{}
\label{sec:ToolMagazine}


A tool storage mechanism that holds any number of tools. Tools are located in \block{Pot}s. \block{Pot}s are moved into position to transfer tools into or out of the \block{ToolMagazine} by an \block{AutomaticToolChanger}.


\paragraph{ToolRack}\mbox{}
\label{sec:ToolRack}


A linear or matrixed tool storage mechanism that holds any number of tools. Tools are located in \block{Station}s.


\paragraph{Turret}\mbox{}
\label{sec:Turret}


A tool mounting mechanism that holds any number of tools. Tools are located in \block{Station}s . Tools are positioned for use in the manufacturing process by rotating the \block{Turret}.


\paragraph{WasteDisposal}\mbox{}
\label{sec:WasteDisposal}


\block{WasteDisposal} represents the information for a unit comprised of all the parts involved in removing manufacturing byproducts from a piece of equipment.



\subsubsection{Axis}
\label{sec:Axis}



\block{Axis} is an abstract \block{Component} that represents linear or rotational motion for a piece of equipment.In robotics, the term \textit{Axis} is synonymous with \textit{Joint}. A \textit{Joint} is the connection between two parts of the structure that move in relation to each other.

\block{Linear} and \block{Rotary} components \textbf{MUST} have a \property{name} attribute that \textbf{MUST} follow the conventions described below. Use the \property{nativeName} attribute for the manufacturer's name of the axis if it differs from the assigned \property{name}.

\gls{MTConnect} has two high-level classes for automation equipment as follows: (1) Equipment that controls cartesian coordinate axes and (2) Equipment that controls articulated axes. There are ambiguous cases where some machines exhibit both characteristics; when this occurs, the primary control system's configuration determines the classification.

Examples of cartesian coordinate equipment are CNC Machine Tools, Coordinate measurement machines, as specified in ISO 841, and 3D Printers. Examples of articulated automation equipment are Robotic systems as specified in ISO 8373.

The following sections define the designation of names for the axes and additional guidance when selecting the correct scheme to use for a given piece of equipment.

\paragraph{Cartesian Coordinate Naming Conventions}
\label{sec:Cartesian Coordinate Naming Conventions}

A Three-Dimensional Cartesian Coordinate control system organizes its axes orthogonally relative to a machine coordinate system where the manufacturer of the equipment specifies the origin. 

\block{Axes} \property{name} \textbf{SHOULD} comply with ISO 841, if possible.

\subparagraph{Linear Motion}\mbox{}
\label{sec:Linear Motion}

A piece of equipment \textbf{MUST} represent prismatic motion using a \block{Linear} axis \block{Component} and assign its \property{name} using the designations \texttt{X}, \texttt{Y}, and \texttt{Z}. A \block{Linear} axis \property{name} \textbf{MUST} append a monotonically increasing suffix when there are more than one parallel axes; for example, \texttt{X2}, \texttt{X3}, and \texttt{X4}. 

\subparagraph{Rotary Motion}\mbox{}
\label{sec:Rotary Motion}

\gls{MTConnect} \textbf{MUST} assign the \property{name} to \block{Rotary} axes exhibiting rotary motion using \texttt{A}, \texttt{B}, and \texttt{C}. A \block{Rotary} axis \property{name} \textbf{MUST} append a monotonically increasing suffix when more than one \block{Rotary} axis rotates around the same \block{Linear} axis; for example, \texttt{A2}, \texttt{A3}, and \texttt{A4}. 

\paragraph{Articulated Machine Control Systems}
\label{sec:Articulated Machine Control Systems}

An articulated control system's axes represent the connecting linkages between two adjacent rigid members of an assembly. The \block{Linear} axis represents prismatic motion, and the \block{Rotary} axis represents the rotational motion of the two related members. The control organizes the axes in a kinematic chain from the mounting surface (base) to the end-effector or tooling.

\paragraph{Articulated Machine Axis Names}
\label{sec:Articulated Machine Axis Names}

The axes of articulated machines represent forward kinematic relationships between mechanical linkages. Each axis is a connection between linkages, also referred to as joints, and \textbf{MUST} be named using a \texttt{J} followed by a monotonically increasing number; for example, \texttt{J1}, \texttt{J2}, \texttt{J3}.  The numbering starts at the base axis connected or closest to the mounting surface, \texttt{J1}, incrementing to the mechanical interface, \texttt{Jn}, where \texttt{n} is the number of the last axis. The chain forms a parent-child relationship with the parent being the axis closest to the base.

A machine having an axis with more than one child \textbf{MUST} number each branch using its numeric designation followed by a branch number and a monotonically increasing number. For example, if \texttt{J2} has two children, the first child branch \textbf{MUST} be named \texttt{J2.1.1} and the second child branch \texttt{J2.2.1}. A child of the first branch \MUST be named \texttt{J2.1.2}, incrementing to \texttt{J2.1.n}, where \texttt{J2.1.n} is the number of the last axis in that branch.


\paragraph{Linear}\mbox{}
\label{sec:Linear}


A \block{Linear} axis represents the movement of a physical piece of equipment, or a portion of the equipment, in a straight line. 


\subsubsection{Chuck}
\label{sec:Chuck}



\block{Chuck} provides the information about a mechanism that holds a part or stock material in place.


\subsubsection{Door}
\label{sec:Door}



\block{Door} represents the information for a mechanical mechanism or closure that can cover.


\subsubsection{Interface}
\label{sec:Interface}



\block{Interface} is a \block{Component} that represents an \gls{Interface} to coordinate actions and activities between pieces of equipment that communicate information between each other.


\subsubsection{Organizer}
\label{sec:Organizer}



{{Organizer}} is an abstract \block{Component} that \glspl{organize} similar {{Component}} types together.


\paragraph{\texttt{Components}}\mbox{}
\label{sec:Components}

\newline 

\subsubsection{Path}
\label{sec:Path}



\block{Path} represents the information for an independent operation or function within a \block{Controller}.


\subsubsection{Power}
\label{sec:Power}



\block{Power} was \textbf{DEPRECATED} in \textit{MTConnect Version 1.1} and was replaced by \block{Availability}.


\subsubsection{Process}
\label{sec:Process}



\block{Process} is an abstract \block{Component} type that represents the manufacturing process being executed on a piece of equipment.



\paragraph{PartOccurrence}\mbox{}
\label{sec:PartOccurrence}


\block{PartOccurrence} \gls{organize} information about a specific part as it exists at a specific place and time, such as a specific instance of a bracket at a specific timestamp.


\subsubsection{Resource}
\label{sec:Resource}



\block{Resource} is an abstract \block{Component} that represents types of items, materials, and personnel that support the operation of a piece of equipment or work to be performed at a location.

\block{Resource} also represents materials or other items consumed or transformed by a piece of equipment for production of parts or other types of goods.


\paragraph{Material}\mbox{}
\label{sec:Material}


\block{Materials} provides information about materials or other items consumed or used by the piece of equipment for production of parts, materials, or other types of goods.


\paragraph{Stock}\mbox{}
\label{sec:Stock}


\block{Stock} represents the information for the material that is used in a manufacturing process and to which work is applied in a machine or piece of equipment to produce parts.


\paragraph{Personnel}\mbox{}
\label{sec:Personnel}


\block{Personnel} provides information about an individual or individuals who either control, support, or otherwise interface with a piece of equipment.



\subsubsection{Structure}
\label{sec:Structure}



\block{Structure} is a \block{Component} that represents the part(s) comprising the rigid bodies of the piece of equipment.


\paragraph{Link}\mbox{}
\label{sec:Link}


\block{Link} is a structural \block{Component} providing a connection between \block{Component}s.


\paragraph{Table}\mbox{}
\label{sec:Table}





\subsubsection{System}
\label{sec:System}



\block{System} is a \block{Component} that represents the major sub-system(s) that are permanently integrated into a piece of equipment.


\paragraph{Coolant}\mbox{}
\label{sec:Coolant}


\block{Coolant} represents the information for a system comprised of all the parts involved in distribution and management of fluids that remove heat from a piece of equipment.


\paragraph{Deposition}\mbox{}
\label{sec:Deposition}


\block{Deposition} represents the information for a system that manages the addition of material or state change of material being performed in an additive manufacturing process.  For example, this could describe the portion of a piece of equipment that manages a material extrusion process or a vat polymerization process.


\paragraph{Dielectric}\mbox{}
\label{sec:Dielectric}


\block{Dielectric} represents the information for a system that manages a chemical mixture used in a manufacturing process being performed at that piece of equipment.


\paragraph{Electric}\mbox{}
\label{sec:Electric}


\block{Electric} represents the information for the main power supply for device piece of equipment and the distribution of that power throughout the equipment.


\paragraph{EndEffector}\mbox{}
\label{sec:EndEffector}


\block{EndEffector} represents the information for those functions that form the last link segment of a piece of equipment. It is the part of a piece of equipment that interacts with the manufacturing process.


\paragraph{Environmental}\mbox{}
\label{sec:Environmental}


\block{Environmental} represents the information for a unit or function involved in monitoring, managing, or conditioning the environment around or within a piece of equipment.


\paragraph{Heating}\mbox{}
\label{sec:Heating}


\block{Heating} is a system used to deliver controlled amounts of heat to achieve a target temperature at a specified heating rate.

Note: As an example, Energy Delivery Method can be either through Electric heaters or Gas burners.


\paragraph{Vacuum}\mbox{}
\label{sec:Vacuum}


\block{Vacuum} is a system that evacuates gases and liquids from an enclosed and sealed space to a controlled negative pressure or a molecular density below the prevailing atmospheric level.


\paragraph{Cooling}\mbox{}
\label{sec:Cooling}


\block{Cooling} is a system used to to extract controlled amounts of heat to achieve a target temperature at a specified cooling rate.

Note: As an example, Energy Extraction Method can be via cooling water pipes running through the chamber.


\paragraph{Pressure}\mbox{}
\label{sec:Pressure}


\block{Pressure} is a system that delivers compressed gas or fluid and controls the pressure and rate of pressure change to a desired target set-point.

Note: For example, Delivery Method can be a Compressed Air or N2 tank that is piped via an inlet valve to the chamber.


\paragraph{Feeder}\mbox{}
\label{sec:Feeder}


\block{Feeder} represents the information for a system that manages the delivery of materials within a piece of equipment.


\paragraph{Hydraulic}\mbox{}
\label{sec:Hydraulic}


\block{Hydraulic} represents the information for a system comprised of all the parts involved in moving and distributing pressurized liquid throughout the piece of equipment.


\paragraph{Lubrication}\mbox{}
\label{sec:Lubrication}


\block{Lubrication} represents the information for a system comprised of all the parts involved in distribution and management of fluids used to lubricate portions of the piece of equipment.


\paragraph{Pneumatic}\mbox{}
\label{sec:Pneumatic}


\block{Pneumatic} is a system that uses compressed gasses to actuate components or do work within the piece of equipment.

Note: Actuation is usually performed using a cylinder.


\paragraph{ProcessPower}\mbox{}
\label{sec:ProcessPower}


\block{ProcessPower} represents the information for a power source associated with a piece of equipment that supplies energy to the manufacturing process separate from the \block{Electric} system.


\paragraph{Protective}\mbox{}
\label{sec:Protective}


\block{Protective} represents the information for those functions that detect or prevent harm or damage to equipment or personnel.


\paragraph{WorkEnvelope}\mbox{}
\label{sec:WorkEnvelope}


\block{WorkEnvelope} organizes information about the physical process execution space within a piece of equipment.

