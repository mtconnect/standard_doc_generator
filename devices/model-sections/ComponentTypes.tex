% Generated 2020-01-29 13:56:33 -0500
\subsection{ComponentTypes} \label{model:ComponentTypes}
\subsubsection[Actuator]{Actuator \\ {\small Subtype of Component}}
  \label{type:Actuator}

\FloatBarrier

Redefined as a piece of equipment with the ability to be represented as a {term:Lower Level} component of a parent {model:Component} element or as a {model:Composition} element. See {model:ACTUATOR}

\FloatBarrier
\subsubsection[Auxiliaries]{Auxiliaries \\ {\small Subtype of Component}}
  \label{type:Auxiliaries}

\FloatBarrier

An XML container used to organize information for {term:Lower Level} elements representing functional sub-systems that provide supplementary or extended capabilities for a piece of equipment, but they are not required for the basic operation of the equipment.

\FloatBarrier
\subsubsection[BarFeeder]{BarFeeder \\ {\small Subtype of Auxiliaries}}
  \label{type:BarFeeder}

\FloatBarrier

{model:BarFeeder} is an XML container that represents the information for a unit involved in delivering bar stock to a piece of equipment.

\begin{table}[ht]
\centering 
  \caption{\texttt{Properties of BarFeeder}}
  \label{properties:BarFeeder}
\tabulinesep=3pt
\begin{tabu} to 6in {|l|l|l|} \everyrow{\hline}
\hline
\rowfont\bfseries {Properties} & {ValueType} & {Multiplicity} \\
\tabucline[1.5pt]{}
\end{tabu}
\end{table}
\FloatBarrier

\FloatBarrier
\subsubsection[Environmental]{Environmental \\ {\small Subtype of Auxiliaries}}
  \label{type:Environmental}

\FloatBarrier

{model:Environmental} is an XML container that represents the information for a unit or function involved in monitoring, managing, or conditioning the environment around or within a piece of equipment.

\begin{table}[ht]
\centering 
  \caption{\texttt{Properties of Environmental}}
  \label{properties:Environmental}
\tabulinesep=3pt
\begin{tabu} to 6in {|l|l|l|} \everyrow{\hline}
\hline
\rowfont\bfseries {Properties} & {ValueType} & {Multiplicity} \\
\tabucline[1.5pt]{}
\end{tabu}
\end{table}
\FloatBarrier

\FloatBarrier
\subsubsection[Loader]{Loader \\ {\small Subtype of Auxiliaries}}
  \label{type:Loader}

\FloatBarrier

{model:Loader} is an XML container that represents the information for a unit comprised of all the parts involved in moving and distributing materials, parts, tooling, and other items to or from a piece of equipment.

\begin{table}[ht]
\centering 
  \caption{\texttt{Properties of Loader}}
  \label{properties:Loader}
\tabulinesep=3pt
\begin{tabu} to 6in {|l|l|l|} \everyrow{\hline}
\hline
\rowfont\bfseries {Properties} & {ValueType} & {Multiplicity} \\
\tabucline[1.5pt]{}
\end{tabu}
\end{table}
\FloatBarrier

\FloatBarrier
\subsubsection[Sensor]{Sensor \\ {\small Subtype of Auxiliaries}}
  \label{type:Sensor}

\FloatBarrier

The {term:sensor unit} is modeled as a {term:Lower Level} {model:Component} called {model:Sensor}.

\begin{table}[ht]
\centering 
  \caption{\texttt{Properties of Sensor}}
  \label{properties:Sensor}
\tabulinesep=3pt
\begin{tabu} to 6in {|l|l|l|} \everyrow{\hline}
\hline
\rowfont\bfseries {Properties} & {ValueType} & {Multiplicity} \\
\tabucline[1.5pt]{}
\end{tabu}
\end{table}
\FloatBarrier

\FloatBarrier
\subsubsection[ToolingDelivery]{ToolingDelivery \\ {\small Subtype of Auxiliaries}}
  \label{type:ToolingDelivery}

\FloatBarrier

{model:ToolingDelivery} is an XML container that represents the information for a unit involved in managing, positioning, storing, and delivering tooling within a piece of equipment.


\begin{table}[ht]
\centering 
  \caption{\texttt{Properties of ToolingDelivery}}
  \label{properties:ToolingDelivery}
\tabulinesep=3pt
\begin{tabu} to 6in {|l|l|l|} \everyrow{\hline}
\hline
\rowfont\bfseries {Properties} & {ValueType} & {Multiplicity} \\
\tabucline[1.5pt]{}
\end{tabu}
\end{table}
\FloatBarrier

\FloatBarrier
\subsubsection[WasteDisposal]{WasteDisposal \\ {\small Subtype of Auxiliaries}}
  \label{type:WasteDisposal}

\FloatBarrier

{model:WasteDisposal} is an XML container that represents the information for a unit comprised of all the parts involved in removing manufacturing byproducts from a piece of equipment.


\begin{table}[ht]
\centering 
  \caption{\texttt{Properties of WasteDisposal}}
  \label{properties:WasteDisposal}
\tabulinesep=3pt
\begin{tabu} to 6in {|l|l|l|} \everyrow{\hline}
\hline
\rowfont\bfseries {Properties} & {ValueType} & {Multiplicity} \\
\tabucline[1.5pt]{}
\end{tabu}
\end{table}
\FloatBarrier

\FloatBarrier
\subsubsection[Axes]{Axes \\ {\small Subtype of Component}}
  \label{type:Axes}

\FloatBarrier

An XML container used to organize the {term:Structural Elements} of a piece of equipment that perform linear or rotational motion.

\FloatBarrier
\subsubsection[Linear]{Linear \\ {\small Subtype of Axes}}
  \label{type:Linear}

\FloatBarrier

A {model:Linear} axis represents the movement of a physical piece of equipment, or a portion of the equipment, in a straight line. 

\begin{table}[ht]
\centering 
  \caption{\texttt{Properties of Linear}}
  \label{properties:Linear}
\tabulinesep=3pt
\begin{tabu} to 6in {|l|l|l|} \everyrow{\hline}
\hline
\rowfont\bfseries {Properties} & {ValueType} & {Multiplicity} \\
\tabucline[1.5pt]{}
\end{tabu}
\end{table}
\FloatBarrier

\FloatBarrier
\subsubsection[Rotary]{Rotary \\ {\small Subtype of Axes}}
  \label{type:Rotary}

\FloatBarrier

A {model:Rotary} axis represents any non-linear or rotary movement of a physical piece of equipment or a portion of the equipment.

\begin{table}[ht]
\centering 
  \caption{\texttt{Properties of Rotary}}
  \label{properties:Rotary}
\tabulinesep=3pt
\begin{tabu} to 6in {|l|l|l|} \everyrow{\hline}
\hline
\rowfont\bfseries {Properties} & {ValueType} & {Multiplicity} \\
\tabucline[1.5pt]{}
\end{tabu}
\end{table}
\FloatBarrier

\FloatBarrier
\subsubsection[Chuck]{Chuck \\ {\small Subtype of Rotary}}
  \label{type:Chuck}

\FloatBarrier

Chuck is an XML container that provides the information about a mechanism that holds a part or stock material in place.

\begin{table}[ht]
\centering 
  \caption{\texttt{Properties of Chuck}}
  \label{properties:Chuck}
\tabulinesep=3pt
\begin{tabu} to 6in {|l|l|l|} \everyrow{\hline}
\hline
\rowfont\bfseries {Properties} & {ValueType} & {Multiplicity} \\
\tabucline[1.5pt]{}
\end{tabu}
\end{table}
\FloatBarrier

\FloatBarrier
\subsubsection[Controller]{Controller \\ {\small Subtype of Component}}
  \label{type:Controller}

\FloatBarrier

An XML container used to organize information about an intelligent or computational function within a piece of equipment.

\FloatBarrier
\subsubsection[Path]{Path \\ {\small Subtype of Controller}}
  \label{type:Path}

\FloatBarrier

{model:Path} is an XML container that represents the information for an independent operation or function within a {model:Controller}.

\begin{table}[ht]
\centering 
  \caption{\texttt{Properties of Path}}
  \label{properties:Path}
\tabulinesep=3pt
\begin{tabu} to 6in {|l|l|l|} \everyrow{\hline}
\hline
\rowfont\bfseries {Properties} & {ValueType} & {Multiplicity} \\
\tabucline[1.5pt]{}
\end{tabu}
\end{table}
\FloatBarrier

\FloatBarrier
\subsubsection[Deposition]{Deposition \\ {\small Subtype of Component}}
  \label{type:Deposition}

\FloatBarrier

{model:Deposition} is an XML container that represents the information for a system that manages the addition of material or state change of material being performed in an additive manufacturing process.  For example, this could describe the portion of a piece of equipment that manages a material extrusion process or a vat polymerization process.

\FloatBarrier
\subsubsection[Door]{Door \\ {\small Subtype of Component}}
  \label{type:Door}

\FloatBarrier

{model:Door} is an XML container that represents the information for a mechanical mechanism or closure that can cover.

\FloatBarrier
\subsubsection[EndEffector]{EndEffector \\ {\small Subtype of Component}}
  \label{type:EndEffector}

\FloatBarrier

{model:EndEffector} is an XML container that represents the information for those functions that form the last link segment of a piece of equipment. It is the part of a piece of equipment that interacts with the manufacturing process.

\FloatBarrier
\subsubsection[Interface]{Interface \\ {\small Subtype of Component}}
  \label{type:Interface}

\FloatBarrier

Each {model:Interface} contains {term:Data Entities} available from the piece of equipment that may be needed to coordinate activities with associated pieces of equipment.

\FloatBarrier
\subsubsection[Interfaces]{Interfaces \\ {\small Subtype of Component}}
  \label{type:Interfaces}

\FloatBarrier

An XML container that organizes information used to coordinate actions and activities between pieces of equipment that communicate information between each other.

\FloatBarrier
\subsubsection[Power]{Power \\ {\small Subtype of Component}}
  \label{type:Power}

\FloatBarrier

{model:Power} was *DEPRECATED* in MTConnect Version 1.1 and was replaced by the {term:Data Entity} called {model:AVAILABILITY}.

\FloatBarrier
\subsubsection[Resources]{Resources \\ {\small Subtype of Component}}
  \label{type:Resources}

\FloatBarrier

An XML container used to organize information for {term:Lower Level} elements representing types of items, materials, and personnel that support the operation of a piece of equipment or work to be performed at a location. {model:Resources} also represents materials or other items consumed or transformed by a piece of equipment for production of parts or other types of goods.

\FloatBarrier
\subsubsection[Materials]{Materials \\ {\small Subtype of Resources}}
  \label{type:Materials}

\FloatBarrier

{model:Materials} is an XML container that provides information about materials or other items consumed or used by the piece of equipment for production of parts, materials, or other types of goods.

\begin{table}[ht]
\centering 
  \caption{\texttt{Properties of Materials}}
  \label{properties:Materials}
\tabulinesep=3pt
\begin{tabu} to 6in {|l|l|l|} \everyrow{\hline}
\hline
\rowfont\bfseries {Properties} & {ValueType} & {Multiplicity} \\
\tabucline[1.5pt]{}
\end{tabu}
\end{table}
\FloatBarrier

\FloatBarrier
\subsubsection[Stock]{Stock \\ {\small Subtype of Materials}}
  \label{type:Stock}

\FloatBarrier

{model:Stock} is an XML container that represents the information for the material that is used in a manufacturing process and to which work is applied in a machine or piece of equipment to produce parts.

\begin{table}[ht]
\centering 
  \caption{\texttt{Properties of Stock}}
  \label{properties:Stock}
\tabulinesep=3pt
\begin{tabu} to 6in {|l|l|l|} \everyrow{\hline}
\hline
\rowfont\bfseries {Properties} & {ValueType} & {Multiplicity} \\
\tabucline[1.5pt]{}
\end{tabu}
\end{table}
\FloatBarrier

\FloatBarrier
\subsubsection[Personnel]{Personnel \\ {\small Subtype of Resources}}
  \label{type:Personnel}

\FloatBarrier

{model:Personnel} is an XML container that provides information about an individual or individuals who either control, support, or otherwise interface with a piece of equipment.


\begin{table}[ht]
\centering 
  \caption{\texttt{Properties of Personnel}}
  \label{properties:Personnel}
\tabulinesep=3pt
\begin{tabu} to 6in {|l|l|l|} \everyrow{\hline}
\hline
\rowfont\bfseries {Properties} & {ValueType} & {Multiplicity} \\
\tabucline[1.5pt]{}
\end{tabu}
\end{table}
\FloatBarrier

\FloatBarrier
\subsubsection[Systems]{Systems \\ {\small Subtype of Component}}
  \label{type:Systems}

\FloatBarrier

An XML container used to organize information for {term:Lower Level} elements representing the major sub-systems that are permanently integrated into a piece of equipment.

\FloatBarrier
\subsubsection[Coolant]{Coolant \\ {\small Subtype of Systems}}
  \label{type:Coolant}

\FloatBarrier

{model:Coolant} is an XML container that represents the information for a system comprised of all the parts involved in distribution and management of fluids that remove heat from a piece of equipment.

\begin{table}[ht]
\centering 
  \caption{\texttt{Properties of Coolant}}
  \label{properties:Coolant}
\tabulinesep=3pt
\begin{tabu} to 6in {|l|l|l|} \everyrow{\hline}
\hline
\rowfont\bfseries {Properties} & {ValueType} & {Multiplicity} \\
\tabucline[1.5pt]{}
\end{tabu}
\end{table}
\FloatBarrier

\FloatBarrier
\subsubsection[Dielectric]{Dielectric \\ {\small Subtype of Systems}}
  \label{type:Dielectric}

\FloatBarrier

{model:Dielectric} is an XML container that represents the information for a system that manages a chemical mixture used in a manufacturing process being performed at that piece of equipment.

\begin{table}[ht]
\centering 
  \caption{\texttt{Properties of Dielectric}}
  \label{properties:Dielectric}
\tabulinesep=3pt
\begin{tabu} to 6in {|l|l|l|} \everyrow{\hline}
\hline
\rowfont\bfseries {Properties} & {ValueType} & {Multiplicity} \\
\tabucline[1.5pt]{}
\end{tabu}
\end{table}
\FloatBarrier

\FloatBarrier
\subsubsection[Electric]{Electric \\ {\small Subtype of Systems}}
  \label{type:Electric}

\FloatBarrier

{model:Electric} is an XML container that represents the information for the main power supply for device piece of equipment and the distribution of that power throughout the equipment.

\begin{table}[ht]
\centering 
  \caption{\texttt{Properties of Electric}}
  \label{properties:Electric}
\tabulinesep=3pt
\begin{tabu} to 6in {|l|l|l|} \everyrow{\hline}
\hline
\rowfont\bfseries {Properties} & {ValueType} & {Multiplicity} \\
\tabucline[1.5pt]{}
\end{tabu}
\end{table}
\FloatBarrier

\FloatBarrier
\subsubsection[Enclosure]{Enclosure \\ {\small Subtype of Systems}}
  \label{type:Enclosure}

\FloatBarrier

{model:Enclosure} is an XML container that represents the information for a structure used to contain or isolate a piece of equipment or area.

\begin{table}[ht]
\centering 
  \caption{\texttt{Properties of Enclosure}}
  \label{properties:Enclosure}
\tabulinesep=3pt
\begin{tabu} to 6in {|l|l|l|} \everyrow{\hline}
\hline
\rowfont\bfseries {Properties} & {ValueType} & {Multiplicity} \\
\tabucline[1.5pt]{}
\end{tabu}
\end{table}
\FloatBarrier

\FloatBarrier
\subsubsection[Feeder]{Feeder \\ {\small Subtype of Systems}}
  \label{type:Feeder}

\FloatBarrier

{model:Feeder} is an XML container that represents the information for a system that manages the delivery of materials within a piece of equipment.

\begin{table}[ht]
\centering 
  \caption{\texttt{Properties of Feeder}}
  \label{properties:Feeder}
\tabulinesep=3pt
\begin{tabu} to 6in {|l|l|l|} \everyrow{\hline}
\hline
\rowfont\bfseries {Properties} & {ValueType} & {Multiplicity} \\
\tabucline[1.5pt]{}
\end{tabu}
\end{table}
\FloatBarrier

\FloatBarrier
\subsubsection[Hydraulic]{Hydraulic \\ {\small Subtype of Systems}}
  \label{type:Hydraulic}

\FloatBarrier

{model:Hydraulic} is an XML container that represents the information for a system comprised of all the parts involved in moving and distributing pressurized liquid throughout the piece of equipment.

\begin{table}[ht]
\centering 
  \caption{\texttt{Properties of Hydraulic}}
  \label{properties:Hydraulic}
\tabulinesep=3pt
\begin{tabu} to 6in {|l|l|l|} \everyrow{\hline}
\hline
\rowfont\bfseries {Properties} & {ValueType} & {Multiplicity} \\
\tabucline[1.5pt]{}
\end{tabu}
\end{table}
\FloatBarrier

\FloatBarrier
\subsubsection[Lubrication]{Lubrication \\ {\small Subtype of Systems}}
  \label{type:Lubrication}

\FloatBarrier

{model:Lubrication} is an XML container that represents the information for a system comprised of all the parts involved in distribution and management of fluids used to lubricate portions of the piece of equipment.

\begin{table}[ht]
\centering 
  \caption{\texttt{Properties of Lubrication}}
  \label{properties:Lubrication}
\tabulinesep=3pt
\begin{tabu} to 6in {|l|l|l|} \everyrow{\hline}
\hline
\rowfont\bfseries {Properties} & {ValueType} & {Multiplicity} \\
\tabucline[1.5pt]{}
\end{tabu}
\end{table}
\FloatBarrier

\FloatBarrier
\subsubsection[Pneumatic]{Pneumatic \\ {\small Subtype of Systems}}
  \label{type:Pneumatic}

\FloatBarrier

{model:Pneumatic} is an XML container that represents the information for a system comprised of all the parts involved in moving and distributing pressurized gas throughout the piece of equipment.

\begin{table}[ht]
\centering 
  \caption{\texttt{Properties of Pneumatic}}
  \label{properties:Pneumatic}
\tabulinesep=3pt
\begin{tabu} to 6in {|l|l|l|} \everyrow{\hline}
\hline
\rowfont\bfseries {Properties} & {ValueType} & {Multiplicity} \\
\tabucline[1.5pt]{}
\end{tabu}
\end{table}
\FloatBarrier

\FloatBarrier
\subsubsection[ProcessPower]{ProcessPower \\ {\small Subtype of Systems}}
  \label{type:ProcessPower}

\FloatBarrier

{model:ProcessPower} is an XML container that represents the information for a power source associated with a piece of equipment that supplies energy to the manufacturing process separate from the {model:Electric} system.

\begin{table}[ht]
\centering 
  \caption{\texttt{Properties of ProcessPower}}
  \label{properties:ProcessPower}
\tabulinesep=3pt
\begin{tabu} to 6in {|l|l|l|} \everyrow{\hline}
\hline
\rowfont\bfseries {Properties} & {ValueType} & {Multiplicity} \\
\tabucline[1.5pt]{}
\end{tabu}
\end{table}
\FloatBarrier

\FloatBarrier
\subsubsection[Protective]{Protective \\ {\small Subtype of Systems}}
  \label{type:Protective}

\FloatBarrier

Protective is an XML container that represents the information for those functions that detect or prevent harm or damage to equipment or personnel.

\begin{table}[ht]
\centering 
  \caption{\texttt{Properties of Protective}}
  \label{properties:Protective}
\tabulinesep=3pt
\begin{tabu} to 6in {|l|l|l|} \everyrow{\hline}
\hline
\rowfont\bfseries {Properties} & {ValueType} & {Multiplicity} \\
\tabucline[1.5pt]{}
\end{tabu}
\end{table}
\FloatBarrier

\FloatBarrier
