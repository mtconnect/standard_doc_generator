% Generated 2020-05-14 16:42:03 -0400
\subsection{DefinitionElements} \label{sec:DefinitionElements}

This section provides detailed descriptions for the elements of a \gls{definition} element for a \gls{dataitem}.


\subsubsection{CellDefinition}
  \label{sec:CellDefinition}


The semantic definition of a \gls{cell}.


\paragraph{Attributes of CellDefinition}\mbox{}
\label{sec:Attributes of CellDefinition}

\tbl{attributes of CellDefinition} lists the attributes of \texttt{CellDefinition}.

\begin{table}[ht]
\centering 
  \caption{Attributes of CellDefinition}}
  \label{table:attributes of CellDefinition}
\tabulinesep=3pt
\begin{tabu} to 6in {|l|l|l|} \everyrow{\hline}
\hline
\rowfont\bfseries {Attribute} & {Type} & {Multiplicity} \\
\tabucline[1.5pt]{}
\texttt{units} & \texttt{UnitEnum} & 0..1 \\
\texttt{key} & \texttt{string} & 1 \\
\texttt{type} & \texttt{DataItemTypeEnum} & 0..1 \\
\texttt{subType} & \texttt{DataItemSubTypeEnum} & 0..1 \\
\end{tabu}
\end{table}
\FloatBarrier


Descriptions for attributes of \texttt{CellDefinition}:

\begin{itemize}
\item \texttt{units} : 
\item \texttt{key} : 
\item \texttt{type} : 
\item \texttt{subType} : 
\end{itemize}

\paragraph{Elements of CellDefinition}\mbox{}
\label{sec:Elements of CellDefinition}

\tbl{elements of CellDefinition} lists the elements of \texttt{CellDefinition}.

\begin{table}[ht]
\centering 
  \caption{Elements of CellDefinition}}
  \label{table:elements of CellDefinition}
\tabulinesep=3pt
\begin{tabu} to 6in {|l|l|l|} \everyrow{\hline}
\hline
\rowfont\bfseries {Association Name} & {Element} & {Multiplicity} \\
\tabucline[1.5pt]{}
\texttt{Description} & \texttt{Description} & 0..1 \\
\end{tabu}
\end{table}
\FloatBarrier


Descriptions for elements of \texttt{CellDefinition}:

\begin{itemize}
\item \texttt{Description} : An element that can contain any descriptive content.
\end{itemize}
\FloatBarrier

\subsubsection{Description}
  \label{sec:Description}


An element that can contain any descriptive content.


\paragraph{Attributes of Description}\mbox{}
\label{sec:Attributes of Description}

\tbl{attributes of Description} lists the attributes of \texttt{Description}.

\begin{table}[ht]
\centering 
  \caption{Attributes of Description}}
  \label{table:attributes of Description}
\tabulinesep=3pt
\begin{tabu} to 6in {|l|l|l|} \everyrow{\hline}
\hline
\rowfont\bfseries {Attribute} & {Type} & {Multiplicity} \\
\tabucline[1.5pt]{}
\texttt{value} & \texttt{string} & 1 \\
\end{tabu}
\end{table}
\FloatBarrier


Descriptions for attributes of \texttt{Description}:

\begin{itemize}
\item \texttt{value} : 
\end{itemize}
\FloatBarrier

\subsubsection{EntryDefinition}
  \label{sec:EntryDefinition}


The semantic definition of an \gls{entry}.


\paragraph{Attributes of EntryDefinition}\mbox{}
\label{sec:Attributes of EntryDefinition}

\tbl{attributes of EntryDefinition} lists the attributes of \texttt{EntryDefinition}.

\begin{table}[ht]
\centering 
  \caption{Attributes of EntryDefinition}}
  \label{table:attributes of EntryDefinition}
\tabulinesep=3pt
\begin{tabu} to 6in {|l|l|l|} \everyrow{\hline}
\hline
\rowfont\bfseries {Attribute} & {Type} & {Multiplicity} \\
\tabucline[1.5pt]{}
\texttt{key} & \texttt{string} & 1 \\
\texttt{units} & \texttt{UnitEnum} & 0..1 \\
\texttt{type} & \texttt{DataItemTypeEnum} & 0..1 \\
\texttt{subType} & \texttt{DataItemSubTypeEnum} & 0..1 \\
\end{tabu}
\end{table}
\FloatBarrier


Descriptions for attributes of \texttt{EntryDefinition}:

\begin{itemize}
\item \texttt{key} : 
\item \texttt{units} : 
\item \texttt{type} : 
\item \texttt{subType} : 
\end{itemize}

\paragraph{Elements of EntryDefinition}\mbox{}
\label{sec:Elements of EntryDefinition}

\tbl{elements of EntryDefinition} lists the elements of \texttt{EntryDefinition}.

\begin{table}[ht]
\centering 
  \caption{Elements of EntryDefinition}}
  \label{table:elements of EntryDefinition}
\tabulinesep=3pt
\begin{tabu} to 6in {|l|l|l|} \everyrow{\hline}
\hline
\rowfont\bfseries {Association Name} & {Element} & {Multiplicity} \\
\tabucline[1.5pt]{}
\texttt{Description} & \texttt{Description} & 0..1 \\
\texttt{CellDefinitions} & \texttt{CellDefinition} & 0..* \\
\end{tabu}
\end{table}
\FloatBarrier


Descriptions for elements of \texttt{EntryDefinition}:

\begin{itemize}
\item \texttt{Description} : An element that can contain any descriptive content.
\item \texttt{CellDefinitions} : \gls{celldefinitions} \glspl{organize} \gls{celldefinition} elements.
\end{itemize}
\FloatBarrier
