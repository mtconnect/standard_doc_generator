% Generated 2021-02-06 01:28:14 +0530
\subsection{Sensor} 


\gls{Sensor} is a unique type of a piece of equipment.  A \gls{Sensor} is typically comprised of two major components: a \gls{sensor unit} that provides signal processing, conversion, and communications and the \glspl{sensing element} that provides a signal or measured value.

The \gls{sensor unit} is modeled as a \gls{Lower Level} \block{Component} called \block{Sensor}.  The \gls{sensing element} may be modeled as a \block{Composition} element of a \block{Sensor} element and the measured value would be modeled as a \block{DataItem} (See \sect{DataItems} for more information on \block{DataItem} elements).  Each \gls{sensor unit} may have multiple \glspl{sensing element}; each representing the data for a variety of measured values.

When a \gls{sensor unit} is modeled as a \block{Component} or as a separate piece of equipment, it may provide additional configuration information for the \glspl{sensor element} and the \gls{sensor unit} itself.  

\block{Configuration} data provides information required for maintenance and support of the sensor.

When \block{Sensor} represents the \gls{sensor unit} for multiple \gls{sensing element}(s), each sensing element is represented by a \block{Channel}.   The \gls{sensor unit} itself and each \block{Channel} representing one \gls{sensing element} \textbf{MAY} have its own configuration data.

\begin{figure}[ht]
  \centering
    \includegraphics[width=1.0\textwidth]{figures/SensorConfiguration.png}
  \caption{SensorConfiguration Diagram}
  \label{fig:SensorConfiguration Diagram}
\end{figure}

\FloatBarrier


Note: See \fig{SensorConfiguration Schema Diagram} for XML schema.



\subsubsection{SensorConfiguration}
\label{sec:SensorConfiguration}



\block{SensorConfiguration} contains configuration information about a \block{Sensor}.


\paragraph{Elements of SensorConfiguration}\mbox{}
\label{sec:Elements of SensorConfiguration}

\tbl{Elements of SensorConfiguration} lists the elements of \texttt{SensorConfiguration}.

\begin{table}[ht]
\centering 
  \caption{Elements of SensorConfiguration}
  \label{table:Elements of SensorConfiguration}
\tabulinesep=3pt
\begin{tabu} to 6in {|l|l|} \everyrow{\hline}
\hline
\rowfont\bfseries {Element} & {Multiplicity} \\
\tabucline[1.5pt]{}
\texttt{CalibrationDate} & 0..1 \\
\texttt{CalibrationInitials} & 0..1 \\
\texttt{FirmwareVersion} & 1 \\
\texttt{NextCalibrationDate} & 0..1 \\
\texttt{Channel} (organized by \block{Channels}) & 0..* \\
\end{tabu}
\end{table}
\FloatBarrier


Descriptions for elements of \block{SensorConfiguration}:

\begin{itemize}

\item \block{CalibrationDate} \newline Date upon which the \gls{sensor unit} was last calibrated.

The value of \block{CalibrationDate} \MUST be \texttt{dateTime}.

\item \block{CalibrationInitials} \newline The initials of the person verifying the validity of the calibration data.

The value of \block{CalibrationInitials} \MUST be \texttt{string}.

\item \block{FirmwareVersion} \newline Version number for the sensor unit as specified by the manufacturer.


The value of \block{FirmwareVersion} \MUST be \texttt{string}.

\item \block{NextCalibrationDate} \newline Date upon which the \gls{sensor unit} is next scheduled to be calibrated.

The value of \block{NextCalibrationDate} \MUST be \texttt{dateTime}.

\item \block{Channels} \newline \block{Channels} \glspl{organize} \block{Channel} elements.

\end{itemize}



\subsubsection{Channel}
\label{sec:Channel}



When \block{Sensor} represents multiple \glspl{sensing element}, each \gls{sensing element} is represented by a \block{Channel} for the \block{Sensor}. 


\paragraph{Attributes of Channel}\mbox{}
\label{sec:Attributes of Channel}

\tbl{Attributes of Channel} lists the attributes of \texttt{Channel}.

\begin{table}[ht]
\centering 
  \caption{Attributes of Channel}
  \label{table:Attributes of Channel}
\tabulinesep=3pt
\begin{tabu} to 6in {|l|l|l|} \everyrow{\hline}
\hline
\rowfont\bfseries {Attribute} & {Type} & {Multiplicity} \\
\tabucline[1.5pt]{}

\property{name}[Channel] & \texttt{NMTOKEN} & 0..1 \\
\property{number}[Channel] & \texttt{NMTOKEN} & 1 \\
\end{tabu}
\end{table}
\FloatBarrier

Descriptions for attributes of \block{Channel}:

\begin{itemize}

\item \property{name}[Channel] \newline The name of an element or a piece of equipment.

\item \property{number}[Channel] \newline A unique identifier that will only refer to a specific \gls{sensing element}.
\end{itemize}


\paragraph{Elements of Channel}\mbox{}
\label{sec:Elements of Channel}

\tbl{Elements of Channel} lists the elements of \texttt{Channel}.

\begin{table}[ht]
\centering 
  \caption{Elements of Channel}
  \label{table:Elements of Channel}
\tabulinesep=3pt
\begin{tabu} to 6in {|l|l|} \everyrow{\hline}
\hline
\rowfont\bfseries {Element} & {Multiplicity} \\
\tabucline[1.5pt]{}
\texttt{CalibrationDate} & 0..1 \\
\texttt{CalibrationInitials} & 0..1 \\
\texttt{NextCalibrationDate} & 0..1 \\
\texttt{Description} & 0..1 \\
\texttt{SensorConfiguration} (organized by \block{Channels}) & 1 \\
\end{tabu}
\end{table}
\FloatBarrier


Descriptions for elements of \block{Channel}:

\begin{itemize}

\item \block{CalibrationDate} \newline Date upon which the \gls{sensor unit} was last calibrated to the \gls{sensor element}.

The value of \block{CalibrationDate} \MUST be \texttt{dateTime}.

\item \block{CalibrationInitials} \newline The initials of the person verifying the validity of the calibration data.

The value of \block{CalibrationInitials} \MUST be \texttt{string}.

\item \block{NextCalibrationDate} \newline Date upon which the \gls{sensor element} is next scheduled to be calibrated with the \gls{sensor unit}.


The value of \block{NextCalibrationDate} \MUST be \texttt{dateTime}.

\item \block{Description} \newline The descriptive content.

\item \block{Channels} \newline \block{Channels} \glspl{organize} \block{Channel} elements.

\end{itemize}


