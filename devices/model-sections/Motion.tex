% Generated 2020-07-27 15:21:27 +0530
\subsection{Motion} \label{sec:Motion}

\subsubsection{Axis}
  \label{sec:Axis}





\paragraph{Attributes of Axis}\mbox{}
\label{sec:Attributes of Axis}

\tbl{attributes of Axis} lists the attributes of \texttt{Axis}.

\begin{table}[ht]
\centering 
  \caption{Attributes of Axis}
  \label{table:attributes of Axis}
\tabulinesep=3pt
\begin{tabu} to 6in {|l|l|l|} \everyrow{\hline}
\hline
\rowfont\bfseries {Attribute} & {Type} & {Multiplicity} \\
\tabucline[1.5pt]{}
\texttt{axis} & \texttt{UnitVector} & 1 \\
\end{tabu}
\end{table}
\FloatBarrier


Descriptions for attributes of \texttt{Axis}:

\begin{itemize}
\item \texttt{axis} : 
\end{itemize}
\FloatBarrier

\subsubsection{Motion}
  \label{sec:Motion}





\paragraph{Attributes of Motion}\mbox{}
\label{sec:Attributes of Motion}

\tbl{attributes of Motion} lists the attributes of \texttt{Motion}.

\begin{table}[ht]
\centering 
  \caption{Attributes of Motion}
  \label{table:attributes of Motion}
\tabulinesep=3pt
\begin{tabu} to 6in {|l|l|l|} \everyrow{\hline}
\hline
\rowfont\bfseries {Attribute} & {Type} & {Multiplicity} \\
\tabucline[1.5pt]{}
\texttt{actuation} & \texttt{ActuationTypeEnum} & 1 \\
\texttt{cooridnateSystemIdRef} & \texttt{IDREF} & 1 \\
\texttt{id} & \texttt{ID} & 1 \\
\texttt{parentIdRef} & \texttt{IDREF} & 0..1 \\
\texttt{type} & \texttt{MotionTypes} & 1 \\
\end{tabu}
\end{table}
\FloatBarrier


Descriptions for attributes of \texttt{Motion}:

\begin{itemize}
\item \texttt{actuation} : 
\tabulinesep = 5pt
\begin{longtabu} to \textwidth {
    |l|X|}
  \caption{ActuationTypeEnum Enumeration}
  \label{enum:ActuationTypeEnum} \\
\hline
Name & Description \\
\hline
\endfirsthead
\hline
\multicolumn{2}{|c|}{Continuation of Table \texttt{ActuationTypeEnum} Enumeration} \\
\hline
Name & Description \\
\hline
\endhead
\texttt{DIRECT} &  \\ \hline
\texttt{DERIVATIVE} &  \\ \hline
\texttt{VIRTUAL} &  \\ \hline
\texttt{FIXED} &  \\ \hline
\end{longtabu}
\FloatBarrier
\item \texttt{cooridnateSystemIdRef} : 
\item \texttt{id} : 
\item \texttt{parentIdRef} : 
\item \texttt{type} : 
\end{itemize}

\paragraph{Elements of Motion}\mbox{}
\label{sec:Elements of Motion}

\tbl{elements of Motion} lists the elements of \texttt{Motion}.

\begin{table}[ht]
\centering 
  \caption{Elements of Motion}
  \label{table:elements of Motion}
\tabulinesep=3pt
\begin{tabu} to 6in {|l|l|l|} \everyrow{\hline}
\hline
\rowfont\bfseries {Association Name} & {Element} & {Multiplicity} \\
\tabucline[1.5pt]{}
\texttt{Axis} & \texttt{Axis} & 1 \\
\texttt{Origin} & \texttt{Origin} & 0..1 \\
\texttt{Transformation} & \texttt{Transformation} & 0..1 \\
\end{tabu}
\end{table}
\FloatBarrier


Descriptions for elements of \texttt{Motion}:

\begin{itemize}
\item \texttt{Axis} : 
\item \texttt{Origin} : The coordinates of the origin position of a coordinate system.
\item \texttt{Transformation} :  The process of transforming to the origin position of the coordinate system from a parent coordinate system using \glselementname{translation event} and \glselementname{rotation event}.
\end{itemize}
\FloatBarrier

\subsubsection{UnitVector}
  \label{sec:UnitVector}




\FloatBarrier

\subsubsection{space}
  \label{sec:space}




\FloatBarrier
