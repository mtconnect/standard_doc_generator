% Generated 2020-08-19 14:44:33 +0530
\subsection{DataItems} \label{sec:DataItems}


\block{DataItems} \glspl{organize} \block{DataItem} elements.


\subsubsection{DataItem}
  \label{sec:DataItem}



\block{DataItem} describes a piece of information reported about a piece of equipment.


\paragraph{Attributes of DataItem}\mbox{}
\label{sec:Attributes of DataItem}

\tbl{attributes of DataItem} lists the attributes of \texttt{DataItem}.

\begin{table}[ht]
\centering 
  \caption{Attributes of DataItem}
  \label{table:attributes of DataItem}
\tabulinesep=3pt
\begin{tabu} to 6in {|l|l|l|} \everyrow{\hline}
\hline
\rowfont\bfseries {Attribute} & {Type} & {Multiplicity} \\
\tabucline[1.5pt]{}
\texttt{category} & \texttt{CategoryEnum} & 1 \\
\texttt{compositionId} & \texttt{ID} & 0..1 \\
\texttt{coordinateSystem} & \texttt{CoordinateSystemEnum} & 0..1 \\
\texttt{discrete} & \texttt{boolean} & 0..1 \\
\texttt{id} & \texttt{ID} & 1 \\
\texttt{name} & \texttt{string} & 0..1 \\
\texttt{nativeScale} & \texttt{integer} & 0..1 \\
\texttt{nativeUnits} & \texttt{NativeUnitEnum} & 0..1 \\
\texttt{sampleRate} & \texttt{float} & 0..1 \\
\texttt{significantDigits} & \texttt{integer} & 0..1 \\
\texttt{statistic} & \texttt{StatisticEnum} & 0..1 \\
\texttt{subType} & \texttt{DataItemSubTypeEnum} & 0..1 \\
\texttt{type} & \texttt{DataItemTypeEnum} & 1 \\
\texttt{units} & \texttt{UnitEnum} & 0..1 \\
\texttt{representation} & \texttt{RepresentationEnum} & 0..1 \\
\texttt{coordinateSystemIdRef} & \texttt{IDREF} & 0..1 \\
\end{tabu}
\end{table}
\FloatBarrier


Descriptions for attributes of \texttt{DataItem}:

\begin{itemize}
\item \texttt{category} : Specifies the kind of information provided by a data item.
\tabulinesep = 5pt
\begin{longtabu} to \textwidth {
    |l|X|}
  \caption{CategoryEnum Enumeration}
  \label{enum:CategoryEnum} \\
\hline
Name & Description \\
\hline
\endfirsthead
\hline
\multicolumn{2}{|c|}{Continuation of Table \texttt{CategoryEnum} Enumeration} \\
\hline
Name & Description \\
\hline
\endhead
\texttt{SAMPLE} &  \\ \hline
\texttt{EVENT} &  \\ \hline
\texttt{CONDITION} &  \\ \hline
\end{longtabu}
\FloatBarrier
\item \texttt{compositionId} : The identifier attribute of the \block{Composition} element that the reported data is most closely associated.
\item \texttt{coordinateSystem} : For measured values relative to a coordinate system like \block{POSITION}, the coordinate system being used may be reported.
\tabulinesep = 5pt
\begin{longtabu} to \textwidth {
    |l|X|}
  \caption{CoordinateSystemEnum Enumeration}
  \label{enum:CoordinateSystemEnum} \\
\hline
Name & Description \\
\hline
\endfirsthead
\hline
\multicolumn{2}{|c|}{Continuation of Table \texttt{CoordinateSystemEnum} Enumeration} \\
\hline
Name & Description \\
\hline
\endhead
\texttt{MACHINE} &  \\ \hline
\texttt{WORK} &  \\ \hline
\end{longtabu}
\FloatBarrier
\item \texttt{discrete} : An indication signifying whether each value reported for the \gls{Data Entity} is significant and whether duplicate values are to be suppressed.
  
 The value defined \textbf{MUST} be either \block{true} or \block{false} - an XML boolean type.
  
 \block{true} indicates that each update to the \gls{Data Entity}'s value is significant and duplicate values \textbf{MUSTNOT} be suppressed.
  
 \block{false} indicates that duplicated values \textbf{MUST} be suppressed.
  
 If a value is not defined for \block{discrete}, the default value \textbf{MUST} be \block{false}.
\item \texttt{id} : The unique identifier for this element.
\item \texttt{name} : The name of an element or a piece of equipment.
\item \texttt{nativeScale} : \block{nativeScale} \textbf{MAY} be used to convert the reported value to represent the original measured value.
\item \texttt{nativeUnits} : The native units of measurement for the reported value of the data item.
\tabulinesep = 5pt
\begin{longtabu} to \textwidth {
    |l|X|}
  \caption{NativeUnitEnum Enumeration}
  \label{enum:NativeUnitEnum} \\
\hline
Name & Description \\
\hline
\endfirsthead
\hline
\multicolumn{2}{|c|}{Continuation of Table \texttt{NativeUnitEnum} Enumeration} \\
\hline
Name & Description \\
\hline
\endhead
\texttt{CENTIPOISE} &  \\ \hline
\texttt{DEGREE/MINUTE} &  \\ \hline
\texttt{FAHRENHEIT} &  \\ \hline
\texttt{FOOT} &  \\ \hline
\texttt{FOOT/MINUTE} &  \\ \hline
\texttt{FOOT/SECOND} &  \\ \hline
\texttt{FOOT/SECOND\^{}2} &  \\ \hline
\texttt{FOOT\textunderscore 3D} &  \\ \hline
\texttt{GALLON/MINUTE} &  \\ \hline
\texttt{HOUR} &  \\ \hline
\texttt{INCH} &  \\ \hline
\texttt{INCH/MINUTE} &  \\ \hline
\texttt{INCH/SECOND} &  \\ \hline
\texttt{INCH/SECOND\^{}2} &  \\ \hline
\texttt{INCH\textunderscore POUND} &  \\ \hline
\texttt{INCH\textunderscore 3D} &  \\ \hline
\texttt{KELVIN} &  \\ \hline
\texttt{KILOWATT} &  \\ \hline
\texttt{KILOWATT\textunderscore HOUR} &  \\ \hline
\texttt{LITER} &  \\ \hline
\texttt{LITER/MINUTE} &  \\ \hline
\texttt{MILLIMETER/MINUTE} &  \\ \hline
\texttt{MINUTE} &  \\ \hline
\texttt{OTHER} &  \\ \hline
\texttt{POUND} &  \\ \hline
\texttt{POUND/INCH\^{}2} &  \\ \hline
\texttt{RADIAN} &  \\ \hline
\texttt{RADIAN/MINUTE} &  \\ \hline
\texttt{RADIAN/SECOND} &  \\ \hline
\texttt{RADIAN/SECOND\^{}2} &  \\ \hline
\texttt{REVOLUTION/SECOND} &  \\ \hline
\end{longtabu}
\FloatBarrier
\item \texttt{sampleRate} : The rate at which successive samples of a data item are recorded by a piece of equipment.
\item \texttt{significantDigits} : The number of significant digits in the reported value.
\item \texttt{statistic} : Describes the type of statistical calculation performed on a series of data samples to provide the reported data value.
\tabulinesep = 5pt
\begin{longtabu} to \textwidth {
    |l|X|}
  \caption{StatisticEnum Enumeration}
  \label{enum:StatisticEnum} \\
\hline
Name & Description \\
\hline
\endfirsthead
\hline
\multicolumn{2}{|c|}{Continuation of Table \texttt{StatisticEnum} Enumeration} \\
\hline
Name & Description \\
\hline
\endhead
\texttt{AVERAGE} &  \\ \hline
\texttt{KURTOSIS} &  \\ \hline
\texttt{MAXIMUM} &  \\ \hline
\texttt{MEDIAN} &  \\ \hline
\texttt{MINIMUM} &  \\ \hline
\texttt{MODE} &  \\ \hline
\texttt{RANGE} &  \\ \hline
\texttt{ROOT\textunderscore MEAN\textunderscore SQUARE} &  \\ \hline
\texttt{STANDARD\textunderscore DEVIATION} &  \\ \hline
\end{longtabu}
\FloatBarrier
\item \texttt{subType} : A sub-categorization of the data item \block{type}.
\item \texttt{type} : The type of either a \gls{Structural Element} or a \block{DataItem} being measured.
\item \texttt{units} : The unit of measurement for the reported value of the data item.
\tabulinesep = 5pt
\begin{longtabu} to \textwidth {
    |l|X|}
  \caption{UnitEnum Enumeration}
  \label{enum:UnitEnum} \\
\hline
Name & Description \\
\hline
\endfirsthead
\hline
\multicolumn{2}{|c|}{Continuation of Table \texttt{UnitEnum} Enumeration} \\
\hline
Name & Description \\
\hline
\endhead
\texttt{AMPERE} & Amps \\ \hline
\texttt{CELSIUS} & Degrees Celsius \\ \hline
\texttt{COUNT} & A count of something. \\ \hline
\texttt{DECIBEL} & Sound Level \\ \hline
\texttt{DEGREE} & Angle in degrees \\ \hline
\texttt{DEGREE\textunderscore 3D} & A space-delimited, floating-point representation of the angular rotation in degrees around the X, Y, and Z axes relative to a cartesian coordinate system respectively in order as A, B, and C. If any of the rotations is not known, it \textbf{MUST} be zero (0). \\ \hline
\texttt{DEGREE/SECOND} & Angular degrees per second \\ \hline
\texttt{DEGREE/SECOND\^{}2} & Angular acceleration in degrees per second squared \\ \hline
\texttt{HERTZ} & Frequency measured in cycles per second \\ \hline
\texttt{JOULE} & A measurement of energy. \\ \hline
\texttt{KILOGRAM} & Kilograms \\ \hline
\texttt{LITER} & Measurement of volume of a fluid \\ \hline
\texttt{LITER/SECOND} & Liters per second \\ \hline
\texttt{MICRO\textunderscore RADIAN} & Measurement of Tilt \\ \hline
\texttt{MILLIMETER} & Millimeters \\ \hline
\texttt{MILLIMETER\textunderscore 3D} & A point in space identified by X, Y, and Z positions and represented by a space-delimited set of numbers each expressed in millimeters. \\ \hline
\texttt{MILLIMETER/REVOLUTION} & Millimeters per revolution. \\ \hline
\texttt{MILLIMETER/SECOND} & Millimeters per second \\ \hline
\texttt{MILLIMETER/SECOND\^{}2} & Acceleration in millimeters per second squared \\ \hline
\texttt{NEWTON} & Force in Newtons \\ \hline
\texttt{NEWTON\textunderscore METER} & Torque, a unit for force times distance. \\ \hline
\texttt{OHM} & Measure of Electrical Resistance \\ \hline
\texttt{PASCAL} & Pressure in Newtons per square meter \\ \hline
\texttt{PASCAL\textunderscore SECOND} & Measurement of Viscosity \\ \hline
\texttt{PERCENT} & Percentage \\ \hline
\texttt{PH} & A measure of the acidity or alkalinity of a solution. \\ \hline
\texttt{REVOLUTION/MINUTE} & Revolutions per minute \\ \hline
\texttt{SECOND} & A measurement of time. \\ \hline
\texttt{SIEMENS/METER} & A measurement of Electrical Conductivity \\ \hline
\texttt{VOLT} & Volts \\ \hline
\texttt{VOLT\textunderscore AMPERE} & The measurement of the apparent power in an electrical circuit, equal to the product of root-mean-square (RMS) voltage and RMS current (commonly referred to as VA). \\ \hline
\texttt{VOLT\textunderscore AMPERE\textunderscore REACTIVE} & The measurement of reactive power in an AC electrical circuit (commonly referred to as VAR). \\ \hline
\texttt{WATT} & Watts \\ \hline
\texttt{WATT\textunderscore SECOND} & Measurement of electrical energy, equal to one Joule \\ \hline
\texttt{REVOLUTION/SECOND} & Revolutions per second. \\ \hline
\texttt{REVOLUTION/SECOND\^{}2} & Revolutions per second squared. \\ \hline
\texttt{GRAM/CUBIC\textunderscore METER} & Gram per cubic meter. \\ \hline
\end{longtabu}
\FloatBarrier
\item \texttt{representation} : Description of a means to interpret data consisting of multiple data points or samples reported as a single value.  
 \block{representation} is an optional attribute.  
 \block{representation} will define a unique format for each set of data.  
 \block{representation} for \block{TIME\textunderscore SERIES}, \block{DISCRETE} \textbf{DEPRECATED:} in \textit{Version 1.5}, \block{DATA\textunderscore SET}, \block{TABLE} and \block{VALUE} are defined in \textbf{CITETITLE:MTCPart2} \textunderscore Section 7.2.2.12\textunderscore .  
 If \block{representation} is not specified, it \textbf{MUST} be determined to be \block{VALUE}.
\tabulinesep = 5pt
\begin{longtabu} to \textwidth {
    |l|X|}
  \caption{RepresentationEnum Enumeration}
  \label{enum:RepresentationEnum} \\
\hline
Name & Description \\
\hline
\endfirsthead
\hline
\multicolumn{2}{|c|}{Continuation of Table \texttt{RepresentationEnum} Enumeration} \\
\hline
Name & Description \\
\hline
\endhead
\texttt{TIME\textunderscore SERIES} &  \\ \hline
\texttt{VALUE} &  \\ \hline
\texttt{DATA\textunderscore SET} &  \\ \hline
\texttt{DISCRETE} &  \\ \hline
\texttt{TABLE} &  \\ \hline
\end{longtabu}
\FloatBarrier
\item \texttt{coordinateSystemIdRef} : The associated \block{CoordinateSystem} context for the \block{DataItem}.
\end{itemize}

\paragraph{Elements of DataItem}\mbox{}
\label{sec:Elements of DataItem}

\tbl{elements of DataItem} lists the elements of \texttt{DataItem}.

\begin{table}[ht]
\centering 
  \caption{Elements of DataItem}
  \label{table:elements of DataItem}
\tabulinesep=3pt
\begin{tabu} to 6in {|l|l|l|} \everyrow{\hline}
\hline
\rowfont\bfseries {Association Name} & {Element} & {Multiplicity} \\
\tabucline[1.5pt]{}
\texttt{Source} & \texttt{Source} & 0..1 \\
\texttt{Constraints} & \texttt{Constraints} & 0..1 \\
\texttt{Filters} & \texttt{Filter} & 0..* \\
\texttt{InitialValue} & \texttt{InitialValue} & 0..1 \\
\texttt{ResetTrigger} & \texttt{ResetTrigger} & 0..1 \\
\texttt{Definition} & \texttt{Definition} & 0..1 \\
\end{tabu}
\end{table}
\FloatBarrier


Descriptions for elements of \texttt{DataItem}:

\begin{itemize}
\item \texttt{Source} : \block{Source} identifies the \block{Component}, \block{DataItem}, or \block{Composition} representing the area of the piece of equipment from which a measured value originates.
\item \texttt{Constraints} : \block{Constraints} \glspl{organize} a set of expected values that can be reported for this \block{DataItem}.
\item \texttt{Filters} : \block{Filters} \glspl{organize} the \block{Filter} elements associated with this \block{DataItem} element. 
\item \texttt{InitialValue} : \block{InitialValue} defines the starting value for a data item as well as the value to be set for the data item after a reset event.
\item \texttt{ResetTrigger} : \block{ResetTrigger} identifies the type of event that may cause a reset to occur.
\item \texttt{Definition} : The \block{Definition} defines the meaning of \block{Entry} and \block{Cell} elements associated with the \block{DataItem} when the \property{representation} is either \block{DATA} or \block{TABLE}.
\end{itemize}
\FloatBarrier
