% Generated 2020-02-05 14:56:53 -0500
\subsection{DataItems} \label{model:DataItems}
\subsubsection{Constraints}
  \label{type:Constraints}

\FloatBarrier

Placeholder for documentation!

\begin{table}[ht]
\centering 
  \caption{\texttt{Properties of Constraints}}
  \label{properties:Constraints}
\tabulinesep=3pt
\begin{tabu} to 6in {|l|l|l|} \everyrow{\hline}
\hline
\rowfont\bfseries {Properties} & {Value} & {Multiplicity} \\
\tabucline[1.5pt]{}
\texttt{<<deprecated>> Filter} & \texttt{FilterEnum} & 0..1 \\
\texttt{Maximum} & \texttt{float} & 0..1 \\
\texttt{Minimum} & \texttt{float} & 0..1 \\
\texttt{Nominal} & \texttt{float} & 0..1 \\
\texttt{Value} & \texttt{float} & 0..* \\
\end{tabu}
\end{table}
\FloatBarrier


\paragraph{\texttt{Filter}}\mbox{}
\newline\tab Placeholder for documentation!

Placeholder for documentation!

\begin{table}[ht]
\centering 
  \caption{\texttt{FilterEnum} Enumeration}
  \label{enum:FilterEnum}
\tabulinesep=3pt
\begin{tabu} to 6in {|l|X|} \everyrow{\hline}
\hline
\rowfont\bfseries {Name} & {Description} \\
\tabucline[1.5pt]{}
\texttt{MINIMUM_DELTA} & Placeholder for documentation! \\
\texttt{PERIOD} & Placeholder for documentation! \\
\end{tabu}
\end{table} 
\FloatBarrier

\paragraph{\texttt{Maximum}}\mbox{}
\newline\tab Placeholder for documentation!

\paragraph{\texttt{Minimum}}\mbox{}
\newline\tab Placeholder for documentation!

\paragraph{\texttt{Nominal}}\mbox{}
\newline\tab Placeholder for documentation!

\paragraph{\texttt{Value}}\mbox{}
\newline\tab Placeholder for documentation!
\FloatBarrier
\subsubsection{DataItem}
  \label{type:DataItem}

\FloatBarrier

Placeholder for documentation!

\begin{table}[ht]
\centering 
  \caption{\texttt{Properties of DataItem}}
  \label{properties:DataItem}
\tabulinesep=3pt
\begin{tabu} to 6in {|l|l|l|} \everyrow{\hline}
\hline
\rowfont\bfseries {Properties} & {Value} & {Multiplicity} \\
\tabucline[1.5pt]{}
\texttt{category} & \texttt{CategoryEnum} & 1 \\
\texttt{compositionId} & \texttt{ID} & 0..1 \\
\texttt{coordinateSystem} & \texttt{CoordinateSystemEnum} & 0..1 \\
\texttt{discrete} & \texttt{boolean} & 0..1 \\
\texttt{id} & \texttt{ID} & 1 \\
\texttt{name} & \texttt{string} & 0..1 \\
\texttt{nativeScale} & \texttt{integer} & 0..1 \\
\texttt{nativeUnits} & \texttt{NativeUnitEnum} & 0..1 \\
\texttt{sampleRate} & \texttt{float} & 0..1 \\
\texttt{significantDigits} & \texttt{integer} & 0..1 \\
\texttt{statistic} & \texttt{StatisticEnum} & 0..1 \\
\texttt{subType} & \texttt{string} & 0..1 \\
\texttt{type} & \texttt{string} & 1 \\
\texttt{units} & \texttt{UnitEnum} & 0..1 \\
\texttt{representation} & \texttt{RepresentationEnum} & 0..1 \\
\texttt{Source} & \texttt{Source} & 0..1 \\
\texttt{Constraints} & \texttt{Constraints} & 0..1 \\
\texttt{Filters} & \texttt{Filter} & 0..* \\
\texttt{InitialValue} & \texttt{InitialValue} & 0..1 \\
\texttt{ResetTrigger} & \texttt{ResetTrigger} & 0..1 \\
\end{tabu}
\end{table}
\FloatBarrier


\paragraph{\texttt{category}}\mbox{}
\newline\tab Specifies the kind of information provided by a data item.

Placeholder for documentation!

\begin{table}[ht]
\centering 
  \caption{\texttt{CategoryEnum} Enumeration}
  \label{enum:CategoryEnum}
\tabulinesep=3pt
\begin{tabu} to 6in {|l|X|} \everyrow{\hline}
\hline
\rowfont\bfseries {Name} & {Description} \\
\tabucline[1.5pt]{}
\texttt{SAMPLE} & Placeholder for documentation! \\
\texttt{EVENT} & Placeholder for documentation! \\
\texttt{CONDITION} & Placeholder for documentation! \\
\end{tabu}
\end{table} 
\FloatBarrier

\paragraph{\texttt{compositionId}}\mbox{}
\newline\tab The identifier attribute of the {model:Composition} element that the reported data is most closely associated.

\paragraph{\texttt{coordinateSystem}}\mbox{}
\newline\tab For measured values relative to a coordinate system like {model:POSITION}, the coordinate system being used may be reported.

Placeholder for documentation!

\begin{table}[ht]
\centering 
  \caption{\texttt{CoordinateSystemEnum} Enumeration}
  \label{enum:CoordinateSystemEnum}
\tabulinesep=3pt
\begin{tabu} to 6in {|l|X|} \everyrow{\hline}
\hline
\rowfont\bfseries {Name} & {Description} \\
\tabucline[1.5pt]{}
\texttt{MACHINE} & Placeholder for documentation! \\
\texttt{WORK} & Placeholder for documentation! \\
\end{tabu}
\end{table} 
\FloatBarrier

\paragraph{\texttt{discrete}}\mbox{}
\newline\tab An indication signifying whether each value reported for the {term:Data Entity} is significant and whether duplicate values are to be suppressed.
  
 The value defined *MUST* be either {model:true} or {model:false} - an XML boolean type.
  
 {model:true} indicates that each update to the {term:Data Entity}'s value is significant and duplicate values *MUSTNOT* be suppressed.
  
 {model:false} indicates that duplicated values *MUST* be suppressed.
  
 If a value is not defined for {model:discrete}, the default value *MUST* be {model:false}.

\paragraph{\texttt{id}}\mbox{}
\newline\tab The unique identifier for this element.

\paragraph{\texttt{name}}\mbox{}
\newline\tab The name of an element or a piece of equipment.

\paragraph{\texttt{nativeScale}}\mbox{}
\newline\tab {model:nativeScale} *MAY* be used to convert the reported value to represent the original measured value.

\paragraph{\texttt{nativeUnits}}\mbox{}
\newline\tab The native units of measurement for the reported value of the data item.

Placeholder for documentation!

\begin{table}[ht]
\centering 
  \caption{\texttt{NativeUnitEnum} Enumeration}
  \label{enum:NativeUnitEnum}
\tabulinesep=3pt
\begin{tabu} to 6in {|l|X|} \everyrow{\hline}
\hline
\rowfont\bfseries {Name} & {Description} \\
\tabucline[1.5pt]{}
\texttt{CENTIPOISE} & Placeholder for documentation! \\
\texttt{DEGREE/MINUTE} & Placeholder for documentation! \\
\texttt{FAHRENHEIT} & Placeholder for documentation! \\
\texttt{FOOT} & Placeholder for documentation! \\
\texttt{FOOT/MINUTE} & Placeholder for documentation! \\
\texttt{FOOT/SECOND} & Placeholder for documentation! \\
\texttt{FOOT/SECOND\^2} & Placeholder for documentation! \\
\texttt{FOOT_3D} & Placeholder for documentation! \\
\texttt{GALLON/MINUTE} & Placeholder for documentation! \\
\texttt{HOUR} & Placeholder for documentation! \\
\texttt{INCH} & Placeholder for documentation! \\
\texttt{INCH/MINUTE} & Placeholder for documentation! \\
\texttt{INCH/SECOND} & Placeholder for documentation! \\
\texttt{INCH/SECOND\^2} & Placeholder for documentation! \\
\texttt{INCH_POUND} & Placeholder for documentation! \\
\texttt{INCH_3D} & Placeholder for documentation! \\
\texttt{KELVIN} & Placeholder for documentation! \\
\texttt{KILOWATT} & Placeholder for documentation! \\
\texttt{KILOWATT_HOUR} & Placeholder for documentation! \\
\texttt{LITER} & Placeholder for documentation! \\
\texttt{LITER/MINUTE} & Placeholder for documentation! \\
\texttt{MILLIMETER/MINUTE} & Placeholder for documentation! \\
\texttt{MINUTE} & Placeholder for documentation! \\
\texttt{OTHER} & Placeholder for documentation! \\
\texttt{POUND} & Placeholder for documentation! \\
\texttt{POUND/INCH\^2} & Placeholder for documentation! \\
\texttt{RADIAN} & Placeholder for documentation! \\
\texttt{RADIAN/MINUTE} & Placeholder for documentation! \\
\texttt{RADIAN/SECOND} & Placeholder for documentation! \\
\texttt{RADIAN/SECOND\^2} & Placeholder for documentation! \\
\texttt{REVOLUTION/SECOND} & Placeholder for documentation! \\
\end{tabu}
\end{table} 
\FloatBarrier

\paragraph{\texttt{sampleRate}}\mbox{}
\newline\tab The rate at which successive samples of a data item are recorded by a piece of equipment.

\paragraph{\texttt{significantDigits}}\mbox{}
\newline\tab The number of significant digits in the reported value.

\paragraph{\texttt{statistic}}\mbox{}
\newline\tab Describes the type of statistical calculation performed on a series of data samples to provide the reported data value.

Placeholder for documentation!

\begin{table}[ht]
\centering 
  \caption{\texttt{StatisticEnum} Enumeration}
  \label{enum:StatisticEnum}
\tabulinesep=3pt
\begin{tabu} to 6in {|l|X|} \everyrow{\hline}
\hline
\rowfont\bfseries {Name} & {Description} \\
\tabucline[1.5pt]{}
\texttt{AVERAGE} & Placeholder for documentation! \\
\texttt{KURTOSIS} & Placeholder for documentation! \\
\texttt{MAXIMUM} & Placeholder for documentation! \\
\texttt{MEDIAN} & Placeholder for documentation! \\
\texttt{MINIMUM} & Placeholder for documentation! \\
\texttt{MODE} & Placeholder for documentation! \\
\texttt{RANGE} & Placeholder for documentation! \\
\texttt{ROOT_MEAN_SQUARE} & Placeholder for documentation! \\
\texttt{STANDARD_DEVIATION} & Placeholder for documentation! \\
\end{tabu}
\end{table} 
\FloatBarrier

\paragraph{\texttt{subType}}\mbox{}
\newline\tab A sub-categorization of the data item {model:type}.

\paragraph{\texttt{type}}\mbox{}
\newline\tab The type of either a {term:Structural Element} or a {model:DataItem} being measured.

\paragraph{\texttt{units}}\mbox{}
\newline\tab The unit of measurement for the reported value of the data item.

Placeholder for documentation!

\begin{table}[ht]
\centering 
  \caption{\texttt{UnitEnum} Enumeration}
  \label{enum:UnitEnum}
\tabulinesep=3pt
\begin{tabu} to 6in {|l|X|} \everyrow{\hline}
\hline
\rowfont\bfseries {Name} & {Description} \\
\tabucline[1.5pt]{}
\texttt{AMPERE} & Amps \\
\texttt{CELSIUS} & Degrees Celsius \\
\texttt{COUNT} & A count of something. \\
\texttt{DECIBEL} & Sound Level \\
\texttt{DEGREE} & Angle in degrees \\
\texttt{DEGREE/SECOND} & Angular degrees per second \\
\texttt{DEGREE/SECOND\^2} & Angular acceleration in degrees per second squared \\
\texttt{HERTZ} & Frequency measured in cycles per second \\
\texttt{JOULE} & A measurement of energy. \\
\texttt{KILOGRAM} & Kilograms \\
\texttt{LITER} & Measurement of volume of a fluid \\
\texttt{LITER/SECOND} & Liters per second \\
\texttt{MICRO_RADIAN} & Measurement of Tilt \\
\texttt{MILLIMETER} & Millimeters \\
\texttt{MILLIMETER_3D} & A point in space identified by X, Y, and Z positions and represented by a space-delimited set of numbers each expressed in millimeters. \\
\texttt{MILLIMETER/REVOLUTION} & Millimeters per revolution. \\
\texttt{MILLIMETER/SECOND} & Millimeters per second \\
\texttt{MILLIMETER/SECOND\^2} & Acceleration in millimeters per second squared \\
\texttt{NEWTON} & Force in Newtons \\
\texttt{NEWTON_METER} & Torque, a unit for force times distance. \\
\texttt{OHM} & Measure of Electrical Resistance \\
\texttt{PASCAL} & Pressure in Newtons per square meter \\
\texttt{PASCAL_SECOND} & Measurement of Viscosity \\
\texttt{PERCENT} & Percentage \\
\texttt{PH} & A measure of the acidity or alkalinity of a solution. \\
\texttt{REVOLUTION/MINUTE} & Revolutions per minute \\
\texttt{SECOND} & A measurement of time. \\
\texttt{SIEMENS/METER} & A measurement of Electrical Conductivity \\
\texttt{VOLT} & Volts \\
\texttt{VOLT_AMPERE} & The measurement of the apparent power in an electrical circuit, equal to the product of root-mean-square (RMS) voltage and RMS current (commonly referred to as VA). \\
\texttt{VOLT_AMPERE_REACTIVE} & The measurement of reactive power in an AC electrical circuit (commonly referred to as VAR). \\
\texttt{WATT} & Watts \\
\texttt{WATT_SECOND} & Measurement of electrical energy, equal to one Joule \\
\end{tabu}
\end{table} 
\FloatBarrier

\paragraph{\texttt{representation}}\mbox{}
\newline\tab Description of a means to interpret data consisting of multiple data points or samples reported as a single value.  
 {model:representation} is an optional attribute.  
 {model:representation} will define a unique format for each set of data.  
 {model:representation} for {model:TIME_SERIES}, {model:DISCRETE ( *DEPRECATED* in _Version 1.5_)}, and {model:VALUE} are defined in *CITETITLE:MTCPart2* _Section 7.2.2.12_.  
 If {model:representation} is not specified, it *MUST* be determined to be {model:VALUE}.

Placeholder for documentation!

\begin{table}[ht]
\centering 
  \caption{\texttt{RepresentationEnum} Enumeration}
  \label{enum:RepresentationEnum}
\tabulinesep=3pt
\begin{tabu} to 6in {|l|X|} \everyrow{\hline}
\hline
\rowfont\bfseries {Name} & {Description} \\
\tabucline[1.5pt]{}
\texttt{TIME_SERIES} & Placeholder for documentation! \\
\texttt{VALUE} & Placeholder for documentation! \\
\texttt{DATA_SET} & Placeholder for documentation! \\
\texttt{DISCRETE} & Placeholder for documentation! \\
\end{tabu}
\end{table} 
\FloatBarrier

\paragraph{\texttt{Source}}\mbox{}
\newline\tab Placeholder for documentation!

\paragraph{\texttt{Constraints}}\mbox{}
\newline\tab Placeholder for documentation!

\paragraph{\texttt{Filters}}\mbox{}
\newline\tab Placeholder for documentation!

\paragraph{\texttt{InitialValue}}\mbox{}
\newline\tab Placeholder for documentation!

\paragraph{\texttt{ResetTrigger}}\mbox{}
\newline\tab Placeholder for documentation!
\FloatBarrier
\subsubsection{Filter}
  \label{type:Filter}

\FloatBarrier

Placeholder for documentation!

\begin{table}[ht]
\centering 
  \caption{\texttt{Properties of Filter}}
  \label{properties:Filter}
\tabulinesep=3pt
\begin{tabu} to 6in {|l|l|l|} \everyrow{\hline}
\hline
\rowfont\bfseries {Properties} & {Value} & {Multiplicity} \\
\tabucline[1.5pt]{}
\texttt{type} & \texttt{FilterEnum} & 1 \\
\end{tabu}
\end{table}
\FloatBarrier


\paragraph{\texttt{type}}\mbox{}
\newline\tab Placeholder for documentation!

Placeholder for documentation!

\begin{table}[ht]
\centering 
  \caption{\texttt{FilterEnum} Enumeration}
\tabulinesep=3pt
\begin{tabu} to 6in {|l|X|} \everyrow{\hline}
\hline
\rowfont\bfseries {Name} & {Description} \\
\tabucline[1.5pt]{}
\texttt{MINIMUM_DELTA} & Placeholder for documentation! \\
\texttt{PERIOD} & Placeholder for documentation! \\
\end{tabu}
\end{table} 
\FloatBarrier
\FloatBarrier
\subsubsection{InitialValue}
  \label{type:InitialValue}

\FloatBarrier

Placeholder for documentation!

\begin{table}[ht]
\centering 
  \caption{\texttt{Properties of InitialValue}}
  \label{properties:InitialValue}
\tabulinesep=3pt
\begin{tabu} to 6in {|l|l|l|} \everyrow{\hline}
\hline
\rowfont\bfseries {Properties} & {Value} & {Multiplicity} \\
\tabucline[1.5pt]{}
\texttt{value} & \texttt{T} & 1 \\
\end{tabu}
\end{table}
\FloatBarrier


\paragraph{\texttt{value}}\mbox{}
\newline\tab Placeholder for documentation!
\FloatBarrier
\subsubsection{ResetTrigger}
  \label{type:ResetTrigger}

\FloatBarrier

Placeholder for documentation!

\begin{table}[ht]
\centering 
  \caption{\texttt{Properties of ResetTrigger}}
  \label{properties:ResetTrigger}
\tabulinesep=3pt
\begin{tabu} to 6in {|l|l|l|} \everyrow{\hline}
\hline
\rowfont\bfseries {Properties} & {Value} & {Multiplicity} \\
\tabucline[1.5pt]{}
\texttt{type} & \texttt{ResetTriggerEnum} & 1 \\
\end{tabu}
\end{table}
\FloatBarrier


\paragraph{\texttt{type}}\mbox{}
\newline\tab Placeholder for documentation!

Placeholder for documentation!

\begin{table}[ht]
\centering 
  \caption{\texttt{ResetTriggerEnum} Enumeration}
  \label{enum:ResetTriggerEnum}
\tabulinesep=3pt
\begin{tabu} to 6in {|l|X|} \everyrow{\hline}
\hline
\rowfont\bfseries {Name} & {Description} \\
\tabucline[1.5pt]{}
\texttt{ACTION_COMPLETE} & Placeholder for documentation! \\
\texttt{ANNUAL} & Placeholder for documentation! \\
\texttt{DAY} & Placeholder for documentation! \\
\texttt{LIFE} & Placeholder for documentation! \\
\texttt{MAINTENANCE} & Placeholder for documentation! \\
\texttt{MONTH} & Placeholder for documentation! \\
\texttt{POWER_ON} & Placeholder for documentation! \\
\texttt{SHIFT} & Placeholder for documentation! \\
\texttt{WEEK} & Placeholder for documentation! \\
\end{tabu}
\end{table} 
\FloatBarrier
\FloatBarrier
\subsubsection{Source}
  \label{type:Source}

\FloatBarrier

Placeholder for documentation!

\begin{table}[ht]
\centering 
  \caption{\texttt{Properties of Source}}
  \label{properties:Source}
\tabulinesep=3pt
\begin{tabu} to 6in {|l|l|l|} \everyrow{\hline}
\hline
\rowfont\bfseries {Properties} & {Value} & {Multiplicity} \\
\tabucline[1.5pt]{}
\texttt{componentId} & \texttt{ID} & 0..1 \\
\texttt{compositionId} & \texttt{ID} & 0..1 \\
\texttt{dataItemId} & \texttt{ID} & 0..1 \\
\texttt{Value} & \texttt{T} & 0..1 \\
\end{tabu}
\end{table}
\FloatBarrier


\paragraph{\texttt{componentId}}\mbox{}
\newline\tab The identifier attribute of the {model:Component} element that represents the physical part of a piece of equipment where the data represented by the {model:DataItem} element originated.

\paragraph{\texttt{compositionId}}\mbox{}
\newline\tab The identifier attribute of the {model:Composition} element that the reported data is most closely associated.

\paragraph{\texttt{dataItemId}}\mbox{}
\newline\tab The identifier attribute of the {model:DataItem} that represents the originally measured value of the data referenced by this data item.

\paragraph{\texttt{Value}}\mbox{}
\newline\tab Placeholder for documentation!
\FloatBarrier
