\tabulinesep = 5pt
\begin{longtabu} to \textwidth {
    |l|X[3l]|X[0.75l]|}
\caption{DataItem Element Filter type} \label{table:dataitem-element-filter-type} \\

\hline
type & Description & Occurrence \\
\hline
\endfirsthead

\hline
\multicolumn{3}{|c|}{Continuation of Table \ref{table:dataitem-element-filter-type}}\\
\hline
type & Description & Occurrence \\
\hline
\endhead
 
\gls{minimumdelta}
&
For a \gls{minimumdelta} type \gls{filter}, a new value \MUSTNOT be reported for a data item unless the measured value has changed from the last reported value by at least the delta given as the \gls{cdata} of this element.
\newline The \gls{cdata} \MUST be an absolute value using the same units as the reported data. 
&
0..1 \notesign \\
\hline

\gls{period}
&
For a \gls{period} type \gls{filter}, the data reported for a data item is provided on a periodic basis. The \gls{period} for reporting data is defined in the \gls{cdata} for the \gls{filter}.
\newline The \gls{cdata} \MUST be an absolute value reported in seconds representing the time between reported samples of the value of the data item.
\newline If the \gls{period} is smaller than one second, the number can be represented as a floating point number. For example, a \gls{period} of 100 milliseconds would be 0.1.
&
0..1 \notesign \\
\hline

\end{longtabu}