\tabulinesep = 5pt
\begin{longtabu} to \textwidth {
    |l|X[3l]|X[0.75l]|}
\caption{Attributes for  Component} \label{table:attributes-for-component} \\

\hline
Attribute & Description & Occurrence \\
\hline
\endfirsthead

\hline
\multicolumn{3}{|c|}{Continuation of Table \ref{table:attributes-for-component}}\\
\hline
Attribute & Description & Occurrence \\
\hline
\endhead
 
\gls{id} 
&
\glsentrydesc{id}
\newline \gls{id} is a required attribute.
\newline An \gls{id} \MUST be unique across all the \gls{id} attributes in the document.
\newline An XML ID-type.
&
1 \\
\hline

\gls{nativename}
&
The common name normally associated with a specific physical or logical part of a piece of equipment.
\newline \gls{nativename} is an optional attribute. 
&
0..1 \\
\hline

\gls{sampleinterval}
&
An optional attribute that is an indication provided by a piece of
equipment describing the interval in milliseconds between the
completion of the reading of the data associated with the \gls{component} element until the beginning of the next sampling of that data. This indication is reported as the number of milliseconds between data captures.
\newline This information may be used by client software applications to understand how often information from a piece of equipment for a specific \gls{component} element is expected to be refreshed.
\newline The refresh rate for data from all \gls{lower level} \gls{component} elements will be the same as for the parent \gls{component} element unless specifically overridden by another \gls{sampleinterval} provided for the \gls{lower level} \gls{component} element.
\newline If the value of \gls{sampleinterval} is less than one millisecond, the value will be represented as a floating-point number. For example, an interval of 100 microseconds would be 0.1.
&
0..1 \notesign \notesign \\
\hline


\deprecated{\gls{samplerate}}
&
\DEPRECATED in MTConnect Version 1.2. Replaced by \gls{sampleinterval}.
&
0..1 \notesign \notesign \notesign \\
\hline

\gls{uuid}
&
A unique identifier for this XML element.
\newline \gls{uuid} is an optional attribute. 
\newline The value provided for the \gls{uuid} \MUST be unique amongst all \gls{uuid} identifiers used in an MTConnect installation. 
\newline For example, this may be a combination of the manufacturer’s code and serial number. The \gls{uuid} \SHOULD be alphanumeric and not exceed 255 characters.
\newline An \gls{nmtoken} XML type.
&
0..1 \notesign \\
\hline

\gls{name}
&
The name of the \gls{component} element.
\newline \gls{name} is an optional attribute.
\newline However, if there are multiple \gls{lower level} components that have the same parent and are of the same component type (example \gls{linear}), then the name attribute \MUST be provided for all \gls{lower level} components of the same element type to differentiate between the similar components.
\newline When provided, name \MUST be unique for all \gls{lower level} components of a parent \gls{component}.
\newline An \gls{nmtoken} XML type.
&
0..1 \\
\hline

\end{longtabu}