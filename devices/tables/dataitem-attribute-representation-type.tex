\tabulinesep = 5pt
\begin{longtabu} to \textwidth {
    |l|X[3l]|}
\caption{DataItem attribute representation type} \label{table:dataitem-attribute-representation-type} \\

\hline
Representation & Description\\
\hline
\endfirsthead

\hline
\multicolumn{2}{|c|}{Continuation of Table \ref{table:dataitem-attribute-representation-type}}\\
\hline
Representation & Description\\
\hline
\endhead
 


\gls{dataset}
&
The reported value(s) are represented as a set of \textit{key-value pairs}.
\newline Each reported value in the \gls{data set} \MUST have a unique key.  \\
\hline

\gls{discrete representation}
&
\DEPRECATED as a \gls{representation} in MTConnect Version. 1.5.  Replaced by the \gls{discrete} attribute for a \gls{data entity} – \sect{discrete-attribute}. \newline \deprecated{A Data Entity where each discrete occurrence of the data may have the same value as the previous occurrence of the data.  There is no reported state change between occurrences of the data.   
\newline In this case, duplicate occurrences of the same data value SHOULD NOT be suppressed. 
\newline An example of a DISCRETE data type would be a parts counter that reports the completion of each part versus the accumulation of parts.   
Another example would be a Message that does not typically have a reset state and may re-occur each time a specific message is triggered.} 
 \\
\hline


\gls{timeseries representation}
&
\glsentrydesc{timeseries representation}
\newline The data is reported for a specified number of samples and each sample is reported with a fixed period.
\\
\hline


\gls{value representation}
&
\glsentrydesc{value representation}
\newline If no representation is specified for a data item, the representation \MUST be determined to be \gls{value representation}.\\
\hline

\end{longtabu}