\tabulinesep = 5pt
\begin{longtabu} to \textwidth {
    |l|X[3l]|X[0.75l]|}
\caption{Elements for DataItem} \label{table:elements-for-dataitem} \\

\hline
Element & Description & Occurrence \\
\hline
\endfirsthead

\hline
\multicolumn{3}{|c|}{Continuation of Table \ref{table:elements-for-dataitem}}\\
\hline
Element & Description & Occurrence \\
\hline
\endhead

\gls{source}
&
\gls{source} is an optional XML element that identifies the \gls{component}, \gls{dataitem}, or \gls{composition} representing the area of the piece of equipment from which a measured value originates.
\newline Additionally, \gls{source} \MAY provide information relating to the identity of a measured value.  This information is reported as CDATA for \gls{source}. (example, a PLC tag)
&
0..1 \\
\hline

\gls{constraints}
&
\glsentrydesc{constraints}
 \gls{constraints} are used by a software application to evaluate the validity of the reported data.
&
0..1 \\
\hline

\gls{filters}
&
An optional container for the \gls{filter} elements associated with this \gls{dataitem} element. 
&
0..1 \\
\hline

\gls{initialvalue}
&
\glsentrydesc{initialvalue}
\newline Only one \gls{initialvalue} element may be defined for a data item. The value will be constant and cannot change.
\newline If no \gls{initialvalue} element is defined for a data item that is periodically reset, then the starting value for the data item \MUST be a value of 0.
&
0..1 \\
\hline

\gls{resettrigger}
&
\glsentrydesc{resettrigger}
&
0..1 \\
\hline

\end{longtabu}