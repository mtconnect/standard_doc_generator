\tabulinesep = 5pt
\begin{longtabu} to \textwidth {
    |l|X[3l]|}
\caption{DataItem type subType for category EVENT} \label{table:dataitem-type-category-event} \\

\hline
DataItem type subType & Description\\
\hline
\endfirsthead

\hline
\multicolumn{2}{|c|}{Continuation of Table 
\ref{table:dataitem-type-category-event}: \nameref{table:dataitem-type-category-event}}\\
\hline
DataItem type subType & Description\\
\hline
\endhead

\gls{activeaxes event} &
\glsentrydesc{activeaxes event}
\newline If this \gls{dataitem} is not provided, it will be assumed that all axes are currently associated with the \gls{controller} \gls{structural element} and with an individual \gls{path}.   
\newline The \gls{valid data value} for \gls{activeaxes event} \should be a space-delimited set of axes reported as the value of the \gls{name} attribute for each axis.  If \gls{name} is not available, the piece of equipment \must report the value of the \gls{nativename} attribute for each axis.
\\ \hline 

\gls{actuatorstate event} &
\glsentrydesc{actuatorstate event}
\newline The \gls{valid data value} \must be \gls{active value} or \gls{inactive value}.
\\ \hline 

\gls{alarm event}
&
\DEPRECATED in Version 1.1. Replaced with \gls{condition category} category.
\\ \hline 

\gls{availability event}
&
\glsentrydesc{availability event}
\newline This \must be provided for a \gls{device} Element and \MAY be provided for any other \gls{structural element}.  The \gls{valid data value} \must be \gls{available value} or \gls{unavailable value}.\\ \hline 

\gls{axiscoupling event} &
\glsentrydesc{axiscoupling event} 
\newline The \gls{valid data value} \must be \gls{tandem value}, \gls{synchronous value}, \gls{master value}, and \gls{slave value}.  
\newline The coupling \must be viewed from the perspective of a specific axis.  Therefore, a \gls{master value} coupling indicates that this axis is the master for the \gls{coupledaxes event}.
\\ \hline 

\gls{axisfeedrateoverride event} & \glsentrydesc{axisfeedrateoverride event}
\newline The value provided for \gls{axisfeedrateoverride event} is expressed as a percentage of the designated feedrate for the axis.   
\newline When \gls{axisfeedrateoverride event} is applied, the resulting commanded feedrate for the axis is limited to the value of the original feedrate multiplied by the value of the \gls{axisfeedrateoverride event}.   
\newline There \MAY be different subtypes of \gls{axisfeedrateoverride event}; each representing an override value for a designated subtype of feedrate depending on the state of operation of the axis.   The subtypes of operation of an axis are currently defined as \gls{programmed subtype}, \gls{jog subtype}, and \gls{rapid subtype}.
\\ \hline 

\quad \gls{jog subtype}
&
The value of a signal or calculation issued to adjust the feedrate of an individual linear type axis when that axis is being operated in a manual state or method (jogging).   \newline When the \gls{jog subtype} subtype of \gls{axisfeedrateoverride event} is applied, the resulting commanded feedrate for the axis is limited to the value of the original \gls{jog subtype} subtype of the \gls{axisfeedrate sample} multiplied by the value of the \gls{jog subtype} subtype of \gls{axisfeedrateoverride event}.
\\ \hline 

\quad \gls{programmed subtype}
&
The value of a signal or calculation issued to adjust the feedrate of an individual linear type axis that has been specified by a logic or motion program or set by a switch. \newline When the \gls{programmed subtype} subtype of \gls{axisfeedrateoverride event} is applied, the resulting commanded feedrate for the axis is limited to the value of the original \gls{programmed subtype} subtype of the \gls{axisfeedrate sample} multiplied by the value of the \gls{programmed subtype} subtype of \gls{axisfeedrateoverride event}. \\ \hline 

\quad \gls{rapid subtype}
&
The value of a signal or calculation issued to adjust the feedrate of an individual linear type axis that is operating in a rapid positioning mode. 
\newline When the \gls{rapid subtype} subtype of \gls{axisfeedrateoverride event} is applied, the resulting commanded feedrate for the axis is limited to the value of the original \gls{rapid subtype} subtype of the \gls{axisfeedrate sample} multiplied by the value of the \gls{rapid subtype} subtype of \gls{axisfeedrateoverride event}. \\ \hline 

\gls{axisinterlock event} 
& 
\glsentrydesc{axisinterlock event} 
\newline The \gls{valid data value} \must be \gls{active value} or \gls{inactive value}.
\\ \hline 

\gls{axisstate event} 
& 
\glsentrydesc{axisstate event}
\newline The \gls{valid data value} \must be \gls{home value}, \gls{travel value}, \gls{parked value}, or \gls{stopped value}.
\\ \hline 

\gls{block event}
& 
\glsentrydesc{block event} 
\newline The value reported for \glselementname{block event} \must include the entire expression for a line of program code, including all parameters.
\\ \hline 

\gls{blockcount event} 
& 
\glsentrydesc{blockcount event} 
\newline \gls{blockcount event} counts blocks of program code executed regardless of program structure (e.g., looping or branching within the program).
\newline The starting value for \gls{blockcount event} \MAY be established by an initial value provided in the Constraint element defined for the data item.
\\ \hline 

\gls{chuckinterlock event} 
& 
\glsentrydesc{chuckinterlock event} 
\newline The \gls{valid data value} \must be \gls{active value} or  \gls{inactive value}.
\\ \hline 

\quad \gls{manualunclamp subtype} & \glsentrydesc{manualunclamp subtype} \\ \hline 

\gls{chuckstate event}
& 
\glsentrydesc{chuckstate event} 
\newline The \gls{valid data value} \must be \gls{open value}, \gls{closed value}, or \gls{unlatched value}.
\\ \hline 

\gls{code event} & \glsentrydesc{code event} \\ \hline 

\gls{compositionstate event}
& 
\glsentrydesc{compositionstate event} 
\newline A \gls{subtype} \must always be specified.
\newline A \gls{compositionid} \must always be specified.
\\ \hline 

\quad \gls{action subtype} & \glsentrydesc{action subtype} \\ \hline 

\quad \gls{lateral subtype} & \glsentrydesc{lateral subtype} \\ \hline 

\quad \gls{motion subtype} & \glsentrydesc{motion subtype} \\ \hline 

\quad \gls{switched subtype} & \glsentrydesc{switched subtype} \\ \hline 

\quad \gls{vertical subtype} & \glsentrydesc{vertical subtype} \\ \hline

\gls{controllermode event}
&
The current mode of the \gls{controller} component. The \gls{valid data value} \MUST be \gls{automatic value}, \gls{manual value}, \gls{manualdatainput value}, \gls{semiautomatic value}, or \gls{edit value}.  \\
\hline 

\gls{controllermodeoverride event} 
& 
\glsentrydesc{controllermodeoverride event} 
\newline A \gls{subtype} \must always be specified.
\\ \hline 

\quad \gls{dryrun subtype} & \glsentrydesc{dryrun subtype} \\ \hline 

\quad \gls{machineaxislock subtype} & \glsentrydesc{machineaxislock subtype} \\ \hline 

\quad \gls{optionalstop subtype} & \glsentrydesc{optionalstop subtype} \\ \hline 

\quad \gls{singleblock subtype} & \glsentrydesc{singleblock subtype} \\ \hline 

\quad \gls{toolchangestop subtype} & \glsentrydesc{toolchangestop subtype} \\ \hline 

\gls{coupledaxes event} 
& 
\glsentrydesc{coupledaxes event} 
\newline The \gls{valid data value} for \gls{coupledaxes event} \should be a space-delimited set of axes reported as the value of the \gls{name} attribute for each axis.  If \gls{name} is not available, the piece of equipment \must report the value of the \gls{nativename} attribute for each axis.
\\ \hline 

\gls{datecode event}
&
The time and date code associated with a material or other physical item.
\newline \gls{datecode event} \MUST be reported in ISO 8601 format. \\
\hline

\quad \gls{manufacture subtype}
&
The time and date code relating to the production of a material or other physical item. \\
\hline

\quad \gls{expiration subtype}
&
The time and date code relating to the expiration or end of useful life for a material or other physical item. \\
\hline

\quad \gls{firstuse subtype}
&
The time and date code relating the first use of a material or other physical item. \\
\hline

\gls{deviceuuid event}
&
The identifier of another piece of equipment that is temporarily associated with a component of this piece of equipment to perform a particular function.
\newline The \gls{valid data value} \MUST be a \gls{nmtoken} XML type. \\
\hline

\gls{direction event} 
& 
\glsentrydesc{direction event} 
  A \gls{subtype} \must always be specified.
\\ \hline 

\quad \gls{linear subtype}
&
The direction of motion of a linear motion. \newline The \gls{valid data value} \must be \gls{positive value} or \gls{negative value}. \\ \hline 

\quad \gls{rotary subtype} & \glsentrydesc{rotary subtype} \\ \hline 

\gls{doorstate event} 
& 
\glsentrydesc{doorstate event} 
\newline The \gls{valid data value} \must be \gls{open value}, \gls{unlatched value}, or \gls{closed value}.
\\ \hline 

\gls{emergencystop event}
& 
\glsentrydesc{emergencystop event} 
\newline The \gls{valid data value} \must be \gls{armed value} (the circuit is complete and the device is allowed to operate) or \gls{triggered value} (the circuit is open and the device must cease operation).
\\ \hline 

\gls{endofbar event} 
& 
\glsentrydesc{endofbar event} 
\newline The \gls{valid data value} \must be expressed as a Boolean expression of YES or NO.
\\ \hline 

\quad \gls{auxiliary subtype} & \glsentrydesc{auxiliary subtype} \\ \hline 

\quad \gls{primary subtype} & \glsentrydesc{primary subtype} \\ \hline 

\gls{equipmentmode event} 
& 
\glsentrydesc{equipmentmode event} 
\newline \gls{equipmentmode event} \MAY have more than one subtype defined.\newline A \gls{subtype} \must always be specified.
\\ \hline 

\quad \gls{delay subtype}
&
An indication that a piece of equipment is waiting for an event or an action to occur.
\\ \hline 

\quad \gls{loaded subtype}
&
An indication that the sub-parts of a piece of equipment are under load. \newline Example: For traditional machine tools, this is an indication that the cutting tool is assumed to be engaged with the part. \newline The \gls{valid data value} \must be \gls{on value} or \gls{off value}.
\\ \hline 

\quad \gls{operating subtype} &
An indication that the major sub-parts of a piece of equipment are powered or performing any activity whether producing a part or product or not.   \newline Example: For traditional machine tools, this includes when the piece of equipment is \gls{working subtype} or it is idle.\newline The \gls{valid data value} \must be \gls{on value} or \gls{off value}.
\\ \hline 

\quad \gls{powered subtype}
&
An indication that primary power is applied to the piece of equipment and, as a minimum, the controller or logic portion of the piece of equipment is powered and functioning or components that are required to remain on are powered. \newline Example: Heaters for an extrusion machine that required to be powered even when the equipment is turned off.\newline The \gls{valid data value} \must be \gls{on value} or \gls{off value}. \\ \hline 

\quad \gls{working subtype}
&
An indication that a piece of equipment is performing any activity  the equipment is active and performing a function under load or not. \newline Example: For traditional machine tools, this includes when the piece of equipment is \gls{loaded subtype}, making rapid moves, executing a tool change, etc.\newline The \gls{valid data value} \must be \gls{on value} or \gls{off value}. \\ \hline

\gls{execution event}
&
The execution status of the \gls{controller}.
\newline The \gls{valid data value} \MUST be \gls{ready value}, \gls{active value}, \gls{interrupted value}, \gls{wait}, \gls{feedhold value}, \gls{stopped value}, \gls{optionalstop value}, \gls{programstopped value}, or \gls{programcompleted value}. \\
\hline 

\gls{functionalmode event}
& 
\glsentrydesc{functionalmode event} 
\newline Typically, the \gls{functionalmode event} \should be modeled as a data item for the Device element, but \MAY be modeled for any \gls{structural element} in the XML document.   
\newline The \gls{valid data value} \must be \gls{production value}, \gls{setup value}, \gls{teardown value}, \gls{maintenance}, or \gls{processdevelopment value}.
\\ \hline 

\gls{hardness event} 
& 
\glsentrydesc{hardness event} 
\newline  The measurement does not provide a unit. \newline A \gls{subtype} \must always be specified to designate the hardness scale associated with the measurement.
\\ \hline 

\quad \gls{brinell subtype} & \glsentrydesc{brinell subtype} \\ \hline 

\quad \gls{leeb subtype} & \glsentrydesc{leeb subtype} \\ \hline 

\quad \gls{mohs subtype} & \glsentrydesc{mohs subtype} \\ \hline 

\quad \gls{rockwell subtype} & \glsentrydesc{rockwell subtype} \\ \hline 

\quad \gls{shore subtype} & \glsentrydesc{shore subtype} \\ \hline 

\quad \gls{vickers subtype} & \glsentrydesc{vickers subtype} \\ \hline 

\gls{interfacestate event} 
& 
The current functional or operational state of an \gls{interface component} type element indicating whether the interface is active or is not currently functioning.
\newline The \gls{valid data value} \must be \gls{enabled value} or \gls{disabled value}.
\\ \hline 

\gls{line event} & 
\deprecated{The current line of code being executed.
\newline The data will be an alpha numeric value representing the line number of the current line of code being executed.}
\newline \glsentrydesc{line event} \\ \hline 

\quad \deprecated{\gls{maximum value}}
&
\deprecated{The maximum line number of the code being executed.} \\ \hline 

\quad \deprecated{\gls{minimum value}}
&
\deprecated{The minimum line number of the code being executed.}\\ \hline 

\gls{linelabel event} & \glsentrydesc{linelabel event} \\ \hline 

\gls{linenumber event} 
& 
\glsentrydesc{linenumber event} 
  The line number \MAY represent either an absolute position starting with the first line of the program or an incremental position relative to the occurrence of the last \gls{linelabel event}. \newline \gls{linenumber event} does not change subject to any looping or branching in a control program.\newline A \gls{subtype} \must be defined.
\\ \hline 

\quad \gls{absolute subtype} & \glsentrydesc{absolute subtype} \\ \hline 

\quad \gls{incremental subtype} & \glsentrydesc{incremental subtype} \\ \hline 

\gls{material event} 
& 
\glsentrydesc{material event} 
\newline The \gls{valid data value} \must be a text string.
\\ \hline 

\gls{materiallayer event}
&
Identifies the layers of material applied to a part or product as part of an additive manufacturing process.
\newline The \gls{valid data value} \MUST be an integer. \\
\hline

\quad \gls{actual subtype}
&
The current number of layers of material applied to a part or product during an additive manufacturing process. \\
\hline

\quad \gls{target subtype}
&
The target or planned number layers of material applied to a part or product during an additive manufacturing process. \\
\hline

\gls{message event} & \glsentrydesc{message event} \\ \hline 

\gls{operatorid event} 
& 
\glsentrydesc{operatorid event} 
\newline \DEPRECATIONWARNING:  May be deprecated in the future.  See \gls{user event} below.
\\ \hline 

\gls{palletid event} 
& 
\glsentrydesc{palletid event} 
\newline The \gls{valid data value} \must be a text string.
\\ \hline 

\gls{partcount}
& 
The current count of parts produced as represented by the \gls{controller} component. \newline The \gls{valid data value} \must be an integer value.
\\ \hline 

\quad \gls{all subtype} & \glsentrydesc{all subtype} \\ \hline 

\quad \gls{bad subtype} & \glsentrydesc{bad subtype} \\ \hline 

\quad \gls{good subtype} & \glsentrydesc{good subtype} \\ \hline 

\quad \gls{remaining subtype}
&
The number of parts remaining in stock or to be produced. \\ \hline 

\quad \gls{target subtype}
&
Indicates the number of parts that are projected or planned to be produced. \\ \hline 

\gls{partdetect event}
&
An indication designating whether a part or work piece has been detected or is present.
\newline The \gls{valid data value} \MUST be \gls{present} or \gls{notpresent}. \\
\hline

\gls{partid event} 
& 
\glsentrydesc{partid event} 
\newline The \gls{valid data value} \must be a text string.
\\ \hline

\gls{partnumber event}
&
An identifier of a part or product moving through the manufacturing process.
\newline The \gls{valid data value} \MUST be a text string. 
\newline \DEPRECATIONWARNING: May be deprecated in the future. \\
\hline 

\gls{pathfeedrateoverride event} 
& 
\glsentrydesc{pathfeedrateoverride event} 
\newline The value provided for \gls{pathfeedrateoverride event} is expressed as a percentage of the designated feedrate for the path. 
\newline When \gls{pathfeedrateoverride event} is applied, the resulting commanded feedrate for the path is limited to the value of the original feedrate multiplied by the value of the \gls{pathfeedrateoverride event}.   
\newline There \MAY be different subtypes of \gls{pathfeedrateoverride event}; each representing an override value for a designated subtype of feedrate depending on the state of operation of the path.   The states of operation of a path are currently defined as \gls{programmed subtype}, \gls{jog subtype}, and \gls{rapid subtype}.
\\ \hline 

\quad \gls{jog subtype}
&
The value of a signal or calculation issued to adjust the feedrate of the axes associated with a \gls{path} component when the axes, or a single axis, are being operated in a manual mode or method (jogging).   \newline When the \gls{jog subtype} subtype of \gls{pathfeedrateoverride event} is applied, the resulting commanded feedrate for the axes, or a single axis, associated with the path are limited to the value of the original \gls{jog subtype} subtype of the \gls{pathfeedrate sample} multiplied by the value of the \gls{jog subtype} subtype of \gls{pathfeedrateoverride event}.
 \\ \hline 

\quad \gls{programmed subtype}
&
The value of a signal or calculation issued to adjust the feedrate of the axes associated with a \gls{path} component when the axes, or a single axis, are operating as specified by a logic or motion program or set by a switch. \newline When the \gls{programmed subtype} subtype of \gls{pathfeedrateoverride event} is applied, the resulting commanded feedrate for the axes, or a single axis, associated with the path are limited to the value of the original \gls{programmed subtype} subtype of the \gls{pathfeedrate sample} multiplied by the value of the \gls{programmed subtype} subtype of \gls{pathfeedrateoverride event}. \\ \hline 

\quad \gls{rapid subtype}
&
The value of a signal or calculation issued to adjust the feedrate of the axes associated with a \gls{path} component when the axes, or a single axis, are being operated in a rapid positioning mode or method (rapid).
\newline When the \gls{rapid subtype} subtype of \gls{pathfeedrateoverride event} is applied, the resulting commanded feedrate for the axes, or a single axis, associated with the path are limited to the value of the original \gls{rapid subtype} subtype of the \gls{pathfeedrate sample} multiplied by the value of the \gls{rapid subtype} subtype of \gls{pathfeedrateoverride event}. \\ \hline 

\gls{pathmode event} 
& 
\glsentrydesc{pathmode event} 
\newline The \gls{valid data value} \must be \gls{independent value}, \gls{master value}, \gls{synchronous value}, or \gls{mirror value}.
\newline The default value \must be \gls{independent value} if \gls{pathmode event} is not specified.
\\ \hline 

\gls{powerstate event} 
& 
\glsentrydesc{powerstate event} 
\newline The \gls{valid data value} \must be \gls{on value} or \gls{off value}.\newline \DEPRECATIONWARNING: May be deprecated in the future.
\\ \hline 

\quad \gls{control subtype} & \glsentrydesc{control subtype} \\ \hline 

\quad \gls{line subtype} & \glsentrydesc{line subtype} \\ \hline 

\gls{powerstatus event} 
& 
\glsentrydesc{powerstatus event} \\ \hline

\gls{processtime event}
&
The time and date associated with an activity or event.
\newline \gls{processtime event} \MUST be reported in ISO 8601 format. \\
\hline

\quad \gls{start subtype}
&
The time and date associated with the beginning of an activity or event. \\
\hline

\quad \gls{complete value}
&
The time and date associated with the completion of an activity or event. \\
\hline

\quad \gls{targetcompletion subtype}
&
The projected time and date associated with the end or completion of an activity or event. \\
\hline

\gls{program event}
&
The identity of the logic or motion program being executed by the piece of equipment.
\newline The \gls{valid data value} \MUST be a text string. \\
\hline 


\quad \gls{schedule subtype}
&
The identity of a control program that is used to specify the order of execution of other programs. \\
\hline

\quad \gls{main subtype}
&
The identity of the primary logic or motion program currently being executed. It is the starting nest level in a call structure and may contain calls to sub programs. \\
\hline

\quad \gls{active value}
&
The identity of the logic or motion program currently executing. \\
\hline

\gls{programcomment event}
&
A comment or non-executable statement in the control program.
\newline The \gls{valid data value} \MUST be a text string. \\
\hline 


\quad \gls{schedule subtype}
&
The identity of a control program that is used to specify the order of execution of other programs. \\
\hline

\quad \gls{main subtype}
&
The identity of the primary logic or motion program currently being executed. It is the starting nest level in a call structure and may contain calls to sub programs. \\
\hline

\quad \gls{active value}
&
The identity of the logic or motion program currently executing. \\
\hline

\gls{programedit event} 
& 
\glsentrydesc{programedit event} 
   \newline The \gls{valid data value} \must be: \newline \gls{active value}:  The controller is in the program edit mode. \newline \gls{ready value}:  The controller is capable of entering the program edit mode and no function is inhibiting a change of mode. \newline \gls{notready value}:  A function is inhibiting the controller from entering the program edit mode.
\\ \hline 

\gls{programeditname event} & \glsentrydesc{programeditname event} \\ \hline 

\gls{programheader event} 
& 
\glsentrydesc{programheader event} 
\newline The \gls{valid data value} \must be a text string.
\\ \hline 

\gls{programlocation event}
&
The Uniform Resource Identifier (URI) for the source file associated with \gls{program event}. \\
\hline

\quad \gls{schedule subtype}
&
An identity of a control program that is used to specify the order of execution of other programs. \\
\hline

\quad \gls{main subtype}
&
The identity of the primary logic or motion program currently being executed. It is the starting nest level in a call structure and may contain calls to sub programs. \\
\hline

\quad \gls{active value}
&
The identity of the logic or motion program currently executing. \\
\hline

\gls{programlocationtype event}
&
Defines whether the logic or motion program defined by \gls{program event} is being executed from the local memory of the controller or from an outside source.
\newline The \gls{valid data value} \MUST be \gls{local} or \gls{external}. \\
\hline

\quad \gls{schedule subtype}
&
An identity of a control program that is used to specify the order of execution of other programs. \\
\hline

\quad \gls{main subtype}
&
The identity of the primary logic or motion program currently being executed. It is the starting nest level in a call structure and may contain calls to sub programs. \\
\hline

\quad \gls{active value}
&
The identity of the logic or motion program currently executing. \\
\hline

\gls{programnestlevel event}
&
An indication of the nesting level within a control program that is associated with the code or instructions that is currently being executed.
\newline If an initial value is not defined, the nesting level associated with the highest or initial nesting level of the program \MUST default to zero (0).
\newline The value reported for \gls{programnestlevel event} \MUST be an integer. \\
\hline

\gls{rotarymode event} 
& 
\glsentrydesc{rotarymode event} 
\newline The \gls{valid data value} \must be \gls{spindle value}, \gls{index value}, or \gls{contour value}.
\\ \hline 

\gls{rotaryvelocityoverride event} 
& 
\glsentrydesc{rotaryvelocityoverride event} 
\newline \gls{rotaryvelocityoverride event} is expressed as a percentage of the programmed \gls{rotaryvelocity sample}.
\\ \hline 

\gls{serialnumber event} & \glsentrydesc{serialnumber event} \\ \hline 

\gls{spindleinterlock event} 
& 
\glsentrydesc{spindleinterlock event} 
\newline The \gls{valid data value} \must be: \newline \gls{active value} if power has been removed and the spindle cannot be operated.  \newline \gls{inactive value} if power to the spindle has not been deactivated.
\\ \hline 

\gls{toolassetid event} & \glsentrydesc{toolassetid event} \\ \hline 

\gls{toolgroup event}
&
An identifier for the tool group associated with a specific tool. Commonly used to designate spare tools. \\
\hline

\deprecated{\mbox{\gls{toolid event}}} & \glsentrydesc{toolid event} \\ \hline 

\gls{toolnumber event} & \glsentrydesc{toolnumber event} \\ \hline

\gls{tooloffset event}
&
A reference to the tool offset variables applied to the active cutting tool.
\newline The \gls{valid data value} \MUST be a text string.
\newline The reported value returned for \gls{tooloffset event} identifies the location in a table or list where the actual tool offset values are stored.
\newline \DEPRECATED in V1.5 \deprecated{A subType \MUST always be specified.} \\
\hline 

\quad \gls{length subtype}
&
A reference to a length type tool offset. \\
\hline

\quad \gls{radial subtype}
&
A reference to a radial type tool offset. \\
\hline 

\gls{user event} 
& 
\glsentrydesc{user event} 
\newline A \gls{subtype} \must always be specified.
\\ \hline 

\quad \gls{maintenance}
&
The identifier of the person currently responsible for performing maintenance on the piece of equipment.
\\ \hline 

\quad \gls{operator subtype} & \glsentrydesc{operator subtype} \\ \hline 

\quad \gls{setup subtype} & \glsentrydesc{setup subtype} \\ \hline 

\gls{variable event}
&
 A data value whose meaning may change over time due to changes in the operation of a piece of equipment or the process being executed on that piece of equipment. \\
\hline

\gls{waitstate event}
&
An indication of the reason that \gls{execution event} is reporting a value of \gls{wait}.
\newline The \gls{valid data value} \MUST be \gls{poweringup}, \gls{poweringdown}, \gls{partload}, \gls{partunload}, \gls{toolload}, \gls{toolunload}, \gls{materialload event}, \gls{materialunload event}, \gls{secondaryprocess}, \gls{pausing}, or \gls{resuming}. \\
\hline

\gls{wire}
&
The identifier for the type of wire used as the cutting mechanism in Electrical Discharge Machining or similar processes. \newline The \gls{valid data value} \must be a text string. \\ \hline 

\gls{workholdingid event} & \glsentrydesc{workholdingid event} \\ \hline 





\gls{workoffset event} 
& 
\glsentrydesc{workoffset event} 
\newline The \gls{valid data value} \must be a text string.\newline The reported value returned for \gls{workoffset event} identifies the location in a table or list where the actual tool offset values are stored.
\\ \hline 
\end{longtabu}