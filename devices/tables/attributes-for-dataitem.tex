\tabulinesep = 5pt
\begin{longtabu} to \textwidth {
    |l|X[3l]|X[0.75l]|}
\caption{Attributes for DataItem} \label{table:attributes-for-dataitem} \\

\hline
Attribute & Description & Occurrence \\
\hline
\endfirsthead

\hline
\multicolumn{3}{|c|}{Continuation of Table \ref{table:attributes-for-dataitem}}\\
\hline
Attribute & Description & Occurrence \\
\hline
\endhead

\gls{name}
&
The name of the data item.
\newline \gls{name} is provided as an additional human readable identifier for this data item in addition to the \gls{id}.
\newline \gls{name} is an optional attribute and will be implementation dependent.
\newline An \gls{nmtoken} XML type.
&
0..1 \\
\hline

\gls{id} 
&
\glsentrydesc{id}
\newline \gls{id} is a required attribute.
\newline The \gls{id} attribute \MUST be unique within the
\gls{mtconnectdevices} document.
\newline An XML ID-type.
&
1 \\
\hline

\gls{type}
&
The type of data being measured.
\newline \gls{type} is a required attribute.
\newline Examples of types are \gls{position sample}, \gls{velocity sample}, \gls{angle sample}, \gls{block event}, and \gls{rotaryvelocity sample}.
&
1 \\
\hline

\gls{subtype}
&
A sub-categorization of the data item \gls{type}.
\newline \gls{subtype} is an optional attribute.
\newline For example, the \gls{subtype} of \gls{position sample} can be \gls{actual subtype} or \gls{commanded subtype}.
\newline Not all \gls{type} attributes have a \gls{subtype}. 
&
0..1 \\
\hline

\gls{statistic}
&
Describes the type of statistical calculation performed on a series of data samples to provide the reported data value.
\newline \gls{statistic} is an optional attribute.
\newline Examples of \gls{statistic} are \gls{average}, \gls{minimum value}, \gls{maximum value}, \gls{rootmeansquare}, \gls{range}, \gls{median}, \gls{mode}, and \gls{standarddeviation}.
&
0..1 \\
\hline

\gls{units}
&
The unit of measurement for the reported value of the data item.
\newline \gls{units} is an optional attribute.
\newline Data items in the \gls{sample} category \MUST report the standard units for the measured values.
\newline See \sect{units Attribute for DataItem} for a list of available standard units identified in the MTConnect Standard.
&
0..1 \\
\hline

\gls{nativeunits}
&
The native units of measurement for the reported value of the data item.
\newline \gls{nativeunits} is an optional attribute.
\newline See \sect{nativeUnits Attribute for DataItem} for a list of available native units identified in the MTConnect Standard.
&
0..1 \\
\hline

\gls{nativescale}
&
The \gls{nativeunits} may not be scaled to directly represent the original measured value. \gls{nativescale} \MAY be used to convert the reported value to represent the original measured value.
\newline \gls{nativescale} is an optional attribute.
\newline As an example, the \gls{nativeunits} may be reported as
\gls{gallonperminute}. The measured value may actually be in 1000  \gls{gallonperminute}. The value of the reported data \MAY be divided by the \gls{nativescale} to convert the reported value to its original measured value and units.
\newline If provided, the value \MUST be numeric.
&
0..1 \\
\hline

\gls{category}
&
\glsentrydesc{category}
\newline \gls{category} is a required attribute.
\newline The available options are \gls{sample}, \gls{event}, or \gls{condition}.
&
1 \\
\hline

\gls{coordinatesystem}
&
\glsentrydesc{coordinatesystem}
\newline \gls{coordinatesystem} is an optional attribute.
\newline The available values for \gls{coordinatesystem} are \gls{work} and \gls{machine}.
&
0..1 \\
\hline

\gls{compositionid}
&
The identifier attribute of the \gls{composition} element that the reported data is most closely associated.
\newline \gls{compositionid} is an optional attribute. 
&
0..1 \\
\hline

\gls{samplerate}
&
The rate at which successive samples of a data item are recorded by a piece of equipment.
\newline \gls{samplerate} is an optional attribute.
\newline \gls{samplerate} is expressed in terms of samples per second.
\newline If the \gls{samplerate} is smaller than one, the number can be represented as a floating point number.
\newline For example, a rate 1 per 10 seconds would be 0.1
&
0..1\\
\hline

\gls{representation}
&
Description of a means to interpret data consisting of multiple data points or as a single value.
\newline     
\newline \gls{representation} is an optional attribute.
\newline \gls{representation} defines the unique format for each set of data.
\newline \gls{representation} for \gls{timeseries representation}, \gls{discrete representation}, \gls{dataset}, and \gls{value representation} are defined in \sect{representation Attribute for DataItem}.
\newline If \gls{representation} is not specified, it \MUST be determined to be \gls{value representation}.
&
0..1 \\
\hline

\gls{significantdigits}
&
The number of significant digits in the reported value.
\newline \gls{significantdigits} is an optional attribute.
\newline This \SHOULD be specified for all numeric values.
&
0..1 \\
\hline



\gls{discrete}
&
An indication signifying whether each value reported for the \gls{data entity} is significant and whether duplicate values are to be suppressed.
\newline The value defined \MUST be either \gls{true removed} or \gls{false removed} - an XML boolean type.
\newline \gls{true removed} indicates that each update to the \gls{data entity}'s value is significant and duplicate values \MUSTNOT be suppressed.
\newline \gls{false removed} indicates that duplicated values \MUST be suppressed.
\newline If a value is not defined for \gls{discrete}, the default value \MUST be \gls{false removed}.
&
0..1 \\
\hline\end{longtabu}