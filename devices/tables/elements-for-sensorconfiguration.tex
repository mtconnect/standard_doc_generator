\tabulinesep = 5pt
\begin{longtabu} to \textwidth {
    |l|X[3l]|X[0.75l]|}
\caption{Elements for SensorConfiguration} \label{table:elements-for-sensorconfiguration} \\

\hline
Element & Description & Occurrence \\
\hline
\endfirsthead

\hline
\multicolumn{3}{|c|}{Continuation of Table \ref{table:elements-for-sensorconfiguration}}\\
\hline
Element & Description & Occurrence \\
\hline
\endhead

\gls{firmwareversion}
&
\glsentrydesc{firmwareversion}
\newline \gls{firmwareversion} is a required element if \gls{sensorconfiguration} is used.
\newline The data value for \gls{firmwareversion} is provided in the \gls{cdata} for this element and \MAY be any numeric or text
content.
&
1 \\
\hline

\gls{calibrationdate}
&
Date upon which the \gls{sensor unit} was last calibrated.
\newline The data value for \gls{calibrationdate} is provided in the \gls{cdata} for this element and \MUST be represented in the W3C ISO 8601 format.
&
0..1 \\
\hline

\gls{nextcalibrationdate}
&
Date upon which the \gls{sensor unit} is next scheduled to be calibrated.
\newline The data value for \gls{nextcalibrationdate} is provided in the \gls{cdata} for this element and \MUST be represented in the W3C ISO 8601 format.
&
0..1 \\
\hline

\gls{calibrationinitials}
&
\glsentrydesc{calibrationinitials}
\newline The data value for \gls{calibrationinitials} is provided in the \gls{cdata} for this element and \MAY be any numeric or text
content.
&
0..1 \\
\hline

\gls{channels}
&
\glsentrydesc{channels}
&
0..1 \\
\hline

\end{longtabu}