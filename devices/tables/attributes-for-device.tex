\tabulinesep = 5pt
\begin{longtabu} to \textwidth {
    |l|X[3l]|X[0.75l]|}
\caption{Attributes for Device} \label{table:attributes-for-device} \\

\hline
Attribute & Description & Occurrence \\
\hline
\endfirsthead

\hline
\multicolumn{3}{|c|}{Continuation of Table \ref{table:attributes-for-device}}\\
\hline
Attribute & Description & Occurrence \\
\hline
\endhead
 
\gls{id} 
&
\glsentrydesc{id}
\newline \gls{id} is a required attribute.
\newline An \gls{id} \MUST be unique across all the \gls{id} attributes in the document.
\newline An XML ID-type.
&
1 \\
\hline

\gls{nativename}
&
The common name normally associated with this piece of equipment.
\newline \gls{nativename} is an optional attribute. 
&
0..1 \\
\hline

\gls{sampleinterval} 
& 
An optional attribute that is an indication provided by a piece of equipment describing the interval in milliseconds between the completion of the reading of the data associated with the \gls{device} element until the beginning of the next sampling of that data. This indication is reported as the number of milliseconds between data captures.
\newline This information may be used by client software applications to understand how often information from a piece of equipment is expected to be refreshed.
\newline The refresh rate for all data from the piece of equipment will be the same as for the \gls{device} element unless specifically overridden by another \gls{sampleinterval} provided for a \gls{component} of the \gls{device} element.
\newline If the value of \gls{sampleinterval} is less than one millisecond, the value will be represented as a floating-point number. For example, an interval of 100 microseconds would be 0.1.
& 
0..1 \notesign \notesign \\
\hline

\deprecated{\gls{samplerate}}
&
\DEPRECATED in MTConnect Version 1.2. Replaced by \gls{sampleinterval}.
&
0..1 \notesign \notesign \notesign \\
\hline


\deprecated{\gls{iso841class}}
&
\DEPRECATED in MTConnect Version 1.1.
&
0..1 \notesign \notesign \notesign \\
\hline

\gls{uuid}
&
A unique identifier for this XML element.
\newline \gls{uuid} is a required attribute. 
\newline The uuid \MUST be unique amongst all uuid identifiers used in an MTConnect installation. 
\newline For example, this may be a combination of the manufacturer’s code and serial number. The \gls{uuid} \SHOULD be alphanumeric and not exceed 255 characters.
\newline An \gls{nmtoken} XML type.
&
1 \notesign \\
\hline

\gls{name}
&
The name of the piece of equipment represented by the \gls{device} element. 
\newline \gls{name} is a required attribute.
\newline This name \MUST be unique for each \gls{device} XML element defined in the \gls{mtconnectdevices} document.
\newline An \gls{nmtoken} XML type.
&
1 \\
\hline

\end{longtabu}