\onesection{Document Style Guidelines}\label{docguide}

The following conventions will be used throughout the document to provide a clear and consistent understanding of the use of each type of data and information used to define the \mtconnect standard and associated data.

\twosection{Elements, Attributes, and Data Items}\label{elements}

\begin{easylist}[enumerate]
	& All tag names (XML Elements) will be:
    	&& specified in Pascal case (first letter of each word is capitalized), Courier New, Size 12. For example: \cvoc{tag}{component events}. 
		&& XML element names will be spelled-out and abbreviations will be avoided. For example: \cvoc{tag}{sequence number} will be used instead of \cvoc{tag}{seq num}. 
		&& Currently the only abbreviation permitted is for ``Reference'' when used to indicate the association to an attribute identifier or another document. For example: \cvoc{tag}{target id ref}, \cvoc{tag}{archetype ref}.
		&& A second exception is with \cfont{Initial_Value} in \cvoc{tag}{constraints}.
    & Attribute names will be:
    	&& Camel case (similar to Pascal case, but the first letter will be lower case), Courier New, Size 12. For example: \cvoc{attribute}{attribute name}, \cvoc{attribute}{serial number}
        && XML attribute names will be spelled-out and abbreviations will be avoided. For example: \cvoc{attribute}{minimum} will be used instead of \cvoc{attribute}{min}.
		&& Currently the only abbreviation permitted is for ``Identifier'' where \cvoc{attribute}{id} will be used.
		&& When used as a specific identifier attribute use, for example, \cvoc{attribute}{targetId}, \cvoc{attribute}{stepId}.
	& When attributes and elements are generally referred to within the text without specific reference to their function within \mtconnect XML, they will use standard body text font Times New Roman, Size 12 and not capitalized -- Unless when used as a generic reference in a Title or beginning of a sentence..
	& All values that are part of a limited or controlled vocabulary (e.g., DataItems) will be in upper case with an \cfont{_} (underscore) separating words, Courier New, Size 12. For example: \cvoc{limited}{on}, \cvoc{limited}{off}, \cvoc{limited}{ACTUAL}, \cvoc{limited}{COUNTER CLOCKWISE}.
	& Within text, DataItem will always be considered one word with both the 'D' and 'I' capitalized.
	& Dates and times will follow the W3C ISO 8601 format with arbitrary decimal fractions of a second allowed. Refer to the following specification for details: \url{http://www.w3.org/TR/NOTE-datetime}. The format will be YYYY-MM-DDThh:mm:ss.ffff, for example 2007-09-13T13:01.213415. The accuracy and number of decimal fractional digits of the timestamp is determined by the capabilities of the device collecting the data. All times will be given in UTC (GMT).
\end{easylist}

\twosection{Table Text}\label{tabletext}

\begin{easylist}[enumerate]
	& First row (header) must be in Times New Roman, Size 12, Bold, White, and Centered with paragraph of 6pt spacing before and 6pt after and single line spacing.
	& Table paragraph rows are 6pt before and 6pt after with single line spacing and 0.03'' indents on both sides.
	& Attributes, Elements, Sub-Types and Values must be in Courier New, Size 10
	& Body test for Description and Occurrence columns must be in Times New Roman, Size 10
	& XML attributes, elements, DataItems and functions in body text will be in Courier New, Size 10.
\end{easylist}

\begin{table}[htb]
	\centering
    \footnotesize
	\begin{tabular}{|c|p{2.75in}|c|}
    	\hline
    	\rowcolor{mtc2}
    	\tblh{Element}	&	\multicolumn{1}{|c|}{\tblh{Description}}	&	\tblh{Occurrence}	\\
    	\hline
    	\cvoc{tag}{archetype ref}	&	An optional reference to the \cvoc{tag}{part archetype} document and \must specify the unique assetId of the archetype as the CDATA. The element can have an option \cfont{xlink:href} and \cfont{xlink:type} attribute that provides the URI to request the asset.	&	0..1 \\
    	\hline
	\end{tabular}
\end{table}

\twosection{Section (Header) Titles}\label{sectiontitles}

\begin{easylist}[enumerate]
	& First level - Size 18, Bold, with the color set at RGB red=43, green=105, blue=145.
    	&& \bfseries\color{mtc1}\Large 2  Part 
	& Second level - Size 14, Bold, with the color set at RGB red=43, green=105, blue=145.
		&& \bfseries\color{mtc1}\large 2.1  Part Attributes
	& Third level - Size 14, Bold, with the color of black.
		&& \bfseries\color{black}\large 2.1.1 Targets Element
	& Forth level - Size 12, Bold, with the color of black.
		&& \bfseries\color{black}\normalsize 2.1.1.1 SubCount Attributes
	& Fifth level - Size 12, Bold, with the color of black.
		&& \bfseries\color{black}\normalsize 2.1.1.1.1 Text
\end{easylist}

\twosection{Figure Titles}\label{figuretitles}

\begin{easylist}[enumerate]
	& Figure titles are Times New Roman font, Size 12, Bold, with the color of black and are represented as - \textbf{Figure XX: Title}. For the number itself use field code of: SEQ Figure \* ARABIC.  For example: \textbf{Figure 23: Part Model}
	& When referring to a Figure within a document, the figure text \must be italicized.
\end{easylist}

\twosection{Code Samples}\label{codesamples}

\begin{easylist}[enumerate]
	& Code samples may be truncated with an ellipsis to show optional CDATA or for excessively long snippets. Excessively long code snippets should be avoided in the first place, if possible.
	& Code samples will always be provided in fixed size 10, Courier New font with line numbers as in:
\code
\begin{lstlisting}
<MTConnectStreams xmlns:m="urn:mtconnect.com:MTConnectStreams:1.1"
    xmlns:xsi="http://www.w3.org/2001/XMLSchema-instance"
    xmlns="urn:mtconnect.com:MTConnectStreams:1.1"
\end{lstlisting}
\end{easylist}

\twosection{Other Conventions}\label{otherconventions}

\begin{easylist}[enumerate]
	& When special emphasis is required on a word or words to differentiate them from other words and to provide additional clarity to the meaning of the standard, these words may be \textit{italicized} or \textbf{bolded} (depending on the context of the surrounding text) to provide emphasis. For example: \textit{sensor element}, \textbf{Deprecated}\ldots \\
    \textbf{Note:} The use of CAPS should be avoided for the purpose of providing emphasis.
	& When referring to the XML elements \cvoc{tag}{Device}, \cvoc{tag}{Component} and \cvoc{tag}{Composition}, it is recommended that they be followed by the word 'element'. For example: \cvoc{tag}{Device} element.
	& Uses of the word 'device' generically should be avoided and replaced by the words 'piece of equipment' or other terms. 
Note: 'Equipment' is a defined word in Part 1 of the standard.
	& References to \doc{Documents} or \doc{Sections} or \doc{Sub-Sections} of this document must be \textit{italicized}. For example: refer to \doc{Part 4.0 -- Assets} of the standard.
	& References to various versions the standard must be italicized and be preceded with the word 'MTConnect'. For example: \ver{1.1}
	& Terms will be written as \textit{Capitalized Words} and placed in \textit{italics}. Examples are as follows: \textit{Structural Elements}, \textit{Asset Archetype}, or \textit{Asset Instance}.
\end{easylist}

\begin{figure}[htb]
	\centering
    \includegraphics[width=0.5\linewidth]{mtconnect-v2.png}
    \figcap{The old \mtconnect logo}
    \label{mtclogo}
\end{figure}

