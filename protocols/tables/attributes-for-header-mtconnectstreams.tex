\tabulinesep = 5pt
\begin{longtabu} to \textwidth {
    |l|X[3l]|X[0.75l]|}
\caption{MTConnectStreams Header} \label{table:attributes-for-header-mtconnectstreams} \\

\hline
Attribute & Description & Occurrence \\
\hline
\endfirsthead

\hline
\multicolumn{3}{|c|}{Continuation of Table \ref{table:attributes-for-header-mtconnectstreams}}\\
\hline
Attribute & Description & Occurrence \\
\hline
\endhead
 
\gls{version}
&
The \gls{major}, \gls{minor}, and \gls{revision} number of the MTConnect Standard that defines the \gls{semantic data model} that represents the content of the \gls{response document}.   It also includes the revision number of the \gls{schema} associated with that specific \gls{semantic data model}.
\newline The value reported for \gls{version} \MUST be a series of four numeric values, separated by a decimal point, representing a \gls{major}, \gls{minor}, and \gls{revision} number of the MTConnect Standard and the revision number of a specific \gls{schema}.  
\newline As an example, the value reported for \gls{version} for a \gls{response document} that was structured based on \gls{schema} revision 10 associated with Version 1.4.0 of the MTConnect Standard would be:  1.4.0.10
\newline \gls{version} is a required attribute.
&
1 \\
\hline

\gls{creationtime}
&
\gls{creationtime} represents the time that an \gls{agent} published the \gls{response document}. 
\newline \gls{creationtime} \MUST be reported in UTC (Coordinated Universal Time) format; e.g., "2010-04-01T21:22:43Z".
\newline Note:  Z refers to UTC/GMT time, not local time.
\newline \gls{creationtime} is a required attribute.
&
1 \\
\hline

\gls{nextsequence}
&
A number representing the \gls{sequence number} of the piece of \gls{streaming data} that is the next piece of data to be retrieved from the \gls{buffer} of the \gls{agent} that was not included in the Response Document published by the \gls{agent}.
\newline If the \gls{streaming data} included in the Response Document includes the last piece of data stored in the \gls{buffer} of the \gls{agent} at the time that the document was published, then the value reported for \gls{nextsequence} \MUST be equal to \gls{lastsequence} + 1.
\newline The value reported for \gls{nextsequence} \MUST be a number representing an unsigned 64-bit integer.
\newline \gls{nextsequence} is a required attribute.
&
1 \\
\hline

\gls{lastsequence}
&
A number representing the \gls{sequence number} assigned to the last piece of \gls{streaming data} that was added to the \gls{buffer} of the \gls{agent} immediately prior to the time that the \gls{agent} published the Response Document.   
\newline The value reported for \gls{lastsequence} \MUST be a number representing an unsigned 64-bit integer.
\newline \gls{lastsequence} is a required attribute.
&
1 \\
\hline

\gls{firstsequence}
&
A number representing the \gls{sequence number} assigned to the oldest piece of \gls{streaming data} stored in the \gls{buffer} of the \gls{agent} immediately prior to the time that the \gls{agent} published the Response Document.   
\newline The value reported for \gls{firstsequence} \MUST be a number representing an unsigned 64-bit integer.
\newline \gls{firstsequence} is a required attribute.
&
1 \\
\hline

\gls{testindicator}
&
A flag indicating that the \gls{agent} that published the \gls{response document} is operating in a test mode.  The contents of the \gls{response document} may not be valid and \SHOULD be used for testing and simulation purposes only. 
\newline The values reported for \gls{testindicator} are:
\newline -	  \gls{true value}:  The \gls{agent} is functioning in a test mode.
\newline -	  \gls{false value}:  The \gls{agent} is not functioning in a test mode.
\newline If \gls{testindicator} is not specified, the value for \gls{testindicator} \MUST be interpreted to be \gls{false value}.
\newline \gls{testindicator} is an optional attribute.
&
0..1 \\
\hline

\gls{instanceid}
&
A number indicating a specific instantiation of the \gls{buffer} associated with the \gls{agent} that published the \gls{response document}.  
\newline The value reported for \gls{instanceid} \MUST be a unique unsigned 64-bit integer.   
\newline The value for \gls{instanceid} \MUST be changed to a different unique number each time the \gls{buffer} is cleared and a new set of data begins to be collected.
\newline \gls{instanceid} is a required attribute.
&
1 \\
\hline

\gls{sender}
&
An identification defining where the \gls{agent} that published the \gls{response document} is installed or hosted.
\newline The value reported for \gls{sender} \MUST be either an IP Address or Hostname describing where the \gls{agent} is installed or the URL of the \gls{agent}; e.g., \cfont{http://<address>[:port]/}. 
\newline Note:  The port number need not be specified if it is the default HTTP port 80.
\newline \gls{sender} is a required attribute.
&
1 \\
\hline

\gls{buffersize}
&
A value representing the maximum number of \glspl{data entity} that \MAY be retained in the \gls{agent} that published the \gls{response document} at any point in time.
\newline The value reported for \gls{buffersize} \MUST be a number representing an unsigned 32-bit integer.
\newline \gls{buffersize} is a required attribute. 
\newline Note 1:  \gls{buffersize} represents the maximum number of  \glspl{sequence number} that \MAY be stored in the \gls{agent}. 
\newline Note 2: The implementer is responsible for allocating the appropriate amount of storage capacity required to accommodate the \gls{buffersize}.
&
1 \\
\hline

\gls{devicemodelchangetime}
&
\glsentrydesc{devicemodelchangetime}
&
1 \\
\hline


\end{longtabu}